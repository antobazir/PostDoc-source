\documentclass[11pt,a4paper]{article}
\usepackage[utf8]{inputenc}
\usepackage{geometry}
\usepackage[dvipsnames]{xcolor}


\begin{document}

\tableofcontents
\section*{Introduction}
This document is supposed to be a bibliography focusing specifically on multicellular layers or mutlilayer cell cultures and their use in the field of tumor growth or more general cancer rresearch. 

Definition found in (2011). Multicellular Layers. In: Schwab, M. (eds) Encyclopedia of Cancer. Springer, Berlin, Heidelberg.: "MCL, multilayered cell cultures (MCC); are planar aggregates of tumor cells grown on porous support membranes in culture medium which are a model for the extravascular compartment of solid tumors considered to more accurately reflect the tumor microenvironment than monolayer cultures. They have a very similar structure to multicellular spheroids often with regions of hypoxia and central necrosis due to diffusion limitations of oxygen, glucose, and other nutrients. Their well-defined planar structure makes them very useful for drug transport studies. Some cell lines that do not grow as spheroids will grow as multilayers. They are more expensive than spheroids with lower throughput and are used mainly for drug transport and binding and drug effect studies." Logical if we consider the focus on drugs.


\section*{Novel application of multicellular layers culture for in situ evaluation of cytotoxicity and penetration of paclitaxel, Al-Abd AM et al.,\textit{Cancer Science}, 2008}
\subsection*{Abstract}
Limited drug penetration into tumor tissue is one of the major factors causing clinical drug resistance in human solid tumors. The multicellular layers (MCL) of human cancer cells have been successfully used to study tissue pharmacokinetics of anticancer drugs. The purpose of this study was to develop a direct and simple method to evaluate vitality changes in situ within MCL using calcein-AM. Human colorectal (DLD-1, HT-29) and bladder (HT-1376, J-82) cancer cells were grown in Transwell inserts to form MCL and subjected to paclitaxel exposure. The drug distribution was evaluated using paclitaxel-rhodamine. Photonic attenuation and limited penetration of calcein-AM prevented cellular vitality evaluation on optical sections under confocal microscopy in DLD-1 MCL. However, direct measurement of the fluorescence intensity on frozen sections of MCL allowed successful vitality assessment in more than 80\% depth for HT-29 and J-82 MCL and in the upper 40\% depth for DLD-1 and HT-1376 MCL. The penetration of paclitaxel-rhodamine was greater in HT-29 than DLD-1 and its distribution pattern was correlated to the spatial profile of vitality deterioration in both MCL, suggesting that tissue penetration may be an important determinant of drug effect in tumors. In conclusion, a novel method for vitality evaluation in situ within MCL was developed using calcein-AM. This method may provide clinically relevant data regarding the spatial pharmacodynamics of anticancer agents within avascular regions of solid tumors. 

\subsection*{Notes}
\begin{itemize}
\item "Wilson et al. introduced the MCL as a 3D in vitro system and used it to study the extravascular pharmacokinetics of anticancer drugs such as tirapazamine and anthracyclines.(12–14) MCL have been especially successful when used to study the pharmacokinetic–pharmacodynamic (PK-PD) relationships of chemotherapeutic agents. (15,16)
\item PK/PD "Pharmacokinetics is the study of how an organism affects a drug, whereas pharmacodynamics (PD) is the study of how the drug affects the organism." (wikipedia)
\item \underline{Pharmacokinetics:} "Process of the uptake of drugs by the body, the biotransformation they undergo, the distribution of the drugs and their metabolites in the tissues, and the elimination of the drugs and their metabolites from the body over a period of time." (IUPAC)
\item \underline{Pharmacodynamics:} "Study of pharmacological actions on living systems, including the reactions with and binding to cell constituents, and the biochemical and physiological consequences of these actions" (IUPAC)
\item "Cells were grown on collagen-coated micro- porous (0.4 \textmu m) membranes in Transwell inserts at a plating density of 3 × 10$^5$ cells/100 \textmu L for colorectal cancer cell lines and 1 × 10$^6$ cells/100 \textmu L for bladder cancer cell lines." Orders of magnitude for cell seeding
\item "Endosomal pH and sequestration significantly influenced the penetration of alkaline drugs such as doxorubicin. (35) The acidic pH and hypoxia of the deep layers of solid tumors caused limited penetration of drugs such as methotrexate, tirapazamine and topoisomerase poisons." reasons why including pH may be necessary
\end{itemize}

\section*{Multicellular membranes as an in vitro model for extravascular diffusion in tumours, D. S. Cowan et al., \textit{Br J Cancer suppl.}, 1996}
\subsection*{Abstract}
Efficient extravascular diffusion is a critical requirement for hypoxic cell radiosensitisers, bioreductive drugs and hypoxic cell markers that must reach cells distant from functional blood vessels in tumours. The diffusion of simple nitroimidazoles with neutral (misonidazole, miso) and basic (pimonidazole, pimo) side chains, as well as 2-nitroimidazoles with acridine (NLA-1) and phenanthridine (2-NLP-3) DNA intercalating moieties was investigated using multicellular membranes (MMs), a new in vitro model for the extravascular compartment of tumours. The diffusion of miso through V79 MMs was concentration independent over the range 0.1-10 mM. Mathematical modelling of the flux kinetics provided a diffusion coefficient in MMs (DMM) of 5.5 x 10(-7) cm2 s-1 which was approximately 13-fold lower than in culture medium. Flux was little affected by the extent of hypoxia in MMs, indicating that hypoxic metabolism does not compromise diffusion of miso over distances in the order of 200 microns. The DMM for pimo was similar to miso, while those for 2-NLP-3 and NLA-1 were both lower. The results demonstrate compromised extravascular diffusion for DNA-intercalating nitroimidazoles, but indicate that this problem is more severe for the basic acridine derivative, NLA-1, than for the phenanthridine, 2-NLP-3. The MM model appears to be well suited to quantitative determination of drug diffusion in a multicellular environment.

\subsection*{Notes}
Uses the insert for better understanding and measurement of diffusion processes but not much more.

\section*{The multilayered postconfluent cell culture as a model for drug screening, Jose Padron et al., Critical Reviews in Oncology Hematology, 2000}
\subsection*{Abstract}
New drug development requires simple in vitro models that resemble the in vivo situation more in order to select active drugs against solid tumours and to decrease the use of experimental animals. In this paper, we review the characteristics and scope of a relatively simple cell-culture system with a three-dimensional organisation pattern — the multilayered postconfluent cell culturemodel. Solid tumour cell lines from diverse origins when grown in V-bottomed microtiter plates reach confluence in 3 – 5 days and then start to form multilayers. The initial exponential growth of the culture is followed by a plateau phase when cells reach confluence. This produces changes in the morphology of the cells. For some cell lines, it is possible to observe cell differentiation. A substantial advantage of the system is the use of the sulforodamine B (SRB) assay to determine relative cell growth or viability, which allows semiautomation of the experiments. Several experiments were performed to assess the differences and similarities between cells cultured as monolayers and multilayers, and eventually, compared with the results for solid tumours and some other models such as spheroids. Cell-cycle analysis for multilayers showed a lower S-phase arrest, which is accompanied by a decrease in the expression of cell-cycle-related proteins and a decrease in cellular nucleotide pools. Gene and protein expression of topoisomerase I, topoisomerase II and thymidylate synthase expression were lower for multilayers, but no substantial changes were observed for the expression of DT-diaphorase. P53 expression increased. Multilayer cultures present distinctive properties for drug transport across the membrane, drug accumulation and retention. In fact, the transport of antifolates across the membrane, accumulation of topotecan and gemcitabine-triphosphate are reduced in multilayers when compared with monolayers, which may be related to a decrease in drug penetration to the inner regions of the multilayers. Alteration of these pharmacodynamic parameters is directly related to a decrease in drug activity. The most powerful application of multilayers is in the assessment of cytotoxicity. Solid tumour cell lines from different origins have been treated with several conventional and investigational anticancer drugs. The data show that multilayers are more resistant to the drugs than the corresponding monolayers, but there are substantial differences between the drugs depending on culture conditions, e.g. the difference was rather small for a drug such as cisplatin, miltefosine and EO9, a drug, which is activated under hypoxic conditions. Gemcitabine was active against ovarian cancer but not against colon cancer, resembling the in vivo situation. This observation was not evident with monolayer experiments. Another interesting application is the possibility to perform drug combination studies. The combination of gemcitabine and cisplatin proved to produce selective cell kill in H322 cells (non-small cell lung cancer cell line). Neither of the drugs was independently able to produce similar effects. In summary, multilayer cultures are relatively simple three-dimensional systems to study the effect of microenvironmental conditions on anticancer drug activity. The model might serve as a base for a more rigorous secondary in vitro screening

\subsection*{Notes}
\begin{itemize}
\item "Solid tumour cells cultured in V-bottomed microtiter plates display a pattern of organisation that mimics
microenvironmental conditions occurring in the in vivo situation. We have observed marked differences in metabolism and sensitivity between the monolayers and multilayers, which are in favour of the multilayer system to predict selective chemosensitivity in solid tumours. "
\item Not read fully so far but the system is very different from what we use so not so useful.
\end{itemize}

\section*{Micropatterning neural cell cultures in 3D with a multi-layered scaffold, Anja Kunze et al., \textit{Biomaterials}, 2010}
\subsection*{Abstract}
Cortical neurons, in their native state, are organized in six different cell layers; and the thickness of the
cell layer ranges from 0.12 mm to 0.4 mm. The structure of cell layers plays an important role in neurodegenerative diseases or corticogenesis. We developed a 3D microfluidic device for creating physiologically realistic, micrometer scaled neural cell layers. Using this device, we demonstrated that (1) agaroseealginate mixture can be gelled thermally, thus an excellent candidate for forming multi-layered scaffolds for micropatterning embedded cells; (2) primary cortical neurons were cultured successfully for up to three weeks in the micropatterned multi-layered scaffold; (3) B27 concentration gradient enhanced neurite outgrowth. In addition, this device is compatible with optical microscopy, the dynamic process of neural growth can be imaged, and density and number of neurites can be quantified. This device can potentially be used for drug development, as well as research in basic neural biology.

\subsection*{Notes}
\begin{itemize}
\item not very useful for me either
\end{itemize}

\section*{The role of bystander effects in the antitumor activity of the hypoxia-activated prodrug PR-104, Foehrenbacher et al., \textit{Frontiers in Oncology}, 2013}
\subsection*{Abstract}
Activation of prodrugs in tumors (e.g., by bioreduction in hypoxic zones) has the potential to generate active metabolites that can diffuse within the tumor microenvironment. Such “bystander effects” may offset spatial heterogeneity in prodrug activation but the relative importance of this effect is not understood. Here, we quantify the contribution of bystander effects to antitumor activity for the first time, by developing a spatially resolved pharmacokinetic/pharmacodynamic (SR-PK/PD) model for PR-104, a phosphate ester pre-prodrug that is converted systemically to the hypoxia-activated prodrug PR-104A. Using Green’s function methods we calculated concentrations of oxygen, PR-104A and its active metabolites, and resultant cell killing, at each point of a mapped three-dimensional tumor microregion. Model parameters were determined in vitro, using single cell suspensions to determine relationships between PR-104A metabolism and clonogenic cell killing, and multicellular layer (MCL) cultures to measure tissue diffusion coefficients. LC-MS/MS detection of active metabolites in the extracellular medium following exposure of anoxic single cell suspensions and MCLs to PR-104A confirmed that metabolites can diffuse out of cells and through a tissue-like environment. The SR-PK/PD model estimated that bystander effects contribute 30 and 50\% of PR-104 activity in SiHa and HCT116 tumors, respectively. Testing the model by modulating PR-104A-activating reductases and hypoxia in tumor xenografts showed overall clonogenic killing broadly consistent with model predictions. Overall, our data suggest that bystander effects are important in PR-104 antitumor activity, although their reach may be limited by macroregional heterogeneity in hypoxia and reductase expression in tumors. The reported computational and experimental techniques are broadly applicable to all targeted anticancer prodrugs and could be used to identify strategies for rational prodrug optimization.

\subsection*{Notes}
\begin{itemize}
\item \underline{Prodrug:} A prodrug is a medication or compound that, after intake, is metabolized (i.e., converted within the body) into a pharmacologically active drug
\item "Here, we utilize a spatially resolved pharmacokinetic/pharmacodynamic (SR-PK/PD) modeling approach to dissect the contribution of bystander effects "
\item [...]multicellular layer (MCL) cultures (27) to determine extravascular transport properties of PR-104A, PR-104H, and PR-10M."
\item Interesting insghts on modelling vascularised systems
\item \textbf{Maybe a better prediction of diffusion in MCL could be an interesting question}
\end{itemize}

\end{document}