\documentclass[11pt,a4paper]{article}
%\usepackage[utf8]{inputenc}
%\usepackage[ascii]{inputenc}
\usepackage{geometry}
\usepackage[dvipsnames]{xcolor}
\usepackage{textcomp}
\usepackage{graphicx}
\usepackage{caption}
\usepackage{subcaption}
\usepackage{amsmath}

\begin{document}
**\textbf{Oxidative Stress in Cancer Cell Metabolism}\\

-"Reactive oxygen species (ROS), the partially reduced metabolites of oxygen that possess strong oxidizing capabilities, are deleterious to cells at high concentrations but at low concentrations, they serve complex signaling functions."\\

"There is also some evidence that cancer cells decrease mitochondrial respiration in the presence of oxygen, which suppresses apoptosis [25]. Under hypoxic conditions, the accelerated metabolism produces ROS in cancer cells that is countered by the increased NADPH which is met by the upregulated glycolysis [26,27]. "\\

-"However, high levels of mtROS are capable of inducing apoptosis by oxidation of the mitochondrial pores and autophagy by the oxidation of autophagy-specific gene 4 (ATG4) [5]. Depending on the tumor cell microenvironment,"\\

-"Latest studies show tumor cells have the capability to carry about both glycolytic and oxidative phosphorylation (OXPHOS) metabolism which makes them resistant to oxidative stress through enhanced antioxidant response and increased detoxification capacity [32]. The changes related to energy metabolism may be correlated to the expression of certain p53 downstream genes regulated by it,"\\

- high levels of ROS have the ability to induce cell cycle arrest, senescence, and cancer cell death either by the initiation of intrinsic apoptotic signaling in the mitochondria or by extrinsic apoptotic signaling by the death receptor pathways [63].\\

-"autophagy plays a pro-tumoral role by eliminating ROS-induced metabolic stress and producing nutrients required for cancer cell survival. The cancer cells under hypoxia induce the formation of ROS which can activate autophagy in neighboring stroma cells which then provide high-energic nutrients, such as lactate or ketones, necessary for cancer cell survival and proliferation in accordance with what we have seen earlier, also termed as “tumor-stromal co-evolution”."\\

**\textbf{Oxidative Phosphorylation: A Target for Novel Therapeutic Strategies Against Ovarian Cancer}\\

-Glycolysis is indicated by high expression of HIF-1$\alpha$ (Hypoxia inducible factor-1$\alpha$) and low levels of phospho-5’ AMP-activated protein kinase (pAMPK), whereas OXPHOS-reliant tumors have low levels of HIF-1$\alpha$ and high levels of pAMPK. Some cancer cells express high levels of both HIF-1$\alpha$ and pAMPK indicating active glycolysis as well as OXPHOS. 


\end{document}