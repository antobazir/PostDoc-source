\documentclass[11pt,a4paper]{article}
%\usepackage[utf8]{inputenc}
%\usepackage[ascii]{inputenc}
\usepackage{geometry}
\usepackage[dvipsnames]{xcolor}
\usepackage{textcomp}
\usepackage{graphicx}
\usepackage{caption}
\usepackage{subcaption}
\usepackage{amsmath}
\usepackage{tikz}

\begin{document}
\section{Introduction}
Cancer cells have become infamous for their versatility which translates into possible metastases and resurgences long after extensive treatments and apparent recoveries. Cancer is a multifactorial disease reacting and interacting with almost every possible aspects of the tumor microenvironment which includes its mechanical and chemical microenvrionment, and the surrounding immune system. The possible response mechanism to changes in either those variables are being studied by biologists and pharmacologists as many possible therapeutic options and metabolic pathways are part of those studies.\\

Cancer cells are known to exhibit a variety of response to various metabolic cues, which are believed to potentially arise from the selective pressure due to the harsh conditions generated by uncontrolled proliferation.\cite{Griguer2005}\cite{Azzalin2020}\cite{Yusuf2022}\cite{Stuart2023} This variety translates into cancer cell lines from a given a tissue being addicted or not to certain substrate such as glucose or glutamine. Addiction being understood here as provoking significant decrease in cell viability in case the nutrient in question goes scarce. Whether a cell is addicted to a given substrate/nutrient depends on its genome, transcriptome and proteome. For example, in some glioblastoma cell lines, the compensation of glucose scarcity through pathways using glutamine as a carbon source are not efficient enough to rescue cells.\cite{Yusuf2022}\\

Modelling studies aiming at reproduction of growth features of tissue and notably spheroids have been performed by several groups.\cite{Cleri2019}\cite{Kempf2015}\cite{Rejniak2012} A variety of modelling tools have been used but this study will focus mostly on the agent-based model variants. The general structure of these studies involve the modelling of nutrient concentrations and their dynamics along with a population of cells that metabolise those nutrients. In most cases, the studied nutrients were glucose and/or oxygen and their abundance would directly influence the modelled growth behavior.\cite{Kempf2005}\cite{Mao2018}\cite{Bull2020}\cite{Jagiella2016} One common feature of those models is that they all relied heavily on experimental data meaning that they do not adress the question on the metabolic variety.\\

There are possible tools to adress the question of this metabolic variety. Studies have included genetic diversity into agent-based models which could be a way of addressing the question.\cite{Sun2017}\cite{Sottoriva2010}\cite{Enderling2009}\cite{Iwasaki2021} Metabolic flux analyses are also a potential tool that can help understand how metabolism can work on the molecular level,\cite{Damiani2017} and a study by Shan and collaborators coupled flux balance analysis to an agent-based model.\cite{Shan2018}\\

The aim of this study is to propose a hybrid agent-based model that addresses the question of metabolic variety. First of all, a general physical study of the most studied subtrate/nutrients, glucose and oxygen will be proposed in order to understand their possible steady state distribution in a model cancer tissue in two different configurations : a sparse matrix/fluid filled tissue with cells, and a dense, spheroid-like kind of tissue. Then, a literature review of possible addictions and "coping mechanism" in various cancer cell lines will be proposed. Once that is done, results for homogeneous tissue models will be proposed and the question of heterogeneous tissue model will be considered.\\

\section{Tissue models, Oxygen and glucose dynamics}
Two different configurations of tissue are explored in this study. The first configuration explored is the dense tumor configuration. The second is the diffuse tumor configuration. What is deisgnated as tissue in this study is the mix of live cells and the surrounding extracellular matrix. At this stage, no stromal cells, blood vessels, or immune system cells are included in the simulation, which makes them closer to in vitro culture models. The physical properties of cells and the surrounding matrix relevant to our model are presented in the following.\\

The physical quantities monitored in this study are the concentration and diffusion of oxygen and glucose. Therefore, the most relevant physical quantities are the diffusion coefficients in tissues of various cell densities, and the average consumption rates of those molecules by cells in the tissue. Cellular consumption rates depend on the studied cell line. The apparent diffusion coefficient of a tissue depend on the cell density and the nature of the extracellular matrix.\\

\subsection{Extracellular matrix structure}
In that subsection the general strcutrure of extracellular matrix is presented. A first important note is that the presented structure to the matrix structure found in connective tissue and not basal lamina which is more specific.\\ 

As explained by Cooper in chapter 16 of his book \textit{The cell: A molecular approach}, "Extracellular matrices are composed of tough fibrous proteins embedded in a
gel-like polysaccharide ground substance"\cite{Cooper2006}. The most abundant component in most types of extracellular matrix is collagen which can be organised in fibers or networks depending on tissue structure. Cells in the tissue are attached to the collagen fibers or networks through fibronectins and integrins, laminins and other membrane proteins, which also provides attachment sites to other proteins. In tissue subjected to extension, elastic fibers are also present and are made of a cross-linked network of a class of protein called elastins. The aforementioned fibers. The gel-like ground structure is made of proteoglycans. "Proteoglycans consist of a central protein “core” to which long, linear chains of disaccharides, called glycosaminoglycans (GAGs), are attached."\cite{Lewin2013} The Proteoglycans play a structural role by retaining water and providing mechanical strength to th matrix along with other fibers. Hyaluronic acid, or Hyaluronan, is a specific case of glycoaminoglycans that does not bind to a central protein core.\\

The difference in structure and concentrations of all the aforementioned components, along with the repartition of cells, is part of what differentiates tissue properties. This structure and its properties can also evolve with time and pathologies such as cancer.\cite{Stewart2010} For example, gliomae present an Collagen-enriched ECM compared with the normally collagen poor healthy brain tissue.\cite{Marino2023}. Karamanos and collaborators show the different known structure of ECM in vairous healthy tissue, illustrating the varying abundance of all the aforementioned components.\cite{Karamanos2021} Therefore, exploring metabolic may also mean accounting for different diffusion behaviors.\\

\subsection{Experimental data on diffusion of glucose and oxygen}
As explained in the previous subsection, the structure and composition of the ECM can vary between tissues. In this section, data from the literature is presented in order to show the spread and sometimes non-intuitive values of diffusion encountered for both glucose and oxygen in relevant experiments.\\

\subsubsection{Diffusion data on glucose}

Experimental data on glucose presented here is summarized in table \ref{diff_glc}.

\begin{table}[h!]
\begin{center}
\begin{tabular}{ |p{18mm}|p{35mm}|p{30mm}|p{7mm}| }
\hline
 \textbf{Molecule}  & \textbf{Diffusion medium} & \textbf{Diffusion\ coeff.} \textmu m$^2$/min  & Ref. \\
 \hline
 \hline
 Glucose & Water & 36000  & \cite{Hober1947} \\
 \hline
   Glucose & Water 27°C & 42000 & \cite{Suhaimi2016}\\
  \hline
   Glucose & Water 37°C & 60000 & \cite{Suhaimi2016}\\
  \hline
 Glucose  & 0.5\% agarose & 36000  &  \cite{Weng2005}\\
 \hline
 Glucose & 2.16 mg /mL(0.21\%) type I \ Coll. gel & 7800  & \cite{Rong2006}\\
 \hline
  Glucose  & 0.5-2.5\% agarose/PBS & 36000-48000  & \cite{Hadler1980}\\
 \hline
   Glucose & 0.5-1\% HA/PBS & 40000-42000 & \cite{Hadler1980}\\
 \hline
   Glucose  & 2.5\% HA/PBS & 120000  & \cite{Hadler1980}\\
 \hline
    Glucose & Dura Mater (20°C) (Collagen) & 9780 & \cite{Bashkatov2003}\\
    \hline
     Glucose & Rat Brain & 8400 & \cite{Pfeuffer2000}\\
       \hline
     Glucose & 50:50 MCF-7/agarose gel  & 54000-84000 & \cite{Zijl1991}\\
       \hline
 Glucose & Human cancer Spheroids & 1200-3600 & \cite{Casciari1988}\\
 \hline     
  Glucose & EMT6/Ro spheroids & 6300  & \cite{Casciari1988}\\
 \hline
  Glucose & HCT116 tissue & 1200 & \cite{Mao2018}\\
 \hline
  Glucose & DMEM 27°C & 30000 & \cite{Suhaimi2016}\\
  \hline
   Glucose & DMEM 37°C & 36000 & \cite{Suhaimi2016}\\
  \hline
    Glucose & Water 27°C + Coll. & 600000 & \cite{Suhaimi2016}\\
  \hline
   Glucose & Water 37°C + Coll. & 660000 & \cite{Suhaimi2016}\\
  \hline
   Glucose & DMEM 27°C + Coll. & 220000 & \cite{Suhaimi2016}\\
  \hline
   Glucose & DMEM 37°C + Coll. & 220000 & \cite{Suhaimi2016}\\
  \hline
    Glucose & Collagen gel & 7800 & \cite{Rong2006}\\
  \hline
    \end{tabular}
\caption{Available data on the diffusion of glucose in tissues and gels \label{diff_glc}}   
\end{center}
\end{table}

The main focus in this study is put both on the matrix and tissue values. The tissue and spheroid value are important and directly useful in the dense tissue configuration. In the sparsely populated tissue the question of the diffusion properties of the matrix becomes more important.\\

In terms of tissue, the values for glucose diffusion coefficient range between 1200 and 8400 µm$^2$/min. The values on spheroids have been measured on various cancer cell lines by Casciari and collaborators.\cite{Casciari1988} The measurements were perfored with radiolabelled L-glucose which means that the diffusion coefficient is not biased by cellular consumption. It also the case for the measurements of Mao and collaborators. For Pfeuffer and collaborators, the measurements were performed on \textit{in vivo} rat brain tissue using Nuclear Magnetic Resonance (NMR). In that case the internalisation of glucose is most likely harder to account for.\\

The measurement of Suhaimi and collaborators on a collagen gel in DMEM and those of Hadler on Hyaluronic acid gels in PBS, show that diffusion parameters can vary significantly with concentration, and possibly in a non monotonous way.\cite{Suhaimi2016}\cite{Hadler1980} In practice, this means that diffusion properties may depend significantly on the type of matrix that surrounds the cells. In order to account, for this possible variety, instead of choosing a single diffusion coefficient for both dense tissue and matrix, the diffusion coefficient of each phase is taken to vary within an order of magnitude.\\ 

\subsubsection{Diffusion data on oxygen}

\begin{table}[h!]
\begin{center}
\begin{tabular}{ |p{18mm}|p{35mm}|p{20mm}|p{7mm}| }
 \hline

 \textbf{Molecule}  & \textbf{Diffusion medium} & \textbf{Diffusion\ coeff.} \textmu m$^2$/min  & Ref. \\
 \hline
  \hline
      Oxygen & water (25°C) & 120000   & \cite{Hober1947}\\
 \hline    
      Oxygen & water (37°C) & 180000   & \cite{Xing2014}\\
 \hline   
       Oxygen  & water (37°C) & 210000   & \cite{Wise1966}\\
 \hline  
 Oxygen  & Rat liver (37°C) & 216000   & \cite{Macdougall1967}\\
 \hline
  Oxygen & DS-Carcinoma (37°C) & 105000   & \cite{Grote1977}\\
 \hline
    Oxygen  & 6\% agarose (25°C) & 120000   & \cite{McCabe1975}\\
 \hline
      Oxygen  & 6\% agarose + 3.5mg/mL HA (25°C) & 114000   & \cite{McCabe1975}\\
 \hline
   Oxygen  & 5\% agarose & 120000   & \cite{Figueiredo2018}\\
 \hline
   Oxygen  & Si-HPMC 1\%+HEPES+NaCl & 18000   & \cite{Figueiredo2018}\\
 \hline
    Oxygen  & Coll I 0.1\% & 150000   & \cite{Figueiredo2018}\\
 \hline
     Oxygen  & agarose 2\% (30°C) & 120000   & \cite{Hulst1987}\\
     \hline
\end{tabular}
\caption{Available data on the diffusion of oxygen in tissues and gels \label{diff_Ox}}   
\end{center}
\end{table}

With the exception of specific gels such as the Si-HPMC tested by Figueiredo and collaborators, most value for hydrogels are between 100000 µm$^2$/min and 150000 µm$^2$/min, with temperatures in the range of 25°C-30°C. It is interesting to note that in the work of Hulst, some gels (not in table \ref{diff_Ox}) display a non-monotonous behavior of the diffusion coefficient with respect to concentration.\cite{Hulst1987} It should also be noted that in live tissues case the influence of oxygen consumption may impact the apparent diffusion coefficient as well.\\

There are no measurements of the oxygen diffusion coefficient on the matrix that is used in the experiments. Therefore, no precise value can be given. The choice here is to consider that the coefficient can vary in the range of 20000-200000 \textmu m$^2$/min. The higher bound is due to the fact that the temperature in the model is taken to be 37°C, which has been shown in the data to be associated with significant increases compared to lower temperature value (60 \% in water). The lower bound is acknowledgement that data on the precise structure of the modelled extracellular matrix is unavailable and that diffusion may be suboptimal in this polymer construct.\\

It also important to note that contrary to tissue presented in that study, the tumor on chip configuration is not a dense one. Therefore the sparse distribution of cell may mean that the diffusion behavior should logically be closer to that of a gel.\\


\subsection{Experimental data on cellular consumption of glucose and oxygen}
The second important physical quantity that needs to be known in order to describe the spatial distribution of a nutrient in the tissue is the average consumption rate of the nutrient. Similar to the case of diffusion, a literature review was performed in order to gather relevant data in tissue or hydrogels in order to later feed this data into the model.\\

\subsubsection{Glucose cellular consumption rate}

In this part, glucose consumption on various cancer cell lines are presented. All values in this table are presented both as they were reported in the original study and  in mM per minute which may in some cases imply conversion from other units.

\begin{table}[h!]
\begin{center}
\begin{tabular}{ |p{18mm}|p{26mm}|p{35mm}|p{20mm}|p{25mm}|p{7mm}| }
 \hline

 \textbf{Cell line}  & \textbf{Measurement method} & \textbf{Reported  value} & Glucose conc. (g/L) &  \textbf{Consumption rate} $\cdot$10$^{17}$ mol cell$^{-1}$ s$^{-1}$  & Ref. \\
 \hline
 HCT116 & Diffusion apparatus & (7.12$\pm$0.82)$\cdot$10$^{-17}$ mol cell$^{-1}$ s$^{-1}$& 0.9-1.8 & 7.1 & \cite{Mao2018}\\
 \hline
  %A431 & Liquid phase culture  & 0.45$\pm$0.05 mg mL$^{-1}$ h$^{-1}$ cell$^{-1}$  & 0.042 & \cite{Pilarek2013}\\
 %\hline
   A431 & Medium measurement/Standard culture  & 4 g/L/24h/1.5$\cdot$10$^{5}$c & 4.5  & 5.4 & \cite{Ang2020}\\
 \hline
    U251 &  GAHK20, Glucose (HK) Assay Kit & 1 mg/mL/4$\cdot$10$^{5}$c/24h & 4.5 (assumed) & 16-48 & \cite{Liu2021}\\
 \hline
     MCF-7 & YSI 2700 Biochemistry Analyzer  & 190 fmol/h/c & 4.5 & 5.2 & \cite{Meadows2008}\\
 \hline
     48R HMEc & YSI 2700 Biochemistry Analyzer  & 590 fmol/h/c & 1.4 & 16.4 & \cite{Meadows2008}\\
 \hline
    MCF-7 & Medium measurement/Standard culture  & 43.8 nmol/h/1$\cdot$10$^{5}$c & 1.0 & 12.6 & \cite{Mazurek1997}\\
 \hline
     " & "  & 13.0 nmol/h/1$\cdot$10$^{5}$c & 0.09 & 3.6 & \cite{Mazurek1997}\\
 \hline
     MDA-MB-453 & "  & 150 nmol/h/2.5$\cdot$10$^{5}$c & 1.0 & 16 & \cite{Mazurek1997}\\
 \hline
     " & "  & 50 nmol/h/2.5$\cdot$10$^{5}$c & 0.09 & 5.5 & \cite{Mazurek1997}\\
 \hline
 	Ishikawa endometrial cancer cells & 2-deoxy-glucose uptake & 5 nmol/min/1$\cdot$10$^{6}$c  & 0.9 & 8.3 & \cite{Medina2004}\\
\hline
 	" & " & 6 nmol/min/1$\cdot$10$^{6}$c & 1.8 & 10 & \cite{Medina2004}\\
\hline
 	" & " & 8 nmol/min/1$\cdot$10$^{6}$c & 5.4 & 13 & \cite{Medina2004}\\
\hline
 HT29 & Medium measurement/Standard culture & 0.55 µmol/hr/mg protein  & 4.5 & 3-6 & \cite{Gauthier1989}\\
 \hline
  A549(non-SP) & Medium measurement/Standard culture & 250 nmol/24 hr/1$\cdot$10$^{5}$c  & 1.26 & 2.9 & \cite{Liu2013}\\
 \hline
   A549(SP) & " & 320 nmol/24 hr/1$\cdot$10$^{5}$c  & 1.26 & 3.7 & \cite{Liu2013}\\
 \hline
 \hline
   MCF-7 &  YSI 2900 biochemistry analyzer  & 10 pmol/min/1$\cdot$10$^{3}$c (glycolytic ATP) & 10(*) & \cite{Prado-Garcia2020}\cite{Gardner2022}\\
 \hline
  %  SU-DIPG-XIII \& IV & ?  & 100-200 pmoles/min (glycolytic stress test) & 0.5-2(*) & \cite{Waker2018}\\
 %\hline
  %   H460 & high-resolution respirometry (Oroboros Oxygraph-O2K)  & 0.13 µmol/100000cells/hr (lact prod) & 5.5 & \cite{Amoedo2011}\\
 %\hline
  %    H460 &  Seahorse XF  & 3 nmol/mg/hr & 0.3 & \cite{Jiang2016}\\
 %\hline
  %    HT29 &  sigma lactate reagent & 30 nmol/min/mg protein & 1.5(*) & \cite{Jordan2005}\\
 %\hline
  %     U251 &  (2-NBDG) & 1.1-1.2 nmoles/hr/mg & 0.1-0.01& \cite{Oliva2011}\\
 %\hline
 %      MCF 10A &  \cite{Frauwirth2002} & 0.26 mg/mL/10$^6$ cells/48hr & ?? & \cite{Bollig2011}\\
 %\hline

 
 \end{tabular}
 \end{center}
 \end{table}
The values here need to be discussed. The various methods through which they were obtained and the different units in which they are expressed means that direct comparison is difficult and requires some calculations and assumptions which are all detailed in the following paragraphs.\\
 
The measurements from Mao and collaborators were obtained by using both  D and L-glucose and using model similar to the one used in this study to fit the results. In terms of culture configuration the cells are placed in a system called the MCL where 10-12 cells layers have been grown on a porous membrane. This can be considered a 3D configuration in that most cells are in contact with other cells in a dense configuration. Aside from the first and second layer which probably behave a bit differently due to the porous substrate it is likely that cells behave as they would in the spheroid. The value they gave is converted in mM/min by assuming a 2 pL cell volume. A value which they used themselves in their study.\\
 
%The value from Pilarek is reported from a specific cell culture method in which culture medium is layed atop a liquid layer of Perfluorodecalin (PFD) in order to increase oxygen availability in culture medium It is also known that cells growing without anchorage, provided they have the ability, consume less nutrient that those with anchorage \cite{Jiang2016}.\\ 

The measurement from Anggayasti and collaborators well illustrate how extra steps are required in order to compare glucose consumption data from different experiments. In their case, they report a concentration decrease of 0.5 g/L for glucose in wells of 96-well plates containing 150000 cells over a time period of 24 hours in control conditions. Their working volume is reported to be 250 µL. Therefore the amount of glucose consumed can be evaluated to be 0.125 mg or $7 \cdot 10^{-7}$ mol. Expressed in mol/cell/s, the glucose consumption becomes $5.4 \cdot 10^{-17}$ mol/cell/s which is 25\% lower than the value found by Mao in HCT116. The cell volume distribution for A431 cells is given by Ng, Keng and Sutherland. In their study they report A431 cell distribution in both spheroids and exponential-phase monolayer. In spheroid the distribution is bimodal with volume ranging from 0.8 pL to 4.8 pL. For monolayer the distribution peaks near 2 pL with significant population up to 4 pL.\cite{Ng1987} \textbf{déplacer ça là où ça sera plus pertinent}\\

In the study of Liu and collaborators, the glucose uptake assay was conducted as follows:"Cells in logarithmic growth phase with good growth status were taken and inoculated into 6 well plates with $4 \cdot 10^5$ cells per well." They measured glucose consumption after 1 day and obtain 1 mg/mL glucose concentration decrease on the wild type U251. The working volume is not reported in their case, but since they work in standard 6-well plate, it is known to be between 1 and 3 mL according to ThermoFisher's website. The cellular glucose consumption can then be estimated to be between 1.6 and 4.8$\cdot 10^{-16}$ mol/cell/s. which is between and 2 and 8 times higher than A431.\\ 

Obtaining a value from the study of Mazurek and collaborators is a less straightforward. The data is taken from the figure where they report the correlation between glucose consumption and lactate production (nmoles/hr/dish). The first interesting information to draw from this study is that MCF-7 is one of the cell lines where glucose consumption and lactate production are well correlated, which is not necessarily the case, as shown on the MDA-MB-453 cell line.\cite{Mazurek1997} The extracted value for MCF-7 is obtained as such. In the "methods" section they report that concentration are measured every 24 hours and the shape of the plot suggest that they do not replenish glucose concentration over time. It is therefore assumed that the maximum value of each concentration is the one that should be kept for our study. It is most likely the value from the first 24 hours meaning that the initial cell number is the value that should be kept in order to get cellular consumption. If anything, the consumption is slightly overestimated due to cell number being lower than reality. However, even if the value is doubled to account for potential divisions(MCF-7 doubling time :  24 hrs \cite{Sutherland1983}) MCF-7 still consume significantly more than the other cell line in that study and that the other cell lines mentioned so far.\\

The lactate production of MCF-7 obtained by Mazurek can be compared to that obtained in  another study by Romero and collaborators. Their maximum reported value is 980 nmoles/h/dish which supposedly contains 1$\cdot 10^{5}$ cells. Romero and collaborators obtained 10 pmol/min/1000 cells of glycolytic ATP in their study.\cite{RomeroAgilent} If the value from Mazurek is converted into the same unit, the result is 16 pmol/min/1000 cells. This corroborates the fact that the glucose consumption values are realistic.\\

For the MCF-7 cell line, another interesting study is that of Bayar and Bildik, in which they measure glucose consumption for MCF-7 cells at different glucose concentrations (5.5, 15 and 55 mM). The glucose consumption cannot be directly extracted because the absolute number of cells is not available. However, the evolution is still interesting. if 5.5 mM is taken as basis, the increase in cellular consumption between 5.5 mM and 15 mM over 24 hours can be calculated to be approximately 31 \%. This value accounts for the increase in cell viability after 24 hours that they report as well. The increase in cellular consumption between 5.5 mM and 55 mM is much more massive at 1270\% which shows that the environment can also influence uptake in various cell lines.\cite{Bayar2021}

The study of Gauthier and collaborators give a value for standard HT29 which is described as highly glycolytic. In order to convert the value given in µmol/hr/mg protein into the same unit used for other studies, the protein content of a cell needs to be known. In that regard, the study of Dolfi and collaborators shows a correlation between cell size and protein content.\cite{Dolfi2013} Tahara and collaborators measured the mean diameter of HT29 cells to be 16.6 µm, which corresponds to a sphere of 2.34 pL of volume.\cite{Tahara2013} which according to Dolfi and collaborators results would match a protein content between 200 pg and 400 pg. This corresponds to a cellular glucose consumption rate between 3 and 6$ \cdot 10^{-17}$ mol/cell/s.

The values obtained by Liu and collaborators on the A549 cell line evidence possible heterogeneity even within a seemingly homogeneous population. However, this heterogeneity is not as significant as the differences that can be observed between cell lines.\cite{Liu2013}
 
The work of Gardner and collaborators on nutrient depletion in Plasmax was also helpful in data gathering and is therefore mentioned along the source publication in the second part of the table and is therefore duly mentioned.\cite{Gardner2022}

In the study of Prado-Garcia and collaborators 
%The data from Waker on DIPG cells also requires some assumptions to obtain an approximation. The glycolytic stress test results are given in pmoles/min which could be either a lactate production or (less likely) a glucose consumption measure.\cite{Waker2018} Using the same hypothesis as in the previous paragraph, and assuming a number of cell per well of 10$^5$, which is the order of magnitude given for the proliferation assay, the glucose consumption for both cell line is estimated to be in the range of 0.5 - 2 mM/min for those cell lines.\\

%For Liu, the volume of fluid inside the 4-well used for the glucose consumption is assumed to be 2 mL. In that case the amount of consumed glucose can be deduced from the concentration change can be deduced and be converted the same unit that was used so far (mM/min).\\

%bollig does not give enough info...  
 
 %\cite{Jiang2016} pour la anchorage independent growth
 %100 à 400 µg /  Mcells -> 200 µg/Mcells alors 1 mg = 5 Mcell 0.3-0.6 nmol/Mcell/h -> 0.3-0.6 fmol/cell/hr -> 5e-3 1e-2 fmol/cell/min -> 5e-18 1e-17 mol/cell/min -> FAIBLE
 
 %OCR H460 0.003 pmol/min/cell -> 3e-15 mol/min/cell -> 1.5 mM/min 
 
 %Liu : concentration fell 1mg/mL in a 2 mL plate -> 1 mg of glucose was consumed -> 
 % 1 µm³ = 1 fL
 
 %garder bayar pour l'impact de la concentration sur la conso l'effet de l'hypoxie est pas claire... à noter
 %Mazurek est utile pour le lien glucose lactate mais pas plus...Ah si
 
 %3.15 g/L pour  le DMEM /F12 chez Azzalin -> 3150 mg/L -> Donc en gros on perd 15 mg/dL/Mcells
 %Meadows a aussi le lactate
\subsubsection{Oxygen cellular consumption rate}

\begin{table}[h!]
\begin{center}
\begin{tabular}{ |p{18mm}|p{26mm}|p{30mm}|p{25mm}|p{7mm}| }
 \hline

 \textbf{Cell line}  & \textbf{Measurement method} & Reported  value & \textbf{Consumption rate} (mM/min)  & Ref. \\
 \hline
 HCT116 & Diffusion apparatus & (6.35$\pm$0.42)$\cdot$10$^{-17}$ mol cell$^{-1}$ s$^{-1}$& 1.8 & \cite{Mao2018}\\
 \hline
  H460 & Seahorse XF &  1-3 pmol 1000 cell$^{-1}$ min$^{-1}$& 0.75-1.5 & \cite{Jiang2016}\\
 \hline
   H460 & high-resolution respirometry (Oroboros Oxygraph-O2K) &  30 pmol 1000000 cell$^{-1}$ s$^{-1}$& 0.9 & \cite{Amoedo2011}\\
 \hline
   HSJD-DIPG-007 &  Seahorse XF & 3000 pmol/min/10$^6$ cells & 1.5(*) & \cite{Shen2020}\\
 \hline
   U87 &  Clark-type oxygen electrode & 28 nmol/min/4*10$^6$ cells & 3.5 & \cite{Zhou2011}\\
 \hline
    MCF-7 &  Seahorse XF & 100 pmoles/min/1000 cells & 50(*) & \cite{Zhuang2014}\\
 \hline

 
 \end{tabular}
 \end{center}
 \end{table}

%Cerebral metabolic rate have also been measured for oxygen in brain tissue in both rat and human tissue. Rhodes and collaborators also gave a measured value for CMRO$_2$ which was approximately 0.5 mM/min/cell. (converted from 1.2 mL/100 mL/min and assuming a brain tissue density of 1050 g/L) \cite{Rhodes1983} A more recent set of value was provided by Rodgers on brain tissue with measured value ranging from 0.1 mM/min to 3 mM/min. Measurements by Shalit yielded values of approximately 0.5 mM/min as well.\cite{Shalit1972} In the model the value of 0.5 mM/min will be used. Kirsch and collaborators also noted that  "Calculations made on the basis of known diffusion and solubility coefficients of oxygen plus tumor oxygen uptake indicate that cells over 200 µm from a capillary source are essentially anaerobic" which gives information on the expected behavior.\cite{Kirsch1978}\\

For DIPG cell, in most cases, the oxygen consumption rate is measured in monolayers through the use of the seahorse XF kit.\cite{RomeroAgilent} Value are generally reported in pmol/min/cell. Conversion from pmol/cell to mM requires knowledge of the cell volume. a cell volume of 2 pL is postulated (after observation of the picture given by L.), corresponding roughly to a cell diamter of 15 µm. In the works of Mbah and Shen, it was reported in a monolayer-like situation as the cells were plated in wells coated in laminin.  Jiang and collaborators also reported on lung cancer cells that oxygen consumption increased in the monolayer situation compared to the spheroid. They showed OCR to be reduced by two thirds in spheroid compared to monolayers.\cite{Jiang2016} The value for OCR is not to be adjusted in this study as it is considered that the contact with the HA/matrigel scaffold produces similar effect to the laminin-coated surfaces used by Shen, Mbah and their collaborators to assess the OCR with the seahorse experiments.\\

The value reported by Shen and collaborators is 4000 pmol/min/$10^6$ cells which corresponds to a consumption of 2 mM/min/cell. This value is reported for HSJD-DIPG-007 cell line cultured as neurospheres.\cite{Shen2019} In the case of Mbah and collaborators, they derived two models from the HSJD-DIPG-007 and SU-DIPG-XIII cell lines: one grown as neurospheres and another cultured as adherent monolayer with serum. The OCR and ECAR were measured in both cases. For HSJD-DIPG-007 the OCR value reported for gliospheres-cultured cells reported is 50 pmol/min/cell. The first thing is to compare this to the value reported by Shen and collaborators which is 0.004 pmol/min/cell. This means that the same experiment on the same cell line yielded results different by 4 orders of magnitude. Other values reported for brain tumor cells by Ruas and collaborators are in the range of 80 pmol/s/$10^6$ cells, which corresponds to 4800 pmol/min/$10^6$ cells measured in suspension on U87 glioma cells. This is much closer to the value of Shen and collaborators. A possible explanation is a conversion error by Mbah and collaborators. If the value reported by Mbah and collaborators is taken not to be 50 pmol/min/cell but 50 pmol/sec/$10^6$ cells then it becomes much closer to the other values. And to further support this point, the results on ECAR are suggested to the same discrepancy and can be corrected in the same way. Now, if the correction is applied to the gliosphere values for HSDJ-DIPG-007 the measured OCR is 50 pmol/s/$10^6$ cells, which corresponds to 3000 pmol/min/$10^6$ cells, and therefore 1.5 mM/min/cell (assuming a 2 pL cell volume). The same operation for SU-DIPG-XIII yields 4.5 mM/min/cell.\\

For the study, it is considered that oxygen consumption can vary between 0.5 and 5 mM/min/cell. This is a broader range than the one found for reported values above. However, it should also be noted that all the reported values were calculated with a constant cell volume of 2 pL. However, cell volume for the studied cell line is not precisely known and maybe slightly higher or lower than the previous value.\\

\subsection{}

\subsection{Lactate}

\section{Metabolic responses} 

\newpage
\bibliographystyle{unsrt}
\bibliography{biblio_synthese}
\end{document}