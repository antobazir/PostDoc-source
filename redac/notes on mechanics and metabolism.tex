\documentclass[11pt,a4paper]{article}
%\usepackage[utf8]{inputenc}
%\usepackage[ascii]{inputenc}
\usepackage{geometry}
\usepackage[dvipsnames]{xcolor}
\usepackage{textcomp}
\usepackage{graphicx}
\usepackage{caption}
\usepackage{subcaption}
\usepackage{amsmath}

\begin{document}
**Extracellular matrix in glioblastoma: opportunities for emerging therapeutic approaches
-Nothing on metabolism

**Mechanical and metabolic interplay in the brain metastatic microenvironment
-Cells within soft ECM facilitate optimal glycolysis by mediating TRIM21 (tripartite motif containing-21, a ubiquitin ligase) and inducing subsequent degradation of phosphofructokinase (PFK), a rate-limiting glycolytic enzyme (54). In contrast, cells surrounded by stiff ECM have increased cell-surface tension, which has promoted highly bundled actin fibers that entrap TRIM21, rendering it inactive and increasing the rate of glycolysis (54). In most cancers, increasing ECM stiffness upregulates the number of glucose transport proteins in the cell membrane, increases glycolytic enzymes and glycose synthase activity, induces the expression of gluconeogenic genes, and enhances the pentose phosphate pathway, all of which increase cancer cell metabolism (47). Thus, changes in cell and/or ECM stiffness—influenced by microenvironmental factors and mechanobiological signaling—play an important role in cell growth, proliferation, migration, and malignancy (36, 55, 56)."
-"he resulting metabolic shift toward aerobic glycolysis in cancer cells supports the production of MMP2, for example, which can help clear a path through the ECM (58). Thus, metabolism contributes to protease-enabled cell migration."
-"In the brain, the mechanical properties of cancer cells and tumors in the metastatic environment are opposing (e.g., tumor versus host stiffness). Cellular stiffness is generally higher than that of the ECM, and the overall tumor generally softens with cancer progression (16)."

**PDTM-35. THERAPEUTIC RELEVANCE OF YAP/TAZ ACTIVITY IN PEDIATRIC HIGH-GRADE GLIOMA.
"in testing this hypothesis, we have generated compelling preliminary data showing that pharmacologic inhibition of YAP/TAZ activity specifically provokes growth arrest and cell death of RTK-overexpressing HGG cells, particularly in the context of H3-K27M mutations. Our ongoing research is focused on identifying the YAP/TAZ transcriptional program that drives tumorigenesis."

**Matrix stiffness induces epithelial–mesenchymal transition and promotes chemoresistance in pancreatic cancer cells
-"Increased matrix rigidity associated with the fibrotic reaction is documented to stimulate intracellular signalling pathways that promote cancer cell survival and tumour growth. "

\end{document}