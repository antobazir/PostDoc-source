\documentclass[11pt,a4paper]{article}
%\usepackage[utf8]{inputenc}
%\usepackage[ascii]{inputenc}
\usepackage{geometry}
\usepackage[dvipsnames]{xcolor}
\usepackage{textcomp}
\usepackage{graphicx}
\usepackage{caption}
\usepackage{subcaption}
\usepackage{amsmath}
\usepackage{tikz}

\begin{document}
**Cytoplasmic p53 couples oncogene-driven glucose metabolism to apoptosis and is a therapeutic target in glioblastoma
-"Glucose, glutamine, and lactate measurements

Cellular glucose consumption and lactate production were measured using a Nova Biomedical BioProfile Basic Analyzer. Briefly, cells were plated in 1 × 10$^5$ cells/ml in 2 mL of gliomasphere conditions and appropriate drug conditions. 12 hrs following drug treatment, 1 ml of media was removed from each sample and analyzed in the Nova BioProfile analyzer. Measurements were normalized to cell number." NOT ATTACHED
-0.006 nmol/cell/12hr -> 8.33 fmol/min/cell -> 4.15 mM/min (GBM39)

**In vitro expansion of U87-MG human glioblastoma cells under hypoxic conditions affects glucose metabolism and subsequent in vivo growth
-Growing U87-MG cells at either 3 or 21 \% O 2 has no
significant impact on cell doubling time calculated,


**Metabolic Alterations in Highly Tumorigenic Glioblastoma
Cells
- Glucose uptake was measured using [ 3H]2-deoxyglucose,
 en CPM donc on peut rien dire
 
 **High Glucose Promotes Human Glioblastoma Cell Growth by Increasing the Expression and Function of Chemoattractant and Growth Factor Receptors
- ON va p'tet pousser le délire sur les U87 vu qu'il y a pas mal de mesure...
- en 6 jours de culture on passe de moins 1e5 à 15e5 en normoxie () et seulement 2.5e5 en hypoxie

**A New Pathway Promotes Adaptation of Human Glioblastoma Cells to Glucose Starvation
-entre 100 et 80 mg/dl/1e6 c (décroît au cours du temps et plus vite dans les sphéroïdes que dans les cellules adhérentes (passe de 100 à 90 ou de 100 à 80 en 8hr)
-U-87 MG, Hu-197, C6 and HeLa cells were grown in DMEM/F12 (Thermo Fisher Scientific, Waltham, MA, USA) supplemented with 10\% FBS, 100 unit/mL penicillin, 0.1 mg/mL streptomycin and 8 μg/mL ciprofloxacin and passaged once they were reaching confluence
-glucose consumption and lactate release into the cell culture medium were measured at the appropriate times by collecting 300 μL of medium from the cultures
- 100 mg/dl/1e6

Dans S1 on a l'uptake 
- on passe 0 à 30 mg/dl/10^6 en 8hr
- 3.75 mg/dl/1e6 c/hr ->  0.021 mmol/dl/1e6 c/hr
Bon je pige pas...


**Multinuclear NMR Studies of an Actively Dividing Artificial
Tumor
-"For glucose consumption measurements, 7e7 A549 cells grown on 100 mg of CultiSphers were removed from spinner culture and placed in an NMR perfusion chamber, as described above."
-" 70 mg of 1-13C-glucose dissolved in 3 ml of glucose-free medium were added
through a 0.22 μm sterilization filter to the 50 ml of medium, resulting in 7.8 mM glucose. " 
- approx 1 mM en 200 mn
- 1 mmol/L * 53e-3 L -> 1e-3*53e-3 -> 53 µmol pour 70 millions de cellules en 200 mn -> 0.0038 µmol/min/million cell -> 3.8 nmol/min/million cell -> 3.8 fmol/min/cell -> 3.8 mM/min

**Bevacizumab and CCR2 Inhibitor Nanoparticles Induce Cytotoxicity-Mediated Apoptosis in Doxorubicin-Treated Hepatic and Non-Small Lung Cancer Cells 
-"Glucose consumption level using Huh-7 and A549 cancer cells were measured upon different nano-treatments and their free counterparts using glucose detection kit. Briefly, the cells were cultured in 96-well plates at a density of 1×10e4 cells/well. In the second day, 5 mM of glucose and different nano-treatments and their free counterparts were added in the media after 2 hours of cells starvation. "
-"Glucose Consumption Rate of A549 and Huh-7 Cancer Cells upon AV, CR, AVNP, CRNP, and DOX+AVCRNP. Mean and standard error (n=3) were represented in the blot. The used concentrations over cells were as follow: 0, 25, and 100 (μM for AV and DOX, and nM for CR). The cells were incubated with the drugs for 24 h"
- well volume 400 µL (working volume 350 µL max)
- 5.5 mM en 24 hr dans 300-350 µL  avec 10000c -> 1.65 à 1.92 µmol par 24hr pour 10 000 c 
- 11 pmol/min/c 
- 57 mM/min... c'est beaucoup...

**The diagnostic value of lower glucose consumption for IDH1 mutated gliomas on FDG-PET
-"Glucose assay
Cells in logarithmic growth phase with good growth status were taken and inoculated into 6-well plates(1 to 3 mL) with 4e5 cells per well. 3–4 multiple Wells were set in each group. After continuous culture for 1 Day, cell glucose consumption was detected, according to the kit operation method (GAHK20, Glucose (HK) Assay Kit)."
-24 hr and 1mg/mL in  U251 wt
- 1 mg/mL -> 5e-3 mmol/mL -> 5 mmol/L -> 10 µmol en 24hr avec 400000 cell
-25 pmol/cell en 24hr -> 17 fmol/min/cell -> 8.5 mM/min


**Central role of lactate and proton in cancer cell resistance to glucose deprivation and its clinical translation
-"It is believed that the amount of glycolytic intermediates entering to biosynthetic pathways is positively correlated with the rate of glycolysis."
-"Lactic acidosis (high lactate concentration with acidic pH) is common in many solid tumors. We found that lactic acidosis, but not lactosis (high lactate concentration with weakly basic pH) or acidosis (low lactate concentration with acidic pH) effectively rescued cancer cells from glucose deprivation.29–31"
-" In a time course experiment, the proliferation cells as assayed by BrdU incorporation were down from 33\% at 2 mM glucose to 3.8\% at 0.2 mM glucose,"
-"Glucose is distributed into three parts, 10\% for OXPHOS, 85\% for lactate generation and the remaining 5\% presumably for biomass synthesis. However, Warburg effect reflects the glucose consumption capacity of cancer cells but this capacity does not necessarily reflect the practical use of glucose in a real tumor (Figure 1)."


**Effects of hyperglycemia on the progression of tumor diseases 


**High Glucose Promotes Pancreatic Cancer Cell
Proliferation via the Induction of EGF Expression and
Transactivation of EGFR
-5.5 25 et 50 et la taux prolif varie d'un facteur 3 sur 48h

**WNT/β-Catenin-Mediated Resistance to Glucose Deprivation in Glioblastoma Stem-like Cells
- "cells were cultured in a medium with either standard or reduced glucose concentrations for various time points (24, 48, and 72 h). Glucose depletion reduced cell viability and facilitated the survival of a small population of starvation-resistant tumor cells."
-"Stress resilience has been attributed to stem-like tumor cells in glioblastomas, with stemness playing an essential
role in promoting the resilience, self-renewal, and metabolic adaptability of glioblastoma cells [9]."
-"In GBM1 and JHH520 cells, reduction in glucose levels (450 mg/dL–45 mg/dL) did not significantly affect cell viability. However, reducing the glucose concentration to 45 mg/dL significantly decreased the viability of BTSC233 cells (p < 0.05)"
-"n addition, glucose-depleted cultures of JHH520 and BTSC233 cells displayed
significantly enhanced invasion after 48 h compared to the control (JHH520 p < 0.05, BTSC233
p < 0.001) (Figure 1B). Interestingly, all cell lines showed a significant increase in clonogenic capacity when cultivated in reduced glucose concentration media (45 mg/dL) as opposed to standard cell culture conditions (450 mg/dL glucose"
-" A small subpopulation of cells managed to survive in a glucose-deprived microenvironment (Figure 1). "
-"This acquisition of an invasive phenotype in nutrient-limiting microenvironments has been shown across various cancer tissues before

**A New Pathway Promotes Adaptation of Human Glioblastoma Cells to Glucose Starvation
-"In glioblastoma cells, PARP1 inhibitor veliparib mimics glucose starvation in enhancing glucose uptake. "

**A New Pathway Promotes Adaptation of Human Glioblastoma Cells to Glucose Starvation
-" Moreover, in glioblastoma, alternative metabolic pathways that use glutamine to compensate for the absence of glucose are poorly
developed"
-"We now show that culture conditions that increase the level of SHC3 in high‐grade glioma cells enhance aerobic glycolysis, as demonstrated by increased glucose consumption and lactate"
-"We also found that depletion of glucose in the culture medium increases the level of SHC3."
production. 

**The Interleukin-11/IL-11 Receptor Promotes Glioblastoma Survival and Invasion under Glucose-Starved Conditions through Enhanced Glutaminolysis
-"Glioblastoma cell lines over-expressing IL-11Rα displayed greater survival, proliferation, migration and invasion in glucose-free conditions compared to their low-IL-11Rα-expressing counterparts, while knockdown of IL-11Rα reversed these pro-tumorigenic characteristics"
-"Overall, our study identified that the IL-11/IL-11Rα pathway promotes glioblastoma cell survival and enhances cell migration and invasion in environments of glucose starvation via glutaminolysis"

**Heterogeneity of Glucose Transport in Lung Cancer
-Km pour les glut en 1.5 et 17 mM

**Low‐frequency mechanical vibration induces apoptosis of A431 epidermoid carcinoma cells
- density 1.5 × 10^5/250 µL/well -> 150000 cells  per well 0.4-0.5 mg/mL de glc en 24hr -> 4 g/L/24hr/150000c

**Cells grown in three-dimensional spheroids mirror in vivo metabolic response of epithelial cells
- Upon contact inhibition, the cyclin-dependent kinase inhibitor p27Kip1 is upregulated9. This causes a change in the metabolic flux, and citrate is not metabolized in the tricarboxylic acid cycle to α-ketoglutarate but rather shuttled into the cytosol where it is involved in the formation of fatty acids or histone modifications10. Furthermore, the rate of glucose uptake and lactate secretion is decreased by 50\% in fibroblasts after contact inhibition as it happens in three-dimensional cell culture.
-  Starving cells modulate mTOR signaling via an activated Her2 and therefore have high autophagic activity that enables the recycling of building blocks to maintain cellular functions
- Cells in such spheroids stop proliferating and start differentiating as evidenced by an accumulation of cells in the G1 phase of the cell cycle (Fig. 1b and Supplementary Fig. 1b) and a diminished Ki67 signal, a bona fide marker for cell proliferation (Fig. 1c).
- The production of macromolecules is one major function of aerobic glycolysis

**Comparison of cancer cells in 2D vs 3D culture reveals differences in AKT–mTOR–S6K signaling and drug responses


**Energy Metabolism in H460 Lung Cancer Cells: Effects of Histone Deacetylase Inhibitors
- 0.13 µmol/100 000 c/hr lactate prod 
- 0.13e-11/cell/hr -> 2.17e-14 mol/cell/min -> 5 mM/min
- 30 pmol/s/Mcells -> 30e-12*60*1e-6
- 0.9 mM/min pour ocr

**The diagnostic value of lower glucose consumption for IDH1 mutated gliomas on FDG-PET
-"Cells in logarithmic growth phase with good growth status were taken and inoculated into 6-well plates with 4 × 105 cells per well."
-"After continuous culture for 1 Day, cell glucose consumption was detected, according to the kit operation method"
- 1 mg/mL U251  1 mg/mL/400000c/24hr -> nope

**Acquisition of Chemoresistance in Gliomas Is Associated with Increased Mitochondrial Coupling and Decreased ROS Production
- 1.1-1.2 nmoles/hr/mg U251
\end{document}
