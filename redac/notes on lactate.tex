\documentclass[11pt,a4paper]{article}
%\usepackage[utf8]{inputenc}
%\usepackage[ascii]{inputenc}
\usepackage{geometry}
\usepackage[dvipsnames]{xcolor}
\usepackage{textcomp}
\usepackage{graphicx}
\usepackage{caption}
\usepackage{subcaption}
\usepackage{amsmath}
\usepackage{tikz}

\begin{document}
\textbf{**Lactate is an Epigenetic Metabolite that Drives Survival In Model Systems of Glioblastoma}\\

-"Here, we demonstrated that lactate rescued patient-derived xenograft (PDX) GBM cells from nutrient deprivation mediated cell death. "\\

-"it is important to note that activity of oxidative metabolism is necessary to enable the biosynthesis of fatty acids and cholesterol from glucose carbons given that acetyl-CoA (from glucose carbons) is generated and converted to citric acid in the mitochondrial matrix first (DeBerardinis and Chandel, 2020)."\\

-"This concept enables the tumor core to rely on glucose, while lactate that in turn along with fatty acids may serve as a substrate for tumor cells in the periphery. "\\

-"Lactate provides carbons to support the intermediary metabolism of GBM cells"\\

- Most likely  cells metabolising glucose are not uptaking lactate which may lead to death if there are both too much glucose and too mcuh lactate...\\

\textbf{** Lactate metabolism in human lung tumors}\\

-"Here we show that lactate is also a TCA cycle carbon source for NSCLC. In human NSCLC, evidence of lactate utilization was most apparent in tumors with high 18fluorodeoxyglucose uptake and aggressive oncological behavior. "\\

-" The data indicate that tumors, including bona fide human NSCLC, can use lactate as a fuel in vivo."\\

-"Excreting lactate through monocarboxylate transporters (e.g. MCT1, 4) eliminates protons arising from the glyceraldehyde 3-phosphate dehydrogenase reaction in glycolysis, thereby maintaining pH homeostasis inside the cell and acidifying the extracellular space."\\

-"Lactate circulates at levels of 1-2 mM and acts as an inter-organ carbon shuttle in mammals (Cori and Cori, 1929). Some cancer cells use lactate as a respiratory substrate and lipogenic precursor in culture (Chen et al., 2016). Blocking lactate uptake with an MCT1 inhibitor reduces respiration and promotes glycolysis in some cancer cell lines, and suppresses xenograft growth in mice (Pavlides et al., 2009; Sonveaux et al., 2008)."\\

-"We also find that lactate's contribution as a respiratory fuel exceeds that of glucose, particularly in tumors growing in the lung."\\

-"Lactate is preferred to glucose as a fuel for the TCA cycle in vivo"\\

-"TP53 and others enhance glycolytic flux, indicating that reprogrammed glucose metabolism is a shared, cell-autonomous consequence of these mutations "\\

-" induces expression of genes involved in oxidative metabolism, promotes lactate oxidation and allows lactate to support cell survival during glucose depletion "\\

-" tissue architecture is related to the relative preference or propensity to use lactate as a fuel."\\

-" It was previously suggested in experimental tumor models that lactate oxidation occurs in well-oxygenated regions as part of symbiotic metabolite exchanges in which hypoxic cancer cells produce lactate and better oxygenated cells take up lactate to fuel respiration (Sonveaux et al., 2008)."\\

-"In tumors with high FDG-PET signal, the availability of extracellular lactate as a respiratory fuel might allow glucose to supply other growth-supporting pathways like the pentose phosphate pathway and synthesis of ribose and hexosamines. "\\



\textbf{**Glucose feeds the TCA cycle via circulating lactate}\\

-"Quantitative analysis reveals that during the fasted state, the contribution of glucose to tissue TCA metabolism is primarily indirect (via circulating lactate) in all tissues except the brain."\\


\textbf{**Glucose Metabolism Heterogeneity in Human and Mouse Malignant Glioma Cell Lines}\\

-"We also report that highly glycolytic glioma cells have differential and specific expression of LDH-B, suggesting the use of extracellular lactate to fuel cell activities"\\



\end{document}