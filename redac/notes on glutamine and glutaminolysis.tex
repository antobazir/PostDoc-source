\documentclass[11pt,a4paper]{article}
%\usepackage[utf8]{inputenc}
%\usepackage[ascii]{inputenc}
\usepackage{geometry}
\usepackage[dvipsnames]{xcolor}
\usepackage{textcomp}
\usepackage{graphicx}
\usepackage{caption}
\usepackage{subcaption}
\usepackage{amsmath}

\begin{document}


**Glutamine Metabolism in Brain Tumors
-"The importance of glutamine-derived nucleobases is underscored by the ability of exogenous nucleotides to rescue glutamine-deprived cancer cells from undergoing cell cycle arrest"
-. Tumor cells mitigate excess ROS and maintain redox homeostasis principally by glutathione synthesis [110,111]. Glutathione is a tripeptide synthesized from glutamate, cysteine, and glycine that actively scavenges ROS."
-"For example, there is a shift from oxidative glutamine metabolism to reductive carboxylation when non-small cell lung cancer (NSCLC) cells transition from a monolayer cultures to spheroids [114]. Under monolayer culture, these cells showed abundant glutamine uptake to drive TCA cycle anaplerosis, which is lost under anchorage-independent growth. "
-"Indeed, expression of PC in tumors is a key determinant of glutamine-independent growth [116,117,118]. Similarly, some tumors can utilize glucose to synthesize glutamate de novo, which can then be converted to glutamine by the enzyme glutamine synthetase (GLUL). Such tumors have high GLUL expression and do not depend on exogenous glutamine for their nutrient needs." LOOK UP PC and GLUL for DIPG-007 It does not seem like DIPG-007 express GLUL...
-"Cancer cells predominantly utilize glucose as the principal source of carbon for lipid and fatty acid synthesis, and its disruption is shown to hinder tumor formation"


**Myc regulates a transcriptional program that stimulates mitochondrial glutaminolysis and leads to glutamine addiction
-"In cancer patients, some tumors have been reported to consume such an abundance of glutamine that they depress plasma glutamine levels (10, 11). Despite these observations, the high rates of glutamine metabolism and addiction exhibited by some cancer cells are poorly understood. Recently, we reported that glioma cells can exhibit glutamine uptake and metabolism that exceeds the cell's use of glutamine for protein and nucleotide biosynthesis (12). In such cells, the excess glutamine metabolites produced were found to be secreted as either lactate or alanine. This high rate of glutaminolysis was found to be beneficial because it provided the cell a high rate of NADPH production that was used to fuel lipid and nucleotide biosynthesis "
-"When SF188 cells were cultured in the presence of 14C-labeled glutamine, <15\% of the glutamine the cells took up from the medium was incorporated into newly synthesized protein (Fig. 1A). Despite the fact that only a small fraction of the glutamine was used for anabolic synthesis, SF188 glioma cells were unable to survive in glutamine-deficient medium despite the presence of 25 mM glucose in the medium (Fig. 1B). "
-"revious 13C-NMR studies found that during glutaminolysis, >60\% of glutamine-derived carbon is released from the cell as either lactate or CO2 (12). Although the TCA cycle was also replenished by glutamine, only 5\% of glutamine fluxing through the TCA cycle was incorporated into fatty acids. Here, we show that only 15\% of the glutamine carbon taken up by the cell is incorporated in protein."
- cell consume between 0.5 and 1.5 mM/min for SF188
- 3-5 mM de glucose par min et 1 mM/min

**Targeting metabolic/epigenetic pathways: a potential strategy for cancer therapy in diffuse intrinsic pontine gliomas
-"Authors found that glutamine or glucose deprivation from cell culture medium could increase the H3K27me3 contents in patient-derived cells (e.g., DIPG-007 and DIPG-IV), and this effect could be abolished by affixion of $\alpha$-KG, which can be assimilated into cells."
-"What’s more, enhanced glutaminolysis was also certificated by isotope tracing, which paralleled with the magnetic resonance spectroscopy (MRS) imaging in samples of patients suffering from high-grade midline gliomas."
_"obviously lowered α-KG/Suc ratios, increased H3K27me3 levels and suppressed H3.3K27M cell growth in vitro and in vivo.

**Glutaminolysis and autophagy in cancer
-"This amino acid is metabolized within the mitochondrion through an enzymatic process termed glutaminolysis, whereby glutamine is converted to $\alpha$-ketoglutarate ($\alpha$-KG), an intermediate of the tricarboxylic acid (TCA) cycle.9 In highly proliferating cells, citrate produced in the TCA cycle is redirected into the cytosol for the production of NADPH and fatty acids. The production of $\alpha$-KG though glutaminolysis replenishes the TCA cycle, a process called anaplerosis.10,11"


*Energetic and morphological plasticity of C6 glioma cells grown on 3-D support; effect of transient glutamine deprivation 
-"Our work demonstrates that glutamine deprivation from the culture medium during 2 to 5 days has no influence on the glycolytic activity of C6 cells."
-"Interestingly, the basal respiratory activity of the cells was lower in the absence (particularly for early culture times) than in the presence of glutamine. Moreover, deprivation of glutamine from the culture medium for 1 to 2 days greatly depressed the total cellular ATP-production flux, without significantly affecting the relative contribution of the oxidative phosphorylation and of the glycolysis to the total ATP synthesis (Fig. 6). This low ATP turnover rate was maintained for longer glutamine deprivation (up to 7 days) (Fig. 6). These results indicate that the energy demand of C6 cells is reduced in the absence of glutamine because of the fact that their growth slows down drastically.

**A metabolic core model elucidates how enhanced utilization of glucose andglutamine, with enhanced glutamine-dependent lactate production, promotes cancer cell growth: The WarburQ effect
-"At high utilization rates of glutamine, oxidative utilization of glucose was decreased, while the production
of lactate from glutamine was enhanced. This emergent phenotype was observed only
when the available carbon exceeded the amount that could be fully oxidized by the available
oxygen."
-Not the same kind of model we do but hey

**Multi-scale computational study of the Warburg effect, reverse Warburg effect and glutamine addiction in solid tumors
-"For example, glutamine is known to be an important nitrogen source in nucleicacids and amino acids synthesis [57,70,71]. Additionally, glutamine contributes to the pool of metabolites that maintains NADPH/NADP+ balance [69,72] and to produce glutathione as an antioxidant to help the cell resist oxidative stress during rapid metabolism [70,72]"

**Reprogramming of glutamine metabolism and its impact on immune response in the tumor microenvironment
-"For example, glutamine metabolites in tumor cells provides energy for tumor progression after entering the TCA cycle [6]. Glutaminolysis generates raw materials for the synthesis of macromolecular substances such as amino acids, nucleotides, fatty acids and hexosamines required by the tumor cells [23]. "
"Glutamine is the main carbon source for the TCA cycle when Ras is activated [21, 35,36,37,38]. HIF-1$\alpha$ and HIF-2α are highly expressed in most tumors [39]. In the human non-small cell lung cancer cell line A549, it was found that silencing HIF-1α expression reduced glutamine consumption in the tumor cells [40]. Furthermore, HIF-2$\alpha$- has been reported to enhance the activity of c-MYC, which in turn drives glutamine catabolism by regulating numerous genes including glutaminase [30, 41]. "


*Rapid Analysis of Glycolytic and Oxidative Substrate Flux
of Cancer Cells in a Microplate
-"They were adapted step-wise to DMEM medium containing 25 mM glucose and 6 mM L-glutamine as a model system for the study of glutamine metabolism as reported[6]. As a control, the parental cells were also adapted in parallel to
DMEM medium containing 5.5 mM glucose and 2 mM L-glutamine. The former acquired a much more rapid growth rate after 4 weeks culture in the medium and was named SF188f (fast). The latter, however, maintained similar growth rates as those parental cells maintained in MEM, and were named SF188s(slow).

*Quantitative modelling of amino acid transport and homeostasis in mammalian cells
- 1.5 mM en 50 h pour la glutamine -> 0.055e-3 mM/min (à 22mM et à 6mM de gluc) bon avec les flux dans les faits on retrouve des valeurs proche -> 0.3 - 4 mM/min pour le glucose et 0.05 mM/min à 0.8 mM/min pour la glutamine
- 5 mM en 50h -> 0.1 mM/h  -> 0.1667e-3 mM/min  (à 6 mM)
-

Quantitative analysis of amino acid metabolism in liver cancer links glutamate excretion to nucleotide synthesis
- 0.05 à 1mM/min sur les hepG2 rapport 5 entre le flux de glucose et de glutamine

**Molecular link between glucose and glutamine consumption in cancer cells mediated by CtBP and SIRT4
- 2 mM au départ sur 5 lignées (HeLa MCF-10A SKOV3 U2OS 293)
- 3 jours pour tout vider sur la lente 1 jour pour la plus rapide (à 25mM de glucose)
- 5 jours pour tout vider sur la plus lente 3 jour pour la plus rapide (à 5mM de glucose)
- 1mM/jr -> 0.694 µM/min
ON est sur les mêmes rythmes que celui juste au dessus.

**A Variant of SLC1A5 Is a Mitochondrial Glutamine Transporter for Metabolic Reprogramming in Cancer Cells
-reports 20 pmole per microgram per hr ->  avec ce que j'ai trouvé on est entre 0.01 mM/min(100 pg/3pL et 0.15 mM/min 500 pg/1pL)

**Targeting Glutamine Addiction in Gliomas
- Donne les réponses à  la déprivation
- " From the opposite perspective, in contrast to SF188 GBM cell line with originally low PC activity, cells with experimentally upregulated PC became glutamine independent and silencing GA expression did not compromise their growth [31]. Therefore, the glutamine-derived αKG remains a major way to supply the TCA cycle."
'"In gliomas, glucose addiction that led to glutamine addiction was shown in GBM cell lines in vitro and in xenograft models [57,58]. When grown in glucose-deficient medium, the SF188 cells, characterized with c-Myc-associated enhanced glutamine metabolism, developed an adaptation manifested as increased activity of GLUD, allowing TCA anaplerosis. The interrelation of glyco- and glutaminolysis was emphasized by the fact that suppressing Akt signaling (i.e., suppressing glycolysis) also activated GLUD [57]. Similar glucose–glutamine relatedness was revealed by Tanaka et al. [58]. Treatment with mTOR inhibitors suppressed glucose consumption and increased glutaminase expression and the activity to take over and sustain the viability of GBM cell lines (U87MG, U251MG, LN229, T98G, A172) in vitro and in U87MG xenografts in mice [58]."

Glutamine metabolism in brain tumors
-"However, cells under hypoxia or cells with
mitochondrial defects begin to utilize glutamine as the source of carbon for making acetyl-coA [67, 68 ]."
-"Thus, glutamine directly contributes to glutathione biosynthesis by
acting as the donor of glutamate, from the GLS reaction, and by enabling uptake of cystine through
SXC. Additionally, glutamine produces reducing equivalents by mediating NADPH synthesis through
GLUD, oxoglutarate dehydrogenase (OGDH), and malate dehydrogenase (MDH)"
-"For example, there is a shift from oxidative glutamine metabolism to reductive carboxylation when
non-small cell lung cancer (NSCLC) cells transition from a monolayer cultures to spheroids [ 114 ]"


**Reductive carboxylation is a major metabolic pathway in the retinal pigment epithelium
-"In mitochondria, citrate can be generated from acetyl CoA and oxaloacetate as part of the TCA cycle. However, under hypoxic conditions, some cells also produce citrate via reductive carboxylation of α-ketoglutarate (αKG) through the action of NADPH-dependent isocitrate dehydrogenases (IDH) (15–17). Reductive carboxylation occurs in a small cohort of cells from liver, heart, brown adipocytes, and quiescent fibroblasts (18–20), where it supports redox homeostasis and synthesis of lipids, nucleotides, and urea (16, 18)."  En gros elles font le cycle à l'envers...

TCA Cycle Defects and Cancer: When Metabolism Tunes Redox State
-"Interestingly, emerging findings from the last year support the hypothesis that, in several cell systems such as (i) cancer cells containing mutations in complex I or complex III of the electron transport chain (ETC), (ii) patient-derived renal carcinoma cells with mutations in FH, (iii) cells with normal mitochondria subjected to acute pharmacological ETC, inhibition, as well as (iv) tumor cells exposed to hypoxia, the first stage of the cycle can proceed in the opposite direction through the reductive carboxylation of α-KG to form citrate. This allows cells to produce acetyl-coenzyme A to support de novo lipogenesis and their viability [4–6]. "
-"Therefore, alternatively or concomitantly to the generation of pseudohypoxic phenotype and the alteration of epigenetic dynamics, the oncometabolites-induced engagement of redox-dependent signaling pathways could contribute both to the neoplastic transformation of healthy cells as well as to the progression of malignancies characterized by germline mutations in SDH and FH and of somatic defects in IDH. "

**Integrated metabolic and epigenomic reprograming by H3K27M mutations in diffuse intrinsic pontine gliomas
-" increasing global H3K27me3 in H3K27M cells are a key therapeutic strategy leading to cell death of H3K27M cells in vitro and in vivo"
- "Isotope tracing revealed both increased glycolysis (Figures S1E–G) and glutaminolysis (Figures S1H–J) in H3.3K27M versus H3WT cells."
- "Western blotting did not reveal elevation of these factors, including Hif-1α in H3.3K27M versus H3.3WT NSC"
- "All three subtypes of malignant cells showed high expression of IDH1 and GLUD1 compared to HK2 and SLC2A3"
- " Glutamine deprivation from cell culture media resulted in increased H3K27me3 levels in H3.3K27M NSC, DIPG-007 and DIPG-IV cells but not SF7761 and DIPG-XIII*P cells"
- "In human cell lines, DIPG-007 cells were more sensitive to glutamine withdrawal than SF7761 cells and showed a partial decrease in proliferation that was reversed by α-KG (Figure 3E)."
 - "Glucose withdrawal did not alter H3K27me3 levels in H3.3K27M and H3.3WT NSC, but it increased H3K27me3 levels in DIPG-007, SF7761, DIPG-IV and DIPG-XIII*P cells (Figure 3F, S3D–L). This effect was observed as early as 24 hours after glucose withdrawal and was rescued by α-KG in DIPG-007 cells (Figures 3G, S3E–F). Moreover, partial glucose withdrawal was sufficient to increase H3K27me3 levels in SF7661 but not in DIPG-007 cells (Figure S3G). "
 - " Moreover, glutaminase (GLS) inhibitors CB-839 and BPTES increased H3K27me3 levels in DIPG-007 cells (Figures S4A–B). DON treatment in vivo, significantly suppressed growth compared to vehicle treated animals in H3.3K27M NSC xenografts (Figures 4E, S4C)."
 
**Reductive carboxylation supports growth in tumour cells with defective mitochondria

**Reductive carboxylation supports redox homeostasis during anchorage-independent growth
-" We observed that detachment from monolayer culture and growth as anchorage-independent tumor spheroids was accompanied by changes in both glucose and glutamine metabolism. Specifically, oxidation of both nutrients was suppressed in spheroids, whereas reductive formation of citrate from glutamine was enhanced."
-"Cells within spheroids proliferated at a reduced rate (Extended Data Fig.2a). Although growth in both conditions required glucose and glutamine (Extended Data Fig.2b), spheroids consumed less of both and secreted less lactate, glutamate and ammonia (Extended Data Fig.2c,d). The ratio of ammonia released to glutamine consumed was comparable between conditions (Extended Data Fig.2d). Spheroids displayed reduced entry of glucose-derived carbon into citrate (Fig.1a) and consumed less oxygen per cell (Fig.1b)."
- Cells in monolyar consume 3 times more oxygen than those in spheres.
- "Reductive carboxylation is enhanced during hypoxia through a HIF1-dependent mechanism that transmits glutamine carbon to fatty acids6." 
- "Together, these data suggest that anchorage loss per se rather than oxygen/nutrient limitation stimulates a mode of reductive metabolism distinct from hypoxia."
-"rotection against oxidative stress is thought to be one aspect of metabolic reprogramming that supports cancer cell fitness17,18. Anchorage independence induces additional oxidative stress resulting in death unless NADPH-producing pathways are engaged2. "
-glucose consumption is low at about 0.052 mM/min


**TAMI-80. CELLULAR METABOLISM IN DIFFUSE INTRINSIC PONTINE GLIOMA
-"glutamine metabolism showed that DIPG cells also have an active TCA cycle metabolism (citrate M+4; 40.07 ± 1.06\%) and moderately active reductive carboxylation pathway (citrate M+5; 10.59 ± 1.13\%)."

**Targeting tumor hypoxia and mitochondrial metabolism with anti-parasitic drugs to improve radiation response in high-grade gliomas
-"he products of TCA cycle, NADH, FADH2 provide electrons for the electron transport chain (ETC) chain. This process is known as oxidative phosphorylation (OXPHOS) and efficiently generates ATP. HIF-1α decreases mitochondrial OXPHOS by activating PDK1, which subsequently inhibits PDH. It also inhibits the excess reactive oxygen species (ROS) produced as a result of inefficient electron transport, therefore, protecting cancer cells against oxidative stress. Under hypoxic conditions and the consequent energy crisis, cancer cells also utilize glutamine to stimulate fatty acid and amino acid biosynthesis for energy production. HIF-2α enhances glutamine uptake which is converted into glutamate and replenishes the TCA cycle. The process of glutaminolysis generates fatty acids and amino acids as an energy source for cancer cells. Glutamate is also utilized for glutathione biosynthesis, which is a major antioxidant and quenches the ROS, therefore providing protection to cancer cells against cytotoxic ROS. "


**Lack of Electron Acceptors Contributes to Redox Stress and Growth Arrest in Asparagine-Starved Sarcoma Cells
-"The mitochondrial electron transport chain (ETC) consists of four enzyme complexes that transfer electrons from donors such as NADH to oxygen (Figure 6a). Chemical disruption of the ETC by complex 1 inhibitors has long been known to impede regeneration of electron acceptors and block cell proliferation "
-""We discovered that asparagine depletion of sarcoma cells causes reductive stress and that exogenous supplementation with the electron acceptor pyruvate [13,14,15] restored the changes in NAD+/NADH ratios, proliferation and viability induced by asparagine deprivation."

**Reductive carboxylation supports growth in tumor cells with defective mitochondria
-"Here we show that tumor cells with defective mitochondria use glutamine-dependent reductive carboxylation rather than oxidative metabolism as the major pathway of citrate formation. This pathway uses mitochondrial and cytosolic isoforms of NADP+/NADPH-dependent isocitrate dehydrogenase, and subsequent metabolism of glutamine-derived citrate provides both the acetyl-CoA for lipid synthesis and the 4-carbon intermediates needed to produce remaining CAC metabolites and related macromolecular precursors. This reductive, glutamine-dependent pathway is the dominant mode of metabolism in rapidly-growing malignant cells containing mutations in complex I or complex III of the ETC, in patient-derived renal carcinoma cells with mutations in fumarate hydratase (FH), and in cells with normal mitochondria subjected to acute pharmacological ETC inhibition."
-"We speculate that reductive carboxylation is stimulated by a disturbance in the redox ratio of the mitochondria caused by ETC impairment, decreasing the NAD+/NADH ratio and rendering oxidative function of the CAC less efficient. "

**An essential role of the mitochondrial electron transport chain in cell proliferation is to enable aspartate synthesis
-"Interestingly, it has long been known that human cells lacking a functional ETC can proliferate if cultured in supra-physiological concentrations of pyruvate (King and Attardi, 1989). While pyruvate can serve as a biosynthetic substrate or affect the redox state of the cell by promoting the regeneration of NAD+ (Harris, 1980; Howell and Sager, 1979), why it reverses the suppressive effects of ETC inhibition on cell proliferation is unknown."
- It is not a complete shift but rather a partial thing.

**NADH Shuttling Couples Cytosolic Reductive Carboxylation of Glutamine with Glycolysis in Cells with Mitochondrial Dysfunction
-"Reductive carboxylation is known to support proliferation of cancer cells with mitochondrial dysfunction (Mullen et al.,2011) or when treated with metformin (Liu et al., 2016), and contributes to de novo lipid synthesis under hypoxia (Metallo et al.,2011). Yet its biochemical determinants remain unclear. In this work, we demonstrate that reductive carboxylation supports metabolic flux through the NADH-consuming MDH1, regenerating cytosolic NADH to support glycolysis. Notably, the requirement of high NADH turnover to support glycolytic flux has been previously hypothesized, especially in conditions in which mitochondrial function is not sufficient to recycle cytosolic NADH and high biomass generation is required (Dai et al.,2016)"

**Increased demand for NAD+ relative to ATP drives aerobic glycolysis
-This argues cells engage in aerobic glycolysis when the demand for NAD+ is in excess of the demand for ATP."
- weird because glycolysis USES NAD+

** NAD+ metabolism: pathophysiologic mechanisms and therapeutic potential
- "Cytosolic pyruvate can also be converted to lactate by LDH, coupled with the oxidation of NADH to NAD+.211 "
- " Thus, the TCA cycle can convert four molecules of NAD+ to NADH using one molecule of pyruvate in the mitochondria under aerobic conditions.212 As an electron donor, NADH produced in the TCA cycle plays a crucial role in ATP synthesis by OXPHOS, which generates most of the energy through the H+ gradient in animal cells.213"
-Both NADH and FADH2 generated in the FAO are utilized to synthesize ATP by the ETC. "

**Enhanced fatty acid oxidation provides glioblastoma cells metabolic plasticity to accommodate to its dynamic nutrient microenvironment
-"This includes serving as a metabolic cue to drive proliferation in nutrient favorable conditions through a β-HB/GPR109A dependent manner, while serving as an efficient, alternate source of ATP only in nutrient unfavorable conditions. Additionally, we identify rational combinatorial strategies designed to target these dynamic roles FAO plays in gliomagenesis, resulting in metabolic synthetic lethality in GBM."

**Pyruvate carboxylase is required for glutamine-independent growth of tumor cells 
-Since there does not seem to be anything about this enzyme I suppose glutamine independant growth does not happen
\end{document}