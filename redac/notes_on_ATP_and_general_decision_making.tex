\documentclass[11pt,a4paper]{article}
%\usepackage[utf8]{inputenc}
%\usepackage[ascii]{inputenc}
\usepackage{geometry}
\usepackage[dvipsnames]{xcolor}
\usepackage{textcomp}
\usepackage{graphicx}
\usepackage{caption}
\usepackage{subcaption}
\usepackage{amsmath}

\begin{document}
This note is here to keep track of all the important aspect of the future model.

\section{ATP reserve and production}
\begin{itemize}
\item We know that prolif cancer cells are more glycoytic but have overall lower ATP reserve than healthy cells. Does it mean that they expanded the reserves and use glycolysis to maintain viability and prolif ? 
\item Cells appears to die of Necrosis if very ATP depleted or of Apoptosis if mildly depleted for long enough.  Where does quiescence fit in all this 
\item Is it achieve at a threshold similar or above apoptosis if Checkpoint is reached ? 
\item Overall ATP increase, meaning there is actually WAY MORE ADP than in healthy cells $\rightarrow$ cancer cells accumulate Phosphore in general
\end{itemize}

\section{Quiescence, Cell death \& energy}
\begin{itemize}
\item Can it be assumed that the pathway for apoptosis is downregulated ? If yes does it mean that cells might continue to proliferate (or try to) even when ATP should not allow for it ?
\item Quiescence is entirely possible in cancer cells. It seems to be triggered by the right set of conditions as in cell sensing nutrient depletion before proceeding to G1 leading to metabolic catastrophe.
\item What level triggers quiescence ? Is there a window of ATP production level that can induce quiescence ?
\item How do we choose that level so that glutamine and autophagy can be acounted for ?
\item Since autophagy is essentially catabolism, how does it come into play into preventing cells from going into necrosis. Can it be used as a crutch to lead cells into dormancy/apoptosis ?

\end{itemize}

\end{document}