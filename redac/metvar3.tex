\documentclass[11pt,a4paper]{article}
%\usepackage[utf8]{inputenc}
%\usepackage[ascii]{inputenc}
\usepackage{geometry}
\usepackage[dvipsnames]{xcolor}
\usepackage{textcomp}
\usepackage{graphicx}
\usepackage{caption}
\usepackage{subcaption}
\usepackage{amsmath}
\usepackage{tikz}



\begin{document}

\tableofcontents
\section{Introduction}
Metabolism is one of the core aspects of cancer cell physiology. Indeed, it is known that metabolic pathways can undergo significant reprogramming during and after tumorigenesis. This reprogramming often includes a change in glucose metabolism especially in conditions of hypoxia.\cite{Cook2021}\cite{Rodriguez2008}\cite{Griguer2005}\cite{DeBerardinis2008}\cite{Stuart2023}\\

But what is interesting in cancer metabolism is not just the reprogramming itself, but the detail of it. An increased glycolysis rate is one of the most widespread characteristic encountered across cancer cell lines. But the magnitude of it vary between cell lines. MCF-7 cells for example are described as more glycolytic than MDA-MB-435.\cite{Mazurek1997}The two aforementioned cell lines are mammary cancer cell lines but similar variability can be encountered in terms metabolism in other types of cancer.\cite{Kammerer2015} Increased glutamine uptake has also been noted in several cell lines.\cite{Natarajan2019}  \\

The difference can also be qualitative and not just quantitative, so to speak. Indeed while some cells will see their proliferation rate increase when subjected hypoxia, such as the aforementioned MCF-7, others will undergo growth arrest.\cite{Bayar2021}\cite{Waker2018}\cite{Hubbi2015} Some cells may be addicted to a given substrate if they lack the molecular apparatus to make up for its absence.\cite{Jiang2016}\\

This study aims at using modelling tools in order to reproduce the main features of the various metabolic behaviors encountered in different cancer cell lines. This takes a different approach to modelling studies which focus on metabolic variety through study of the molecular detail with, for example, flux balance analysis.\cite{Orth2010}\cite{Ng2022}\cite{Damiani2017} In terms of approach, this study uses agent-based model and encodes different behaviors in terms on consumption and proliferation for cells placed in a evolving nutritional environment.\\

It should be noted that this model does not address mechanical aspects in any specific way compared to other studies which focused on their interplay with nutrient dynamics.\cite{Milotti2010}\cite{Bull2020} The novelty is considered to be the way in which the interplay between nutrients and products of different type can impact overall growth dynamics. Immune system is also left out. But in both cases it is possible that this impacts may later be included by creating corresponding behaviors and adding agents.\\

\section{The model}

\subsection{The diffusion reaction model}
In that model, the goal is not so much to model precisely one case as it is to explore possible configurations in "controlled fashion". In order to do this, the nutrients dynamics is heavily simplified based on hypotheses that are detailed in the following.\\

Reading the literature, four great families of active substance have been delineated by the authors. This  classification arose from the both the observed dynamics and response to different substances observed in tissues. The four families are: Substrate (S),  Product (P), Oxygen (O) and Kine (K). Substrate represents anything used by the cell to function, proliferate and move. Product represents one or several species that are rejected into the extracellular medium as a results of metabolic processes. The product species may also be consumed or degraded in some cases. Oxygen is represented specifically for its impact on the electron transport chain both in terms in biosynthesis and energy production, and is therefore not treated symmetrically to substrate as will be shown later. Kine is a family of substances that represents the substance that are only released in the medium at cell death but that are nonetheless metabolically active. They are similar to product but the temporal and spatial dynamics of their release is completely different. \\

In order to model the evolution of the concentration of those species, the model we use is the reaction-diffusion equation, which has been used in several similar agent-based model involving nutrient dynamics and interactions with cells.\cite{Bull2020} 
\cite{Mao2018}\cite{Cleri2019}\cite{Kempf2005}\cite{Kempf2015}
The equation is usually written as follows : 
\[ \frac{d [C]}{\partial t}  =  D \Delta [C] + k_C \frac{[C]}{C_m + [C]} \]

In this reaction-diffusion equation, the consumption term is modulated by Monod function on nutrient concentration. This is not an active response of the cell, but the manifestation of the impossibility to consume nutrient that is not there. In others terms the Hill function represents the chemical regulation of consumption while $k_C$ represents the biological regulation.\\

The equation can be non-dimensionalised by multiplying it by a constant homogeneous to a concentration over time. let $d_0$ be the size of the modelled "cell"  and $D$ the diffusion coefficient of the medium. Then, one can write $\tau  = \frac{d_0^2}{D_{med}}$. The concentration can then be normalized by the external value $[C_{ext}]$. Now, if $C = \frac{[C]}{[C_{ext}]}$ then, 

\[ \tau \frac{d C}{\partial t}  = \tau \nabla D \nabla C + \tau  D  \Delta C + \frac{\tau k_C}{C_{ext}} \frac{C}{\frac{Cm}{C_{ext}} + C}  \]
\[ \rightarrow \tau \frac{d C}{\partial t}  = d_0^2 \nabla \frac{D}{D_{med}} \nabla C + d_0^2 \frac{D}{D_{med}} \Delta C + \frac{\tau k_C}{C_{ext}} \frac{C}{\frac{Cm}{C_{ext}} + C}  \]

written this way, $\tau$ is the characteristic diffusion timescale for a cell, $d_0^2$ is the size of square cell and $\frac{\tau k_C}{C_{ext}}$ scales the consumption relative to the external concentration.\\


\subsection{Behavior of the Monod function}
The Monod function has been proposed by Jacques Monod as a mathematical tool for the description of bacteria substrate consumption with respect to the substrate concentration. The Monod function has the following form : \\

\[V = V_{max} \frac{[C]}{M_c + [C]}\]\\

with $V$ the consumption rate of the species, $V_{max}$ the maximum value of said consumption, $[C]$ the concentration and $M_c$ the half velocity constant. An  example is plotted in figure \ref{monod}.\\

\usetikzlibrary {datavisualization.formats.functions}
\begin{figure}[ht!]
\begin{center}
\begin{tikzpicture}
  \datavisualization [
    scientific axes=clean,
    x axis = {label={$[C]$},length=8cm, ticks=few},
    y axis = {label={$V$}, ticks=few},
    all axes={grid},
    visualize as  line = hf, hf ={style = blue}]
    data [format=function] {
      var x : interval [0:1] samples 50;
      func y = (\value x /(0.2 + \value x));%*(1+0.05/1);
    };
\end{tikzpicture}
\end{center}
\caption{Monod function with $V_{max} =1 $ and $M_c = 0.2$  \label{monod}}
\end{figure}

This realisation of the Monod function uses the parameters chosen for the substrate simulation. The first observation is that at concentration of 1 which is the maximum value possible in the model, the function is not equal to 1. This is adressed in the model by normalising the function by its maximum.\\

Besides this an interesting thing to note is also the fact that depending on condition the maximum value for consumption may change in the cell but not the rest of the Monod function. it is thus interesting to also plot.\\

\begin{figure}[ht!]
\begin{center}
\begin{tikzpicture}
  \datavisualization [
    scientific axes=clean,
    x axis = {label={$[C]$},length=8cm, ticks=few},
    y axis = {label={$V$}, ticks=few},
    all axes={grid},
    visualize as line/.list = {m1,m2,m3,m4,m5},
    m1 ={style = blue},
    m1 ={style = dashed},
    m2 ={style = cyan},
    m2 ={style = dashed},
    m3 ={style = red},
    m3 ={style = thick},
    m4 ={style = red},
    m4 ={style = thick},
    m5 ={style = red},
    m5 ={style = thick},
    data/format = function]
    data [set=m1] {
      var x : interval [0:1] samples 50;
      func y = (\value x /(0.2 + \value x))*(1+0.2/1);
    }
     data [set=m2] {
      var x : interval [0:1] samples 50;
      func y = 0.3*(\value x /(0.2 + \value x))*(1+0.2/1);
    }
     data [set=m3] {
      var x : interval [0:0.2] samples 20;
      func y = 0;
      }
     data [set=m4] {
      var x : interval [0.2:0.4] samples 20;
      func y = 0.3*(\value x /(0.2 + \value x))*(1+0.2/1);
      }
     data [set=m5] { 
      var x : interval [0.4:1] samples 60;
      func y = (\value x /(0.2 + \value x))*(1+0.2/1);
    };
\end{tikzpicture}
\end{center}
\caption{Monod function with $V_{max} =1 $ and $M_c = 0.2$ (blue) and  $V_{max} =0.3 $ and $M_c = 0.2$ (cyan) normalised by value at 1 and actually nutrient response in red \label{monod2}}
\end{figure}

As cells can go back and forth between the quiescent and proliferating state, so will the consumption go back and forth between the two types of Monod functions shown in figure \ref{monod2}. But will irreversibly stay 0 if it falls below 0.2. This can lead to interesting behavior in terms of nutrient dynamics because it completely non-linear in nature.\\

\subsection{Behavior of the reaction diffusion model}
In order to assess the behavior of the reaction-diffusion model its results will be presented in simple configurations in this section. Some important parameters of the model will also be justified and presented.\\

The reaction-diffusion equation is solved on a square grid. To know the correct size for the grid, both the spatial step size and the maximum size of the agregate being modelled need to be determined. Experiments and modelling on spheroids are generally performed on spheroids with a diameter between 300 and 1000 \textmu m.\cite{Freyer1986}\cite{MullerKlieser1986}\cite{Mao2018} Considering agregate may grow past this diameter in some conditions and the fact there needs to be an area with medium, the modelled area needs to have a length of 2-3 mm.\\

For the spatial step size, the admitted range of value for eukaryotic cell diameter is 1-100 \textmu m.\cite{Cooper2006} It should be noted that this  range of values represent a wide range of cell lines. Specific cell lines fall in  that range but their size distribution is narrower than the aforementioned range. MCF-7 cells, for example,  have a size distribution that peaks at 20 \textmu m.\cite{Moon2013}\\

Before choosing the size it is important to understand its impact on computational resource. glucose has a diffusion constant in dense tissue that has been measured between 1000 and 10000 \textmu m$^2$/min.\cite{Casciari1992}\cite{Mao2018} Therefore, the timescale for glucose diffusion over 1000 \textmu m of dense tissue will be between 100 and 1000 minutes. The model therefore needs to run for at least 100 mn. On the opposite side, there is the Courant-Friedrichs-Lewy condition which states :\\

\[ \frac{2 D \Delta t}{\Delta x^2} < 1 \] \\

with $D$ being the diffusion coefficient, and $\Delta t$ and $\Delta x$ being the spatial and temporal steps in the finite difference model. Since the same grid size is used for all species, it needs to be respected for oxygen primarily.\textbf{[Finish this]}\\
\subsection{Modelled Species}

%how do you handle the product when it reaches the border then...

%PMID: 33212982 extracellular ATP impacts biology
%PMID: 27321181 Tumor cell killing by mmol/l ATP concentrations is a long-standing observation
\subsubsection{Substrate}
The question of which species to model in that simplified model was the first coming to the mind of the authors. Glucose seemed like an obvious choice. It is the best known nutrient and the one on which the most data has been gathered both in terms of diffusion and consumption in tissues.\\

However, while all cells consume glucose when it is available, none of them can survive and thrive on glucose only. Cells need sources of glucose, nitrogen, oxygen, phosphate and sulfur in order to maintain and replicate. This is why culture medium for mammalian cells contains many different molecules and vitamins that act both as nutrient sources and also as regulatory cues that ensure proper survival and proliferation for cultivated cells.\\

All the aforementioned nutrients are present and consumed by living cells for their physiological needs. In some cases, one can replace the other. The most well known case being that glutamine can replace glucose as a carbon source.\cite{Stuart2023}\cite{Mazurek1997} However, in most cases, if for some reason glucose has depleted significantly, glutamine has also been consumed in the meantime, and even with the switch, is likely to depleted in short order as well. Besides, when medium is supplied it is generally supplied with all nutrients. It can therefore appear irrelevant to treat them separately, and a model that would model every nutrient would have prohibitive computational cost anyway.\\

In order to account for all the previous facts, it has been chosen to encompass all nutrients (sugars, amino acids and vitamins) within a single effective species that diffuses roughly like glucose and is consumed roughly at the same speed. It will be designated as substrate. Its role in the model is to support both maintenance and proliferation.\\

\subsubsection{Oxygen}
Molecular oxygen has a specific role in cell metabolism. It is used as a terminal electron acceptor in the electron transport chain (ETC). Proper function of the electron transport chain is important for ATP production but also for proliferation as some cellular building block synthesis are possible only if the ETC functions.\cite{Martinez2020}\\

It is also known that in case of hypoxia some other molecules can serve as terminal electron acceptor, meaning that proliferation and ATP generation can be maintained in the mitochondria, albeit at a reduced rate, and possibly with adverse effects if the degraded condition lasts too long.\cite{Spinelli2021}\\

Oxygen is thus modelled mainly as an energy booster. Behavior where the lack of oxygen is toxic on its own will be treated but in most cases, lack of oxygen will not be treated as lethal.\\

Oxygen is taken as the fastest diffusing species in the model and therefore in the normalised model its diffusion constant is taken to be one with other species diffusion constants being expressed as a fraction of that oxygen.\\

\subsubsection{Product}
Cells consume substrate/nutrients in order to generate energy and replicate themselves. However, if anabolism can be roughly described as the process of using energy and small molecules in order to form bigger molecules, catabolism is the process by which a molecule is broken in order to extract its chemical potential, possibly through the phosphorylation of ADP into ATP.\\

A good example is the pyruvate that is produced by glycolysis, which is a shorter molecule than glucose. The conversion between the two resulting in the phosphorylation of two ADP into ATP. Glutaminolysis is another possible example. Product may also represent other species such as reactive oxygen species (ROS). In that case, the product may mediate some form of long-distance interactions.\\

Product represent the facts that catabolism generates some molecules that may be released in the extracellular space and possibly used by certain cells provided they have the adequate molecular machinery. This makes it different to the other two species in its implementation.\\

Product starts with a concentration of zero and is produced over time. Since setting the Hill function on product concentration would result in product production staying at zero, this point must be handled differently. As said before product represent the general concept of catabolism where bigger molecules are broken into smaller ones to extract energy from them. Therefore the product increase will be logically coupled to the substrate concentration.\\

The link between maximum product generation and substrate consumption is not trivial. It can be a fraction of the substrate if most of it goes into biosynthesis. But if substrate is not majorly redirected to biosynthesis, then product is produced with a proportionality factor that can be above one. (ex : pyruvate to lactate conversion)\\

\subsubsection{Kine}
The "kines" are species produced by general metabolism with  different dynamics than that of the products. Indeed, Product is generated continuously as long as cells live and proliferate and therefore stops when a cell dies. Kines are the functional opposite: They are released in fixed amount when a cell dies and then diffuse to interact with remaining living cells. Their initial level is thus close to zero initially and increases gradually with cell deaths. \\  

A typical example of such molecule is eATP which is released mostly by stressed or dying cell whose membrane permeability have been compromised. eATP is known to trigger proliferation at intermediate levels and cell death at high levels \cite{Vultaggio2020}\\

\subsection{General considerations}
In terms of nutrients dynamics two different cases are considered. The first and simpler one is the case where fresh medium constantly circulated, that is to say the concentration outside of the sphere is always equal to one for the exogeneous species while the endegeneous species are always equal to zero. In the second case, the medium is refreshed after each division cycle.\\

In the first configuration where medium circulates constantly, after each division round, the reaction diffusion is solved for both substrate and oxygen until it reaches stability for all modelled species. Cell states are then updated before the next cycle of divisions.\\

In the configuration with depletion the reaction-diffusion equation is solved for the same duration that lead to stability in the previous configuration between each division round.\\


\subsection{The cellular model}
The cells are modelled as part of a square grid of the same size as the ones used for the substrate, oxygen and products. The main feature of this part is the behavior of cell. In order to model this, cells  are treated as agents which can react to the concentration levels of the species around them.\\

\subsubsection{Metabolic response}
The specific behaviors implemented will not be listed here, however, sepcific features that they have in common as well as the rationale  behind the design of this aspect of the study can be detailed. \\

First, the change between behavior will be determined by thresholds rather than mapped to continuous fonction. For example, if hypoxia is taken to reduce proliferation, implementation will be :"below concentration X, proliferation stops. Above X, proliferation resumes/continues". The number of possibilites is already significant in that case and this type of model is less computationally demanding.\\

Second, as much as possible the responses to the various thresholds will also be encoded in binary fashion when possible. For example proliferation will not vary continually with the level of factor that may impact but rather stop or resume whenever the threshold is passed.\\

Finally, there must be a response time to behavior change. Cells cannot instantly mobilize the molecular machinery required to operate a change in metabolism. In the model this is achieved by setting a timer for state changes. When a cell changes its consumption state, the timer starts and consumption is incremented or decremented linearly to the targeted value. This avoid unrealistic high amplitude oscillations of concentration in the tissue.
 
\subsubsection{Division}
The most obvious aspect of such a cellular model is division. In our model, cell divides once the system has reached steady-state in terms of concentration. The time needed to reach the steady-state is determined by the balance between the consumption and diffusion of glucose.\\

If one or several free neighbors are available near the mother cell, then daughter cell is placed either randomly in one of the available empty spaces or in the single free neighboring space. If no free space is available then a direction in the 2D-plane is chosen at random and all cells are shifted in the direction of division.\\

It should also be noted that in order to account for possible variation in proliferation rate due to varying conditions, cells are given a division score. If a cell has a division score of 1, it will divide once per day. If it has a division score of 2, it will divide twice a day, and so on.\\

\subsubsection{Cell movements}
The way division is implemented means that if division alone is performed multiple times, the resulting agregate will be ragged and not close to circular shape as a 2D slice of spheroid would be. For this reason after each division round (where all cells that can divide in the model do) a round of migration is applied.\\

This round of migration is used to represent passive and active processes leading to agregates of roughly spherical shapes such as surface tension and local short-scale migration of cells on the surface of the agregates. This routine provides 2D-agregates with roughly circular shape.\\

It is also known that spheroids shed cells sduring their growth.\cite{Freyer1988}\cite{Menchon2009} This shedding is touted to be part of the mechanisms behind metastases. It has been evaluated on various cell line to be of the order  100 cells per mm$^2$ per hour.\\

%The way it is implemented in the model is by removing randomly some cells from the outer layer with a a given probability. Not one, but two shedding round are performed in this manner because a loss rate of 100 cells per mm$^2$ per hour means that a spheroid of 500 \textmu m of radius looses roughly its entire outer layer each day. it was deemed more realistic to potentially have two rounds with probablity of 0.5 rather than a single with a probability of 1.

Shedding is however not implemented in the first runs of the model. Indeed, in several studies factors, such as hypoxia, lack of substrate or other regulatory cues such as detection of growth factors or other species for which receptors are expressed may or may not trigger migration, which here will be included in the form of cell shedding.\\

\subsection{Cell Death}
In this model, the only type of cell death represented so far is necrosis. However, even necrosis is not straightforward to include as it requires the answer to the question : "what happens to necrotic tissue ?"\\

It turns out that necrosis can go two ways. Necrotic debris can be dissolved and removed from the tissue, most likely through a mix of autophagy and diffusion, forming a liquid cavity. But in other cases, necrotic material cannot evacuated leading to calcification.\cite{Thim2010}\cite{YuMi2017}\\

For "liquid" necrosis the choice is made to mark the cell as dying for a given time once it crosses starvation threshold. The consumption then gradually goes down until it reaches zero. The cell then undergoes lysis and is consequently removed from the simulation, freeing the site it occupied.\\

Calcification being the formation of a solid core the step after consumption diminution should be solidification which will also be set on a timer. The site will thus be given a "solid" state with diffusion coefficient set to zero.\\

\subsection{Dense vs sparse configuration}
the model will be a radial geometry on a square grid. In order to save computation time, instead of modelling the full 3D setup, only the central largest layer of the would-be 3D culture is represented.  The general goal is to recapitulate cancer growth and see what the effect of different metabolic behaviors at the cell level have on size and structure.\\

In the dense configuration, cells are considered to be aggregated in circular shape and growing as such with a diffusion coefficient significantly lower in tissue than in medium. That lower diffusion account for both the reduced diffusion due to interaction with the matrix and the tortuosity induced by the presence of cells in said matrix which reduced diffusion further for species that are not membrane permeable. \textbf{[ref]}  

\section{Results: First simple cases}
\subsection{The reference configuration}
In order to illustrate the interplay of the module a reference configuration is studied in which no biological response to nutrients consumption is implemented. Cell multiply, regardelss of nutrient concentration.\\

The reference configuration that does not represent a realistic case as much as it gives an effective reference point to compare other situations against it.\\


\begin{figure}[ht!]
\begin{subfigure}{0.5\textwidth}
	\centering
	\input{ref_conf_area.tex}
	\caption{ \label{ref_conf_area}}
\end{subfigure}
~~
\begin{subfigure}{0.5\textwidth}
	\centering
	\input{ref_conf_Grid.tex}
	\caption{\label{ref_conf_Grid}}
\end{subfigure}
\caption{(a) Number of cells in the inactive as a function of time (b) cell map for the reference configuration at the end of the simulation \label{area_Grid}}
\end{figure}

In that case, as shown in figure \ref{ref_conf_area}, nothing impacts growth and so the cell number increases following an exponential trend. This is due to all cells having the same cycle duration and only the first 16 cells having different timers. Therefore, cells mutiply in batches. The duration of the cell cycle is 500 minutes.\\

As can be seen in fig.\ref{ref_conf_Grid}, cells are organised in a roughly spherical agregate. The color gives the index of the cells. Younger cells are yellow while older cells are blue. As can be seen the division method tend to leave older cells at the center even though some blue spots can be spotted near the rim. \\ 

\begin{figure}[ht!]
\begin{subfigure}{0.5\textwidth}
	\centering
	\input{ref_SO_ctr.tex}
	\caption{ \label{ref_SO_ctr}}
\end{subfigure}
~~
\begin{subfigure}{0.5\textwidth}
	\centering
	\input{ref_SO_ctrline.tex}
	\caption{\label{ref_SO_ctrline}}
\end{subfigure}
\caption{(a) Concentration of substrate (blue) and oxygen (red) vs time (b) Concentration on the midline of the agregate at the end of the simulation \label{ref_SO}}
\end{figure}

In terms of nutrients, the value for $kS$ and $kO$ were 0.1 and 0.7 respectively. The diffusion of subtrate was set to be 60 \% of oxygen in water. The diffusion of oxygen in tissue was set to be 50 \% of that in water and the diffusion of substrate was set to be 5 \% of that oxygen in water.\\

These values were set to ensure that spatial availability of the two nutrients would be nearly the same at steady state. This can be verified in figure \ref{ref_SO_ctrline}. \\

The center of the agregate gets fully depleted of subtrate in $\approx$ 1700 minutes and of oxygen in $\approx$ 2100 minutes. Depletion occurs for a radius of 420 \textmu m which is of the order observed in experiments on spheroids.\cite{MullerKlieser1986}\cite{Freyer1988}\\

\subsection{A simple response: Starvation}
To illustrate the behavior module of the model, a simple case is treated first. The cells are to die if they cross the starvation threshold. The threshold for cell death was set at 30 \% of the external concentration.
\\

As a reminder, cell death in this model is implemented by decreasing the consumption gradually to zero and removing the cell from the simulation subsequently, leaving the grid site unoccupied.\\


\begin{figure}[ht!]
\begin{subfigure}{0.5\textwidth}
	\centering
	\input{starv_area.tex}
	\caption{ \label{starv_area}}
\end{subfigure}
~~
\begin{subfigure}{0.5\textwidth}
	\centering
	\input{starv_Grid.tex}
	\caption{\label{starv_Grid}}
\end{subfigure}
\caption{(a) Surface of the agregate (blue), number of live cells (red) and number of dead cells (yellow) vs time. (b) cell map for the starvation configuration at the end simulation \label{starv_Grid}}
\end{figure}

As expected, the number of live and dead cells is not monotonous. When cells cross the threshold, the initial exponential trend undergoes an inflexion.\\

\begin{figure}[ht!]
\begin{subfigure}{0.5\textwidth}
	\centering
	\input{starv_SO_ctr.tex}
	\caption{ \label{starv_SO_ctr}}
\end{subfigure}
~~
\begin{subfigure}{0.5\textwidth}
	\centering
	\input{starv_SO_ctrline.tex}
	\caption{\label{starv_SO_ctrline}}
\end{subfigure}
\caption{(a) Concentration of substrate (blue) and oxygen (red) vs time (b) Concentration on the midline of the agregate at the end of the simulation \label{ref_SO}}
\end{figure}

As can be seen un figure \ref{starv_SO_ctr}, once the threshold is crossed, the consumption of cells start decreasing. The decrease of consumption leads to an increase of concentration which impacts the dynamics of behavior. This results in visible oscillations of the concentration as cells on the outer rim keep dividing and can impact the concentration as well.\\

The asymmetry observed in figure \ref{starv_Grid} results in a significant nutrient distribution asymmetry for both oxygen and substrate that can be observed \ref{starv_SO_ctrline}

\section{Substrate and oxygen configurations}
The main characteristics of the model have been illustrated in the preceding section. The principle of behavior being regulated by well-defined threshold is now applied to both substrate and oxygen.\\

Applying conditions on both nutrients increases the number of cases to be treated. Indeed, which nutrient is going to be abundant or not. Abundant here meaning that it is present in sufficient amount to not trigger any pallative mechanism in the cell. If one nutrient becomes scarce, then what is the response ?. These study aims at exploring the different answers to these questions.\\

\subsection{Studied configurations}
Contrary to the previous section, three configurations are now explored systematically:
\begin{itemize}
\item \textbf{Id}: The configuration from before where both nutrients have the same spatial availability in steady-state.
\item \textbf{Su}: The substrate-rich configuration. The consumption of subsrate in that configuration is divided by 10.
\item \textbf{Ox}: The oxygen-rich configuration. The consumption of subsrate in that configuration is divided by 10.
\end{itemize}

in terms of behavior both substrate and oxygen must be adressed:
\begin{itemize}
\item \textbf{Subtrate}: can be in mild shortage ($<$ 60\%) or severe shortage ($<$ 30\%)

\begin{itemize}
\item \textbf{Mild shortage}: may trigger compensation by increasing oxygen consumption and may decrease substrate consumption
\item \textbf{Severe shortage}: triggers cell death irreversibly
\end{itemize}

\item  \textbf{Oxygen}: either above or below an hypoxia threshold ($<$ 40\%)
\begin{itemize}
\item \textbf{Hypoxia}: may trigger compensation by increasing substrate consumption, increase or decrease proliferation
\end{itemize}
\end{itemize}

At this point no mechanism account for the possibility of "oxygen-scarcity-induced-cell-death". This will be included later.

\subsection{Population}



\newpage
\bibliographystyle{unsrt}
\bibliography{/home/antonybazir/Documents/Post-doc/Redac/biblio_synthese}
\end{document}