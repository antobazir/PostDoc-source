\documentclass[11pt,a4paper]{article}
%\usepackage[utf8]{inputenc}
%\usepackage[ascii]{inputenc}
\usepackage{geometry}
\usepackage[dvipsnames]{xcolor}
\usepackage{textcomp}
\usepackage{graphicx}
\usepackage{caption}
\usepackage{subcaption}
\usepackage{amsmath}

\begin{document}
**DIPG-67. HYPOXIA-INDUCIBLE FACTORS REGULATE DIFFUSE INTRINSIC PONTINE GLIOMA GROWTH IN NORMOXIC CULTURE
-In both ambient and hypoxic conditions, HIF2-alpha activity may oppose DIPG growth.
- All three DIPG cultures retained stable expression of HIF1-
alpha and HIF2-alpha protein at ambient oxygen tension, 
-"Hypoxia-mimetics increase cultured DIPG glycolytic rate and enzyme expression, and decrease
proliferation"
- On note que la proliferation des lignées des DIPG peut varier du simple au double dans les mêmes conditions IV prolifère plus vite que XIII et que l'hypoxie ralentit cette prolif
- On apprend deux choses 1) l'hypoxie augmente leur glycolyse et elles sont MAIS la suprression d'une partie de la production d'ATP mitochondriale n'etraine pas d'ajustement notable.

** Hypoxia Inducible Factors’ Signaling in Pediatric High-Grade Gliomas: Role, Modelization and Innovative Targeted Approaches
-"In fact, HIF-1α is dedicated to acute hypoxia, whereas HIF-2α seems to be expressed during chronic phase of hypoxia to maintain immature cells and probably pHGG stem cell niches [27,29,35,53,54]."
-"Finally, the only way to better reproduce the global hypoxic conditions in cells is using a hypoxic
chamber or incubator in which oxygen levels can be regulated from 21% (normoxia) to the most extreme
hypoxia encountered in HGG with O2 less than 0.1\%"
- Bon c'est complexe..
-"Glutamine addiction is usually associated to glycolysis impairment and Akt induction in those HGG cells [6]."
Acute hypoxia acts on peripheral cells that are incompletely and intermittently oxygenated by an aberrant neovascularization. The temporal scale for acute hypoxia is generally considered from minutes to several hours. After 24 h of poor oxygen levels, the tumor cells are considered in a chronic hypoxic state and usually located in the furthest part of the tumor from the blood vessels. Lastly, the cyclic hypoxia might be considered as a kind of intermediary between acute and chronic hypoxia. "
-"To regulate cell cycle progression, HIF-1α inhibits MYC, while HIF-2α promotes MYC/MYCN stabilization. On cell lines with poor MYC expression but high HIF-2α expression, this latter was sufficient to promote cell proliferation in the absence of MYC/MYCN deregulation."

**Hypoxia, metabolism, and the circadian clock: new links to overcome radiation resistance in high-grade gliomas
-"Furthermore, hypoxia activates the hypoxia-inducible factor 1 (HIF-1) pathway which favors the survival of tumor cells by increasing their glucose uptake and utilization via altered glucose metabolism [6]. It also induces angiogenesis [7], creates an acidic microenvironment and promotes proliferation [8]" WHAT ?

**Acute vs. Chronic vs. Cyclic Hypoxia: Their Differential Dynamics, Molecular Mechanisms, and Effects on Tumor Progression
- Figure 1 en cas d'acute hypoxia -> HIF1 et HIF2 et hypoxie chronique HIF2 tout seum
-"In terms of oxygen concentration, HIF-2α is seen to be more stable compared to HIF-1α at higher oxygen levels (5\% O2) in neuroblastoma cell lines SK-N-BE(2)C and KCN-69n. In these cell lines at 1\% O2, both HIF-1α and HIF-2 α get stabilized; while HIF-1α levels stay high to mediate acute response and decay during prolonged hypoxia, HIF-2α accumulates to regulate cellular response under prolonged hypoxia [38,39]. "
-neuroblastoma and lung cancer cells but hey 


**Therapeutic targeting of hypoxia and hypoxia-inducible factors in cancer
-"Hypoxia can result in decreased cell proliferation, cell cycle arrest, and/or apoptosis. "
-"Hypoxia-induced HIF-1α inhibits c-Myc transcription and suppresses proliferation (Koshiji et al., 2004), and HIF-1α can also promote c-Myc degradation (Zhang et al., 2007). In contrast, HIF-2α can enhance c-Myc activity and promote cell cycle progression (Gordan et al., 2007a)."

**HGG-12. HYPOXIA SEEMS TO BE FREQUENTLY UPREGULATED IN THE PEDIATRIC HIGH GRADE GLIOMA AND DIPG
-"We, previously,
observed that the resistance to mTor and HIF1 inhibitions was completely
linked to HIF2 hyperexpression spontaneously in the pHGG and DIPG
cell lines""
-"The DIPG subgroup was statistically associated with HIF2 hyperexpression and the complete absence of mTor expression. "

**Frequently asked questions in hypoxia research
If we take as a likely approximation that typical cell culture media have properties similar to blood plasma, the plasma O2 solubility of 1.26 μM O2 per 1 mmHg at 37°C1 would result in 1.26 μM/mmHg ×141 mmHg = 177.66 μM O2 concentration under normoxic incubator conditions

**Hypoxia Inducible Factors’ Signaling in Pediatric High-Grade Gliomas: Role, Modelization and Innovative Targeted Approaches
-"However, HIF-1α and HIF-2α led the cells to a final complete metabolic reprogramming with the activation of glutaminolysis, serinolysis and/or phospholipid metabolism [24,25]. "

**Constitutive Expression of Hypoxia-inducible Factor-1α Renders Pancreatic Cancer Cells Resistant to Apoptosis Induced by Hypoxia and Nutrient Deprivation
- pancreatic cancer cell lines express HIF1-alpha in normoxia

**Myc regulates a transcriptional program that stimulates mitochondrial glutaminolysis and leads to glutamine addiction
-"In contrast, high level expression of Myc was required to maintain the glutaminolytic phenotype and addiction to glutamine as a bioenergetic substrate"

**Characterization of the transcriptional and metabolic responses of pediatric high grade gliomas to mTOR-HIF-1α axis inhibition
-"It can also be stabilized in normoxic conditions through different mechanisms of dysregulation at the transcriptional, translational and post-translational levels, leading to a pseudo-hypoxic phenotype [13–15]. HIF-1α and HIF-2α, the two most studied isoforms, activate numerous target genes. Noticeably, HIF-1α can activate the majority of genes involved in glycolysis, but, also, those involved in pH regulation, angiogenesis and metastatic progression [14, 15]. HIF-2α acts synergistically with HIF-1α to regulate many of these genes and its expression seems to be correlated with a more aggressive phenotype [16–18]. "
-" When mTOR is hyperactivated, it promotes HIF-1α protein activation, counteracting its degradation in normoxic conditions, leading to pseudo-hypoxia. Essential for tumor growth and survival, mTOR is also a sensor of nutrient level, "
-"Sensitivity to mTOR/HIF-1α inhibition correlates positively with AKT/PI3K activation and negatively with HIF-2α accumulation"

- Shen states that hypoxia increases glycolysis and glucose uptake compared to normoxic cell culture:"Furthermore, hypoxia activates the hypoxia-inducible factor 1 (HIF-1) pathway which favors the survival of tumor cells by increasing their glucose uptake and utilization via altered glucose metabolism [6]."

**Evofosfamide Is Effective against Pediatric Aggressive Glioma Cell Lines in Hypoxic Conditions and Potentiates the Effect of Cytotoxic Chemotherapy and Ionizing Radiations
-"HIF-1α expression was higher when cells were incubated under hypoxia compared to normoxic conditions, indicating that they actually sense our low-oxygen in vitro conditions (Figure 1a, uncropped version in Figure S1). Nevertheless, we did not observe any cell culture suffering up to 96 h under these hypoxic conditions,"

**Cycling hypoxia increases U87 glioma cell radioresistance via ROS induced higher and long-term HIF-1 signal transduction activity 
-"Our results demonstrated that cycling hypoxia induced higher and longer term HIF-1 signal transduction activity via reactive oxygen species (ROS) in U87 cells compared with non-interrupted hypoxia."

**Cycling hypoxia induces chemoresistance through the activation of reactive oxygen species-mediated B-cell lymphoma extra-long pathway in glioblastoma multiforme
-"Cycling hypoxia-induced Bcl-xL expression via ROS-mediated HIF-1α and NF-κB activation plays an important role in the tumor microenvironment-promoted anti-apoptosis and chemoresistance in glioblastoma."

\end{document}