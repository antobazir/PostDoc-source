\documentclass[11pt,a4paper]{article}
%\usepackage[utf8]{inputenc}
%\usepackage[ascii]{inputenc}
\usepackage{geometry}
\usepackage[dvipsnames]{xcolor}
\usepackage{textcomp}
\usepackage{graphicx}
\usepackage{caption}
\usepackage{subcaption}
\usepackage{amsmath}

\begin{document}
Lire tout ça et voir si on peut en tirer quelque chose au global

**Measuring and modeling energy and power consumption in living microbial cells with a synthetic ATP reporter
%-"We found that cellular power consumption varies significantly from approximately 0.8 and 0.2 million ATP/s for a tested strain during lag and stationary phases to 6.4 million ATP/s during exponential phase, indicating approx 8 30 fold changes of metabolic rates among different growth phases"

**Metabolic control of the cell cycle

**Monitoring and modeling of lymphocytic leukemia cell bioenergetics reveals decreased ATP synthesis during cell division
-". We confirmed this result using population-based measurements of bioenergetics, which suggested that mitotic cells have lower ATP synthesis rates as well as ATP and ADP levels in comparison to G2 cells"
-"suggest that overall cellular ATP synthesis is decreased by approx 50\% in mitosis when compared to G2. Similarly, ATP levels decreased  approx 40\% in mitosis (Fig. 5d)." 

**Adenosine Triphosphate and Synchronous Mitosis in Physarum polycephalum
-"The experiments in this paper show that the pool of ATP decreases during mitosis. At this time, the rate of macromolecular synthesis is low: DNA is not synthesized (14), and ribonucleic acid synthesis (13) and protein (12) synthesis occur at much lower rates than in the premitotic period. In contrast, the ATP pool is high throughout interphase, when macromolecular synthesis proceeds at high rates, A possible correlation between the size of the ATP pool and macromolecular synthesis would have ATP increase when synthetic rates are low and decrease when synthetic rates are high. Our data do not follow this pattern. The depletion of ATP may reflect the energy requirements of
mitosis"

**Glucose uptake in the cell cycle of Saccharomycescerevisie
-two peaks in the cell cycle glucose uptake. One before and one after division


**Cell Cycle-Dependent Regulation of Cellular ATP Concentration, and Depolymerization of the lnterphase Microtubular Network Induced by Elevated Cellular ATP Concentration in Whole Fibroblasts
- "it is seen that the average ATP concentration of the cell population increases as the cells enter late S-, and G,+M-phase, and reaches its maximum at about 25 h. This coincides with the peak of the G, + M-fraction, indicating that the ATP concentration is high during G, + M-phase. The ATP concentration of the cell population decreases after completion of cell division."

**A Restriction Point for Control of Normal Animal Cell Proliferation
- "The restriction point control is proposed to permit normal cells to retain viability by a shift to a minimal metabolism upon differentiation in vivo and in vitro when conditions are suboptimal for growth. Malignant cells are proposed to have lost their restriction point control. Hence, under very adverse conditions, as in the presence of antitumor agents, they stop randomly in their division cycle and die."

**Quantitative analysis of cell cycle phase durations and PC12 differentiation using fluorescent biosensors
- G1 and S phase have 400 mn G2 is roughly 100 and M is 40 mn

**Fueling the Cell Division Cycle
-" Both the initiation and completion of DNA synthesis and mitosis
are energy-dependent and require oxidative phosphorylation (OXPHOS) in plant and animal cells [7,8]. In plant cells, the energy requirements were reported to be higher before and during entry into S- or M-phases than during DNA synthesis or chromosome segregation itself, in agreement with the relevance of the G1/S and G2/M transitions [9]. Aerobic glycolysis was also reported to peak during S-phase entry in lymphocytes [10], whereas the consumption of O 2 and lactate production specifically increased during G2/M in Ehrlich ascites carcinoma cells [11]." IMPORTANT

**Nutrient starvation induces apoptosis and autophagy in C6 glioma stem-like cells 


**Resistance to glucose starvation as metabolic trait of platinum-resistant human epithelial ovarian cancer cells
-"Tumor cells from GNA patients rely more on autophagy compared to GA patients"
-"GNA samples are characterized by low proliferation and high MDR pump expression"

**Optimisation of viability assay for DIPG patient-derived cell lines
 CellTiter-Glo
 
 **Understanding tumor anabolism and patient catabolism in cancer-associated cachexia
-Regarding glucose and glutamine, most cancer cells develop high avidity for their consumption to generate energy and to build macromolecules for tumor progression [6], which, together with the constitutive activation of signaling pathways downstream of diverse growth factor receptors (even without circulating growth factors), doubles their total biomass to generate daughter cells [3,7].

**How does mTOR sense glucose starvation? AMPK is the usual suspect
- dditionally, this is likely explained by the observation that mTORC1 inhibition mediates tumor cells protection against conditions of glucose deprivation7,8, commonly encountered within the tumor microenvironment.

**https://www.aatbio.com/resources/faq-frequently-asked-questions/What-are-the-growth-phases-of-culture-cells
-Lag phase: At this stage cells do not divide. It is the period when cells are adjusting to the culture condition and preparing for the cell division.
    Log phase: It is also called logarithmic phase or exponential phase, when cells actively proliferate and the cell density increases exponentially. It is recommended to assess cellular function at this stage since the cell population is most viable. Cells are also generally passaged at late log phase, because passaging cells too late can lead to overcrowding, apoptosis and senescence.
    Stationary phase (or plateau phase): Cell proliferation slows down due to a growth-limiting factor such as the depletion of an essential nutrient and/or the formation of an inhibitory product, resulting in a situation in which growth rate and death rate are equal. Cells are most susceptible to injury at this stage.
    Death phase (or decline phase): Cell death predominates at this phase and the number of viable cells reduces.
Associé à la CULTURE et au collectif et pas à la cellule elle-même.

**G1 phase wikipedia
-"The first restriction point is growth-factor dependent and determines whether the cell moves into the G0 phase, while the second checkpoint is nutritionally-dependent and determines whether the cell moves into the S phase.[3][4]"
-"This transition is essentially irreversible; after passing the restriction point, the cell will progress through S-phase even if environmental conditions become unfavorable.

**The Restriction Point of the Cell Cycle
- "Although S-phase is independent of growth factors, massive DNA damage or deprivation of nucleotides forces a cell to be arrested in S-phase, but such arrest is usually accompanied by cell death."

**Mechanisms of Cell Cycle Arrest and Apoptosis in Glioblastoma
The entrance into the quiescence state allows for resistance to overcome stress and toxic stimuli. After tolerating and repairing the cellular damage, the cells may re-enter a novel cell cycle upon stimulation by specific growth factors, such as cyclin-dependent kinase-2 (CDK2) and E2F [7,8,9,10,11,12,13].
Unlike quiescence, senescence is a state of permanent cell cycle arrest with high cellular metabolism. Both quiescence and senescence are triggered by external and internal signals, such as ionizing radiation, DNA and chromatin damage, endogenic replication stress, cellular stress from reactive oxygen species (ROS), serum starvation, contact inhibition, etc., whereas persistent damage and stress signaling often favor senescence

**Understanding cell cycle and cell death regulation provides novel weapons against human diseases
- inactivation of caspases or a significant decrease in the level of ATP can lead to a shift from apoptosis to necrosis, or to a mixture of these two types of cell death. 
- In this view, autophagy acts as a prosurvival mechanism, mainly in adverse conditions such as nutrient and oxygen deprivation. However, this process has a clear self-limiting character and may lead to cell death with autophagic features (or programmed cell death type II) when at high levels or duration [42]

**Metabolic Stress in Autophagy and Cell Death Pathways
-"Cells that do not receive proper growth factor signals typically atrophy, lose the ability to uptake and use extracellular nutrients, and instead induce the self-digestive process of autophagy as an intracellular energy source before ultimately undergoing programmed cell death. Cancer cells, in contrast, often become independent of extracellular growth signals by gaining mutations or expressing oncogenic kinases to drive intrinsic growth signals that mimic growth factor input, which can be the source of oncogene addiction. Growth factor input or oncogenic signals often drive highly elevated glucose uptake and metabolism "
-"If glucose metabolism remains insufficient or disrupted, the cells can switch to rely on mitochondrial oxidation of fatty acids and amino acids, which are energy rich but do not readily support cell growth and can lead to potentially dangerous levels of reactive oxygen species (Wellen and Thompson 2010). Amino acid deficiency can directly inhibit components of the signaling pathways downstream from growth factors and activate autophagy (Lynch 2001; Beugnet et al. 2003; Byfield et al. 2005; Nobukuni et al. 2005). Finally, hypoxia induces a specific pathway to increase nutrient uptake and metabolism via the hypoxia-inducible factor (HIF1/2α) that promotes adaptation to anaerobic conditions, but may lead to apoptosis if hypoxia is severe"
-"In cases in which these apoptotic pathways are suppressed, metabolic stress can instead lead to necrotic cell death"*****
-"Ultimately, necrosis may occur when cells do not meet their minimal bioenergetic demands (Jin et al. 2007). In these instances, a collapse of ATP levels may lead to failure of ATP-dependent sodium/potassium exchangers and osmotic stress and cell rupture. "
- "If DNA damage is extensive, PARP can deplete NAD+, causing a collapse of glycolysis, loss of ATP production, and subsequent necrosis"

**Tumor necrosis: a synergistic consequence of metabolic stress and inflammation
-"Both hypoxia and glucose deprivation have been shown to increase intracellular ROS production in cancer cells, leading to mitochondrial dysfunction, energy depletion, and eventual cellular demise [12, 29]."

**A reduced model of cell metabolism to revisit the glycolysis-OXPHOS relationship in the deregulated tumor microenvironment
- A model with lactate pH glucose and oxygen. Based on casciari Jagiella etc...
- ils mettent la quiescence aussi !

**Cooper
-"In the absence of growth factors, cells are unable to pass the restriction point and instead become quiescent, frequently entering the resting state known as G 0 , from which they can re-enter the cell cycle in response to growth factor stimulation."
-"Loss of p53 function as a result of these mutations prevents cell cycle arrest in response to DNA damage, so the damaged DNA is replicated and passed on to daughter cells instead of being repaired."

**In Vitro/In Silico Study on the Role of Doubling Time ..
 \end{document}