\documentclass[11pt,a4paper]{article}
%\usepackage[utf8]{inputenc}
%\usepackage[ascii]{inputenc}
\usepackage[margin=0.7in]{geometry}
%\usepackage{geometry}
\usepackage[dvipsnames]{xcolor}
\usepackage{textcomp}
\usepackage{graphicx}
\usepackage{caption}
\usepackage{subcaption}
\usepackage{amssymb}
\usepackage{amsmath}
\usepackage{tikz}

\begin{document}
\section{Glucose}
\textbf{**Glucose Metabolism Heterogeneity in Human and Mouse Malignant Glioma Cell Lines}\\

-"The D-54MG and GL261 glioma cell lines displayed an oxidative phosphorylation (OXPHOS)-dependent phenotype, characterized by extremely long survival under glucose starvation, and low tolerance to poisoning of the electron transport chain (ETC). Alternatively, U-251MG and U-87MG glioma cells exhibited a glycolytic-dependent phenotype with functional OXPHOS. These cells displayed low tolerance to glucose starvation and were resistant to a ETC blocker. Moreover, these cells could be rescued in low glucose conditions by oxidative substrates (e.g., lactate, pyruvate). Finally, these two phenotypes could be distinguished by the differential expression of LDH isoforms. OXPHOS-dependent cells expressed both LDH-A and -B isoforms whereas glycolytic-dependent glioma cells expressed only LDH-B. In the latter case, LDH-B would be expected to be essential for the use of extracellular lactate to fuel cell activities."\\

- Check the metabolome and proteomics of the DIPG article\\

**\textbf{Glucose transporter Glut1 controls diffuse invasion phenotype with perineuronal satellitosis in diffuse glioma microenvironment}


**\textbf{Proteomics and metabolomics approach in adult and pediatric glioma diagnostics}\\

-Super complet et intéressant même si ça répond pas à ma question.\\


\textbf{**Glioblastoma cells require glutamate dehydrogenase to survive impairments of glucose metabolism or Akt signaling}\\

-Oncogenes influence nutrient metabolism and nutrient dependence. The oncogene c-Myc stimulates glutamine metabolism and renders cells dependent on glutamine to sustain viability (“glutamine addiction”), suggesting that treatments targeting glutamine metabolism might selectively kill c-Myc-transformed tumor cells.\\

-conso glutamine 100e9 mol/hr/1e6 cell pour les SF188

\section{Electron Transport chain blocking}
\textbf{**Metabolic Reprogramming in Glioma}\\
-"meanwhile fatty acids are used both as energetic substrates and as raw materials for lipid membranes."\\

-"A further complication—alongside intracellular metabolic complexity—is potential heterogeneity in metabolic strategies across different cell types within the tumor."\\

-"When this hypothesis was tested in primary-cultured human glioblastoma cells, it was observed that cells were highly oxidative and largely unaffected by treatment with glucose or inhibitors of glycolysis (Lin et al., 2017). Thus, it appears that substrate oxidation can co-exist with aerobic glycolysis and lactate release."\\

-"For example, a recent study showed that glioma stem cells (GSCs) are less glycolytic than differentiated glioma cells, consuming lower levels of glucose and producing lower amounts of lactate while maintaining higher ATP levels compared with their differentiated progeny. The notorious radio-resistance of this cell population is correlated with higher mitochondrial reserve capacity, leading the authors to conclude that GSCs primarily rely upon oxidative metabolic strategies and will not be vanquished by therapies aiming to inhibit glycolysis (Vlashi et al., 2011)."\\

-"Specifically, highly proliferative cells have elevated PPP enzymes and lower expression of glycolytic enzymes, while highly migratory cells have a reverse profile. Thus, it appears that metabolic specialization within tumors prize nucleic acid generation in dividing cell types. Mechanisms controlling this behavior in glioma cells are understudied."\\

-"Glycine and serine levels increase in cultured rat glioma cells exposed to oxygen and glucose deprivation (Fuchs et al., 2012) and enzymes within this pathway are highly expressed in pseudopalisading cells surrounding necrotic foci (Kim et al., 2015)."\\

-"Recent results from our lab and other groups have demonstrated that glioma cells primarily use fatty acids as a substrate for energy production. Specifically, human glioma cells primary-cultured under serum-free conditions oxidize fatty acids to maintain both respiratory and proliferative activity (Lin et al., 2017). 13C in vivo radiolabelling studies conducted in orthotopic mouse models of malignant glioma show that acetate contributes over half of oxidative activity within these tumors, while glucose contributes only a third (Maher et al., 2012; Mashimo et al., 2014)."\\

-"Glucose can be transported into the cells, converted to fatty acids by the enzyme fatty acid synthase (FASN), then imported into the mitochondria for beta-oxidation (a process known as a Futile Cycle, see next section). Glioma cells contain FASN, and indeed the expression of this enzyme increases with tumor malignancy (Tao et al., 2013). "\\

-"Fatty acid synthesis has been shown to continue under low-oxygen tension and low-nutrient conditions (Lewis et al., 2015), a process which is activated by HIF1$\alpha$ signaling. Fatty acids are shuttled into lipid droplets upon hypoxia in order to support cell growth and survival upon re-oxygenation (Bensaad et al., 2014)."\\

-"Fourthly, the reliance of human glioma cells on fatty acid oxidation is abrogated after serum exposure (Lin et al., 2016, 2017), a commonly-used culture method which alters the characteristics of brain-derived cancer cells (Pollard et al., 2009). Studying cells under these conditions may therefore cause an underestimation of oxidative activity."\\

-il dit aussi que l'hypoxie légère provoque plutot la prolifération

\end{document}