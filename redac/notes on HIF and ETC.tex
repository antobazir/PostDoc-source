\documentclass[11pt,a4paper]{article}
%\usepackage[utf8]{inputenc}
%\usepackage[ascii]{inputenc}
\usepackage{geometry}
\usepackage[dvipsnames]{xcolor}
\usepackage{textcomp}
\usepackage{graphicx}
\usepackage{caption}
\usepackage{subcaption}
\usepackage{amsmath}

\begin{document}
** Hypoxia Inducible Factors Modulate Mitochondrial Oxygen Consumption and Transcriptional Regulation of Nuclear-Encoded Electron Transport Chain Genes
-bon difficile de conclure mais a priori, oui.

-Galactose is a C-4 epimer of glucose (OH and H group swapped). This sugar must be converted into glucose before being metabolised
-"A change from Glc to Gal in culture media increases the respiration rate and cytosolic pH, increases the rate of cellular oxygen consumption, and results in mitochondrial remodeling.35,36 In human primary myotubes, Gal medium enhances aerobic metabolism and decreases the rate of anaerobic glycolysis."


**Hypoxia. 2. Hypoxia regulates cellular metabolism
-  Hypoxia diminishes ATP production in part by lowering the activity of the electron transport chain through activation of the transcription factor hypoxia-inducible factor-1.


** Mitochondrial reactive oxygen species trigger hypoxia-induced transcription
Thus, hypoxia activates transcription via a mitochondria-dependent signaling process involving increased ROS, whereas CoCl2 activates transcription by stimulating ROS generation via a mitochondria-independent mechanism.

**Mitochondrial composition and function under the control of hypoxia
-"A lack of oxygen induces reductive carboxylation, which was shown to increase ROS production [20], [21], [22], [23]. Moreover, hypoxia causes multiple changes in the composition of ETC complexes. These changes are required to keep mitochondria intact under low oxygen conditions and to prevent excessive ROS formation. Most of the changes in complex composition occur within complex I, a dominant acceptor of electrons, and complex IV, facilitating electron transport to molecular oxygen. Some of these alterations comprise the exchange of subunits within ETC complexes others modify complex structure, while subunit depletion also occurs."

**Cellular adaptation to hypoxia through hypoxia inducible factors and beyond
-"It is important to note that intracellular O2 levels at 0.3\% begin to become rate limiting for ETC activity"
-"COX has a high affinity for O2 and an apparent Km close to 0.1\% O2; thus, the ETC can function at near anoxic levels93 and cells largely maintain their ATP levels during hypoxia."
-"Interestingly, starting at 3\% O2, when O2 is not limiting for ETC function, cells have developed mechanisms to decrease their cellular metabolic demand94"
-"One of the most prominent adaptations in O2 starved [cancer] cells is increased glucose uptake and elevated glycolytic flux to promote glucose catabolism."
-"Strikingly, human non-small-cell lung cancers not only employ lactate as a fuel source, but they incorporate more lactate-derived carbons into TCA cycle intermediates than those from glucose, indicating that lactate can be a preferred anaplerotic substrate143. " THE FUCK
-"Similarly, hypoxia induces the uptake of glutamine,"

**AMPK-mTOR Signaling and Cellular Adaptations in Hypoxia
-"As a terminal ETC component, Complex IV (cytochrome c oxidase, COX) stops electron flows by delivering electrons to oxygen, producing two molecules of water in the process. It has a high affinity for oxygen (Km close to 0.1\% oxygen), therefore, the ETC can function at near anoxic levels (around 0.5\% oxygen) and cells can maintain minimum ATP levels to survive during hypoxia [172]."
-"n response to acute or mild hypoxia, cells turn on the mechanisms to decrease cellular ATP demand by suppressing ATP-consuming processes, thereby decelerating oxygen consumption [178]."
" Aspartate is necessary for nucleotide synthesis, suggesting that aspartate can be a limiting metabolite for tumor growth [131]."

**Aspartate is a limiting metabolite for cancer cell proliferation under hypoxia and in tumours 
-"As oxygen is essential for many metabolic pathways, tumour hypoxia may impair cancer cell proliferation1-4. "
-"Cell lines least sensitive to ETC inhibition maintain aspartate levels by importing it through an aspartate/glutamate transporter, SLC1A3."
\end{document}