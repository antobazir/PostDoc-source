\documentclass[11pt,a4paper]{article}
%\usepackage[utf8]{inputenc}
%\usepackage[ascii]{inputenc}
\usepackage{geometry}
\usepackage[dvipsnames]{xcolor}
\usepackage{textcomp}
\usepackage{graphicx}
\usepackage{caption}
\usepackage{subcaption}
\usepackage{amsmath}

\begin{document}
** Hypoxia Inducible Factors Modulate Mitochondrial Oxygen Consumption and Transcriptional Regulation of Nuclear-Encoded Electron Transport Chain Genes
-bon difficile de conclure mais a priori, oui.

-Galactose is a C-4 epimer of glucose (OH and H group swapped). This sugar must be converted into glucose before being metabolised
-"A change from Glc to Gal in culture media increases the respiration rate and cytosolic pH, increases the rate of cellular oxygen consumption, and results in mitochondrial remodeling.35,36 In human primary myotubes, Gal medium enhances aerobic metabolism and decreases the rate of anaerobic glycolysis."


**Hypoxia. 2. Hypoxia regulates cellular metabolism
-  Hypoxia diminishes ATP production in part by lowering the activity of the electron transport chain through activation of the transcription factor hypoxia-inducible factor-1.


** Mitochondrial reactive oxygen species trigger hypoxia-induced transcription
Thus, hypoxia activates transcription via a mitochondria-dependent signaling process involving increased ROS, whereas CoCl2 activates transcription by stimulating ROS generation via a mitochondria-independent mechanism.

**Mitochondrial composition and function under the control of hypoxia
-"A lack of oxygen induces reductive carboxylation, which was shown to increase ROS production [20], [21], [22], [23]. Moreover, hypoxia causes multiple changes in the composition of ETC complexes. These changes are required to keep mitochondria intact under low oxygen conditions and to prevent excessive ROS formation. Most of the changes in complex composition occur within complex I, a dominant acceptor of electrons, and complex IV, facilitating electron transport to molecular oxygen. Some of these alterations comprise the exchange of subunits within ETC complexes others modify complex structure, while subunit depletion also occurs."
\end{document}