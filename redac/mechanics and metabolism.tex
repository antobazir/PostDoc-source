\documentclass[11pt,a4paper]{article}
%\usepackage[utf8]{inputenc}
%\usepackage[ascii]{inputenc}
\usepackage{geometry}
\usepackage[dvipsnames]{xcolor}
\usepackage{textcomp}
\usepackage{graphicx}
\usepackage{caption}
\usepackage{subcaption}
\usepackage{amsmath}

\begin{document}

Is it possible to understand through cadherin integrins and CAMs the link between mechanics and metabolism

**Crosstalk between mechanotransduction and metabolism
- "however, forces applied at different adhesions have different biological effects, as indicated, for
example, by the growth-promoting effect of cell–ECM. contacts and by the growth-inhibiting effect of cell–cell
adhesions."
-"On a stiff ECM, the ubiquitin ligase TRIM21 is trapped by stress fibres (TaBle 1), and can-
not induce the degradation of PFK 40 (Fig. 2a). On a soft  ECM, where stress fibres are inhibited, TRIM21 is able
to target PFK and direct it to degradation, thus reducing glycolysis40. The reduced glycolysis rate was not accom-
panied by a compensatory increase in mitochondrial respiration, indicating a decoupling between the two main energy-producing pathways that should limit the total energy production capacity in cells on a soft ECM."
-"Indeed, oncogene transformation overrides this regulation and uncouples ECM stiffness from glycolysis, leading to aberrant metabolism of glucose on soft substrata 40 , a key hallmark of cancer (Box 2)."
-"Intracellular pH is regulated by integrin-mediated cell spreading, and thus by the level
of cellular tension 46–49 . Intracellular pH is more basic, indicating more active H + extrusion, in spread cells,
where cell tension is higher, compared with rounded cells, where tension is lower, and this is due to differential NHE1 activity (Fig. 2a). "
-The first example of YAP/TAZ- dependent mechanotransduction regulating metabolism was the obser-
vation that ECM stiffness promotes glycolysis and the concerted use of glutamine for the synthesis of aspartate,
a precursor of nucleotides needed for cell proliferation79."
-" Overall,YAP/TAZ- driven glutamate production matches the mediated induction of autophagy and resistance to
nutrient starvation was partially shown 88, and the metabolic effects of such regulation await better characterization. Moreover, this effect is opposite to the increased autophagy observed on soft ECM by others90. The reason for such discrepancy remains unknown and may be due to differences in ECM composition and/or 2D versus
3D cell organization."
-"The complete absence of adhesion (cell detachment) can be compared, to some extent, with an extremely soft ECM"*******
-"lung cancer cells cultured as spheroids, characterized by low or absent cell–ECM adhesions, rewired the use of glucose and glutamine to increase the production of antioxidant molecules within the mitochondria101 (Fig. 4b). "
-"or instance, detachment from ECM made of hyaluronic acid leads to increased glucose metabolism 103 , which is opposite to other observations described above and suggests that cells likely integrate various cues to determine their (metabolic) fate."******
-" Extensive evi-
dence indicates that ECM detachment can induce
autophagy to delay cell death normally associated with
lack of attachment 105–108 (Fig. 4c). "
-Ferroptosis is a form of regulated cell death occurring
when the ability of cells to scavenge lipid peroxides and
prevent the resulting oxidative stress is impaired 114 .
-"Under nutrient- rich conditions, cells
produce new ECM components and stiffen the ECM,
while under nutrient- deprived conditions, cells con-
sume the ECM. As this may have a measurable impact
on ECM mechanics, at least locally, it will be interesting
to investigate whether changes in ECM composition
of nutrient-deprived cells affect the activity of mech-
anotransduction pathways in these cells and if so, how
such differences translate to phenotypic alterations,
including cancer progression"

**Extracellular-matrix mechanics regulate cellular metabolism: A ninja warrior behind mechano-chemo signaling crosstalk
-stiff ECM -> more glycolysis

** Mechano-induced cell metabolism promotes microtubule glutamylation to force metastasis
-"we show that matrix stiffening rewires glutamine metabolism to promote MT glutamylation and force MT stabilization, thereby promoting cell invasion."

**A Feedforward Mechanism Mediated by Mechanosensitive Ion Channel PIEZO1 and Tissue Mechanics Promotes Glioma Aggression 
 PIEZO1 is overexpressed in aggressive human gliomas and its expression inversely correlates with patient survival. Deleting PIEZO1 suppresses the growth of glioblastoma stem cells, inhibits tumor development, and prolongs mouse survival. Focal mechanical force activates prominent PIEZO1-dependent currents from glioma cell processes, but not soma. PIEZO1 localizes at focal adhesions to activate integrin-FAK signaling, regulate extracellular matrix, and reinforce tissue stiffening. In turn, a stiffer mechanical microenvironment elevates PIEZO1 expression to promote glioma aggression.
 
 **Bioengineered Models to Study Microenvironmental Regulation of Glioblastoma Metabolism
 -"The GBM ECM displays a wide array of aberrantly expressed ligands and signaling molecules that modulate tumor cell proliferation, invasiveness, and aggressiveness. Elevated hyaluronic acid (HA) increases tumor stiffness, and higher levels of CSPGs, such as brevican and versican, promote invasiveness, and proliferation (61)."
 -"Up-regulation of quiescent GBM cells was identified with up-regulation of laminin, collagens, tenascin C, and integrin α3 (62)."
"Furthermore, GSCs and differentiated tumor cells exhibit differing metabolic profiles in soft versus stiff microenvironments. Stiffened microenvironments can activate the PI3K/Akt pathway, which represent a mechanotransductive link to increased glycolysis (77). Recent work has also shown the impact of mechanosensation, differentiation, and metabolism in GBM. Hughes et al (78) showed that the mesenchymal growth factor, BMP4 promotes GSC differentiation and leads to reduced oxidative phosphorylation and differential cell spreading on soft and stiff substrates. Inhibition of oxidative phosphorylation surprisingly disrupts cell protrusive extensions and spread area, underscoring a complex interaction between GBM metabolism, cell mechanotransduction, and TME biophysical characteristics"

**Reciprocal H3.3 gene editing identifies K27M and G34R mechanisms in pediatric glioma including NOTCH signaling 
 The K27M and G34R mutations induced several of the same pathways suggesting key shared oncogenic mechanisms including activation of neurogenesis and NOTCH pathway genes. 
H3.3 mutant gliomas are also particularly sensitive to NOTCH pathway gene knockdown and drug inhibition, reducing their viability in culture.

**Tissue mechanics regulate brain development, homeostasis and disease
ECM stiffness influences GBM invasion by facilitating the binding between CD44 and hyaluronic acid, which results in pro-migration signaling downstream and also influences the binding of integrins to their ECM substrates (Kim and Kumar, 2014; Knupfer et al., 1999). Enhanced ECM stiffness also drives GBM cell proliferation and a phenotype reminiscent of EMT, which further enhances GBM invasion (Ulrich et al., 2009; Cancer Genome Atlas Research et al., 2015). 

**
-"We have identified high-level expression of stem cell factors such as BMI1, SOX2, and nestin in primary DIPG tumors (2)."
-"In light of the high level of stem cell marker expression in DIPG, we hypothesized that DIPG would also have significant NOTCH pathway activity. NOTCH is a stem cell pathway critical for the development of the nervous system that also is implicated in multiple cancer types (21). "
-Ok but since they are not  contact snesitive wtf ?


**β1 integrins activate a MAPK signalling pathway in neural stem cells that contributes to their maintenance
-These experiments showed that, for spheres prepared from either rat or mouse, cells expressing high levels of β1 were present only on the edge of the sphere (Fig. 3A,B).
-Just as in the intact developing CNS, we found that laminin α2 was present in the region containing the cells that express high levels of β1 (Fig. 4A,D). Laminin 1, by contrast, was expressed in the central regions of the sphere (Fig. 4C), and fibronectin was diffusely localised in a speckled pattern (not shown) similar to that seen in the intact CNS. 
-when spheres grown in EGF alone were sectioned and immunolabelled with anti-EGF
receptor antibodies, they showed low but distinct labelling at the edges of the spheres in the region shown above to contain the EGF receptor-expressing cells (Fig. 5A). 

**Epidermal growth factor receptor in glioblastoma
Les glioblastome surexpriment EGFR et ça promeut la croissance et l'invasion et supprime l'apoptose fig 1

**Effects of fibronectin and laminin on structural, mechanical and transport properties of 3D collageneous network
- For laminin (LM) and fibronectin (FN)-based semi-interpenetrating polymer networks (semi-IPNs), fibronectin and laminin-1 (Sigma) were added to collagen solution (1.2 mg/ml) before fibrillogenesis was initiated. The final concentrations of fibronectin and laminin were 10, 50, 100 μg/ml.
- diffusion coefficient of Dextran (50 kDa) collagen (Laminin/fibronectin +) gel is between  50\% and 80\% of the value in water and increases with the value of  laminin/fibronectin

**A Hydrogel Platform that Incorporates Laminin Isoforms for Efficient Presentation of Growth Factors – Neural Growth and Osteogenesis
-Currently, the most commonly used ECM mimetic material is Matrigel, an undefined mix of proteins and GFs derived from Engelbreth–Holm–Swarm mouse sarcoma cells that provide excellent biological functionality.[2

**Engineering growth factors for regenerative medicine applications
-Protein stability and half-life
A major limitation of natural growth factors is their short effective half-life due to poor stability or fast blood clearance. 
-As further demonstration of this concept, an EGF variant containing a Y13G mutation reduces EGFR binding affinity by 50-fold, but exhibits greater potency in vitro due to decreased receptor downregulation and reduced ligand depletion [26]
- Hence, to increase growth factor efficacy, mutations can be introduced into a ligand to drive its receptor dissociation once the internalized complex reaches the endosomal compartment. This concept is based on TGFα, which naturally promotes EGFR recycling and is therefore a more potent mitogen than EGF [52, 53]. 

-Glycosaminoglycans (GAGs), including heparan sulfate-GAGs (HS-GAGs), are main components of the ECM that serve to effectively immobilize growth factors [66]. 

- In addition to participating in growth factor sequestration, matrix proteins also contain cell-adhesion sites, in particular, integrin-binding domains. The spatial proximity between growth factor and integrin binding sites can lead to the formation of molecular clusters at the cell surface, which are able to strongly modulate growth factor signaling

**Biomaterials for Sequestration of Growth Factors and Modulation of Cell Behavior
-For example, the half-life of
vascular endothelial growth factor (VEGF) after intravenous injection is only 50 minutes, while that
of FGF-2 is merely 3 minutes.[5] 
-Glycosaminoglycans (GAGs – e.g. heparin), their proteoglycan derivatives and glycoproteins (e.g.
fibronectin, laminin) are key mediators in this process. This is mainly driven by the negatively
charged sulfate groups widely present on their sugar monomers, which can establish strong
electrostatic interactions with many soluble factors.[23
-A good example of the complexity of regulatory processes is the different ways in which cells
respond to soluble vs. immobilized GFs, possibly derived from different interactions between these
molecules and cell surface receptors. While soluble factors are recognized and internalized by cells
as a substrate-receptor complex, immobilized factors cannot be endocytosed. This results in a
sustained intracellular signaling and, therefore, in a distinct phenomenon termed “artificial juxtacrine
signaling.”[26] 


**Influence of Hyaluronic Acid Transitions in Tumor Microenvironment on Glioblastoma Malignancy and Invasive Behavior
-he brain ECM has minimal fibrillary structures and is mainly composed of hyaluronic acid (HA, also called hyaluronan, or hyaluronate) (Bonneh-Barkay and Wiley, 2009; Sivakumar et al., 2017).
-Metabolic Activity of GBM39 PDX Cells Cultured in GelMA Hydrogels Is Sensitive to the Molecular Weight of Matrix Bound HA
-The Molecular Weight of Matrix-Bound HA Significantly Affects Invasion
-ere, we highlight the impact of matrix-bound HA MW on GBM cell malignancy. Cells cultured in hydrogels containing 500 kDa matrix-immobilized HA, with controlled physical properties, showed less invasive potential than those in hydrogels containing matrix immobilized 10 or 60 kDa HA.

**Hedgehog-responsive candidate cell of origin for diffuse intrinsic pontine glioma
-"We found that all cells in the culture expressed the intermediate filament proteins Nestin, GFAP, and Vimentin"
-Cells do seem to be moving within the sphere or at least have a rather weird geometry for non-migrating cells

**The emerging role of NG2 in pediatric diffuse intrinsic pontine glioma
- does not seem to report any laminin...
\end{document}