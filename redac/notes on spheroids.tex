\documentclass[11pt,a4paper]{article}
%\usepackage[utf8]{inputenc}
%\usepackage[ascii]{inputenc}
\usepackage{geometry}
\usepackage[dvipsnames]{xcolor}
\usepackage{textcomp}
\usepackage{graphicx}
\usepackage{caption}
\usepackage{subcaption}
\usepackage{amsmath}
\usepackage{tikz}

\begin{document}
**Comparison of cancer cells in 2D vs 3D culture reveals differences in AKT–mTOR–S6K signaling and drug responses
-All cell lines formed spheroids within 24–48 h after seeding.Spheroid morphology varied from a compact appearance (DLD-1, HT29) and less-condensed spheroids with smooth (HCT116) or irregular surfaces (LS174T) to loose aggregates (SW620) and adenomatous cell clusters (Caco-2) (Fig. 1A). 
-A significant reduction (−50\%) of cells in S-phase in 3D culture compared to those in 2D culture was observed in HT29, HCT116, Caco-2 and DLD-1 cells.
A small reduction (−20\%, Fig. 1C) in the number of cells in S-phase was observed in LS174T and SW620 cells.

**Reductive carboxylation supports redox homeostasis during anchorage-independent growth
-"Cells within spheroids proliferated at a reduced rate (Extended Data Fig.2a). Although growth in both conditions required glucose and glutamine (Extended Data Fig.2b), spheroids consumed less of both and secreted less lactate, glutamate and ammonia (Extended Data Fig.2c,d). The ratio of ammonia released to glutamine consumed was comparable between conditions (Extended Data Fig.2d)."
3 nmol/mg/h en monolayer vs 1.75 nmol/mg/h
\end{document}