\documentclass[11pt,a4paper]{article}
\usepackage[utf8]{inputenc}
\usepackage{geometry}
\usepackage[dvipsnames]{xcolor}



\begin{document}

\tableofcontents
\addcontentsline{toc}{section}{Unnumbered Section}
\section*{Introduction}
This document is supposed to be a bibliography focusing specifically on agent-based models and their use in the field of tumor growth or more general tumor modelling. If found, references on agent-base models used outside of cancer research may also be included. Contrary to the previous document, this one will focus primarily on recent papers.

\section{General Bibliography}
A broad research without clear selection in time and or subject matter. More focus brought on growth question but a generally open reading.

\section*{Quantitative agent-based modeling reveals mechanical stress response of growing tumor spheroids is predictable over various growth conditions and cell lines, Van Liedekerke et al., BioRxiv preprint,2018}
\subsection*{Abstract}
Model simulations indicate that the response of growing cell populations on mechanical stress follows the same functional relationship and is predictable over different cell lines and growth conditions despite the response curves look largely different. We develop a hybrid model strategy in which cells are represented by coarse-grained individual units calibrated with a high resolution cell model and parameterized measurable biophysical and cell-biological parameters. Cell cycle progression in our model is controlled by volumetric strain, the latter being derived from a bio-mechanical relation between applied pressure and cell compressibility. After parameter calibration from experiments with mouse colon carcinoma cells growing against the resistance of an elastic alginate capsule, the model adequately predicts the growth curve in i) soft and rigid capsules, ii) in different experimental conditions where the mechanical stress is generated by osmosis via a high molecular weight dextran solution, and iii) for other cell types with varying doubling times. Our model simulation results suggest that the growth response of cell population upon externally applied mechanical stress is the same, as it can be quantitatively predicted using the same growth progression function.

\subsection*{Abstract notes}
\begin{itemize}
\item Very long paper but potentially VERY informative
\item Read:  "Simulating tissue mechanics with Agent Based Models: concepts and perspectives" From same author
\item abstract read : This model focuses more specifically on mechanics
\end{itemize}

\section*{A quantitative high-resolution computational mechanics cell model for growing and regenerating tissues, Van Liedekerke et al., Biomechanics and Modeling in Mechanobiology ,2019}
\subsection*{Abstract}
mathematical models are increasingly designed to guide experiments in biology, biotechnology, as well as to assist in medical decision making. They are in particular important to understand emergent collective cell behavior. For this purpose, the models, despite still abstractions of reality, need to be quantitative in all aspects relevant for the question of interest. This paper considers as showcase example the regeneration of liver after drug-induced depletion of hepatocytes, in which the surviving and dividing hepatocytes must squeeze in between the blood vessels of a network to refill the emerged lesions. Here, the cells’ response to mechanical stress might significantly impact the regeneration process. We present a 3D high-resolution cell-based model integrating information from measurements in order to obtain a refined and quantitative understanding of the impact of cell-biomechanical effects on the closure of drug-induced lesions in liver. Our model represents each cell individually and is constructed by a discrete, physically scalable network of viscoelastic elements, capable of mimicking realistic cell deformation and supplying information at subcellular scales. The cells have the capability to migrate, grow, and divide, and the nature and parameters of their mechanical elements can be inferred from comparisons with optical stretcher experiments. Due to triangulation of the cell surface, interactions of cells with arbitrarily shaped (triangulated) structures such as blood vessels can be captured naturally. Comparing our simulations with those of so-called center-based models, in which cells have a largely rigid shape and forces are exerted between cell centers, we find that the migration forces a cell needs to exert on its environment to close a tissue lesion, is much smaller than predicted by center-based models. To stress generality of the approach, the liver simulations were complemented by monolayer and multicellular spheroid growth simulations. In summary, our model can give quantitative insight in many tissue organization processes, permits hypothesis testing in silico, and guide experiments in situations in which cell mechanics is considered important

\subsection*{Abstract notes}
\begin{itemize}
\item abstract read : model focused specifically on liver tissue. "capable of supplying information at subcellular scales".
\item different to center based models... \textbf{interesting to see how that works focus on the cell mechanics and modelling}
\item "quantitative insight into in many tissue organization problems" may be of used when we try to assess the impact of the chip
\end{itemize}

\subsection*{Full article notes}
\begin{itemize}
\item "The cells were approximated as spheres (the shape a cell adopts in isolation), while the forces between them are simulated as forces between the cell centers, which is why these models are often termed “center-based model” (CBM)."
\item \textbf{Center-based models useful for cell organisation processes}
\item "extensive simulated sensitivity analyses had been performed varying each parameter of the model within its physiological range"
\item "A large category of models called lattice-free, force-based “deformable cell models” (DCMs)
has been developed to meet these needs(the shortcomings of CBM)"
\item "Our cell model builds upon earlier work by Odenthal et al. (2013) and Van Liedekerke et al. (2019), a DCM type whereby the cell surface is triangulated and the nodes are connected by viscoelastic elements, representing the cell membrane and actin cortical cytoskeleton and homogeneous cytoplasm."
\item The triangulation method is interesting an in case more mechanical questions are treated this could be very useful. However, Reading is so far stopped at section 2 (07/11/2022 08:00) 
\end{itemize}



\section*{An agent-based model for drug-radiation interactions in the tumour microenvironment: Hypoxia-activated prodrug SN30000 in multicellular tumour spheroids, Xinjian Mao et al., PLoS Comp. Biol., 2018}
\subsection*{Abstract}
Multicellular tumour spheroids capture many characteristics of human tumour microenvironments, including hypoxia, and represent an experimentally tractable in vitro model for studying interactions between radiotherapy and anticancer drugs. However, interpreting spheroid data is challenging because of limited ability to observe cell fate within spheroids dynamically. To overcome this limitation, we have developed a hybrid continuum/agent-based model (ABM) for HCT116 tumour spheroids, parameterised using experimental models (monolayers and multilayers) in which reaction and diffusion can be measured directly. In the ABM, cell fate is simulated as a function of local oxygen, glucose and drug concentrations, determined by solving diffusion equations and intracellular reactions. The model is lattice-based, with cells occupying discrete locations on a 3D grid embedded within a coarser grid that encompasses the culture medium; separate solvers are employed for each grid. The generated concentration fields account for depletion in the medium and specify concentration-time profiles within the spheroid. Cell growth and survival are determined by intracellular oxygen and glucose concentrations, the latter based on direct measurement of glucose diffusion/reaction (in multilayers) for the first time. The ABM reproduces known features of spheroids including overall growth rate, its oxygen and glucose dependence, peripheral cell proliferation, central hypoxia and necrosis. We extended the ABM to describe in detail the hypoxia-dependent interaction between ionising radiation and a hypoxia-activated prodrug (SN30000), again using experimentally determined parameters; the model accurately simulated clonogenic cell killing in spheroids, while inclusion of reversible cell cycle delay was required to account for the marked spheroid growth delay after combined radiation and SN30000. This ABM of spheroid growth and response exemplifies the utility of integrating computational and experimental tools for investigating radiation/drug interactions, and highlights the critical importance of understanding oxygen, glucose and drug concentration gradients in interpreting activity of therapeutic agents in spheroid models.

\subsection*{Notes}
\begin{itemize}
\item abstract read 
\item  "Hybrid continuum/agent-base model for HCT116 tumour spheroids", "monolayers and multilayers"
\item I'll read this one in full. Questions really close to what Fabrizio adresses.
\end{itemize}

\subsection*{Full article notes}
\begin{itemize}
\item "In vitro three-dimensional (3D) cell cultures, including multicellular tumour spheroids and
multicellular layers (MCLs), capture many features of real tumours, and have advantages over
monolayer cell culture for developing drugs."
\item "MCLs are ideal models for quantifying diffusion and its coupling with reaction in the tumour microenvironment (e.g.prodrug metabolism)." explains the chip
\item continuum models were used at the beginning but were missing key features...
\item "Agent-based models (ABM) for the growth of tumour spheroids can be roughly classified as on-lattice (this includes the cellular Potts model and approaches often called “cellular automata”) or lattice-free, according to how space is treated."
\item "There are several excellent reviews of tumour modelling, discussing the wide range of meth-
ods and assumptions adopted: [14–23]."
\item "Very few ABM studies have simulated the action of HAPs or their combination with radia-
tion.
\item "The model was first calibrated for growth of HCT116 spheroids, based on measurements of diameter, cell number, viable cell fraction, hypoxic fraction and S-phase fraction of cells, employing flow cytometry and histology."
\item \textbf{The MCL (or on-chip) experiements are VERY useful to capture the diffusive part of the physics} (eed to understand why)
\item "Inside the cell rates of change in constituent concentrations of nutrients and drugs are determined by the balance of the trans-membrane transport rates and reaction rates described by ODEs." Similar to Fabrizio
\item "In the extracellular domain,[...] concentration fields are determined by solving PDEs for diffusion, with cells acting as sinks for nutrients and sinks for drugs and sources of drug metabolites."
\item  In this model, no cell migration other than division
\item "A cell divides when its volume reaches a specified threshold, giving rise two cells each with half the volume." Compare with Fabrizio algorithm
\item "Oxygen concentration is specified at the upper boundary of the domain by the gas phase oxygen level."
\item They model the whole growth configuration in the fluid 
\item "Parameters were fitted using the monolayer ABM and Matlab MCL program as described in methods." Look this up as we are more likely modelling a MCL
\item \textbf{How does the cell vary in size ? Is it the result of calculation based on concentrations ?} $\rightarrow$ "Oxygen and glucose consumption determines the cell growth rate."
\item L-glucose is metabolised, D-glucose is not. Therefore, they conducted diffusion with L-glucose
\item Why is the cell volume fraction 0.5 ?
\item The part on SN30000 will be read later... But overall, good results on pure growth.
\item "Powathil models (PMs) simulate a 2D slice of a real tumour, our model simulates a complete 3D tumour
spheroid growing in a tissue culture plate, and as a consequence the boundary conditions are
very different."
\item "Needing to simulate cell cycle to account for phase-dependent effects, the PMs employ a system of six ODEs, but in the SABM there is no model for cell cycle, time to divide depending simply on growth rate, a function of oxygen and glucose concentrations." so they do not account for cell phases... Except I saw it basically everywhere before (especially when response to radiation counts)
\item few adjustments needed, that's a plus.
\item  They are currently implementing cell cycle
\item Methods section to keep for future reference.
\item \textbf{Check the most recent review(s)}
\end{itemize}

\section*{Bystander Effects of Hypoxia-Activated Prodrugs: Agent-Based Modeling Using Three Dimensional Cell Cultures,C. Chong et al., Frontiers in Pharmacology, 2018}
\subsection*{Abstract}
Intra-tumor heterogeneity represents a major barrier to anti-cancer therapies. One strategy to minimize this limitation relies on bystander effects via diffusion of cytotoxins from targeted cells. Hypoxia-activated prodrugs (HAPs) have the potential to exploit hypoxia in this way, but robust methods for measuring bystander effects are lacking. The objective of this study is to develop experimental models (monolayer, multilayer, and multicellular spheroid co-cultures) comprising ‘activator’ cells with high expression of prodrug-activating reductases and reductase-deficient ‘target’ cells, and to couple these with agent-based models (ABMs) that describe diffusion and reaction of prodrugs and their active metabolites, and killing probability for each cell. HCT116 cells were engineered as activators by overexpressing P450 oxidoreductase (POR) and as targets by knockout of POR, with fluorescent protein and antibiotic resistance markers to enable their quantitation in co-cultures. We investigated two HAPs with very different pharmacology: SN30000 is metabolized to DNA-breaking free radicals under hypoxia, while the dinitrobenzamide PR104A generates DNA-crosslinking nitrogen mustard metabolites. In anoxic spheroid co-cultures, increasing the proportion of activator cells decreased killing of both activators and targets by SN30000. An ABM parameterized by measuring SN30000 cytotoxicity in monolayers and diffusion-reaction in multilayers accurately predicted SN30000 activity in spheroids, demonstrating the lack of bystander effects and that rapid metabolic consumption of SN30000 inhibited prodrug penetration. In contrast, killing of targets by PR104A in anoxic spheroids was markedly increased by activators, demonstrating that a bystander effect more than compensates any penetration limitation. However, the ABM based on the well-studied hydroxylamine and amine metabolites of PR104A did not fit the cell survival data, indicating a need to reassess its cellular pharmacology. Characterization of extracellular metabolites of PR104A in anoxic cultures identified more stable, lipophilic, activated dichloro mustards with greater tissue diffusion distances. Including these metabolites explicitly in the ABM provided a good description of activator and target cell killing by PR104A in spheroids. This study represents the most direct demonstration of a hypoxic bystander effect for PR104A to date, and demonstrates the power of combining mathematical modeling of pharmacokinetics/pharmacodynamics with multicellular culture models to dissect bystander effects of targeted drug carriers.

\subsection*{Notes}
\begin{itemize}
\item Predates and seems to complete previous paper. Will be read if not enough is found in the previous one
\end{itemize}

\section*{An on-lattice agent-based Monte Carlo model simulating the growth kinetics of multicellular tumor spheroids, S. Ruiz-Arrebola et al., Physica Media, 2020}
\subsection*{Abstract}
\subsubsection*{Purpose}
To develop an on-lattice agent-based model describing the growth of multicellular tumor spheroids using simple Monte Carlo tools.

\subsubsection*{Methods}
Cells are situated on the vertices of a cubic grid. Different cell states (proliferative, hypoxic or dead) and cell evolution rules, driven by 10 parameters, and the effects of the culture medium are included. About twenty spheroids of MCF-7 human breast cancer were cultivated and the experimental data were used for tuning the model parameters.

\subsubsection*{Results}
Simulated spheroids showed adequate sizes of the necrotic nuclei and of the hypoxic and proliferative cell phases as a function of the growth time, mimicking the overall characteristics of the experimental spheroids. The relation between the radii of the necrotic nucleus and the whole spheroid obtained in the simulations was similar to the experimental one and the number of cells, as a function of the spheroid volume, was well reproduced. The statistical variability of the Monte Carlo model described the whole volume range observed for the experimental spheroids. Assuming that the model parameters vary within Gaussian distributions it was obtained a sample of spheroids that reproduced much better the experimental findings.

\subsubsection*{Conclusions}
The model developed allows describing the growth of in vitro multicellular spheroids and the experimental variability can be well reproduced. Its flexibility permits to vary both the agents involved and the rules that govern the spheroid growth. More general situations, such as, e. g., tumor vascularization, radiotherapy effects on solid tumors, or the validity of the tumor growth mathematical models can be studied.

\subsection*{Abstract notes}
\begin{itemize}
\item "Monte Carlo tools" so same solver as Fabrizio ?
\item No delaunay voronoi tesselation. try to figure out why 
\item find the 10 parameters
\end{itemize}

\subsection*{Notes}
\begin{itemize}
\item "[...], at distances between 50 and 250 \textmu m from the spheroid surface, cells are in a hypoxic, or even necrotic, state due to nutrients and/or oxygen deficits"
\item "Within this group one finds the cellular automata (CA) and the agent-based models (ABM)."
\item "we choose an on-lattice ABM because it permitted us to study MTS with sizes of the order of a few mm or larger and, in addition, include diffusion processes of nutrients and O 2 in an easy way, without the need of
a hybrid approach." so their model is not hybrid...$\rightarrow$ \textbf{Check what a hybrid model is...}
\item "it is decided by random sampling if peripheral cells disappear or not from the MTS via exfoliation;" What is the rate ?  How was it determined ?
\item program written in Fortran 77 (very old...)
\item "The maximum number of proliferative ( L$_p$ ) and hypoxic ( L$_h$ ) layers are input data of the simulation: these values are maintained throughout the whole simulation." Once again I don't understand where this hypothesis comes from...
\item So a cell that divides does not necessarily spawn its daughter next to itself but somewhere else in the layer... I guess that is not necessarily problematic, if unintuitive... It's used as a time saving measure.
\item "In our simulation, each iteration corresponds to a complete cellular cycle and to permit the comparison with the experiment we assumedthat it elapsed 24 h."
\item Apparently all parameters that  I found weird were taken from experiments...
\item \textbf{Another important aspect is how they deal with timescales}
\item "To [compare experimental and simulated variablities] the former was shifted in time until the average volume of the experimental MTS in the last day of measurement and the average volume of the simulated MTS in a given instant of their growth coincided." I wish the shifting was better justified... but I guess I understand
\item They captured the necrotic core growth properly with their function $g$
\item \textbf{for each model it is important to note how they model but also what they capture}
\item They need to include parameter variability beyond the random variable already there to achieve enough variability
\item This paper carries a pretty reductionist approach... not necessarily wrong for the questions it adresses.
\end{itemize}

\section*{Mathematical modelling reveals cellular dynamics within tumour spheroids, Joshua A. Bull et al., \textbf{PLoS Comp. Biol.}, 2020}
\subsection*{Abstract}
Tumour spheroids are widely used as an in vitro assay for characterising the dynamics and response to treatment of different cancer cell lines. Their popularity is largely due to the reproducible manner in which spheroids grow: the diffusion of nutrients and oxygen from the surrounding culture medium, and their consumption by tumour cells, causes proliferation to be localised at the spheroid boundary. As the spheroid grows, cells at the spheroid centre may become hypoxic and die, forming a necrotic core. The pressure created by the localisation of tumour cell proliferation and death generates an cellular flow of tumour cells from the spheroid rim towards its core. Experiments by Dorie et al. showed that this flow causes inert microspheres to infiltrate into tumour spheroids via advection from the spheroid surface, by adding microbeads to the surface of tumour spheroids and observing the distribution over time. We use an off-lattice hybrid agent-based model to re-assess these experiments and establish the extent to which the spatio-temporal data generated by microspheres can be used to infer kinetic parameters associated with the tumour spheroids that they infiltrate. Variation in these parameters, such as the rate of tumour cell proliferation or sensitivity to hypoxia, can produce spheroids with similar bulk growth dynamics but differing internal compositions (the proportion of the tumour which is proliferating, hypoxic/quiescent and necrotic/nutrient-deficient). We use this model to show that the types of experiment conducted by Dorie et al. could be used to infer spheroid composition and parameters associated with tumour cell lines such as their sensitivity to hypoxia or average rate of proliferation, and note that these observations cannot be conducted within previous continuum models of microbead infiltration into tumour spheroids as they rely on resolving the trajectories of individual microbeads.

\subsection*{Abstract notes}
\begin{itemize}
\item Cellular flow from rim towards spheroid core 
\item "off-lattice hybrid agent-based model" find out what this means
\end{itemize}

\subsection*{Notes}
\begin{itemize}
\item " it is common in mathematical models of tumour spheroids to simplify these complex metabolic processes while retaining the qualitative behaviour of the experimental observations."
\item "We refer the interested reader to [31] for a comparison of five ABMs (CA, cellular Potts models, overlapping spheres[32], Voronoi tesselation [33] and vertex-based methods)." \textbf{check these references}
\item So they focus on mechanics to explain bead flow
\item "We note that while this model can simulate spheroid growth in three dimensions, here we restrict
attention to 2D simulations to reduce computational time."
\item "We assume that oxygen is maintained at a constant level in the culture medium surrounding the tumour spheroid and, hence, by continuity that the oxygen concentration on the spheroid boundary is also maintained at this
constant value, $\omega_{\infty}$."
\item "We first show that our ABM qualitatively reproduces the dynamics and changing spatial struc-
ture that characterises spheroid growth in vitro."
\item "When performing parameter sensitivity analyses, three model parameters were varied: the average cell cycle
length, $\tau$, the oxygen threshold for quiescence, $\omega_{q}$, and the oxygen threshold for hypoxia, $\omega_{h}$"
\item \textbf{ "By appropriately tuning the proliferation rate and oxygen thresholds at which cells become quiescent or die, we can generate synthetic spheroids of the same equilibrium size which possess different internal compositions. This suggests that spheroid composition cannot be predicted by observing the overall growth dynamics alone."}
\item \textbf{" Identifying the relationships between models of tumour spheroid growth implemented in 2D or 3D, or implemented within different software frameworks, remains an open problem."}
\item The cellular flow is induced by pressure gradients due to cell death and proliferation
\end{itemize}

\section*{Comparing individual-based approaches to modelling the self-organization of multicellular tissues, James M. Osborne t al., \textbf{PLoS Comp. Biol.}, 2017}
\subsection*{Abstract}
The coordinated behaviour of populations of cells plays a central role in tissue growth and renewal. Cells react to their microenvironment by modulating processes such as movement, growth and proliferation, and signalling. Alongside experimental studies, computational models offer a useful means by which to investigate these processes. To this end a variety of cell-based modelling approaches have been developed, ranging from lattice-based cellular automata to lattice-free models that treat cells as point-like particles or extended shapes. However, it remains unclear how these approaches compare when applied to the same biological problem, and what differences in behaviour are due to different model assumptions and abstractions. Here, we exploit the availability of an implementation of five popular cell-based modelling approaches within a consistent computational framework, Chaste (http://www.cs.ox.ac.uk/chaste). This framework allows one to easily change constitutive assumptions within these models. In each case we provide full details of all technical aspects of our model implementations. We compare model implementations using four case studies, chosen to reflect the key cellular processes of proliferation, adhesion, and short- and long-range signalling. These case studies demonstrate the applicability of each model and provide a guide for model usage.

\subsection*{Notes}
\begin{itemize}
\item "In the latter (Voronoi Tesselation),the shape of each cell is defined to be the set of points in space that are nearer to the centre of the cell than the centres of any other cell; a Delaunay triangulation is performed to connect those cell centres that share a common face, thus determining the neighbours of each cell [9] (Fig 1(d))." Used by Fabrizio
\item \textbf{check ref 10 to 15}
\item  "Aegerter-Wilmsen et al. coupled a vertex model of cell proliferation and rearrangement with a differential algebraic equation model for a protein regulatory network to describe the interplay between mechanics and signalling in regulating tissue size in the Drosophila wing imaginal disk.[17]" \textbf{Check to  see if cell-cell signalling is included}
\item  Adhesion : I assume he picked values that \underline{should} lead to engulfment and he notes that it ends up not being the case for the OS model. Fluctuation also play a large role.
\item in many cases differentiation is often either directly linked to a concentration level or simply a rule
\item \underline{Short range signalling} : "This example demonstrates how intercellular signalling may be incorporated within each cell-based model."
\item They model Notch signalling by an equation
\item \underline{Long-range signalling} : As our final case study, we simulate the growth of an epithelial tissue in which cell proliferation is coupled to the level of a diffusible morphogen. 
\item they link the division probability to the morphogen concentration 
\item integrating the gradient to the ABM is not easy
\item \textbf{The question for us will be what can kind of signalling should we integrate}
\item Voronoi Tesselation is highly appropriate for proliferation and signalling
\item \textbf{\underline{Perspective}: Cross platform modelling, 3D models comparison, or 2D/3D comparison}
\end{itemize}

\section*{PhysiCell: An open source physics-based cell simulator for 3-D multicellular systems;A. Ghaffarizadeh et al., \textbf{PLoS Comp. Biol.}, 2018}
\subsection*{Abstract}
Many multicellular systems problems can only be understood by studying how cells move, grow, divide, interact, and die. Tissue-scale dynamics emerge from systems of many interacting cells as they respond to and influence their microenvironment. The ideal “virtual laboratory” for such multicellular systems simulates both the biochemical microenvironment (the “stage”) and many mechanically and biochemically interacting cells (the “players” upon the stage). PhysiCell—physics-based multicellular simulator—is an open source agent-based simulator that provides both the stage and the players for studying many interacting cells in dynamic tissue microenvironments. It builds upon a multi-substrate biotransport solver to link cell phenotype to multiple diffusing substrates and signaling factors. It includes biologically-driven sub-models for cell cycling, apoptosis, necrosis, solid and fluid volume changes, mechanics, and motility “out of the box.” The C++ code has minimal dependencies, making it simple to maintain and deploy across platforms. PhysiCell has been parallelized with OpenMP, and its performance scales linearly with the number of cells. Simulations up to 105-106 cells are feasible on quad-core desktop workstations; larger simulations are attainable on single HPC compute nodes. We demonstrate PhysiCell by simulating the impact of necrotic core biomechanics, 3-D geometry, and stochasticity on the dynamics of hanging drop tumor spheroids and ductal carcinoma in situ (DCIS) of the breast. We demonstrate stochastic motility, chemical and contact-based interaction of multiple cell types, and the extensibility of PhysiCell with examples in synthetic multicellular systems (a “cellular cargo delivery” system, with application to anti-cancer treatments), cancer heterogeneity, and cancer immunology. PhysiCell is a powerful multicellular systems simulator that will be continually improved with new capabilities and performance improvements. It also represents a significant independent code base for replicating results from other simulation platforms. The PhysiCell source code, examples, documentation, and support are available under the BSD license at http://PhysiCell.MathCancer.org and http://PhysiCell.sf.net.

\subsection*{Notes}
\begin{itemize}
\item Seems to compare stochastic and deterministic models
\item They coded the BioFVM biotransport platform  an now they complete it an off-lattice ABM
\item \textbf{He does a general presentation of the tools available at that point $\rightarrow$ Could be very useful for later}
\item  "[The sequential approach to PDE] approach is not expected to efficiently scale to 3-D simulations with many diffusible factors—a key requirement in reconciling secretomics with single-cell and multicellular systems biology, particularly as we work to understand cell-cell communication involving many cell-secreted factors."
\item "Following our work in [14–16], we include basic models of gradual cell volume changes, rather than static cell volumes. This avoids non-physical “jumps” in cell velocity following cell division events: the sudden localized doubling of cell density causes cells to overlap, leading to large, temporary, and non-physical “repulsive” forces that can manifest as non-physical “tears” in simulated tissues." Interesting for handling of division
\item "Through BioFVM, PhysiCell can couple cell phenotype to many diffusible substrates. It is the only simulation package to explicitly model the cell’s fluid content—a key aspect in problems such as cryobiology "
\item "PhysiCell takes advantage of this by using three separate time step sizes ($\Delta t_{diff}$, ($\Delta t_{mech}$, and ($\Delta t_{cells}$). In particular, the cell phenotypes and arrangement (operating on slow timescales) can be treated as quasi-static when advancing the solution to the biotransport PDEs, so BioFVM can be called without modification with the cell arrangements fixed."
\item They parallelise everything except cell interactions data.
\item "converged with first-order accuracy" find what it means (check numerical recipes)
\item "In both models, the innermost portion of the necrotic core developed a network of fluid voids or cracks. This phenomenon emerges from competing biophysical effects of the multicellular system and its cell-scale mechanical details: necrotic cells lose volume, even as they continue to adhere, leading to the formation of cracks. To our knowledge, this is the only model that has predicted this necrotic tumor microarchitecture, which would be very difficult to simulate by continuum methods except with very high-resolution meshes comparable to the * 1 to 10 \textmu m feature size. These cracked necrotic core structures have been observed with in vitro hanging drop spheroids (e.g., [5, 38, 39]). See Fig 2."
\textmu \textbf{Immune response: They modelled immunogenicity, then they modelled immune cells as biorobots increasing apoptosis probability}
\item Physicell not great for morphogenetic mechanism but it's ok.
\item check the "system Biology Marker language" it is used on COPASI Complex Pathway Simulator and might be used in the future on PhysiCell
\end{itemize}

\section*{Simulating Cancer Growth with Multiscale Agent-Based Modeling, Wang et al., \textit{Semin Cancer Biol}, 2016}
\subsection*{Abstract}
There have been many techniques developed in recent years to in silico model a variety of cancer behaviors. Agent-based modeling is a specific discrete-based hybrid modeling approach that allows simulating the role of diversity in cell populations as well as within each individual cell; it has therefore become a powerful modeling method widely used by computational cancer researchers. Many aspects of tumor morphology including phenotype-changing mutations, the adaptation to microenvironment, the process of angiogenesis, the influence of extracellular matrix, reactions to chemotherapy or surgical intervention, the effects of oxygen and nutrient availability, and metastasis and invasion of healthy tissues have been incorporated and investigated in agent-based models. In this review, we introduce some of the most recent agent-based models that have provided insight into the understanding of cancer growth and invasion, spanning multiple biological scales in time and space, and we further describe several experimentally testable hypotheses generated by those models. We also discuss some of the current challenges of multiscale agent-based cancer models.

\subsection*{Notes}
\begin{itemize}
\item \underline{Stroma}: "Stromal tissue is primarily made of extracellular matrix containing connective tissue cells. Extracellular matrix is primarily composed of ground substance - a porous, hydrated gel, made mainly from proteoglycan aggregates - and connective tissue fibers. There are three types of fibers commonly found within the stroma: collagen type I, elastic, and reticular (collagen type III) fibres." wikipedia 
\item Maybe read ref.9
\item \underline{Molecular signalling}: "This cross-talk pathway was described using a system of 26 ODEs solved simultaneously to determine signaling molecule concentrations at each time step for each individual cell. "
\item "Accumulation of H+ lowers the pH in the tumor; H+ leaves the tumor through diffusion to the vasculature and is then removed with other cellular waste products, or it infiltrates into the surrounding tissues. In turn, lowering pH in surrounding tissue can acidify the ECM and lead to collagen degradation, and has been shown to increase tumor invasiveness into the acidified regions [35]."
\item \textbf{Gatenby et al. have extensively studied the effects of glycolytic metabolism in the presence
of hyperplasia and increased acid resistance in cells through a hybrid ABM [37, 38]." Good potential read on the effect of pH.}
\item "Low oxygen supply favors a transition to anaerobic metabolism, which leads to higher glucose consumption and increased H+ levels"
\item "Oxygen levels are known to be much lower in tumors than healthy tissues, with levels dropping as low as 5\%-30\% in the necrotic center [42]."
\item "Gerlee and Anderson have investigated how the evolutionary dynamics of tumor growth respond to the microenvironment, using the output of an intracellular neural network to determine cell phenotype [40, 43, 44]."
\item "Transition to the carcinogenic phenotype is a complex process, involving changes in focal adhesion kinase,
integrin composition and expression, as well as degradation of ECM through the production
of specific enzymes [53]."
\textbf{\item "In this model, cell growth and apoptosis were determined by external forces experienced by the cell, where tension forces increased the likelihood of cellular proliferation and compression increased the likelihood of apoptosis. This model demonstrated the biologically observed behavior that a genetic change is not necessary for cancerous behavior in cells, and that some cancer cells transplanted into a healthy tissue environment can revert back to a healthy, normal phenotype."}
\item \underline{Important for radioresistance}:" CSCs are resistant to chemotherapy, and are therefore thought
of as a potential cause of tumor recurrence and metastasis after therapy [63]. Recently, the role of CSCs in resistance to chemotherapy has been investigated experimentally in more detail [64, 65]."
\item "Their[Enderling] simulations showed that tumors formed without unlimited replicative potential cancer cells will inevitably die out; hence, CSCs are necessary for tumor initiationand continued malignant growth."
\item "Other than the heterogeneous multiscale method [87] and the equation-free approach [88] that have
been discussed elsewhere, Lowengrub et al. have been developing a new theoretical upscaling framework based on dynamic density functional theory (DDFT) [89] to analyze and quantify the complexity associated with correlations in heterogeneous tumor tissues." Depending on our goal and constraints this might be interesting....
\end{itemize}

\section*{Agent Based Modelling and Simulation tools: A review of the state-of-art software, Abar et al., \textit{Computer Science Review}, 2017}
\subsection*{Abstract}
The key intent of this work is to present a comprehensive comparative literature survey of the state-of-art in software agent-based computing technology and its incorporation within the modelling and simulation domain. The original contribution of this survey is two-fold: (1) Present a concise characterization of almost the entire spectrum of agent-based modelling and simulation tools, thereby highlighting the salient features, merits, and shortcomings of such multi-faceted application software; this article covers eighty five agent-based toolkits that may assist the system designers and developers with common tasks, such as constructing agent-based models and portraying the real-time simulation outputs in tabular/graphical formats and visual recordings. (2) Provide a usable reference that aids engineers, researchers, learners and academicians in readily selecting an appropriate agent-based modelling and simulation toolkit for designing and developing their system models and prototypes, cognizant of both their expertise and those requirements of their application domain. In a nutshell, a significant synthesis of Agent Based Modelling and Simulation (ABMS) resources has been performed in this review that stimulates further investigation into this topic.

\subsection*{Notes}
\begin{itemize}
\item A bit too general for now... 27/11/2022
\end{itemize}

\section{"Recent" Bibliography}
This section focuses more on paper that were released in the 5 past years dealing with ABM in the field of biology. All scales and systems may be covered. The method was to check the first 100 reference of the last 5 years on pubmed and select the most appealing title. The most relevant and interesting abstract are kept and read

\section*{Computational Modeling of Vascular Adaptation: A Systems Biology Approach Using Agent-Based Models, Corti A. et al., Front Bioeng Biotechnol., 2021}
\subsection*{Abstract}
The widespread incidence of cardiovascular diseases and associated mortality and morbidity, along with the advent of powerful computational resources, have fostered an extensive research in computational modeling of vascular pathophysiology field and promoted in-silico models as a support for biomedical research. Given the multiscale nature of biological systems, the integration of phenomena at different spatial and temporal scales has emerged to be essential in capturing mechanobiological mechanisms underlying vascular adaptation processes. In this regard, agent-based models have demonstrated to successfully embed the systems biology principles and capture the emergent behavior of cellular systems under different pathophysiological conditions. Furthermore, through their modular structure, agent-based models are suitable to be integrated with continuum-based models within a multiscale framework that can link the molecular pathways to the cell and tissue levels. This can allow improving existing therapies and/or developing new therapeutic strategies. The present review examines the multiscale computational frameworks of vascular adaptation with an emphasis on the integration of agent-based approaches with continuum models to describe vascular pathophysiology in a systems biology perspective. The state-of-the-art highlights the current gaps and limitations in the field, thus shedding light on new areas to be explored that may become the future research focus. The inclusion of molecular intracellular pathways (e.g., genomics or proteomics) within the multiscale agent-based modeling frameworks will certainly provide a great contribution to the promising personalized medicine. Efforts will be also needed to address the challenges encountered for the verification, uncertainty quantification, calibration and validation of these multiscale frameworks.

\section*{Using machine learning as a surrogate model for agent-based simulations, Angione C et al., PLoS One., 2022}
\subsection*{Abstract}
In this proof-of-concept work, we evaluate the performance of multiple machine-learning methods as surrogate models for use in the analysis of agent-based models (ABMs). Analysing agent-based modelling outputs can be challenging, as the relationships between input parameters can be non-linear or even chaotic even in relatively simple models, and each model run can require significant CPU time. Surrogate modelling, in which a statistical model of the ABM is constructed to facilitate detailed model analyses, has been proposed as an alternative to computationally costly Monte Carlo methods. Here we compare multiple machine-learning methods for ABM surrogate modelling in order to determine the approaches best suited as a surrogate for modelling the complex behaviour of ABMs. Our results suggest that, in most scenarios, artificial neural networks (ANNs) and gradient-boosted trees outperform Gaussian process surrogates, currently the most commonly used method for the surrogate modelling of complex computational models. ANNs produced the most accurate model replications in scenarios with high numbers of model runs, although training times were longer than the other methods. We propose that agent-based modelling would benefit from using machine-learning methods for surrogate modelling, as this can facilitate more robust sensitivity analyses for the models while also reducing CPU time consumption when calibrating and analysing the simulation.

\subsection*{Notes}
Might prove very interesting if I need to start studies that require a lot of simulations.

\section*{An overview of agent-based models in plant biology and ecology. Zhang B. et al. Ann Bot. 2020}
\subsection*{Abstract}
Agent-based modelling (ABM) has become an established methodology in many areas of biology, ranging from the cellular to the ecological population and community levels. In plant science, two different scales have predominated in their use of ABM. One is the scale of populations and communities, through the modelling of collections of agents representing individual plants, interacting with each other and with the environment. The other is the scale of the individual plant, through the modelling, by functional–structural plant models (FSPMs), of agents representing plant building blocks, or metamers, to describe the development of plant architecture and functions within individual plants. The purpose of this review is to show key results and parallels in ABM for growth, mortality, carbon allocation, competition and reproduction across the scales from the plant organ to populations and communities on a range of spatial scales to the whole landscape. Several areas of application of ABMs are reviewed, showing that some issues are addressed by both population-level ABMs and FSPMs. Continued increase in the relevance of ABM to environmental science and management will be helped by greater integration of ABMs across these two scales.

\subsection*{Notes}
I'll probably read this one out of curiosity more than for deeper knowledge. May still give some interesting ideas to branch out.

\section*{PhysiBoSS: a multi-scale agent-based modelling framework integrating physical dimension and cell signalling., Letort G. et al. Bioinformatics., 2019}
\subsection*{Abstract}
\subsubsection*{Motivation}
Due to the complexity and heterogeneity of multicellular biological systems, mathematical models that take into account cell signalling, cell population behaviour and the extracellular environment are particularly helpful. We present PhysiBoSS, an open source software which combines intracellular signalling using Boolean modelling (MaBoSS) and multicellular behaviour using agent-based modelling (PhysiCell).

\subsubsection*{Results}
PhysiBoSS provides a flexible and computationally efficient framework to explore the effect of environmental and genetic alterations of individual cells at the population level, bridging the critical gap from single-cell genotype to single-cell phenotype and emergent multicellular behaviour. PhysiBoSS thus becomes very useful when studying heterogeneous population response to treatment, mutation effects, different modes of invasion or isomorphic morphogenesis events. To concretely illustrate a potential use of PhysiBoSS, we studied heterogeneous cell fate decisions in response to TNF treatment. We explored the effect of different treatments and the behaviour of several resistant mutants. We highlighted the importance of spatial information on the population dynamics by considering the effect of competition for resources like oxygen.

\subsubsection*{Availability and implementation}
PhysiBoSS is freely available on GitHub (https://github.com/sysbio-curie/PhysiBoSS), with a Docker image (https://hub.docker.com/r/gletort/physiboss/). It is distributed as open source under the BSD 3-clause license.

\subsubsection*{Supplementary information}
Supplementary data are available at Bioinformatics online 

\section*{Computational modelling unveils how epiblast remodelling and positioning rely on trophectoderm morphogenesis during mouse implantation, Dokmegang et al., PLoS One., 2021}
\subsection*{Abstract}
Understanding the processes by which the mammalian embryo implants in the maternal uterus is a long-standing challenge in embryology. New insights into this morphogenetic event could be of great importance in helping, for example, to reduce human infertility. During implantation the blastocyst, composed of epiblast, trophectoderm and primitive endoderm, undergoes significant remodelling from an oval ball to an egg cylinder. A main feature of this transformation is symmetry breaking and reshaping of the epiblast into a “cup”. Based on previous studies, we hypothesise that this event is the result of mechanical constraints originating from the trophectoderm, which is also significantly transformed during this process. In order to investigate this hypothesis we propose MG\# (MechanoGenetic Sharp), an original computational model of biomechanics able to reproduce key cell shape changes and tissue level behaviours in silico. With this model, we simulate epiblast and trophectoderm morphogenesis during implantation. First, our results uphold experimental findings that repulsion at the apical surface of the epiblast is essential to drive lumenogenesis. Then, we provide new theoretical evidence that trophectoderm morphogenesis indeed can dictate the cup shape of the epiblast and fosters its movement towards the uterine tissue. Our results offer novel mechanical insights into mouse peri-implantation and highlight the usefulness of agent-based modelling methods in the study of embryogenesis.

\section*{Multiscale Agent-Based and Hybrid Modeling of the Tumor Immune Microenvironment, Norton et al., Processes (Basel). 2019}
\subsection*{Abstract}
Multiscale systems biology and systems pharmacology are powerful methodologies that are playing increasingly important roles in understanding the fundamental mechanisms of biological phenomena and in clinical applications. In this review, we summarize the state of the art in the applications of agent-based models (ABM) and hybrid modeling to the tumor immune microenvironment and cancer immune response, including immunotherapy. Heterogeneity is a hallmark of cancer; tumor heterogeneity at the molecular, cellular, and tissue scales is a major determinant of metastasis, drug resistance, and low response rate to molecular targeted therapies and immunotherapies. Agent-based modeling is an effective methodology to obtain and understand quantitative characteristics of these processes and to propose clinical solutions aimed at overcoming the current obstacles in cancer treatment. We review models focusing on intra-tumor heterogeneity, particularly on interactions between cancer cells and stromal cells, including immune cells, the role of tumor-associated vasculature in the immune response, immune-related tumor mechanobiology, and cancer immunotherapy. We discuss the role of digital pathology in parameterizing and validating spatial computational models and potential applications to therapeutics.

\section*{Predictive landscapes hidden beneath biological cellular automata, Koopmans et al., \textit{J. Biol Phys.}, 2021}
\subsection*{Abstract}
To celebrate Hans Frauenfelder’s achievements, we examine energy(-like) “landscapes” for complex living systems. Energy landscapes summarize all possible dynamics of some physical systems. Energy(-like) landscapes can explain some biomolecular processes, including gene expression and, as Frauenfelder showed, protein folding. But energy-like landscapes and existing frameworks like statistical mechanics seem impractical for describing many living systems. Difficulties stem from living systems being high dimensional, nonlinear, and governed by many, tightly coupled constituents that are noisy. The predominant modeling approach is devising differential equations that are tailored to each living system. This ad hoc approach faces the notorious “parameter problem”: models have numerous nonlinear, mathematical functions with unknown parameter values, even for describing just a few intracellular processes. One cannot measure many intracellular parameters or can only measure them as snapshots in time. Another modeling approach uses cellular automata to represent living systems as discrete dynamical systems with binary variables. Quantitative (Hamiltonian-based) rules can dictate cellular automata (e.g., Cellular Potts Model). But numerous biological features, in current practice, are qualitatively described rather than quantitatively (e.g., gene is (highly) expressed or not (highly) expressed). Cellular automata governed by verbal rules are useful representations for living systems and can mitigate the parameter problem. However, they can yield complex dynamics that are difficult to understand because the automata-governing rules are not quantitative and much of the existing mathematical tools and theorems apply to continuous but not discrete dynamical systems. Recent studies found ways to overcome this challenge. These studies either discovered or suggest an existence of predictive “landscapes” whose shapes are described by Lyapunov functions and yield “equations of motion” for a “pseudo-particle.” The pseudo-particle represents the entire cellular lattice and moves on the landscape, thereby giving a low-dimensional representation of the cellular automata dynamics. We outline this promising modeling strategy.

\subsection*{Notes}
\begin{itemize}
\item "The chemical reaction network theory makes general statements, such as the number and stability of equilibrium (fixed) points of a dynamical system that represents a network of chemical reactions."
\item "Several other approaches, without involving cellular automata, also mitigate the parameters problem while still using a system of nonlinear equations [10, 11]." check that maybe
\item "For example, a fixed point of a gene network dynamics can exhibit excitability whereby a small perturbation in expression level leads to an excursion far away from the fixed point in the phase space that lasts for a long time — the initial perturbation does not exponentially decay over time — before the system (gene expression levels) returns to the fixed point [13, 14]. Similar phenomena also arise in physical oscillators and laser dynamics. "
\item "Between the two challenges — the lack of any obvious symmetry and being out of thermal equilibrium — researchers have made progress on the latter by exploiting the idea of quasi-steady states. We turn to this in the next section."
\item \underline{Lyapunov functions:} in the theory of ordinary differential equations (ODEs), Lyapunov functions, named after Aleksandr Lyapunov, are scalar functions that may be used to prove the stability of an equilibrium of an ODE. Lyapunov functions (also called Lyapunov’s second method for stability) are important to stability theory of dynamical systems and control theory. (wikipedia)
\item "in CPM, adjacent cells mechanically interact by exerting force on one another, through the cell–cell contacts. A Hamiltonian captures the cell–cell interaction energies in a simplified and insightful way. The CPM then evolves over time with the rule that the Hamiltonian must be non-increasing over time, leading to a final configuration of the tissue in which the Hamiltonian takes on a locally minimal value."
\end{itemize}

\section*{Cellular Automata Modeling of Stem‐Cell‐Driven Development of Tissue in the Nervous System, Lehotzky et al., Dev. Neurobiol., 2019}
\subsection*{Abstract}
Mathematical and computational modeling enables biologists to integrate data from observations and experiments into a theoretical framework. In this review, we describe how developmental processes associated with stem‐cell‐driven growth of tissue in both the embryonic and adult nervous system can be modeled using cellular automata. A cellular automaton is defined by its discrete nature in time, space, and state. The discrete space is represented by a uniform grid or lattice containing agents that interact with other agents within their local neighborhood. This possibility of local interactions of agents makes the cellular‐automata approach particularly well suited for studying through modeling how complex patterns at the tissue level emerge from fundamental developmental processes (such as proliferation, migration, differentiation, and death) at the single‐cell level. As part of this review, we provide a primer for how to define biologically inspired rules governing these processes so that they can be implemented into a cellular automata model. We then demonstrate the power of the cellular automata approach by presenting simulations (in the form of figures and movies) based on building models of three developmental systems: the formation of the enteric nervous system through invasion of neural crest cells; the growth of normal and tumorous neurospheres induced by proliferation of adult neural stem/progenitor cells; and the neural fate specification through lateral inhibition of embryonic stem cells in the neurogenic region of Drosophila. 

\subsection*{Notes}
\begin{itemize}
\item "We believe that this low number is, at least in part, due to the lack of familiarity of most biologists with the CA-modeling approach, and the difficulty of mathematicians and systems engineers to extract from biological data the parameters and rules necessary to build biologically meaningful CA models."
\item "this lattice is two-dimensional and consists most commonly of squares, but sometimes of hexagonal elements." Seem to believe it's only 2D models
\item "In CA models, such a mechanism can be implemented by assuming that a cell is able to move to each of its empty neighboring grid positions with equal probability, but it cannot move to a grid position that is already occupied by another cell (Fig. 3)." Interesting to see how they model migration with very simple arguments
\end{itemize}

\section*{A Multipurpose In Situ Adenocarcinoma Simulation Model with Cellular Automata and Parallel Processing ,Antonio J. Tomeu et al., IEEE Latin America Transctions, 2020}
\subsection*{Abstract}
Adenocarcinomas are tumors that originate in the lining epithelium of the ducts that form the endocrine glands of the human body. In addition to the enormous morbidity and mortality that they imply for patients, the economic impact of the treatments is very high. The early detection of the disease when it has not yet acquired infiltrating character is therefore of great interest. Among the many different tools that contribute to it, computational simulation is one more that is increasingly consolidated over the traditional ones. The objective pursued by the research is to have fast and efficient computations that allow simulating the dynamics of the tumour in situ under a wide range of different parameters. This work proposes a simulation model that can be generalized to the most frequent types of in situ adenocarcinomas (CIS) based on cellular automata, applies it to the study of the case of ductal adenocarcinoma in situ of the breast, parallels them, and studies the acceleration achieved.

\subsection*{Notes}
\begin{itemize}
\item "Carcinogenesis is a phenomenon in which one or multiple mutations in certain genes allow cells to reproduce and survive abnormally, under a selection process that leads to uncontrolled tumor growth of an infiltrating nature."
\item "The genetic load of a cell is then modelled by an ordered tuple of the form GC=(brca1, brca2, pten, tp53)"
\item" We see that in silico simulation is compatible with biological reality as well as the histological relaity, with an acceptable degree of fidelity to the global dynamics of neoplastic growth in situ. " Wonder how they measured the experimental ones...

\end{itemize}

\section*{Influence of nutrient availability and quorum sensing on the formation of metabolically inactive1 microcolonies within structurally heterogeneous bacterial biofilms: An individual-based 3D cellular automata model, Lakshmi Machineni et al., Bull. Math Biol, 2017}
\subsection*{Abstract}
The resistance of bacterial biofilms to antibiotic treatment has been attributed to the emergence of structurally heterogeneous microenvironments containing metabolically inactive cell populations. In this study, we use a three-dimensional individual-based cellular automata model to investigate the influence of nutrient availability and quorum sensing on microbial heterogeneity in growing biofilms. Mature biofilms exhibited at least three structurally distinct strata: a high-volume, homogeneous region sandwiched between two compact sections of high heterogeneity. Cell death occurred preferentially in layers in close proximity to the substratum, resulting in increased heterogeneity in this section of the biofilm; the thickness and heterogeneity of this lowermost layer increased with time, ultimately leading to sloughing. The model predicted the formation of metabolically dormant cellular microniches embedded within faster-growing cell clusters. Biofilms utilizing quorum sensing were more heterogeneous compared to their non-quorum sensing counterparts, and resisted sloughing, featuring a cell-devoid layer of EPS atop the substratum upon which the remainder of the biofilm developed. Overall, our study provides a computational framework to analyze metabolic diversity and heterogeneity of biofilm-associated microorganisms and may pave the way toward gaining further insights into the biophysical mechanisms of antibiotic resistance.

\subsection*{Notes}
\begin{itemize}
\item "Quorum sensing (QS) is a mechanism of intercellular communication used to collectively28
coordinate group behaviors based on population density [32-35]. This process relies on the production, release, and group-wise detection of signal molecules called autoinducers (e.g. acyl-homoserine lactones in Gram-negative bacteria) which rapidly diffuse in the liquid phase and across cell populations, and accumulate in the biofilm over time. "
\item "Here, we present a prototype individual-based 3D computational cellular automata model to simulate biofilm growth, and quantify heterogeneity as a function of growth phase, space, and time."
\item "The model incorporates the following processes: nutrient diffusion,reaction, and convection; biomass growth kinetics, cell division, death, and dispersal; autoinducer production, and transport; and EPS production"
\item "Each of the bacterial cells that lie between the mother cell and the closest bacterium-free element is then shifted by one grid element – away from the mother cell, and towards the empty element – creating a bacterium-free element in the Moore neighborhood of the mother cell."
\end{itemize}

\section*{A review of spatial computational models for multi-cellular systems, with regard to intestinal crypts and colorectal cancer development, G. De Matteis et al., 2013, J. Math. Biol}
\subsection*{Abstract}
Colon rectal cancers (CRC) are the result of sequences of mutations which lead the intestinal tissue to develop in a carcinoma following a "progression" of observable phenotypes. The actual modeling and simulation of the key biological structures involved in this process is of interest to biologists and physicians and, at the same time, it poses significant challenges from the mathematics and computer science viewpoints. In this report we give an overview of some mathematical models for cell sorting (a basic phenomenon that underlies several dynamical processes in an organism), intestinal crypt dynamics and related problems and open questions. In particular, major attention is devoted to the survey of so-called in-lattice (or grid) models and off-lattice (off-grid) models. The current work is the groundwork for future research on semi-automated hypotheses formation and testing about the behavior of the various actors taking part in the adenoma-carcinoma progression, from regulatory processes to cell-cell signaling pathways. 

\subsection*{Notes}
\begin{itemize}
\item Nothing really new in this one compared to Osborne
\end{itemize}

\section*{Modeling and Computer Simulation of Viscoelastic Crypt Deformation, E. P. Oliveira et al., Trends in Computational and Applied Mathematics, 2022}
\subsection*{Abstract}
Colorectal cancer morphogenesis begins at the cellular level from cell mutations in the in-
testinal epithelium cavities called crypts. These mutations lead to a pressure difference in the epithelium
crypt walls, which can cause deformation and generate visible abnormalities in the epithelium. The geo-
metrical modeling of these crypts and the mathematical modeling of the biomechanical process that leads
to deformations can be simulated by using a Finite Element Method. The method solves numerically the
system of partial differential equations (PDEs) that governs this phenomenon and permits to estimate the
deformations of the crypt walls. In this work we simulate the crypt deformation when the cell mutations
appear in several regions of the crypt epithelium.

\subsection*{notes}
\begin{itemize}
\item Not an agent based model : "The choice to preferring a spatial continuum model with respect to a cell-based model is in the possibility of extend its application to simulate cell dynamics in millions of crypts that
can be done using a multiscale strategy,"
\item A read for later but not very useful now
\end{itemize}

\section*{Experimental and modeling study of the formation of cell aggregates with differential substrate adhesion Adenis L. et al., PLoS One, 2020}
\subsection*{Abstract}
The study of cell aggregation in vitro has a tremendous importance these days. In cancer biology, aggregates and spheroids serve as model systems and are considered as pseudo-tumors that are more realistic than 2D cell cultures. Recently, in the context of brain tumors (gliomas), we developed a new poly(ethylene glycol) (PEG)-based hydrogel, with adhesive properties that can be controlled by the addition of poly(L-lysine) (PLL), and a stiffness close to the brain's. This substrate allows the motion of individual cells and the formation of cell aggregates (within one day), and we showed that on a non-adhesive substrate (PEG without PLL is inert for cells), the aggregates are bigger and less numerous than on an adhesive substrate (with PLL). In this article, we present new experimental results on the follow-up of the formation of aggregates on our hydrogels, from the early stages (individual cells) to the late stages (aggregate compaction), in order to compare, for two cell lines (F98 and U87), the aggregation process on the adhesive and non-adhesive substrates. We first show that a spaceless model of perikinetic aggregation can reproduce the experimental evolution of the number of aggregates, but not of the mean area of the aggregates. We thus develop a minimal off-lattice agent-based model, with a few simple rules reproducing the main processes that are at stack during aggregation. Our spatial model can reproduce very well the experimental temporal evolution of both the number of aggregates and their mean area, on adhesive and non-adhesive soft gels and for the two different cell lines. From the fit of the experimental data, we were able to infer the quantitative values of the speed of motion of each cell line, its rate of proliferation in aggregates and its ability to organize in 3D. We also found qualitative differences between the two cell lines regarding the ability of aggregates to compact. These parameters could be inferred for any cell line, and correlated with clinical properties such as aggressiveness and invasiveness. 

\subsection*{Notes}
\begin{itemize}
\item "we developed a minimal off-lattice agent-based model, whose rules are defined in order to reproduce the important phenomena that drive the behavior of cell assemblies: cell and aggregate motion, cell-cell adhesion, cell proliferation and aggregate compaction"
\item "Each agent is a cell, modeled by a disk. The disk radius is the same for all cells in a given simulation and its value stays constant during the simulation. " + "In order to model the fact that cells are deformable and can be
organized in three dimensions in aggregates, cells are allowed to partially superimpose in the model." $\rightarrow$ overlapping disk model
\item " To model all these stages, without describing precisely the shape of the cells, we estimated that a cellular Potts model was less adapted to our problem, compared to a classical agent-based cellular automaton. We introduced four rules: the motion rule (for individual cells, for cells inside an aggregate and for aggregates, in the presence or not of a flux), the superimposition rule, the proliferation rule and the compaction rule." except no
\end{itemize}

\section*{Modelling collective cell motion: are on- and off-lattice models equivalent?, Josué Manik Nava-Sedeño et al., Philos Trans R Soc Lond B Biol Sci. ,2020}
\subsection*{Abstract}
Biological processes, such as embryonic development, wound repair and cancer invasion, or bacterial swarming and fruiting body formation, involve collective motion of cells as a coordinated group. Collective cell motion of eukaryotic cells often includes interactions that result in polar alignment of cell velocities, while bacterial patterns typically show features of apolar velocity alignment. For analysing the population-scale effects of these different alignment mechanisms, various on- and off-lattice agent-based models have been introduced. However, discriminating model-specific artefacts from general features of collective cell motion is challenging. In this work, we focus on equivalence criteria at the population level to compare on- and off-lattice models. In particular, we define prototypic off- and on-lattice models of polar and apolar alignment, and show how to obtain an on-lattice from an off-lattice model of velocity alignment. By characterizing the behaviour and dynamical description of collective migration models at the macroscopic level, we suggest the type of phase transitions and possible patterns in the approximative macroscopic partial differential equation descriptions as informative equivalence criteria between on- and off-lattice models. This article is part of the theme issue 'Multi-scale analysis and modelling of collective migration in biological systems'. 

\section*{Pulling in models of cell migration, George Chappelle et al., Phys Rev E, 2019}
\subsection*{Abstract}
There are numerous biological scenarios in which populations of cells migrate in crowded environments. Typical examples include wound healing, cancer growth, and embryo development. In these crowded environments cells are able to interact with each other in a variety of ways. These include excluded-volume interactions, adhesion, repulsion, cell signaling, pushing, and pulling. One popular way to understand the behavior of a group of interacting cells is through an agent-based mathematical model. A typical aim of modellers using such representations is to elucidate how the microscopic interactions at the cell-level impact on the macroscopic behavior of the population. At the very least, such models typically incorporate volume-exclusion. The more complex cell-cell interactions listed above have also been incorporated into such models; all apart from cell-cell pulling. In this paper we consider this under-represented cell-cell interaction, in which an active cell is able to "pull" a nearby neighbor as it moves. We incorporate a variety of potential cell-cell pulling mechanisms into on- and off-lattice agent-based volume exclusion models of cell movement. For each of these agent-based models we derive a continuum partial differential equation which describes the evolution of the cells at a population level. We study the agreement between the agent-based models and the continuum, population-based models and compare and contrast a range of agent-based models (accounting for the different pulling mechanisms) with each other. We find generally good agreement between the agent-based models and the corresponding continuum models that worsens as the agent-based models become more complex. Interestingly, we observe that the partial differential equations that we derive differ significantly, depending on whether they were derived from on- or off-lattice agent-based models of pulling. This hints that it is important to employ the appropriate agent-based model when representing pulling cell-cell interactions. 

\end{document}