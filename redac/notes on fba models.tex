\documentclass[11pt,a4paper]{article}
%\usepackage[utf8]{inputenc}
%\usepackage[ascii]{inputenc}
\usepackage{geometry}
\usepackage[dvipsnames]{xcolor}
\usepackage{textcomp}
\usepackage{graphicx}
\usepackage{caption}
\usepackage{subcaption}
\usepackage{amsmath}

\begin{document}
**A metabolic core model elucidates how enhanced utilization of glucose and glutamine with enhanced glutainme-dependent lacate production promotes cancer cell growth: the WarburQ effect
-we simulated the response of 50,000 wirings to a modulation of either the glucose or glutamine uptake flux, while the maximal oxygen consumption rate (i.e., oxygen availability) was kept constant across all the experiments.
-"indicating that exploitation of reduc-
tive carboxylation of glutamine supports cell growth."
- Ils ont PAVE l'espace des flux possibles en regardant divers outputs dont la production de biomasse et ils ont éliminé possibilité qui collaient pas.
- "The model also correctly predicted that the lactate/glucose ratio exceeds 2 at high glutamine (Fig 3A), indicating that some glutamine is converted to lactate."
- "Our prediction of a decrease in succinate level did not match the experimental result (Fig 3C). The inaccurate prediction for this metabolite may be due to intrinsic limits of FBA, which is not able to capture regulatory effects [45,46], such as the responsiveness of Complex II (succinate dehydrogenase) respiration flux to changes in the ATP/ADP ratio (see S2 Text for a more accurate but complex tentative prediction)."

**Multi-scale computational study of the Warburg effect, reverse Warburg effect and glutamine addiction in solid tumors
-"In the reverse Warburg hypothesis, oxidative tumor cells
have been observed to uptake lactate as a carbon source in addition to glucose (Fig 1A, 2)
[22,23,51]."

**Metabolic requirements for cancer cell proliferation

A LIREs
\end{document}