\documentclass[11pt,a4paper]{article}
%\usepackage[utf8]{inputenc}
%\usepackage[ascii]{inputenc}
\usepackage{geometry}
\usepackage[dvipsnames]{xcolor}
\usepackage{textcomp}
\usepackage{graphicx}
\usepackage{caption}
\usepackage{subcaption}
\usepackage{amsmath}

\begin{document}
**Glucose Metabolism Heterogeneity in Human and Mouse Malignant Glioma Cell Lines
-he D-54MG and GL261 glioma cell lines displayed an oxidative phosphorylation (OXPHOS)-dependent phenotype, characterized by extremely long survival under glucose starvation, and low tolerance to poisoning of the electron transport chain (ETC). Alternatively, U-251MG and U-87MG glioma cells exhibited a glycolytic-dependent phenotype with functional OXPHOS. These cells displayed low tolerance to glucose starvation and were resistant to a ETC blocker. Moreover, these cells could be rescued in low glucose conditions by oxidative substrates (e.g., lactate, pyruvate). Finally, these two phenotypes could be distinguished by the differential expression of LDH isoforms. OXPHOS-dependent cells expressed both LDH-A and -B isoforms whereas glycolytic-dependent glioma cells expressed only LDH-B. In the latter case, LDH-B would be expected to be essential for the use of extracellular lactate to fuel cell activities. 
- Check the metabolome and proteomics of the DIPG article

**Glucose transporter Glut1 controls diffuse invasion phenotype with perineuronal satellitosis in diffuse glioma microenvironment


**Proteomics and metabolomics approach in adult and pediatric glioma diagnostics 
-Super complet et intéressant même si ça répond pas à ma question.

**Glioblastoma cells require glutamate dehydrogenase to survive impairments of glucose metabolism or Akt signaling
-Oncogenes influence nutrient metabolism and nutrient dependence. The oncogene c-Myc stimulates glutamine metabolism and renders cells dependent on glutamine to sustain viability (“glutamine addiction”), suggesting that treatments targeting glutamine metabolism might selectively kill c-Myc-transformed tumor cells.
-conso glutamine 100e9 mol/hr/1e6 cell pour les SF188

\end{document}