\documentclass[11pt,a4paper]{article}
%\usepackage[utf8]{inputenc}
%\usepackage[ascii]{inputenc}
\usepackage{geometry}
\usepackage[dvipsnames]{xcolor}
\usepackage{textcomp}
\usepackage{graphicx}
\usepackage{caption}
\usepackage{subcaption}
\usepackage{amsmath}

\begin{document}
**Glucose Metabolism Heterogeneity in Human and Mouse Malignant Glioma Cell Lines
-he D-54MG and GL261 glioma cell lines displayed an oxidative phosphorylation (OXPHOS)-dependent phenotype, characterized by extremely long survival under glucose starvation, and low tolerance to poisoning of the electron transport chain (ETC). Alternatively, U-251MG and U-87MG glioma cells exhibited a glycolytic-dependent phenotype with functional OXPHOS. These cells displayed low tolerance to glucose starvation and were resistant to a ETC blocker. Moreover, these cells could be rescued in low glucose conditions by oxidative substrates (e.g., lactate, pyruvate). Finally, these two phenotypes could be distinguished by the differential expression of LDH isoforms. OXPHOS-dependent cells expressed both LDH-A and -B isoforms whereas glycolytic-dependent glioma cells expressed only LDH-B. In the latter case, LDH-B would be expected to be essential for the use of extracellular lactate to fuel cell activities. 
- Check the metabolome and proteomics of the DIPG article



**Glucose transporter Glut1 controls diffuse invasion phenotype with perineuronal satellitosis in diffuse glioma microenvironment


**Proteomics and metabolomics approach in adult and pediatric glioma diagnostics 
-Super complet et intéressant même si ça répond pas à ma question.

**Glioblastoma cells require glutamate dehydrogenase to survive impairments of glucose metabolism or Akt signaling
-Oncogenes influence nutrient metabolism and nutrient dependence. The oncogene c-Myc stimulates glutamine metabolism and renders cells dependent on glutamine to sustain viability (“glutamine addiction”), suggesting that treatments targeting glutamine metabolism might selectively kill c-Myc-transformed tumor cells.
-conso glutamine 100e9 mol/hr/1e6 cell pour les SF188

**Starvation-dependent differential stress resistance protects normal but not cancer cells against high-dose chemotherapy
-"The normal physiological blood glucose level for both mice and humans is ≈1.0 g/liter but can reach 0.5 g/liter after starvation. Therefore, we tested the effect of normal glucose (1.0 g/liter), low glucose (0.5 g/liter), and high glucose (3.0 g/liter) on oxidative stress. All cell lines were grown until confluence to minimize proliferation and differences in proliferation between the primary and cancer cells and then switched to medium containing different glucose concentrations with 1\% serum. Low serum was used to minimize the addition of serum glucose, which is ≈1.0 g/liter. After a 24-h glucose treatment, cells were challenged with two different oxidants, H2O2 and menadione, for 24 h."
- suggère que pour 24h passer à 0.5 g/L de glucose hors hypoxie ne pose pas de problème dans les gliomes
-" Primary rat glial cells, rat glioma cell lines (C6, A10-85, RG2, and 9L), a human glioma cell line (LN229), and a human neuroblastoma cell line (SH-SY5Y) were tested for glucose restriction-induced DSR. Cells were incubated in low glucose (0.5 g/liter, STS), normal glucose (1.0 g/liter), or high glucose (3.0g/liter), supplemented with 1\% serum, for 24 h."
- C'est pas sur

**Induction of vascular endothelial growth factor expression by hypoxia and by glucose deficiency in multicell spheroids:Implications for tumor angiogenesis
-"After reaching 0.6-0.8 mm in diameter, spheroids were fixed and processed for in situ analyses. In situ analysis of cell proliferation was carried out by
visualizing cells that have incorporated BrdUrd into newly replicated DNA. Results showed that cell proliferation was mostly confined to the outermost three to five cell layers, while the remainder of viable tumor cells ('70 \%) were quiescent."

**The Interleukin-11/IL-11 Receptor Promotes Glioblastoma Survival and Invasion under Glucose-Starved Conditions through Enhanced Glutaminolysis
-"Since the brain microenvironment is rich in glutamine [2], glioblastoma cells can take advantage of glutamine catabolism (termed glutaminolysis) as an additional or alternative energy source especially when glycolytic energy production is low due to phases when glucose levels are depleted"
-"Thus, glucose and glutamine can compensate for each other to maintain TCA cycle function, promoting cell survival [11]."
-Chercher si c-MYC est lié aux mutations dans la DIPG-XIII
-"Metabolic shift or enhanced glutamine metabolism is believed to occur in response to oncogenes such as c-MYC and pro-inflammatory cytokines in glioblastoma"
-"High MYC expression is required for glutaminolysis and addiction to glutamine metabolism 

**Phosphorus-31 nuclear magnetic resonance of C6 glioma cells and rat astrocytes
-Although anaerobic glycolysis can provide energy to the cells in an efficient way, as demonstrated by the maintainance of the energy level under anoxia, aerobic metabolism
can provide the energy to the cells in the absence of glucose (and glycogen), probably by utilizing pyruvate or glutamine as carbon sources (the concentrations of which are 1 mM and 4 mM, respectively, in the culture medium). Under the most drastic starvation conditions (Dulbecco’s modified Eagle’s medium free of glucose, pyruvate and glutamine), the survival of C6 glioma cells with a good energy status at least duringan 8-h period, suggests the ultimate utilization, before death of the endogenous pool of fatty acids. These results illustrate the ability of the cells to modulate the activity of their metabolic pathways as a function of the substrate content of their
external medium.

**Autophagy mediates glucose starvation-induced glioblastoma cell quiescence and chemoresistance through coordinating cell metabolism, cell cycle, and survival
-"We set off to determine the cell cycle and metabolic states by flow cytometry analyses of Hoechst 33342 and Pyronin Y, a standard cell cycle analysis method capable to discern G1 and G0 states (Fig. 2a). With 2 days of treatment, while under normal growth condition (4.5 g/L glucose), only 18\% GBM cells exited cell cycle, glucose starvation induced 46\% to exit cell cycle and enter quiescence. Consistent with the cell cycle analysis, Ki67 staining showed a substantial decrease by 30\% (from 80 to 50\%) with the glucose starvation (Fig. 2b and Supplementary Figure 1a, *P < 0.05)
\end{document}