\documentclass[11pt,a4paper]{article}
%\usepackage[utf8]{inputenc}
%\usepackage[ascii]{inputenc}
\usepackage{geometry}
\usepackage[dvipsnames]{xcolor}
\usepackage{textcomp}
\usepackage{graphicx}
\usepackage{caption}
\usepackage{subcaption}
\usepackage{amsmath}

\begin{document}
**Mechanoresponsive metabolism in cancer cell migration and metastasis
-As such, metabolic signaling for migration and proliferation can be considered to be separate and distinct.
-"Estimates suggest that up to approximately 50\% of ATP is used to support the actin cytoskeleton (Bernstein, 2003; Daniel et al., 1986). Consequently, intracellular ATP:ADP ratio has been positively correlated with migration potential (Zanotelli et al., 2018)."
-"Migratory cells have been suggested to favor mitochondrial respiration for increased ATP production"
-"Glycolysis can help cells respond to fluctuating energetic demands in the membrane (Epstein et al., 2014), hence an increase in glucose uptake is often observed when the energy cost associated with migration increases rapidly (Zanotelli et al., 2019; Zhang et al., 2019). While glycolysis can synthesize ATP up to 100 times quicker than OXPHOS, the energetic yield is very low and presents a thermodynamic trade-off between rate and yield (Martinez-Outschoorn et al., 2017)."
- leader consume more ATP than followers (fig1)

**Cell cycle during neuronal migration and neocortical lamination \\
-"Furthermore, the migrating neurons in the neocortex were Cyclin D1- (G1 phase-specific marker) positive, suggesting that they were in the G1 phase."

**Notch controls the cell cycle to define leader versus follower identities during collective cell migration
- Fucci shows that migrating cells can be in G1 S or G2 and that leaders tend to be in S phase rather than G1 (for zebrafish trunk neural crest) (80\% in fig 10)


**Co-ordination of cell cycle, migration and stem cell-like activity in breast cancer
-"G0/1 Stem-like cells have increased migratory activity"
-"G0/1 cells breast cancer cells show increased mammosphere formation and migration"

**Cell migration in paediatric glioma; characterisation and potential therapeutic targeting
- "HSJD-DIPG-007, were first examined for their ability to form spheroids in culture. Similar to Vinci et al (2012), we noted that all three cell lines readily formed round dense spheroids within 24 h when cultured in low adherence 96-well round bottomed plates (Figure 1A)."
-"HSJD-DIPG-007 migrated by extending flattened protrusions and spreading in a sheet-like manner. "
-"SF188 cells (control 0.295±0.0018 μm min−1" (in spheroids)
-"In adult glioma models, LiCl has been shown to increase β-catenin reporter activity and β-catenin knockdown has been demonstrated to rescue the anti-migratory effects of BIO "
- beta catenin are low and are involved in cell contact

**Cell Cycle and Cell Migration 
-"Mitogenic stimulation of VSMCs initiates cell cycle progression and cell migration. However, cells in the late S or G2/M phase do not migrate.4 There is a window of opportunity in the G1 to G1/S transition where VSMCs are able to migrate in response to mitogenic stimuli"

**Live-cell time-lapse imaging and single-cell tracking of in vitro cultured neural stem cells – Tools for analyzing dynamics of cell cycle, migration, and lineage selection
-fig1. Oligodendroglial cell track speed 0.003 0.006 µm/s -> 0.180-0.360 µm/min

**Extracellular ATP and adenosine in tumor microenvironment: Roles in epithelial–mesenchymal transition, cell migration, and invasion 
-eATP may promote migration.


**Migrating oligodendrocyte progenitor cells swell prior to soma dislocation 
-"The migration of OPCs has already been investigated in detail by video time lapse microscopy. OPCs move in a saltatory manner alternating between a resting and a moving state with a mean velocity of 10 μm/h ± 7 μm/h and a maximum velocity of 120 μm/h to 140 μm/h (both on poly-L-lysine)48. It has been reported that both increases and decreases of the basal internal calcium level impair the migration of OPCs49 and that the migration correlates with internal calcium transients50. Furthermore, the golli proteins which regulate the migration of OPCs also regulate the expression of the TRP family member TRPC 151 which is proposed to be a component of store operated calcium channels52."


**Diabetes Mellitus-Related Neurobehavioral Deficits in Mice Are Associated With Oligodendrocyte Precursor Cell Dysfunction
-"Hyperglycemia Eliminated the Oligodendrocyte Precursor Cell Migration and Survival Ability"
-Diabetes studies but STILL


** Intracellular Signaling Mechanisms Directing Oligodendrocyte Precursor Cell Migration
-"By identifying key components of the intracellular signaling cascade triggered by PDGF, Miyamoto and colleagues (2008) have provided further insight into the processes governing OPC migration."
-"conclude that PDGF promotes migration by initiating a signaling cascade involving Fyn, Cdk5, and WAVE2. In their discussion, however, they also acknowledge that Fyn could independently stimulate the Rho GTPases to modulate the cytoskeletal changes leading to migration."
- Fyn is related to integrin therefore there needs to be matrix contact.

**Brainstem glioma: a review -> Nada
**Diffuse Intrinsic Pontine Glioma: Time for Cautious Optimism -> Nada

**TRIM11 is over-expressed in high-grade gliomas and promotes proliferation, invasion, migration and glial tumor growth
 -"These findings suggest that TRIM11 might be an indicator of glioma malignancy, and has an oncogenic function mediated through the EGFR signaling pathway."
 
**Post-transcriptional regulation of cytokine and growth factor signaling in cancer
la figure 2 montre bien que l'absence des growth factors va inhiber la migration

**Hypoxia enhances migration and invasion in glioblastoma by promoting a mesenchymal shift mediated by the HIF1α-ZEB1 axis
- The hypoxia-induced mesenchymal shift is associated with enhanced migration and invasion capacity of GBM cells.

**Hypoxia Can Induce Migration of Glioblastoma Cells Through a Methylation-Dependent Control of ODZ1 Gene Expression
Hypoxia triggers a complex tumor cell response that enables migration and invasion of the surrounding parenchyma through activation of multiple molecular pathways, as PI3K/Akt, Wnt/ß-catenin, Hedegehog, TGFß, and Tyrosine kinase receptors, among others (10–12).

**Hypoxia can induce c-Met expression in glioma cells and enhance
SF/HGF-induced cell migration
-The c-Met receptor and its ligand scatter factor/hepatocyte
growth factor (SF/HGF) are strongly overexpressed in malignant
gliomas.
-Signaling through c-Met as well as exposure to hypoxia
can stimulate glioma cell migration and invasion.

** Signaling through c-Met as well as exposure to hypoxia
can stimulate glioma cell migration and invasion.

**Cell migration in multicell spheroids: Swimming against the tide
- a priori bien une migration active.

**Unification of aggregate growth models by emergence from cellular and intracellular mechanisms 
- à garder pour comparaison mais pas le sujet ici.

**Cell migration in multicell spheroids: Swimming against the tide
-both bull and him state that the migration is mostly due to pressure gradients
- experimtent show they make appoximately 100 µm radial in 4 days -> approx 1 µm/hr
\end{document}