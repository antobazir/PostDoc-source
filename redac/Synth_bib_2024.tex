\documentclass[11pt,a4paper]{article}
%\usepackage[utf8]{inputenc}
%\usepackage[ascii]{inputenc}
\usepackage[margin=0.7in]{geometry}
%\usepackage{geometry}
\usepackage[dvipsnames]{xcolor}
\usepackage{textcomp}
\usepackage{graphicx}
\usepackage{caption}
\usepackage{subcaption}
\usepackage{amssymb}
\usepackage{amsmath}
\usepackage{tikz}

\begin{document}
\tableofcontents

\section{Introduction}
This document aims at digesting and organising the knowledge accumulated during literature review on various subjects related to cell metabolism. There are two mains goals in this task : 
\begin{enumerate}
\item Keep track of what has been read and potentially retrieve lost/forgotten important pieces of information
\item Organizing the information to make the writing of an article/synthesis faster in the future
\end{enumerate}
To this end relevant and linked pieces of information will be placed together in section with corresponding references.


\section{ATP and energy management}
\begin{itemize}
\item We know that prolif cancer cells are more glycoytic but have overall lower ATP reserve than healthy cells. Does it mean that they expanded the reserves and use glycolysis to maintain viability and prolif ? 
\item Cells appears to die of Necrosis if very ATP depleted or of Apoptosis if mildly depleted for long enough.  Where does quiescence fit in all this 
\item Is it achieve at a threshold similar or above apoptosis if Checkpoint is reached ? 
\item Overall ATP increase, meaning there is actually WAY MORE ADP than in healthy cells $\rightarrow$ cancer cells accumulate Phosphore in general
\item "These data suggest that when demand for NAD+ to support oxidation reactions exceeds the rate of ATP turnover in cells, NAD+ regeneration by mitochondrial respiration becomes constrained, promoting fermentation, despite available oxygen. This argues that cells engage in aerobic glycolysis when the demand for NAD+ is in excess of the demand for ATP."\cite{Luengo2020}
\item Puschel et al. show that the upregulation of chemokines and cytokines in absence of nutrients takes hours to occur at least 4 to 8 and more... \cite{Puschel2020}

\item "The viability of control cells and cells treated with siRNA for 72 h was analyzed. There was no significant impact of GLUT1 downregulation on the viability of FTC-133 cells growing in hypoglycemia, normoglycemia or hyperglycemia (Fig. 7A). Cells (8305c) treated with siRNA and growing in normoxic conditions irrespective of the glucose concentration showed decreased cell viability and proliferation by ~20\%. However, GLUT1 downregulation did not affect the viability of cells growing in hypoxia conditions."\cite{Jozwiak2014}
\end{itemize}
\textbf{[Need to find the references for this]}

\section{Quiescence, Cell death \& energy}
\begin{itemize}
\item Can it be assumed that the pathway for apoptosis is downregulated ? If yes does it mean that cells might continue to proliferate (or try to) even when ATP should not allow for it ? $\rightarrow$ yes I suppose
\item Quiescence is entirely possible in cancer cells. It seems to be triggered by the right set of conditions as in cell sensing nutrient depletion before proceeding to G1 leading to metabolic catastrophe.
\item What level triggers quiescence ? Is there a window of ATP production level that can induce quiescence ? $\rightarrow$ Quiescence can also be triggered by regulatory cues not linked to nutrient availability
\item How do we choose that level so that glutamine and autophagy can be acounted for ? $\rightarrow$ In the final model a single "effective" nutrient is used representing basically everything
\item Since autophagy is essentially catabolism, how does it come into play into preventing cells from going into necrosis. Can it be used as a crutch to lead cells into dormancy/apoptosis ?$\rightarrow$ I suppose but do not have evidence for that so far

\item "When mouse hematopoietic cells or lymphocytes were not dividing, they exhibited little glucose uptake, performed reduced amounts of glycolysis, secreted less lactate, and instead, relied on oxidative phosphorylation as their major source of energy [9]. When stimulated to divide in response to growth factors or cytokines, mouse hematopoietic cells and lymphocytes exhibited a surprisingly strong shift to increased glucose consumption and elevated rate of glycolysis [9]."\cite{Coller2019}

\item "When the oxygen supply is insufficient, tumor cells survive by entering the G0 phase and may return to the G1 phase when oxygen is replenished [10,15]. Cancer cell lines can be experimentally exposed to hypoxia [8,15,27,28] through the use of a hypoxia chamber [8,15,26,27,28] or treatment with CoCl2 (concentration ranges from 100 \textmu M to 500 \textmu M) [15]. The degree of hypoxia in hypoxia chamber is severe, typically 1\% oxygen [26,27,28] or 0.1\% oxygen [15] for 5 [26] to 14 days [27]. ."\cite{Nabil2021}

\item "Detachment from the substratum, low levels of nutrients, and hypoxia would all be expected to be characteristics of the tumor microenvironment."\cite{Valcourt2012}

\item  "We find that the duration of spontaneous quiescence in untransformed and cancer cells is heterogeneous and that a portion of this heterogeneity results from asynchronous proliferation-quiescence decisions in pairs of daughters after mitosis, where one daughter cell enters or remains in temporary quiescence while the other does not."\cite{Pulianmackal2021}

\item "it is also possible to achieve a state of cellular quiescence in which glucose uptake, glycolysis and flux through central carbon metabolism are not reduced."\cite{Valcourt2012}

\item "In 1974, Pardee provided evidence for a distinct quiescent state and demonstrated the existence of a restriction point (R-point) in G1 that determines cell fates: cells in G1 can become quiescent before the R-point but commit to enter a mitotic cell cycle after the R-point"\cite{Cheung2013}

\item " For instance, proliferating HSCs uptake three times more glucose than quiescent HSCs " \cite{Marescal2020}

\item "Indeed, in vitro models using aggregates of squamous carcinoma cells grown in a nutrient-deprived suspension, have been shown to arrest via growth factor-independent epidermal growth factor receptor (EGFR)-Y1086 autophosphorylation, which leads to reduced AKT signalling (Fig. 1b) and reduced CyclinD [35]."

\item it seeems that even in monolayers quiescence is not something that happens uniformly. and the percentage seems to be 40-60 \% in common cell lines.There also seems to be 10-15\% of quiescent cells even in full medium \cite{Wang022018}\cite{Hu2011}\cite{Nabil2021}\cite{Koshiji2004}

\item "contact-inhibited fibroblasts maintain comparable metabolic rates to proliferating fibroblasts despite not actively"proliferating. 220 Glucose uptake and lactate secretion rates were only 2-fold lower in contact-inhibited compared with proliferating fibroblasts."\cite{Valcourt2012}

\item "Early studies showed that lectin stimulation of lymphocytes led to increased glucose uptake, and an increased rate of glycolysis and pentose phosphate pathway (PPP) activities [6],[7]. More recent experiments have focused on a murine pro-B cell lymphoid cell line, FL5.12, that proliferates in response to the cytokine interleukin IL-3 [8]. IL-3 stimulation results in an 8-fold increased glycolytic flux. IL-3 also induces the cells to consume less oxygen per glucose consumed, and to excrete much more lactate, indicating a shift away from oxidative toward glycolytic metabolism. For human peripheral blood T lymphocytes, stimulation resulted in a 30-fold increase in glycolysis [9]; for thymocytes, the increase was 50-fold [10]. These differences in quiescent and proliferating lymphocytes have played a pivotal role in our understanding of the quiescent state, and experiments with lymphocytes as a model system have been important contributors to the development of the idea that quiescence is characterized by decreased metabolic activity."\cite{Lemons2010}

\item "This deviation can be attributed to the cell-type specific function of fibroblasts as primary synthesizers of extracellular matrix (ECM). Whereas most other quiescent cells lack biosynthetic function, fibroblasts must constantly secrete proteins and other molecules needed for ECM formation (Lemons et al., 2010)."\cite{Marescal2020}

\end{itemize}

\section{Necrosis}
\begin{itemize}
\item Two models for necrosis liquid and solid with calcification \cite{Thim2010}\cite{YuMi2017}
\item The cell death after tagging takes 24hrs in Mao's model and it's measured \cite{Mao2018}
\item Cell death removal rate is 0.2/day in Kempf first mode \cite{Kempf2005}
\item "The loss of tissue and cellular profile occurs within hours in liquefactive necrosis." \cite{Adigun2024}
\end{itemize} 

\section{Reactive Oxygen Species (ROS) \& Cell metabolism}
\begin{itemize}
\item "Excessive concentrations of ROS result in cell‐cycle arrest and apoptosis." \cite{Nakamura2021}
\item "The concentration of ROS in tumor tissues is typically higher compared with that recorded in surrounding normal tissues (23). In the TME with persistently high ROS levels, both immune and tumor cells are affected. The increase of ROS is one of the main causes of immunosuppression in the TME (106). "\cite{Liang2021}
\item it seems ROS can only be released by necrosis...
\item "ROS are deleterious (cell cycle arrest or cell death) at high concentrations but can trigger cell response at low concentrations" \cite{Arfin2021}
\item "Cancer cells downregulate OXPHOS even with oxygen. Accelerated metabolism $\rightarrow$ ROS (countered by NADPH also produced by accelerated glycolysis)"\cite{Arfin2021}
\item "Glutamate is also utilized for glutathione biosynthesis, which is a major antioxidant and quenches the ROS, therefore providing protection to cancer cells against cytotoxic ROS."\cite{Mudassar2020}
\item "The electron transport chain couples the transfer of charge across the inner mitochondrial membrane with the production of ATP. The leakage of ROS and protons from Complexes I-IV of the electron transport chain represents a self-regulating system to reduce oxidative stress, whereby ROS themselves induce proton leak and decrease ROS generation (Brookes, 2005). "\cite{Strickland2017}

\item "In the rapidly proliferating SF-188 pediatric glioblastoma cell line there is a shift away from glycolysis toward oxidative glutamine metabolism represented by increased oxygen consumption and increased ROS"\cite{Strickland2017}

\item "Cancer cells show an increased tolerance for oxidative stress whereby moderate ROS levels promote proliferation and differentiation, whilst excessive ROS exposure causes oxidative damage and induces apoptosis (Jin et al., 2015; Rinaldi et al., 2016)."\cite{Strickland2017} 

\item "Hypoxia often arises concurrently with increased ROS/NOS production, which can act indirectly to stabilize HIF complexes (Chandel et al., 2000, Figure 14B)."\cite{Strickland2017} 
\item Is it the case for all cells or only for those that do not constitutively express HIF1-$\alpha ?$
\end{itemize}

\section{Oxidative Phosphorylation (OXPHOS) \& Electron Transport Chain (ETC)}
\begin{itemize}
\item OXPHOS is the part where NADPH and FADH$_2$ are used to cancel the proton gradient and make ATP in the mitochondria \cite{Berg2006}

\item Mechanism of energy production within mitochondria that is more ATP efficient that glycolysis through the electron transport chain (ETC) \cite{Berg2006}

\item "The mitochondrial electron transport chain (ETC) consists of four enzyme complexes that transfer electrons from donors such as NADH to oxygen (Figure 6a). Chemical disruption of the ETC by complex 1 inhibitors has long been known to impede regeneration of electron acceptors and block cell proliferation ”

\item "Glycolysis is indicated by high expression of HIF-1$\alpha$ (Hypoxia inducible factor- 1$\alpha$) and low levels of phospho-5’ AMP-activated protein kinase (pAMPK), whereas
OXPHOS-reliant tumors have low levels of HIF-1$\alpha$ and high levels of pAMPK. Some cancer cells express high levels of both HIF-1$\alpha$ and pAMPK indicating active glycolysis
as well as OXPHOS" \cite{Nayak2018}

\item ”The mitochondrial electron transport chain (ETC) consists of four enzyme complexes that transfer electrons from donors such as NADH to oxygen (Figure 6a). Chemical disruption of the ETC by complex 1 inhibitors has long been known to impede regeneration of electron acceptors and block cell proliferation.” Bauer 2021

\item "Interestingly, it has long been known that human cells lacking a functional ETC can proliferate if cultured in supra-
physiological concentrations of pyruvate (King and Attardi, 1989). While pyruvate can serve as a biosynthetic substrate or affect the redox state of the cell by promoting the regeneration of NAD+ (Harris, 1980; Howell and Sager, 1979), why it reverses the suppressive effects of ETC inhibition on cell proliferation is unknown."\cite{Birsoy2015}

\item "the electron transport chain produces ATP and reactive oxygen species (ROS) which act as signaling molecules"\cite{Strickland2017}

\item "Glioma cells demonstrate alterations to mitochondrial morphology, with some cells containing healthy electron-dense mitochondria and others exhibiting mitochondria with extensive cristolysis; these characteristics are thought to correlate with hypoxia-resistant and hypoxia-sensitive cell types (Arismendi-Morillo and Castellano-Ramirez, 2008)."

\item "The difference between basal respiration and
respiration at its maximal level constitutes the mitochondrial reserve"\cite{Marchetti2020}

\item "Moreover, hypoxia causes multiple changes in the
composition of ETC complexes. These changes are required to keep mitochondria in tact under low oxygen conditions and to prevent excessive ROS formation. Most of the changes in complex composition occur within complex I, a dominant acceptor of electrons, and complex IV, facilitating electron transport to molecular oxygen. Some of these alterations comprise the exchange of subunits within ETC complexes others modify complex structure, while subunit depletion also occurs." \cite{Fuhrmann2017}

\item The work of Spinelli et al. also demonstrated that some cells can tolerate molecular O$_{2}$ absence for proliferation through reduction of fumarate with the SDH complex working in reverse. \cite{Spinelli2021}
\end{itemize}

\section{Metabolism \& Mechanics}
\begin{itemize}
\item  "our group reveals that hepatoma MCS formation is a three-step process [6] (Fig. 1)."\cite{Lin2008}

\item "Long-chain ECM fibers with multiple RGD motifs for cell-surface integrin binding initially provide rapid aggregation of dispersed cells. A delay phase follows this aggregation
showing up-regulated cadherin expression. Finally, homophilic cadherin-cadherin binding between two cells confers strong cell adhesion, forming MCS."\cite{Lin2008}

\item "[Dr. Hynes] found that ECM components in tumor matrix
are derived from cancer cells and stromal cells,
and many of them are only expressed by cancer
cells, including Col19A1, Col22A1, Col7A1, LAMA4,
LAMB1, LTBP1, LTBP3, LTBP4, TINAGL1, and ECM
regulators galectin 1 (LGALS1) and PLOD1.[20,21] "\cite{Xiong2016}
\item "Cell-cell adhesion is mediated by transmembrane proteins called cadherins that interact through extracellular domains [Citation125]. Cadherin bonds are stabilized by the cortical actin network, with the interaction between actin and cadherins being dynamic and mechanoresponsive [Citation126–128]."\cite{Boot2021}

\item "Manning et al. developed a model that explicitly showed how the overall surface tension of a multicellular aggregate is determined by the ratio of adhesion tension to cortical tension, indicating a crossover from adhesion-dominated to cortical tension-dominated behavior [Citation20]."\cite{Boot2021}

\item "Such variability is a result of innate differences within cell lines, in cell-cell adhesion, thus affecting the overall MCT formation. Cell lines with compact spheroid formation indicated high levels of E-cadherins and those with tight spheroid formation expressed high levels of N-cadherin molecules." \textbf{Facellitate: https://facellitate.com/the-role-of-cell-type-in-multicellular-tumor-spheroid-formation/}

\item i) Dispersed cells initially are drawn closer to form loose aggregates due to their long-chain ECM fibres with multiple RGD motifs that can bind tightly to the integrin on the cell membrane surface. Direct cell–cell contact due to initial aggregation results in upregulated cadherin expression. (ii) Cadherin is accumulated at the membrane surface. (iii) Cells are compacted into solid aggregates to form MCSs due to the homophilic cadherin–cadherin binding [13] (figure 1). The ECM fibres and cadherin type and concentration may vary for different type of cells [14–16]."\cite{Cui2017}

\item " Meanwhile N-cadherin mediates the spontaneous formation of spheroids in MDA-MB-435S [15]. MDA-MB-231 and SK-BR-3 cells form spheroid structure due to collagen I/integrin $\beta$ interaction without cadherin involvement [18]."\cite{Cui2017}

\item Careful... it is not as simple as loads of Cadherin...

\item "The cytoskeleton also plays an important role in MCS formation (figure 2, the actin filaments in the cytoskeleton of the spheroids [21]). The actin filaments undergo significant changes during MCS formation. The expanded microfilaments as stress fibres become localized along the cell periphery. The cytoskeleton as a force generation structure performs as a continuous pre-stressed lattice that keeps cellular structural stability. Morphogenetic phenomena promote the emergence of ordered structures, resulting in the MCSs formation [22]. "\cite{Cui2017}

\item "We identified that E-cadherin mediates the spontaneous formation of spheroids in MCF7, BT-474, T-47D and MDA-MB-361 cells, whereas N-cadherin is responsible for tight packing of MDA-MB-435S cells. In contrast, the matrix protein-induced transformation of 3D aggregates into spheroids in MDA-MB-231 and SK-BR-3 cells is mediated primarily by the collagen I/integrin $\beta$1 interaction with no cadherin involvement."\cite{Ivascu2008}

\item "There was no correlation between the expression levels of integrin ß1 and the 3D morphology of the breast tumor cell lines."\cite{Ivascu2008}

\item "In conclusion, it can be hypothesized that compaction of HT-29 spheroids is mediated by the reorganization of E-cadherin/$\beta$-catenin complexes on the plasma membrane and that this compaction may be responsible for the increase in resistance of HT-29 spheroids to ionizing radiation."\cite{Ferrante2006}

\item "This protocol gives rise to sub-populations of colon cancer cells with stable loss of cell-cell adhesion. SW620 cells lacked E-cadherin, DLD-1 cells lost $\alpha$-catenin and HCT116 cells lacked P-cadherin in the NSF state."\cite{Stadler2018}

\item "However, we show here for the first time that HCT116 spheroid formation is independent from E-cadherin expression but critically dependent on the presence of P-cadherin. "\cite{Stadler2018}

\item "In two other colon cancer cell lines (LS174T and HT29), which were not included in the initial study, ablation of E-cadherin expression also induced a NSF phenotype (Supplemental Figure S7)."\cite{Stadler2018}

\item "Indeed, HT-29 cell spheroids appeared like loose cell aggregates where single cells can be still distinguishable. In turn, SW620 and DLD-1 spheroids displayed a smooth surface and oval (SW620) or round (DLD-1 and HCT-15) shape."\cite{Sargenti2020}

\item "A significant difference was found only for epithelial cell adhesion molecule EpCAM: its expression was more than 3-fold higher in spheroid cells."\cite{Gisina2020}

\item Stiffer extracellular matrix (ECM) upregulates glucose transport proteins (GLUTs), increases glycolytic enzymes and the PPP pathway which is used for nucleic acids, aromatic acids and NADPH. \cite{Onwudiwe2022}
\item Stiffer ECM also enables proteases which helps degrading the matrix and therefore migration. \cite{Onwudiwe2022}
\item "In the brain, [...]Cellular stiffness is generally higher than that of the ECM, and the overall tumor generally softens with cancer progression (16).”\cite{Onwudiwe2022}
\item So stiffer ECM is good at the beginning and then softens because it is degrade, but it means it boosts cancer-favourable metabolism \cite{Onwudiwe2022}\cite{Oh2017}\cite{Romani2020}
\item "there is a shift from oxidative glutamine metabolism to reductive carboxylation when non-small cell lung cancer (NSCLC) cells transition from a monolayer cultures to spheroids [114]." \cite{Natarajan2019}\cite{Jiang2016}
\item "”On a stiff ECM, the ubiquitin ligase TRIM21 is trapped by stress fibres (TaBle 1),
and can- not induce the degradation of PFK 40 (Fig. 2a). On a soft ECM, where stress fibres are inhibited, TRIM21 is able to target PFK and direct it to degradation, thus reducing glycolysi$s^{40}$. The reduced glycolysis rate was not accompanied by a compensatory increase in mitochondrial respiration, indicating a decoupling between the two main energy-producing pathways that should limit the total energy production capacity in cells on a soft ECM.”\cite{Romani2020}
\item "Indeed, oncogene transformation overrides this regulation and uncouples ECM stiffness from glycolysis, leading to aberrant metabolism of glucose on soft substrata$^{40}$ , a key hallmark of cancer (Box 2)."\cite{Romani2020}
\item ”Intracellular pH is regulated by integrin-mediated cell spreading, and thus by the level of cellular tension 46–49 . Intracellular pH is more basic, indicating more active H
+ extrusion, in spread cells, where cell tension is higher, compared with rounded cells, where tension is lower, and this is due to differential NHE1 activity (Fig. 2a). ”\cite{Romani2020}
\textbf{”The complete absence of adhesion (cell detachment) can be compared, to some extent, with an extremely soft ECM”}\cite{Romani2020}
\item "lung cancer cells cultured as spheroids, characterized by low or absent cell–ECM
adhesions, rewired the use of glucose and glutamine to increase the production of antioxidant molecules within the mitochondria" \cite{Romani2020}
\textbf{\item ”For instance, detachment from ECM made of hyaluronic acid leads to increased glu-
cose metabolism 103 , which is opposite to other observations described above and sug-
gests that cells likely integrate various cues to determine their (metabolic) fate.”}\cite{Romani2020}
\item "Metabolic flux analysis revealed a significant increase in the glycolysis capacity of dendritic cells (DCs) cultured on 50 kPa hydrogel and rigid plastic substrates compared to culture on the soft substrates."\cite{Liao2022}
\item "Increased substrate stiffness has a positive effect on glycolytic gene expression, and glucose uptake, and also promotes the increase of intermediates in the pentose phosphate pathway (PPP) and tricarboxylic acid cycle pathway (TCA) " \cite{Liao2022}
\item ”Up-regulation of quiescent GBM cells was identified with up-regulation of laminin, collagens, tenascin C, and integrin $\alpha$3 (62).”\cite{Chen2021}
\item "ECM composition is also related to GBM cell phenotype. Up-regulation of quiescent GBM cells was identified with up-regulation of laminin, collagens, tenascin C, and integrin $\alpha$3 (62)."\cite{Chen2021}
\item "ECM composition is not homogenous or static, in fact, ECM composition varies between different TME regions; periostin and MMP-2/9 are expressed in hypoxic regions (67), whereas type I collagen, tenascin C, laminin, integrin-$\alpha$6, and fibronectin are more abundant in vascularized regions (Table)."\cite{Chen2021}
\item "These spheroid-based models demonstrated significant interplay between spheroid edges and HA altering not only cell morphology of the cells but also metabolic profiles through enriched glycolysis and fatty acid oxidation and synthesis"\cite{Chen2021}
\item \textbf{Look up if somebody compared spheroid in suspension and in scaffold of relevant matrix}
\item "Patient-derived cells maintained under serum-free conditions in neurospheres or laminin-attached monolayers have been shown to retain the original characteristics of the human tumors (Lee et al., 2006; Pollard et al., 2009) and have a more oxidative phenotype (Lin et al., 2017)." \cite{Strickland2017}

\item "This may indicate that tissue architecture is related to the relative preference or propensity to use lactate as a fuel."\cite{Faubert2017}

 \item"A. Lactate production was significantly reduced and glutamine was consumed as an alternative substrate for oxidative metabolism. Longterm adapted cells formed exclusively monolayers, while they normally grow in multilayers forming tumor spheroids. Also, longterm adapted cells proliferated significantly faster."\cite{Weber2002} 
\end{itemize}

\section{Glutamine \& Glutaminolysis}
\begin{itemize}
\item Glutamine can provide energy but is also used to provide nucleotides to cancer cells. \cite{Natarajan2019}\cite{Ma2022}
\item " In the present study, we observed a switch in substrate oxidation between glutamine and glucose, two critical nutrients tied to cancer metabolism [59]. Oxidation of glucose increased while glutamine oxidation decreased in both cell types over time, suggesting a metabolic switch activated by adhesion. While the glucose levels had no effect, hypoxia increased glucose utilization in both cell lines but more so in the MOSE-LTICv spheroids." \cite{Compton2022}
\item ”Tumor cells mitigate excess ROS and maintain redox homeostasis principally by
glutathione synthesis [110,111]. Glutathione is a tripeptide synthesized from glutamate,
cysteine, and glycine that actively scavenges ROS.”\cite{Natarajan2019} This can mean that removing ROS resistance may mimick a glutamine-related defect
\item "Under monolayer culture, these cells showed abundant glutamine uptake to drive TCA cycle anaplerosis, which is lost under anchorage-independent growth." \cite{Natarajan2019}
\item "In cancer patients, some tumors have been reported to consume such an abun-
dance of glutamine that they depress plasma glutamine levels (10, 11). Despite these
observations, the high rates of glutamine metabolism and addiction exhibited by some
cancer cells are poorly understood. Recently, we reported that glioma cells can exhibit
glutamine uptake and metabolism that exceeds the cell’s use of glutamine for protein and
nucleotide biosynthesis (12). In such cells, the excess glutamine metabolites produced
were found to be secreted as either lactate or alanine. This high rate of glutaminolysis
was found to be beneficial because it provided the cell a high rate of NADPH production
that was used to fuel lipid and nucleotide biosynthesis" \cite{Wise2008} 
\item NADPH can balance ROS and support biosynthesis. So in some cases if glutamine is not present enough glucose may not be enough to support massive growth
\item SF188 which are addicted to Glutamine consume between 0.5 and 1.5 mM (in control condition ?) \cite{Wise2008}
\item "At high utilization rates of glutamine, oxidative utilization of glucose was decreased, while the production of lactate from glutamine was enhanced. This emergent phenotype was observed only when the available carbon exceeded the amount that could be fully oxidized by the available oxygen" Modelling results, careful...\cite{Damiani2017}
\item Glutamine is the main carbon source for the TCA cycle when Ras is activated \cite{Ma2022}
\item In the human non-small cell lung cancer cell line A549, it was found that silencing HIF-1$\alpha$ expression reduced glutamine consumption in the tumor cells [40]. Furthermore, HIF-2$\alpha$-
has been reported to enhance the activity of c-MYC, which in turn drives glutamine
catabolism by regulating numerous genes including glutaminase [30, 41]. ”\cite{Ma2022}
\item "HIF-2$\alpha$ enhances glutamine uptake which is
converted into glutamate and replenishes the TCA cycle." \cite{Mudassar2020}
\item "Oncogenes influence nutrient metabolism and nutrient dependence. The oncogene c-Myc stimulates glutamine metabolism and renders cells dependent on glutamine to sustain viability (“glutamine addiction”), suggesting that treatments targeting glutamine metabolism might selectively kill c-Myc-transformed tumor cells."\cite{Yang2009}
\item "Interestingly, glutamine and glutamate are actually released by glioma cells, affecting the surrounding neural tissues (Buckingham et al., 2011)."\cite{Strickland2017}
\item ”Since the brain microenvironment is rich in glutamine [2], glioblastoma cells can take advantage of glutamine catabolism (termed glutaminolysis) as an additional or alternative energy source especially when glycolytic energy production is low due to phases when glucose levels are depleted”\cite{Stuart2023}
\item ”Glutamine addiction is usually associated to glycolysis impairment and Akt induction in those HGG cells [6].”\cite{Fuchs2020}
\item Consos dans "notes on deprivation" et "notes on glutamine and glutaminolysis" 
\end{itemize}

\section{The Citric Acid Cycle}
\begin{itemize}
\item "The products of TCA cycle, NADH, FADH2 provide electrons for the electron transport chain (ETC) chain. This process is known as oxidative phosphorylation (OXPHOS) and efficiently generates ATP."\cite{Mudassar2020}
\item "Mitochondrial metabolism provides precursors to build macromolecules in growing cancer cells1,2." \cite{Mullen2011} 
\item "there is a shift from oxidative glutamine metabolism to reductive carboxylation when non-small cell lung cancer (NSCLC) cells transition from a monolayer cultures to spheroids [114]." \cite{Natarajan2019}\cite{Jiang2016}
\item "This reductive, glutamine-dependent pathway is the dominant mode of metabolism in rapidly growing malignant cells containing mutations in complex I or complex III of the ETC, in patient-derived renal carcinoma cells with mutations in fumarate hydratase, and in cells with normal mitochondria subjected to acute pharmacological ETC inhibition." \cite{Mullen2011}
\item  In highly proliferating cells, citrate produced in the TCA cycle is redirected into the cytosol for the production of
NADPH and fatty acids. The production of $\alpha$-KG though glutaminolysis replenishes
the TCA cycle, a process called anaplerosis.10,11 \cite{Villar2015}
\item  "These results indicate that the energy demand of C6 cells is reduced in the absence of glutamine because of the fact that their growth slows down drastically."[...]"without significantly affecting the relative contribution of the oxidative phosphorylation and of the glycolysis to the total ATP synthesis" It  is \textbf{rat glioma}, though....\cite{Martin1999}
\item "The Kreb's cycle is therefore a flexible central provider for the catabolic and anabolic needs of the cancer cell."\cite{Strickland2017}
"The hypoxia-induced acidosis of the cancer microenvironment, which is caused by an increased production and secretion of lactate is also important [66,68]. Lactate causes tumor immune evasion and neoplastic cell migration."\cite{Korbecki2021}
\end{itemize}


\section{Lactate}
\begin{itemize}
\item "Excreting lactate through monocarboxylate transporters (e.g. MCT1, 4) eliminates protons arising from the glyceraldehyde 3-phosphate dehydrogenase reaction in glycolysis, thereby maintaining pH homeostasis inside the cell and acidifying the extracellular space"\cite{Faubert2017}
\item Some cancer cells use lactate as a respiratory substrate and lipogenic precursor in culture (Chen et al., 2016). Blocking lactate uptake with an MCT1 inhibitor reduces respiration and promotes glycolysis in some cancer cell lines, and suppresses xenograft growth in mice"\cite{Faubert2017}
\item "We also find that lactate’s contribution as a respiratory fuel exceeds that of glucose, particularly in tumors growing in the lung."\cite{Faubert2017}
\item "Lactate [can be] preferred to glucose as a fuel for the TCA cycle in vivo" \cite{Faubert2017}
\item "This may indicate that tissue architecture is related to the relative preference or propensity to use lactate as a fuel."\cite{Faubert2017}
\item "At high utilization rates of glutamine, oxidative utilization of glucose was decreased, while the production of lactate from glutamine was enhanced. This emergent phenotype was observed only when the available carbon exceeded the amount that could be fully oxidized by the available metabolites. oxygen"\cite{Damiani2017} Modelling results, careful...
\item "Although lactate is generally considered a waste product, we now show that it is a prominent substrate that fuels the oxidative metabolism of oxygenated tumor cells. There is therefore a symbiosis in which glycolytic and oxidative tumor cells mutually regulate their access to energy." \cite{Sonveaux2009}
\item "Lactate rescues a broad array of different GBM model systems from nutrient deprived mediated reduction in cellular viability"\cite{Torrini2022}
\item "Lactate activates oxidative metabolism and thereby facilitates promotion of proliferation and survival of GBM cells"\cite{Torrini2022}
\item "Consistently, lactate renders GBM cultures more sensitive to the cytotoxic actions by inhibitors of oxidative phosphorylation and the electron transport chain,"\cite{Torrini2022}
\item "Here we show that lactate is also a TCA cycle carbon source for NSCLC. In human NSCLC, evidence of lactate utilization was most apparent in tumors with high 18fluorodeoxyglucose uptake and aggressive oncological behavior." \cite{Faubert2017}
\end{itemize}

\section{Flux-Balance analysis}
\begin{itemize}
\item FBA allows to identify a phenotype that maximizes a certain objective (typically growth) among all the possible flux patterns compatible with the steady state assumption"\cite{Damiani2017}
\item Sampling of the feasible flux space allowed us to obtain a large number of randomly mutated cells simulated at different glutamine and glucose uptake rates. We observed that, in the limited subset of proliferating cells, most displayed fermentation of glucose to lactate in the presence of oxygen.\cite{Damiani2017}
\item Need for a good network establishment...
\item It can be used in conjunction with ABM. By fixing the maximum  growth rate, Shan \textit{et al.} can use the flux in the diffusion-reaction equation and get to the concentration of each metabolite (They still had to reduce the network).\cite{Shan2018}
\item "We demonstrated with FBA that
under our interpretation, glutamine addiction led to an increase uptake of oxygen (i.e., lower
yields on oxygen) in glutamine-addicted tumor cells to maintain their redox balance and to
meet the energy demand; this lower yield on oxygen represents a cost of using glutamine in
the TCA cycle. We see the impact of this lower yield on oxygen in the reduced growth rate of
glutamine-addicted tumor cells relative to Warburg tumor cells."\cite{Shan2018}

\end{itemize}


\section{Hypoxia \& Hypoxia-Inducible Factors}
\begin{itemize}
\item In the human non-small cell lung cancer cell line A549, it was found that silencing HIF-1$\alpha$ expression reduced glutamine consumption in the tumor cells [40]. Furthermore, HIF-2$\alpha$ has been reported to enhance the activity of c-MYC, which in turn drives glutamine catabolism by regulating numerous genes including glutaminase [30, 41]. ”\cite{Ma2022}

\item "HIF-1$\alpha$ decreases mitochondrial OXPHOS by activating PDK1, which subsequently inhibits PDH. It also inhibits the excess reactive oxygen species (ROS) produced as a result of inefficient electron transport, therefore,protecting cancer cells against oxidative stress."\cite{Mudassar2020}

\item "HIF-2$\alpha$ enhances glutamine uptake which is converted into glutamate and replenishes the TCA cycle." \cite{Mudassar2020}

\item "HIF-1$\alpha$ mediates multiple cellular adaptations present in GBM under hypoxic conditions, such as cell differentiation, inflammatory/immune response modulation, and metabolic reprogramming (47, 48)." \cite{Chen2021}

\item "the electron transport chain produces ATP and reactive oxygen species (ROS) which act as signaling molecules"\cite{Strickland2017}

\item "Glioma cells demonstrate alterations to mitochondrial morphology, with some cells containing healthy electron-dense mitochondria and others exhibiting mitochondria with extensive cristolysis; these characteristics are thought to correlate with hypoxia-resistant and hypoxia-sensitive cell types (Arismendi-Morillo and Castellano-Ramirez, 2008)."

\item "Fatty acid synthesis has been shown to continue under low-oxygen tension and low-nutrient conditions (Lewis et al., 2015), a process which is activated by HIF1$\alpha$ signaling. Fatty acids are shuttled into lipid droplets upon hypoxia in order to support cell growth and survival upon re-oxygenation (Bensaad et al., 2014)." (Glioma cells \cite{Strickland2017}

\item "Hypoxia causes: activation of Hypoxia Inducible Factors (HIF; the subject of this section), facilitation of adaptive metabolism (such as formation of lipid droplets), release of vascular endothelial growth factor (which influences the surrounding tissue to support neo-angiogenesis), and death (if resultant signaling and nutrient restoration does not promote survival). " \cite{Strickland2017}

\item "By recapitulating reductive metabolism in the form of glycolysis and glutaminolysis constitutive activation of HIF-1 in cancer can contribute to cell growth (Lemaire et al., 2015)." \cite{Strickland2017}

\item "Lactate production, due to HIF metabolic remodeling, also has a role in creating a favorable environment for glioma invasion (Figure 14). Lactate causes a decrease in extracellular pH forming an acidic microenvironment, which promotes the death of surrounding tissue, ECM degradation and subsequent localized migration" \cite{Strickland2017}

\item ”Hypoxia-mimetics increase cultured DIPG glycolytic rate and enzyme expression, and decrease proliferation” \cite{Waker2018}

\item "Acute hypoxia acts on peripheral cells that are incompletely and intermittently oxygenated by an aberrant neovascularization. The temporal scale for acute hypoxia is generally considered from minutes to several hours. After 24 h of poor oxygen levels, the tumor cells are considered in a chronic hypoxic state and usually located in the furthest part of the tumor from the blood vessels. Lastly, the cyclic hypoxia might be considered as a kind of intermediary between acute and chronic hypoxia.”\cite{Fuchs2020}

\item ”Furthermore, hypoxia activates the hypoxia-inducible factor 1 (HIF-1) pathway which favors the survival of tumor cells by increasing their glucose uptake and utilization via altered glucose metabolism [6]. It also induces angiogenesis [7], creates an acidic microenvironment and promotes proliferation [8]”\cite{Shen2020}

\item HOWEVER, the paper cited from Gatenby only seems to say that hypoxia does not stop proliferation when it should... also 
" Note that the original model assumed central tumour necrosis was caused by a decrease in glucose concentration. The model simulation's lower panel, however, demonstrates that this component of the hypothesis is incorrect since small declines in glucose concentrations are observed. Instead, it appears that central necrosis is due to a combination of hypoxia and acidosis."\cite{Gatenby2007}

\item "Moreover, hypoxia causes multiple changes in the composition of ETC complexes. These changes are required to keep mitochondria in tact under low oxygen conditions and to prevent excessive ROS formation. Most of the changes in complex composition occur within complex I, a dominant acceptor of electrons, and complex IV, facilitating electron transport to molecular oxygen. Some of these alterations comprise the exchange of subunits within ETC complexes others modify complex structure, while subunit depletion also occurs." \cite{Fuhrmann2017}

\item There seem to be mechanism that prevent the Depletion of NADH \cite{Yan2020}

\item The work of Spinelli et al. also demonstrated that some cells can tolerate molecular O$_{2}$ absence for proliferation through reduction of fumarate with the SDH complex working in reverse. \cite{Spinelli2021}

\item "hypoxia can inhibit cell proliferation by inducing cell cycle arrest [15,16,17]. However, tumour cells often adapt to survive in such hypoxic conditions." \cite{Druker2021}

\item "Tumors are associated with two kinds of hypoxia—chronic hypoxia (also known as continuous or noninterrupted hypoxia) and cycling hypoxia (intermittent or transient hypoxia).[...]  One such cycle of hypoxia lasts from several minutes [12] to 4 h [13], depending on the selected research model."\cite{Korbecki2021}

\item "The mechanisms of activation, however, are different. In chronic hypoxia, the main role is played by a reduction in oxygen levels which triggers a drop in the activity of oxygen-dependent enzymes [15]. In cycling hypoxia, transcription factors are activated mainly by reactive oxygen species (ROS) [16,17]."\cite{Korbecki2021}

"The hypoxia-induced acidosis of the cancer microenvironment, which is caused by an increased production and secretion of lactate is also important [66,68]. Lactate causes tumor immune evasion and neoplastic cell migration."\cite{Korbecki2021}

\item "Notably, such hypoxia-promoted tumor cell growth seemed to be present in MCF-7 TRCs rather than their differentiated counterparts, because in hypoxic conditions, bulk MCF-7 tumor cells cultured in rigid plate were found to cease growth (Fig. 2g). By contrast, few hypoxic MCF-7 TRCs underwent apoptosis, similar to those TRCs in normoxia (Fig. 2h)." \cite{Tang2019}

\item "A hypoxic environment significantly increased the phosphorylation of Akt and PDK1 in mitochondria. The hypoxia-induced accumulation of p-Akt in mitochondria activated PDK1 phosphorylation, promoted the expression of HIF1$\alpha$, and the expression of YKL-40. The overexpression of YKL-40 promoted the proliferation, migration, invasion and tubule formation of CL1-5 cells." \cite{Miao2020}

\item "Hypoxia treatment promoted the cell proliferation, mobility, and colony growth abilities of the two ovarian cancer cell lines HO-8910 and A2780."\cite{Li2023}

\item " Hypoxia inducible factor (HIF) is responsible for the transcription activation of a number of glycolytic genes as well as GLUT1 (17)."\cite{Jozwiak2014}

\item The smaller doubling time (td) of 23.5 h in normoxic A549 cells compared to 33.3 h in H358 cells indicated the higher growth rate of A549 cells during the exponential growth phase (Table 1). Incubation at 1\% and 0.1\% O2 increased td in A549 cells by 32\% and 128\% in comparison to normoxic controls, indicating slower growth under hypoxia. Similarly, H358 cells showed increases in td of 57\% and 106\% in comparison to normoxic controls when incubated at 1\% and 0.1\% O2, respectively. Hypoxia significantly slowed the growth rate in both cell lines (Table 2) and it also significantly decreased the maximal cell density reached in the saturation phase of the growth curve (Figure 3 and Table 2, 144 h and 168 h values)."\cite{Nisar2023}

\item So far the only boost scenario is Pham. on adipose-derived stem cell which therefore is not cancer cells... \cite{Pham2016}

\item The figures in the works of Chabaud show that bladder cnacner cell lines do at best keep their doubling time in hypoxia SW1, and at worst see it go up to 48 hours.\cite{Chabaud2022}

\item "A similar pattern for cell cycle times was found for JIMT-1
cells cultured in normoxia and hypoxia as described for L929 cells
i.e the cell cycle time for individual JIMT-1 cells cultivated in nor-
moxia did not vary much during the observation time (Fig. 3C,
upper panel, almost horizontal regression line) while it increased
for cells cultured in hypoxia (Fig. 3D, upper panel) during the
experimental period shown by the positive slope of the regression
line."\cite{Kamlund2017}

\item Kamlund study is interesting as it shows that human cancer cells can cycle rapidly (e.g less than 20hours) while the majority of the population is stopped which results in an increased doubling time."\cite{Kamlund2017}

\item "The population doubling time, obtained from growth
curves, is derived from both cycling and non-cycling cells. Our
data clearly show that there are sub-populations of rapidly
dividing cells hiding in population-based data such as the pop-
ulation doubling time. Thus, by only drawing conclusions
based on population data, important biological processes on
individual cell levels may be neglected [...] Altogether these data
reflect a phenotypic and genotypic diversity in cell popula-
tions resulting in heterogeneity of cell responses to a com-
mon stimulus. " \cite{Kamlund2017}

\item JIMT-1 which are human breast cancer cell line have a doubling time of 20 h in normoxia but only 36\% divide which yields à 27.hours doubling time. in hypoxia, it's 62.5 but but cells cycle in  30 hours.

\item "This may include three different cell populations. The first type is chronically hypoxic cells. If left in situ, the cells die. These “doomed” cells survive alone after the necessary excision in the cell survival assay and does not affect the response of the tumor when left in situ, which is considered the main cause of cell necrosis in the central region of solid tumors. The second type is the chronically hypoxic cells that are viable if left in situ. These cancer cells are stimulated by hypoxia to promote proliferation, alter gene expression, and enhance cellular drug resistance and radioresistance.16,44,45,46,47,48,49 The third type is transient hypoxic cells, which are expressed closer to functional blood vessels, where the duration of hypoxia is short. Based on current evidence, tumor cells adapt to hypoxia by altering their signaling pathways. Hypoxia promotes malignant behavior of cancer cells, including proliferation, migration, infestation and epithelial-mesenchymal transition (EMT), and enhances immunotherapy, chemotherapy, and radiotherapy tolerance.
" \cite{Chen2023}

\item On HepG2, Hypoxia 1\% does not seem to decrease cell viability with time \cite{Cunha2019}

\item Zhu et al. report a 3 fold increase of OCR beteen hypoxic and normoxic phases \cite{Zhu2020} and Chen \cite{Chen2015} reports a 50 \% decrease (from 480 to 250) at 5 \% pO2
Papandreou reports similar decreases showing OCR goes from 2.5 in both human and murine cells to  1-1.5 nmol/min/Mcells

\item "Acute hypoxia is defined by short (from a few minutes to a few hours) exposure to low oxygen levels that can be reversed by regained blood flow, whereas chronic hypoxia is defined by long exposure to low oxygen levels." \cite{Liu2022}

\item "Importantly, the required concentration to induce hypoxia varies among cell types, some cell types are hypoxic at 5\% O2, while others require less than 1 \% O2" \cite{Liu2022}

\item In human breast tissue, physiological oxygen levels are around 8.5 \% O2 whereashypoxia in human breast cancer has been determined to be around 1.5\% O2 [39]. Cells have different responses to low oxygen concentrations. At 1–5\% O2 the canonical HIF pathway is activated, and other non-canonical pathways can be stimulated to produce the hypoxic response [40]. At around 0.5\% O2, the cell undergoes reduced mRNA translation, which is the most energetically costly process [40, 41]. At around 0.1\% O2 and lower, there is reduced respiration and cell cycle arrest [42]. \cite{Liu2022} 
\end{itemize}

\section{Fatty acid Oxidation}
\begin{itemize}
\item "more recent work has revealed that GBM utilizes bioenergetic substrates such as amino acids, nucleic acids, and fatty acids, with emerging evidence that fatty acid metabolism is the primary substrate for energy production"\cite{Chen2021}
\item "ncreasingly, it is appreciated that fatty acids can act as critical bio-energetic substrates within the glioma cell (Figure 3). Recent results from our lab and other groups have demonstrated that glioma cells primarily use fatty acids as a substrate for energy production. Specifically, human glioma cells primary-cultured under serum-free conditions oxidize fatty acids to maintain both respiratory and proliferative activity (Lin et al., 2017)."\cite{Strickland2017}
\item "3C in vivo radiolabelling studies conducted in orthotopic mouse models of malignant glioma show that acetate contributes over half of oxidative activity within these tumors, while glucose contributes only a third (Maher et al., 2012; Mashimo et al., 2014)." \cite{Strickland2017}
\item "Fatty acids pass easily through the plasma membrane, and this may indeed be a nutrient source in vivo, but these substrates are not made available in cell culture. However, cells do have access to high concentrations of glucose both in vitro and in vivo. "\cite{Strickland2017}
\item "Fatty acid synthesis has been shown to continue under low-oxygen tension and low-nutrient conditions (Lewis et al., 2015), a process which is activated by HIF1$\alpha$ signaling. Fatty acids are shuttled into lipid droplets upon hypoxia in order to support cell growth and survival upon re-oxygenation (Bensaad et al., 2014)."\cite{Strickland2017}
\end{itemize}

\section{pH and metabolism}
\begin{itemize}
\item The collective metabolic process of GBM ultimately leads to an acidic environment that is harmful to normal cells but has minimal effect on cancer cells, thus supporting tumor progression (20).\cite{Chen2021}
\item ”Intracellular pH is regulated by integrin-mediated cell spreading, and thus by the level of cellular tension 46–49 . Intracellular pH is more basic, indicating more active H
+ extrusion, in spread cells, where cell tension is higher, compared with rounded cells, where tension is lower, and this is due to differential NHE1 activity (Fig. 2a). ”\cite{Romani2020}
\item "Notably, low tissue pH and lactic acidosis, a hallmark of GBM that is intimately tied to hypoxia and glycolysis, impacts tumor cell metabolism and survival."\cite{Chen2021}
\item "It is therefore generally accepted that cancer cells undergo aerobic glycolysis so that the NADH by-product of lactate production can be used to fuel biomass production and lactate can be utilized to acidify the microenvironment, facilitating invasion" \cite{Strickland2017}
\item "GSCs have high mitochondrial reserves compared with differentiated cell types; inhibiting neither glycolysis nor oxidative phosphorylation in this cell type has significant effects on energy production (Vlashi et al., 2011)."\cite{Strickland2017}
\item "During aerobic and/or anaerobic glycolysis, glutaminolysis, and ATP hydrolysis, hydrogen ions are formed which are actively transported outside the cell. Via the interstitial space The H+ ions finally reach the blood vessels and are removed from Ihe tissue thereafter by convective transport. If a high glycolytic rate and a high lactic acid production coincide with an insufficient drainage by convective and/or diffusive transport, H+ ions accumulate in the respective tissue. " \cite{Vaupel1990}
\end{itemize}

\section{Extracellular ATP}
\begin{itemize}
\item "In addition to its fundamental role in cellular bioenergetics, the purine nucleotide adenosine triphosphate (ATP) plays a crucial role in the extracellular space as a signaling molecule [1,2]. ATP is actively released in the pericellular environment in response to several stimuli, including (1) inflammation-related biological processes, (2) cellular stress and tissue damage during tumorigenesis, (3) cells undergoing apoptosis, and (4) exosomes secreted by cancer cells [3,4,5]. Extracellular ATP can also be secreted during the process of immunogenic cell death induced by chemotherapeutics or released during necrosis [6,7]. "\cite{Dillard2021}
\item " In healthy tissues, extracellular ATP concentration is very low (in the nM range). However, its concentration can reach hundreds of \textmu M at sites of damaged or inflammatory tissues, as well as in the tumor microenvironment (TME) or at site of metastases [2]."\cite{Dillard2021}
\item "several reports have shown that high concentrations of extracellular ATP or adenosine can directly act on cancer cells by inducing apoptosis [34,35,36,37]. "\cite{Dillard2021}
\item "P2RYs have thus been reported to support growth, invasiveness and metastatic spreading suggesting that the increased ATP content in the TME might drive cancer cell proliferation [2,40,41,42]. However, opposite outcomes of P2RY1 or P2RY2 activation have also been described in different settings [43,44]. "\cite{Dillard2021}
\item "Interestingly, the anti-proliferative activities of both purines were confirmed for the HT29, LS513 and LS174T cell lines, with a marked slowdown of spheroid growth."\cite{Dillard2021}
\item "Today there is a wide consensus that eATP and other nucleotides, and their plasma membrane receptors play a central role in tumor cell proliferation and immune cell regulation [9,10]."\cite{Vultaggio2020}
\item "In spite of the high eATP levels of the TME, the much higher intracellular ATP concentration (in the 5–10 millimolar range) generates an outward-directed gradient for ATP, thus facilitating passive efflux."\cite{Vultaggio2020}
\item "ATP generated inside the cell can be actively released through plasma membrane-derived microvesicles,"\cite{Vultaggio2020}
\item "The basic assumption is that low ATP levels promote tumor proliferation and immunosuppression, while high ATP levels activate infiltrating inflammatory cells and promote antitumor immunity [9]. "\cite{Vultaggio2020}
\end{itemize}

\section{Migration}
\textbf{Does Migration occur in spheroids ?}
\begin{itemize}
\item ”Estimates suggest that up to approximately 50\% of ATP is used to support the actin cytoskeleton (Bernstein, 2003; Daniel et al., 1986). Consequently, intracellular ATP:ADP ratio has been positively correlated with migration potential (Zanotelli et al., 2018).”\cite{Zanotelli2021}
\item ”Glycolysis can help cells respond to fluctuating energetic demands in the membrane (Epstein et al., 2014), hence an increase in glucose uptake is often observed when the
energy cost associated with migration increases rapidly (Zanotelli et al., 2019; Zhang et al., 2019). While glycolysis can synthesize ATP up to 100 times quicker than OXPHOS,
the energetic yield is very low and presents a thermodynamic trade-off between rate and yield (Martinez-Outschoorn et al., 2017).”
\item "The essential observation to be
made is that the inert polystyrene microspheres are driven towards the center of the spheroid while the labelled cells, although exhibiting a similar tendency for internalization, persisted near the periphery of the spheroid occupied by proliferating cells. Consequently a distinctly bimodal distribution of labelled cells can be seen. These experiments are taken by us to imply that there are both passive and active mechanisms governing cell migration."
\item Besides the paper from Dorie$^{\cite{Dorie1986}}$, There is EXACTLY two papers from the same team that adress the question...No compelling evidence of migration in spheroids
\item "In general, cell migration is a stochastic process [75], and in the absence of an external gradient or directional cue, cells migrate randomly [76]. However, when presented with a directional cue, such as a chemokine gradient, the internal signaling machinery becomes polarized and cells migrate with directional bias [77]."\cite{Polacheck2012}
\item "Despite the growing body of work on the molecular mechanisms behind directional sensing and migration in cancer cells, little is known about the chemo-mechanical dynamics of chemotaxis in 3D environments. Furthermore, tumor cells secrete chemokines, and autocrine chemokine gradients can guide cell migration in the absence of an externally applied chemokine gradient [148]." \cite{Polacheck2012}
\item Four types of migration : amoeboid, mesenchymal, clusters/cohorts, strands/sheet\cite{Friedl2003}
\item for Mesenchymal cells : "Focal contacts
form and are turned over in the range of 10–120
minutes 11,31 , resulting in relatively slow migration
velocities of 0.1–2 \textmu m/min in 3D models 33 ."\cite{Friedl2003}
\item There is no direct link between nutrient levels and migration mode or speed or...
\item To explore this issue, we analyzed secretomes from glucose-deprived cells, which revealed up-regulation of multiple cytokines and chemokines, including IL-6 and IL-8\cite{Puschel2020}
\item The low-pH environment of solid tumors caused by elevated glycolysis can also provide a favorable environment for invasion (Martínez-Zaguilán et al., 1996
) and metastasis (Schlappack et al., 1991
). Acid-mediated invasion has been observed in vivo, where highest tumor invasion occurred in areas with the lowest pH and no invasion was observed into peritumoral regions with approximately normal pH (Estrella et al., 2013
). An acidic extracellular pH can promote migration through the activation of proteases including MMPs (Kato et al., 2007
) and cysteine cathepsins (Mohamed and Sloane, 2006
). \cite{Zanotelli2021}
\item . Glutamine metabolism promotes invasion through activation of STAT3 (Yang et al., 2014
), is correlated with increased risk of metastasis and mortality in colorectal cancer (Xiang et al., 2019
), and drives metastasis to liver, lung, and kidney in glioblastoma (Shelton et al., 2010
).\cite{Zanotelli2021}
\item sphere and spheroid can be formed from Epithelial and mesenchymal cells
\item it seems that epithelial cells migrate in groups
\item "Mesenchymal cells, exhibiting elongated morphology, can move forward by generating traction force via cytoskeletal contractility and integrin-mediated ECM-adhesion [4]. Proteolysis-dependent ECM degradation is also required for mesenchymal tumor cells to generate paths for their migration. EMT and hybrid EMT have been identified as key pathways for epithelial tumor cells to gain mesenchymal phenotypes [5]. Conversely, amoeboid cells with rounded and deformable morphology can squeeze through narrow spaces and smaller pores of the ECM in the absence of proteolysis-dependent ECM remodeling [6,7]. During this type of movement, the cells exhibit bleb-like protrusions driven by actomyosin contractility and maintain weak and dynamic cell adhesion to ECM, resulting in high-speed movement [8]. Distinct from single-cell motility, collective cell migration is a movement pattern of multiple cells that retain cell-cell connections and migrate coordinately [9,10]. This type of tumor cell movement depends on actin dynamics, integrin-based ECM adhesion, and proteolytic cleavage of ECM."\cite{Wu2021}
\item amoeboid motion is faster than mesenchymal motion \cite{Wu2021}
\item None of the mode seems more costly so far but mesenchymal is slower and degrades the matrix
\end{itemize}

\section{Spheroids}
\begin{itemize}
\item "We observe that spheroids grow to a limiting size that is independent of the number of cells used to initiate the experiment (Figure 1a–f), leading us to hypothesise that spheroids have a limiting structure (Folkman and Hochberg, 1973). This behaviour is consistent with untested predictions of mathematical models of tumour progression"\cite{Browning2021}

\item "It is widely accepted that the eventual inhibition of spheroid growth arises through three phases (Figure 1g and i; Wallace and Guo, 2013; Spoerri et al., 2017; Flegg and Nataraj, 2019). During phase 1, for spheroids that are sufficiently small, we observe cycling cells throughout. In phase 2, spheroids develop to a size where cells in the spheroid centre remain viable but enter cell cycle arrest, potentially due to a higher concentration of metabolites in the spheroid centre (Weiswald et al., 2015; Masuda et al., 2016). Finally, during phase 3 the spheroid develops a necrotic core. Eventually, the loss of cells within the spheroid balances growth at the spheroid periphery, stalling net overall growth."\cite{Browning2021}
\item "Following Greenspan, 1972, we make two minimal assumptions regarding growth inhibition and necrosis (Figure 1h). Firstly, that growth inhibition, or cell cycle arrest, is a result of a chemical inhibitor that originates from the metabolic waste of living cells (Laurent et al., 2013)."\cite{Browning2021}

\item Experiments shown that reoxygenation of spheroids could lead to necrotic core removal as a single object in extreme cases and could at least lose their symetry...\cite{Murphy2023}

\item "The human melanoma cell lines established from primary (WM793b) and metastatic cancer sites (WM983b, WM164)"\cite{Murphy2023}

\item "Moreover, our analysis suggests that growth and formation of the necrotic core is reasonably described by oxygen diffusion and consumption, whereas the growth and formation of the inhibited region is more accurately described by waste production and diffusion."\cite{Murphy2023}

\item "High-attenuation regions above the threshold highlighted in red (H, L) were detected as the necrotic cores in the tumor spheroids."\cite{Huang2017}
\item "We defined necrotic cores as areas in which the extracellular matrix was lacking (total loss of collagen by picrosirius red staining) and replaced by dead cells and cellular debris (no or fragmented nuclei by hematoxylin and
eosin staining). "\cite{Thim2010}

\item "In some cases, the necrotic cells undergo dissolution
and are removed from the system entirely, leading to the formation of an inner cavity filled with liquid and eventually some necrotic debris (see Fig. 1-a), while, in other cases, the necrotic core is separated from the
rest of the tumour, so that the clearance of the necrotic material is prevented and the whole region undergoes dystrophic calcification (see Fig. 1-b) [15, 35, 36]. Calcification has been detected in many cancers, such as
in breast cancer and in particular ductal carcinoma in situ (DCIS) [41, 68], in glioblastoma multiforme [45], in colorectal and ovarian cancer [15, 53] and in a wide variety of epithelial, mesenchymal, lymphoid, or germ
cell neoplasms [15]. However the pathophysiology of calcification in primary and metastatic malignancies is extremely heterogeneous and not completely understood [15]"\cite{Giverso2018}

\item "HT29 cells formed spheroids with round type, smooth surface and compact morphology after 96 h incubation as hanging drops and became more compact and dense during 10 days of culture (Fig.1b,c). By contrast, Caco2 cells spontaneously started to form spheroids with round-shape structure, and after cultivation for 3 days; they generated hollow spheroids with bubblelike structures (Fig.1e,f). Furthermore, the size of Caco2 spheroids was significantly smaller than HT29 spheroids after 4 days of incubation (pvalue< 0.01)."\cite{Gheytanchi2021}

\item "Manning et al. developed a model that explicitly showed how the overall surface tension of a multicellular aggregate is determined by the ratio of adhesion tension to cortical tension, indicating a crossover from adhesion-dominated to cortical tension-dominated behavior [Citation20]."\cite{Boot2021}

\item i) Dispersed cells initially are drawn closer to form loose aggregates due to their long-chain ECM fibres with multiple RGD motifs that can bind tightly to the integrin on the cell membrane surface. Direct cell–cell contact due to initial aggregation results in upregulated cadherin expression. (ii) Cadherin is accumulated at the membrane surface. (iii) Cells are compacted into solid aggregates to form MCSs due to the homophilic cadherin–cadherin binding [13] (figure 1). The ECM fibres and cadherin type and concentration may vary for different type of cells [14–16]."\cite{Cui2017}

\item  "Actin cytoskeleton is crucial in adhesion, mediation of cell shape, migration, and spreading. Furthermore, actin skeleton plays an important role in spheroids formation. Blocking polymerization of actin filaments reduces aggregation of T47D, HC11, and 4T1 cells strongly. Microtubules also take part in cell aggregation and the growth of spheroids. Interference with the polymerization of microtubules slows down the aggregation of cells or results in the decrease of compaction of spheroids in HC11 cells [23]. "\cite{Bialkowska2020}

\item "E-cadherin is indispensable for spheroid formation" in the 3 studied cell line.\cite{Smyrek2018}

\item "An intact actin cytoskeleton is indispensable for spheroid formation" \cite{Smyrek2018}

\item "Depolymerisation of microtubules primarily decelerates the aggregation and compaction of spheroid formation" \cite{Smyrek2018}

\item "We found that fibronectin was expressed in spheroids of all cell lines at both time points.[...] These data prove that cellular spheroids produce ECM. This and the capability of cells to differently adhere to the ECM may have an influence on the aggregation and compaction of spheroids. "\cite{Smyrek2018}

\item "We demonstrate that the cell-cell adhesion protein E-cadherin and the desmosome proteins DSG2 and DSC2 are important for aggregation. Furthermore, we show that inhibition or silencing of myosin IIa enhances aggregation, suggesting that cytoskeleton tension inhibits tumor cell aggregation. "\cite{Saias2015}

\item "  We have shown that low glucose availability is detected within 3 h of shortage which is then translated into variable expression of genes for cell-to-cell adhesion such as cadherins and Ig-like cell adhesion molecules, and matrix-associated genes such as integrins and metalloproteases. We also found that low glucose concentrations induced cell adhesion, whereas higher concentrations stimulated cell migration." (MCF-7)\cite{Aftab2021}

\item " In the present study, we observed a switch in substrate oxidation between glutamine and glucose, two critical nutrients tied to cancer metabolism [59]. Oxidation of glucose increased while glutamine oxidation decreased in both cell types over time, suggesting a metabolic switch activated by adhesion. While the glucose levels had no effect, hypoxia increased glucose utilization in both cell lines but more so in the MOSE-LTICv spheroids." \cite{Compton2022}

\item "By measuring glucose dynamics in hanging-drop compartments populated by cancer spheroids of various sizes, we could infer glucose distributions within the spheroid, which will help translate in vitro 3D tissue model results to in vivo. "\cite{Rousset2022}

\item "In HeLa and Hek293 young-spheroids, the OxPhos flux and cytochrome c oxidase protein content and activity were similar to those observed in monolayer cultured cells, whereas the glycolytic flux increased two- to fourfold; the contribution of OxPhos to ATP supply was 60 \%. In contrast, in old-spheroids, OxPhos, ATP content, ATP/ADP ratio, and phosphorylation potential diminished 50–70\%, as well as the activity (88\%) and content (3 times) of cytochrome c oxidase. Glycolysis and hexokinase increased significantly (both, 4 times); consequently glycolysis was the predominant pathway for ATP supply (80\%). These changes were associated with an increase
(3.3 times) in the HIF-1a content."\cite{Rodriguez2008}

\item "In low-oxygen conditions, as found in the core of large spheroids, glutamine also makes aerobic glycolysis more effective and can sustain cell growth even as the exclusive substrate [33]. In support of this, we found that there was a decrease in glutamine concentration in the cell media for our 2D and 3D grown cell lines, although the 3D spheroids seem to be better at exploiting this substrate for energy conversion in glucose-free conditions. "\cite{Tidwell2022}
\end{itemize}

\section{Modelling Cell Mechanics}
\begin{itemize}
\item "Recently, Moeendarbary et al.22 experimentally tested the poroelastic nature of cells and showed that cells indeed behave like poroelastic materials at short timescales, and exhibits a power‐law response at long time scales."\cite{Rajagopal2017}
\item "Actin filament network stiffness increases with both filament lengths98 and density.99 Biochemical and mechanical signals regulate the lengths and density of the actin filaments within the cytoskeleton and create a wide variety of network morphologies, broadly categorized into branched, parallel and antiparallel bundles. These morphologies modulate the emergent mechanical behavior of the cytoskeleton composite.100" \cite{Rajagopal2017}
\item "For instance, in one study, cancer cells subjected to external heterogeneous osmotic stress showed cellular responses independent of their metabolic conditions, indicating that short-time responses of cells are physical rather than chemical [157]. The authors demonstrated that the physical cellular responses can be explained better by a theoretical framework called the fluid-filled sponge model that combines mechanics and hydraulics with the poroelastic description of the cytoplasm." \cite{Jung2020}
\item" A study using AFM found that the actin cytoskeleton and myosin II play a major role in cell poroelasticity, whereas MTs and IFs are much less important [98]."\cite{Jung2020}
\item "Unlike soft tissues, many parts of intracellular spaces do not consist of dense networks. Therefore, it is expected that poroelastic effects would emerge only in specific parts of cells or arise more under certain conditions."\cite{Jung2020}
\item  "Some predictions of DAH theory were confirmed in in vitro experiments with cells expressing different levels of N-, P-, or E-cadherins showing that surface tension is a function of cadherin expression level (Foty and Steinberg 2005). The Differential Surface Contraction Hypothesis (DSCH) proposes that surface tension arises primarily from differences in actomyosin-driven cell cortical contractility (Harris 1976). Finally, the Differential Interfacial Tension Hypothesis (DITH) states that differences in interfacial tensions depend both on intercellular adhesion and cell contractility with some competing effects, thus combining ideas of the DAH and the DSCH (Brodland 2002; Manning et al. 2010)."
\item Cell nucleus diameter reported to be around 10 \textmu m  in HCT116 \cite{Kang2010}
\item "The prevailing view suggests that at the mesoscale level, chromatin fills the entire volume of the nucleus20,31,40,46,47 except in some unique cases48,49 where the chromatin is peripheral.Each chromosome is maintained within a specific terittory,[46] mingling with adjacent chromosomes, but without fiber entanglment.[50] "\cite{Lorber2022}
\item "Chromatin volume fraction is defined as the ratio between the volume of chromatin to the volume of the nucleus,62 and the values reported in the literature range between 15\% [63] and 65\%. [64] "\cite{Lorber2022}
\end{itemize}

\section{Cytoskeleton}
\begin{itemize}
\item actin is now known to be an extremely abundant protein (typically 5–10\% of total protein) \cite{Cooper2006}
\item wtf
\end{itemize}

\section{Metabolic variety and plasticity}
\begin{itemize}
\item "To meet their demand for rapid growth, tumour biomacromolecules can be synthesised using intermediate metabolites generated during aerobic glycolysis (Vaupel, Schmidberger \& Mayer, 2019). However, not all tumours share this property of aerobic glycolysis."\cite{Zhang2023}

\item "For example, glioblastoma stem cells (GSCs) in perivascular regions exhibit robust glycolytic metabolism based on blood glucose availability, whereas cells in hypoxic regions display high levels of metabolic flexibility and are fueled by lactate, lipids, and amino acids (21)."\cite{Chen2021}

\item "In addition to these region-specific changes, a metabolic interplay has been posited in the perivascular and hypoxic regions where lactate release from hypoxic regions are metabolized through oxidative phosphorylation by perivascular cells and glucose release from the perivascular cells fuel glycolysis in the hypoxic space (60)."\cite{Chen2021}

\item "Tumors contain oxygenated and hypoxic regions, so the tumor cell population is heterogeneous. Hypoxic tumor cells primarily use glucose for glycolytic energy production and release lactic acid, creating a lactate gradient that mirrors the oxygen gradient in the tumor. By contrast, oxygenated tumor cells have been thought to primarily use glucose for oxidative energy production."\cite{Sonveaux2009}

\item "The D-54MG and GL261 glioma cell lines displayed an oxidative phosphorylation (OXPHOS)-dependent phenotype, characterized by extremely long survival under glucose starvation, and low tolerance to poisoning of the electron transport chain (ETC). Alternatively, U-251MG and U-87MG glioma cells exhibited a glycolytic-dependent phenotype with functional OXPHOS."\cite{Griguer2005}

\item "While the Warburg Effect has been observed in gliomas and other tumors (Oudard et al., 1996), it has been noted that aerobic glycolysis does not account for the total ATP production in many types of cancer cells—both immortalized cell lines and primary cultures (Vander Heiden et al., 2009)—suggesting that other substrates are being oxidized. When this hypothesis was first formally tested in MCF-7 breast cancer cells in 2002, it was discovered that total ATP turnover was 80\% oxidative and 20\% glycolytic (Guppy et al., 2002)."\cite{Strickland2017}

\item "In particular, cancer stem cells which propagate tumor growth and allow recurrence after resection or chemotherapeutic treatment, may exploit different metabolic strategies than other cells within the tumor. For example, a recent study showed that glioma stem cells (GSCs) are less glycolytic than differentiated glioma cells, consuming lower levels of glucose and producing lower amounts of lactate while maintaining higher ATP levels compared with their differentiated progeny. The notorious radio-resistance of this cell population is correlated with higher mitochondrial reserve capacity, leading the authors to conclude that GSCs primarily rely upon oxidative metabolic strategies and will not be vanquished by therapies aiming to inhibit glycolysis (Vlashi et al., 2011)."\cite{Strickland2017}

\item "The Kreb's cycle is therefore a flexible central provider for the catabolic and anabolic needs of the cancer cell." \cite{Strickland2017}

\item "Interestingly, glutamine and glutamate are actually released by glioma cells, affecting the surrounding neural tissues (Buckingham et al., 2011)."\cite{Strickland2017}

\item "In the 1960s and 1970s, Randle showed that NADH and acetyl-CoA produced during beta-oxidation both inhibit the activity of pyruvate dehydrogenase (PDH), thereby promoting the conversion of pyruvate to lactate. Well-characterized in diabetes, the work of Randle and his colleagues reveals that non-oxidative glycolysis can occur alongside the oxidation of other substrates, particularly fatty acids" (Randle et al., 1963).\cite{Strickland2017}

\item "A more recent study has established the Corbet-Feron Effect, where lactate-induced acidification of the microenvironment over a period of weeks leads to adaptation of the cancer cell population, promoting beta-oxidation as a metabolic strategy (Corbet et al., 2016). "\cite{Strickland2017}

\item "In conclusion, the metabolic state of GSCs seems to differ substantially from the metabolic state of differentiated glioma cells, and it correlates with resistance to ionizing radiation. The ability of GSCs to use multiple pathways to produce energy renders them resistant to therapies that target individual metabolic pathways, suggesting that targeting specific metabolic pathways in glioblastoma may spare GSCs."\cite{Vlashi2011}

\item "In the rapidly proliferating SF-188 pediatric glioblastoma cell line there is a shift away from glycolysis toward oxidative glutamine metabolism represented by increased oxygen consumption and increased ROS"\cite{Strickland2017}

\item "In fact it has been proposed that varying oxygen levels within a tumor make metabolic heterogeneity inevitable (Strickaert et al., 2016)."\cite{Strickland2017}

\item "Metabolic shift or enhanced glutamine metabolism is believed to occur in response to oncogenes such as c-MYC and pro-inflammatory cytokines in glioblastoma" \cite{Stuart2023}

\item "Although anaerobic glycolysis can provide energy to the cells in an efficient way, as demonstrated by the maintainance of the energy level under anoxia, aerobic metabolism can provide the energy to the cells in the absence of glucose (and glycogen), probably by utilizing pyruvate or glutamine as carbon sources (the concentrations of which are 1 mM and 4 mM, respectively, in the culture medium). Under the most drastic starvation conditions (Dulbecco’s modified Eagle’s medium free of glucose, pyruvate and glutamine), the survival of C6 glioma cells with a good energy status at least duringan
8-h period, suggests the ultimate utilization, before death of the endogenous pool of fatty acids. These results illustrate the ability of the cells to modulate the activity of their metabolic pathways as a function of the substrate content of their external medium."\cite{Piannet1991}

\item ”Furthermore, hypoxia activates the hypoxia-inducible factor 1 (HIF-1) pathway which favors the survival of tumor cells by increasing their glucose uptake and utilization via altered glucose metabolism [6]. It also induces angiogenesis [7], creates an acidic microenvironment and promotes proliferation [8]”\cite{Shen2020}

\item "HIF-1 $\alpha$ expression was higher when cells were incubated under hypoxia compared to normoxic conditions, indicating that they actually sense our low-oxygen in vitro conditions (Figure 1a, uncropped version in Figure S1). Nevertheless, we did not observe any cell culture suffering up to 96 h under these hypoxic conditions"\cite{Bailleul2021}

\item ” It was previously suggested in experimental tumor models that lactate oxidation occurs in well-oxygenated regions as part of symbiotic metabolite exchanges in which hypoxic cancer cells produce lactate and better oxygenated cells take up lactate to fuel respiration (Sonveaux et al., 2008).”\cite{Faubert2017}

\item \textbf{Glucose feeds the TCA cycle via circulating lactate}  \cite{Hui2017}
 
\item ”In GBM1 and JHH520 cells, reduction in glucose levels (450 mg/dL–45 mg/dL) did not significantly affect cell viability. However, reducing the glucose concentration to 45 mg/dL significantly decreased the viability of BTSC233 cells (p < 0.05)” \cite{Yusuf2022}

\item The work of Spinelli et al. also demonstrated that some cells can tolerate molecular O$_{2}$ absence for proliferation through reduction of fumarate with the SDH complex working in reverse. \cite{Spinelli2021}

\item "Furthermore, a subclass of glioma cells which utilize glycolysis preferentially (i.e., glycolytic gliomas) can also switch from aerobic glycolysis to OXPHOS under limiting glucose
conditions [7,8]"\cite{Jose2010}

\item " In the present study, we observed a switch in substrate oxidation between glutamine and glucose, two critical nutrients tied to cancer metabolism [59]. Oxidation of glucose increased while glutamine oxidation decreased in both cell types over time, suggesting a metabolic switch activated by adhesion. While the glucose levels had no effect, hypoxia increased glucose utilization in both cell lines but more so in the MOSE-LTICv spheroids." \cite{Compton2022}

\item "All the experiments here have been carried out in low glucose as this is one way to achieve a more physiological cell culture environment. To ensure the cultures are not being starved of nutrients and for more insight into nutrient utilization, the levels of glucose, glutamine, and lactate in the culture media were tested (Fig. 6). All cell lines exhibit higher glucose consumption, lactate production, and glutamine consumption in 2D cultures when comparing absolute levels (Figure S6). However, when normalizing to surface area exposed to media, consistent differences disappear. HCT116 in 3D has almost double the amount of glucose consumed per mm2 than in 2D, but the other cell lines are quite similar between 2D and 3D cultures. The increase in HCT116 in 3D is also present for glutamine consumption and lactate production. "\cite{Tidwell2022}

"In low-oxygen conditions, as found in the core of large spheroids, glutamine also makes aerobic glycolysis more effective and can sustain cell growth even as the exclusive substrate [33]. In support of this, we found that there was a decrease in glutamine concentration in the cell media for our 2D and 3D grown cell lines, although the 3D spheroids seem to be better at exploiting this substrate for energy conversion in glucose-free conditions. "\cite{Tidwell2022}

\item They indicate that metastases-forming 4T1 cells are
more adept at adjusting their metabolism in response to environmental stress than isogenic, nonmetastatic 67NR cells. We suggest that the metabolic plasticity and adaptability are more important to the metastatic breast cancer phenotype than rapid
cell proliferation alone, which could 1) provide a new biomarker for early detection of this phenotype, possibly at the time of
diagnosis, and 2) lead to new treatment strategies of metastatic breast cancer by targeting mitochondrial metabolism." \item{Simoes2015}

\item "For instance, 4T1 cells, which have higher basal levels of LDH-A
expression than 67NR cells [17], significantly increased their OCR in
response to LDH-A knockdown and reduced aerobic glycolysis [23]." \cite{Simoes2015}

\item "The HIF proteins are induced in hypoxic areas of tumors, and HIF-$\alpha$ subunit accumulation has been linked to poor prognosis in a variety of cancer settings (Nakazawa et al. 2016). HIF target genes facilitate the adaptation to growth and survival in low oxygen, which involves a cell-intrinsic increase in glucose uptake and glycolysis, with a corresponding attenuation of mitochondrial oxidative metabolism (Nakazawa et al. 2016). This metabolic change allows tumor cells to maintain the energy production needed for survival under hypoxia. "\cite{Torrence2018}

\item The study of Freischel on competition between MCF-7 and MDA MB 234 could be reproduced in the sphere model. \cite{Freischel2021}
\end{itemize}

\section{Consumptions}
\begin{itemize}
\item The article from Kammerer sums up nicely why changing the subtrate consumption five-fold.

\item CMRO2 were found between 0.5 mM/min/cell and 3 mM/min/cell in studies \cite{Rhodes1983}\cite{Shalit1972}\cite{Kirsch1978}

\item Radde reports a 2.5-ratio between MCF-7 and T47D.

\item OCR may well vary within a factor of 5 as well 

\item "As expected, p53-/- cells showed lower O2 consumption than p53+/+ cells under fed conditions. Surprisingly, p53-/- cells responded to serine starvation with increased O2 consumption, whereas p53+/+ cells showed lower O2 consumption (Fig. 2c)."\cite{Maddocks2012}

\item "Glucose uptake and lactate secretion rates were only 2-fold lower in contact-inhibited compared with proliferating fibroblasts."\cite{Valcourt2012}

\item "HepG2 cells had increased mitochondrial content, OXPHOS levels, and decreased glycolysis levels under aglycemic conditions than under hyperglycemic condition (Domenis et al., 2012)."\cite{Zhang2023}

\item Important: Not all O2 consumption is mitochondrial

\item "All the experiments here have been carried out in low glucose as this is one way to achieve a more physiological cell culture environment. To ensure the cultures are not being starved of nutrients and for more insight into nutrient utilization, the levels of glucose, glutamine, and lactate in the culture media were tested (Fig. 6). All cell lines exhibit higher glucose consumption, lactate production, and glutamine consumption in 2D cultures when comparing absolute levels (Figure S6). However, when normalizing to surface area exposed to media, consistent differences disappear. HCT116 in 3D has almost double the amount of glucose consumed per mm2 than in 2D, but the other cell lines are quite similar between 2D and 3D cultures. The increase in HCT116 in 3D is also present for glutamine consumption and lactate production. "\cite{Tidwell2022}

\item Results from Han suggest that from 1 to 2.5 mM the overall uptake varies 50\% overall which even if the cell number is the same is no dramatic evolution \cite{Han062015}

\item  OCR is typically measured with electrodes \cite{Divakaruni2022} 

\item Glucose is measured by checking supernatant concentration at given time 

\item OCR is mitochondrial and non-mitochondrial. The mitochondrial part has a proton leak and ATP linked part (which can be removed with oligomycin) \cite{Hill2012}

\item reserve capacity indicates how close a cell is to operating at its bioenergetic limit. \cite{Hill2012}
\end{itemize}

\section{Proliferation}
\begin{itemize}
\item "Cell proliferation rates (i.e., doubling times) of HepG2, grown in the three different culture conditions, were compared. HepG2 cells divided faster in the high glucose (25 mM) medium (doubling time: 19 $\pm$ 0.8 h) than in the intermediate (11 mM) glucose condition (doubling time: 30.7 $\pm$ 0.5 h). In galactose medium, the doubling time was approximately two times longer (50 $\pm$ 1.2 h). No significant change in cell morphology or cell viability was observed in HepG2 grown under the three different conditions."\cite{Domenis2012}

\item "HeLas divided rapidly with a doubling time of $\approx$1 day (23.9 $\pm$ 1 h). In galactose/glutamine medium, the doubling time was three times longer (58 $\pm$ 2 h)."\cite{Rossignol2004}

\item "In summary, cells such as HeLas, osteosarcomas 143B, and fibroblasts can live using glycolysis or oxidative phosphorylation, but HeLa and osteosarcoma cells do so by altering significantly mitochondrial composition and form to facilitate optimal use of the available substrate."\cite{Rossignol2004}

\item " Display of cell cycle arrest at 48 h by MDA-MB-231 cells, in G0/G1 and G 2 /M phases due to GS and GS’ conditions, respectively, and in both, G 0 /G 1 plus G 2 /M phases after exposure to HG’, PG’, Hypo G’, Py 1 mM and Q 4 mM conditions can be seen as an attempt to cope with nutritional stress and survive. Colombo et al. (2011), highlighted the role of 6-phosphofructo-2-kinase/fructose-2,6-bisphosphatase, isoform 3 (PFKFB3) and GLS1
in causing S and G0/G1 arrest in conditions similar to GS and PG’, respectively, at 18 h working with synchronized HeLa cells. However, our study revealed S-phase arrest in both conditions at 24 h and G 0 /G 1 and S-phase arrest in GS and PG’ conditions, respectively, at 48 h. Our finding of S-phase arrest in HG’ and PG’ is corroborated by the findings of
Sun et al. (2019), who also reported S-phase arrest upon a similar condition of Q-deprivation for 48 h in T24 bladder cancer cell"\cite{Prasad2023}

\item Consumption goes from 30 to 10 fmol/cell/day from growth to stagnant phase \cite{Zhang2011}

\item Glucose consumption measured by 2NBDG fluo varies WITHIN cell population. \cite{Hassanein2011}

\item Similar to previous results in fibroblasts (Lemons et al., 2010), glucose consumption and lactate secretion were only slightly reduced in quiescent (day 15) compared to proliferating (day 5) MCF-10A cells (Figure 1E). \cite{Coloff2016}

\item The paper from Shen does not show decrease  of glycolytic atp when exposed to metformin and phenformin
\end{itemize}

\section{Glucose}
\begin{itemize}
\item "Glucose deprivation of non-insulin-responsive cells. Cells where glucose transport is not regulated by insulin do not express the GLUT4 transporter. According to their response to glucose deprivation, these cells can be subdivided into those where GLUT1 mRNA is not altered and those where this transcript is elevated. In the first group we find fibroblasts of diverse origin and vascular endothelial cells. The second group comprises glial cells, fetal lung epithelial cells, and endothelial microvessel cells." \cite{Klip1994}

\item Consumption goes from 30 to 10 fmol/cell/day from growth to stagnant phase in NIH 3T3 rat cells. \cite{Zhang2011}

\item long term glucose deprivation reduces proliferation to 80\% max in some cell lines after a 10-15 days.\cite{Mathews2020}

\item data does not seemingly support very significant cellular increase in consumption
uptake
\end{itemize}


\section{Cell cycle}
\begin{itemize}
\item There is dependance between serum concentration and cecll cycle spread (which is log normal and centered on 27h in HeLa cells)\cite{Govindaraj2022}
\item 50\% de densité max dans les 5 heures quand même. \cite{Govindaraj2022}

\item Cells within a subpopulation and in cell lineages show distinct cell cycle period and phase duration pattern\cite{Govindaraj2022}

\item "First, we demonstrated that the
transcription rate variability in different cell lineages present within the cellular population is
responsible for the high correlations in TCC among sister pairs across cell lineages"\cite{Govindaraj2022}

\item Cell cycle phase at birth formula given by Mosheiff on fera tn plus bruit blanc et fini

\item Cells can also undergo G0 arrest spontaneously in response to cell-intrinsic factors like replication stress [23–25].\cite{Wiecek2023} 

\item TP53-WT RKO 37,5\% of G0 cells in "normal conditions"
25\% for G0 TP53-MT SW480 and 37,5\% for PC9 NSCLC. However I couldn't find culture conditions for those cell lines in the paper...
\end{itemize}

\section{Nutrients supply and vasculature}
The goal of this section is to compile facts relating to how nutrient varies during tumor growth in order see if we can model the impact.
\begin{itemize}
\item "[..] structural abnormalities contribute to spatial and temporal heterogeneity in tumorblood flow. "\cite{Jain2005}
\item Blood flow in vasculature can vary by two orders of magnitude in tumors \cite{Vaupel1990}
\end{itemize}

\section{Liver}
\begin{itemize}
\item  "A characteristic feature of stellate cells is the deposition of vitamin A in lipid droplets. In the liver, hepatic stellate cells (HSC) are normally located within perisinusoidal and portal areas and can constitute as much as 15\% of liver mass. Hepatocellular carcinoma (HCC) is the predominant form of liver cancer and HSCs function to promote crosstalk within the TME. A key signaling molecule, TGF-$\beta$ is produced by HCC and triggers HSCs to become activated. Once activated HSCs modify the ECM and produce proangiogenic factors such as VEGF-A and MMP-2. Lipid droplets are critical structures in HSCs used to produce new ECM and remodel it through the production of MMPs. Pancreatic ductal adenocarcinoma is the most common form of pancreatic cancer (95\%), characterized by dense fibrotic tissue or desmoplasia. When pancreatic stellate cells (PSCs) are quiescent, they contribute to ECM modification through the production of ECM proteins (e.g. desmin, vimentin) and degradation enzymes." \cite{Anderson2020}
\end{itemize}

\section{Consumption response to environment}
\begin{itemize}
\item "The expression of the two glucose transporters was regulated in the opposite direction in response to glucose concentration in the culture medium. GLUT-1 was more effectively induced when glucose was low, and GLUT-2 expression was more pronounced when glucose was high in the culture media. Another difference between the two glucose transporters was that GLUT-2 expression was increased while GLUT-1 expression was decreased as culturing continued as long as 7 days. Thus, after 7 days of culture GLUT-2 expression in beta-cells was nearly the same at low and high glucose, whereas GLUT-1 was practically absent no matter what the glucose level was. In attempts to correlate GLUT-1 and GLUT-2 expression to beta-cell function glucose uptake and glucose-stimulated insulin release in fresh and cultured islets were measured. In freshly isolated islet glucose uptake was estimated to be 100-fold in excess of actual glucose use. Glucose uptake was reduced by 7-day culture to about one-third of that observed in freshly isolated islets no matter what the glucose concentration of the culture media. We conclude that in the present experimental system GLUT-1 and GLUT-2 expression and function are not closely associated with glucose usage rates or the secretory function of beta-cells."\cite{Tal1992}
\item "GLUT1 expression is mainly regulated by blood glucose concentration, cell signaling mechanisms and hormones. In hypoglycaemic states, there is an upregulation of GLUT1 in tissues such as brain where it helps in providing a major source of energy."\cite{Pragallapati2019}
\item "GLUT2 has a very low affinity for glucose with a Km for 3-O-methylglucose of 40 mM (Gould et al, 1991). Since normal circulating glucose concentration is 3.9-5.6 mM, the rate of transport will be directly proportional to glucose concentration. Therefore, in the postprandial state, when circulating glucose levels are high, there is a net flux of glucose into hepatocytes and pancreatic $\beta$-cells. In contrast, when circulating glucose levels are low, intracellular glucose concentration will increase as a result of glycogenolysis and gluconeogenesis. When the intracellular glucose concentration exceeds the plasma concentration GLUT transports glucose from the liver into the circulation. GLUT2 also functions as a low-affinity fructose transporter, which is consistent with the liver being the primary site for fructose metabolism (Gould et al, 1991). GLUT2 is further involved in the anterior transport of glucose supplied by choroidal circulation from the early stages of retinal development (Watanabe et al, 1999). "
\item "Cells can also increase
glucose uptake when ATP production through the mitochondrial
respiratory chain is limited by decreased oxygen availability or produces
harmful levels of reactive oxygen species. When electron build-up in the
respiratory chain results in high mitochondrial production of reactive
oxygen species and accumulation of TCA cycle intermediates, activity of
the transcription factor hypoxia-inducible factor 1 $\alpha$ (HIF1$\alpha$) increases,
which in turn suppresses entry of glucose carbons into the TCA cycle
and decreases oxidative phosphorylation 32,33 . Concomitantly, HIF1$\alpha$
induces the expression of GLUT1 and lactate dehydrogenase, thereby
increasing glucose uptake and glycolysis as an alternative means to
generate ATP34–36."\cite{Palm2017}
\item "To assess the role of glucose and glutamine in colorectal cancer cells, a proliferation assay was performed under various media conditions (Fig. 1A and Supplementary Fig. S1A). For the assay, we confirmed that DLD1 and HCT116 cells had a KRAS mutation at codon 13 involving a nucleo-
tide change from GGC to GAC, and that HT29 and CaR1 cells did not have this KRAS mutation (Fig. 1B and Supplementary Fig. S1B). Notably, DLD1, HCT116, and CaR1 cells could survive under the glucose-deprived conditions (Fig. 1C–E and Supplementary Fig. S1C–E). Furthermore, DLD1 cells that had strong resistance to the condition of glucose depletion were able to survive for 14 days (Fig. 1F and G), and the passage of DLD1 cells was possible under that condition. The rate of apoptotic cells under the glucose-deprived conditions was lower in DLD1 cells than in HT29 cells (1.5\% vs. 24.7\%, respectively) (Fig. 1H). These findings show that colorectal cancer cells can survive under conditions of glucose depletion (glutamine sufficiency), which is profoundly different from pancreatic cancer cells in which both nutrients are indispensable."\item{Miyo2016}
\end{itemize}

\section{Culture Methods}
In this section we look for hanging drop culture to reduce the simulation volume.
\begin{itemize}
\item constant flux of fresh medium in 100 \textmu m tube you can model with finite elements with 20 \textmu m/s in the well. It also shows flows (advective and rotative) in the flowless inverted drop  \cite{HuangS2020}
\item hanging drops are 10 to 60 µL \cite{Foty2011}\cite{Jeong2022} 
\end{itemize}

\section{Reponse time}
\begin{itemize}
\item Results from Miyo and Simoes suggest a 8-12h response time \cite{Simoes2015}\cite{Miyo2016}
\end{itemize}

\section{Doubling time}
formula on doubling time : $t_2 - t_1 \cdot \frac{\log(2)}{\log(q_2/q_1)}$
\begin{itemize}
\item https://bionumbers.hms.harvard.edu/files/Doubling\%20time\%20of\%20human\%20cancerous\%20cell\%20lines.pdf

\item while the previous link 13.4 hours  others report 25 to 35 on HCT116(https://www.cytion.com/Knowledge-Hub/Cell-Line-Insights/HCT116-Cell-Line-A-Comprehensive-Guide-to-Colorectal-Cancer-Research/) with the following condition McCoys 5a medium, supplemented with 3.0 g/L L-glucose, 1.5 mM L-glutamine, 3.0 g/L NaHCO3, and 10\% fetal bovine serum, is optimal for HCT116 cell cultivation. It is advisable to renew the medium 1 to 2 times per week.

\item in RPMI 1640 (1.8- 2.2 g/L)/(9.9 mM 12.1 mM) the doubling time on a plate iof HCT116 on a plate is 19.6h \cite{Lei2013}

\item they report 21-23h in "ATCC standard conditions" 3g/L of glucose (16.5 mM) \cite{Ribas2003}

\item two different strain of SF188 (s \& f) cycle at 23.5 and 9.9 hours respectively in 25 mM glucose 2 mM glutamine and 5.5/2 the one that grew faster were in the "super medium" (after 4 weeks) \cite{Winer2014} 

\item Oriaopoulou is onto this thing with her paper but does not model cell cycle heterogeneity specifically. \cite{Oraiopoulou2017}

\item So far only high levels of glucose and serum have yielded shorter doubling times and never hypoxia (in cancer cells)
\end{itemize}
\textbf{Is it the proportion of live cells or the cell cycle length that has the most impact ?}

\section{MCF-7}
\begin{itemize}
\item Mazurek reports 27h at 5 and 90 h at 0.5 mM glucose with 20\% serum \cite{Mazurek1997}

\item Hou reports 50-60h at 25 and 90 h at 5 mM glucose with 2\% serum \cite{Hou2017}

\item Sun reports 13h at 25 and150 h at 5.5 mM glucose with 10\% serum \cite{SunS2019}

\item A-427 and MCF-7 but Not A-549 Cells Cultured Under Normoxia Consumed Lactate Independently of the pH \cite{Romero-Garcia2019}

\item Guppy gives the whole energetic makeup of MCF7 \cite{Guppy2002}

\item Azevedo and colleagues show uptake/consumption grows exponentially \cite{Azevedo2015}

\item Smith1998 says serum deprivation did not impact glucose uptake in mcf7 \cite{Smith1998}

\item Burgman shows taht hypoxia (in fact anoxia of 0.0002\% O2) impacts glucose uptake (average 2.5 times over nine experiments compared to normoxic conditions) which is apparently not linked to GLUT1 being more present. Severe hypoxia increases GLUT activity through reduction of thiol groups \cite{Burgman2001}

\item Aftab and colleagues found that MCF-7 cells took 3 hours to detect low glucose conditions and that glucose uptake fall from 1.4 to 1.2 (14\% decrease)

\item glucose inbreast cancer pdf \cite{Shin2021}
\end{itemize}
\section{Necrosis in spheroids}
\begin{itemize}
\item While it is not outright stated there seem to be small necrotic core in  500 \textmu m diameter HCT116 spheroids in 25 mM glucose. \cite{Klaudia2023}

\item In 600 \textmu m diameter spheroids the necrotic core is around 300-400 \textmu m in HCT116 mccoy5A grown medium with 16.5 mM glucose  \cite{Huang2017}

\item HCT116 600 µm in RMPI 1640 has a clear necrotic core of around 200 \textmu m in RPMI 1640 which is 25 mM. \cite{Schaefer2023}

\item OVCAR spheroids of around 600 \textmu m has a rather larger necrotic core in 11 mM RPMI medium and viable rim that is obviously less than 200\textmu m  \cite{Yan2021} 

\item MCF-7 spheroids cultured in (supposedly standard conditions) DMEM and no necrotic core before day 7 which correspond to a diameter 600 \textmu m then necrosis occurs with a 200 \textmu m core forming \cite{Lee2010}


%\item Spheroids above 500 \textmu m in diameter commonly display a layer-like structure comprising a necrotic core surrounded by a viable rim, which consists of an inner layer of quiescent cells and an outer layer of proliferating cells . in 11 or 25 mM and normoxia \cite{Palma2016}

\item A549 has a 200 \textmu prolif rim in 600 \textmu m diameter. in a concentration of 7 mM (F12K medium) in hanging drop ?\cite{Zanoni2016}

\item "that necrosis in spheroids may
develop despite sufficient 02 supply."\cite{MullerKlieser1986}

\item "In normal medium containing glucose (1 g/liter; 5.5 mM), spheroid diameter increased linearly with time, growing from approximately 400 micron to approximately 1200 micron in 8 days, and most spheroids did not develop central necrosis. Increase in glucose concentration up to 5 g/liter had no effect on spheroid growth. Lower glucose concentration decreased the rate of spheroid growth, but large effects were observed only at glucose concentration lower than 100 mg/liter. Spheroids developed central necrosis at 2-4 days after transfer to glucose-deficient medium, and the diameter of the necrotic center increased more rapidly than the diameter of the spheroid. There was an approximately linear relationship between thickness of the viable rim in 5-6-day spheroids and glucose concentration in the range of zero (rim thickness, approximately 150 micron) to 500 mg/liter (rim thickness, approximately 400 micron). The presence or absence of pyruvate (110 mg/liter) in the medium had no effect on spheroid growth or formation of necrosis. These results suggest that limited penetration of glucose may be one of the factors that contribute to cell death in solid tumors. "\cite{Tannock1986}

\item at 25 mM viable rim is 300 \textmu m... \cite{Walenta2000}

\item  4-5 mM glucose with bladder cancer cell :"
The width of the viable rim of spheroids grown in spinner culture was maintained at approximately 190 microns over a wide range of spheroid diameters (400 to 1000 microns)"\cite{Erlichman1986}

\item consumption was at 17 mM,  5 to 15 mM/10000/24hr
and specifically 12 for HCT116 -> \cite{Kammerer2015}

\item Mao reports 150 \textmu m at 5 mM
\end{itemize}

\section{blood vessel model}
\begin{itemize}
\item Chapter 1 - Physiology and Pathophysiology
David Sidebotham, Ian J. Le Grice state that for 100 µm we have 0.1cm/s -> 100 µm/s ->6000 µm/min
\end{itemize}

\section{DIPG metabolism \& model}
\begin{itemize}
\item
"The aforementioned is supported by the notion that cancer cells display high metabolic plasticity and can alter their metabolic phenotypes under various selection pressures [53]. A recent study by Shen et al. 2020 reported that although pediatric HGGs exhibit a glycolytic phenotype with reduced reliance on mitochondrial OXPHOS, inhibition of glycolysis by PDK1 inhibitor Dichloroacetate, stimulated OXPHOS by increasing PDH activity [54]. Another study found that various glioma cell lines had high dependency on mitochondrial OXPHOS for ATP production [55], and that glioma cells with preference for aerobic glycolysis could transition to OXPHOS under glucose deprived conditions [56]."\cite{Mudassar2020}
\item The study on Papandreou clearly shows OCR goes from 2.5 in both human and murine cells to  1-1.5 nmol/min/Mcells
\item Demuth \cite{Demuth2000} reports 0 to 24 µm/h of velocity for glioma cells which is also reported by Sengul \cite{Sengul2021} and Friedl \cite{Friedl2003} 

\item Water has a limited ability to dissolve and transport oxygen, having an approximate oxygen permeability of 80 Dk units (Fatt 1986). 
\end{itemize}
\newpage
\bibliographystyle{unsrt}

\bibliography{/home/antonybazir/Documents/Post-doc/Redac/biblio_synthese}

\end{document}