\documentclass[11pt,a4paper]{article}
%\usepackage[utf8]{inputenc}
%\usepackage[ascii]{inputenc}
\usepackage{geometry}
\usepackage[dvipsnames]{xcolor}
\usepackage{textcomp}
\usepackage{graphicx}
\usepackage{caption}
\usepackage{subcaption}
\usepackage{amsmath}

\begin{document}


**The Paradox of Metabolism in Quiescent Stem Cells A coller
"Subsequent studies revealed that mitogen stimulation of human lymphocytes [5], mouse lymphocytes [6], and rat thymocytes [7, 8] all result in both increased glucose uptake and more excretion of lactate." -> So more glycolysis in proliferative cells than in quiescent ones"
"When mouse hematopoietic cells or lymphocytes were not dividing, they exhibited little glucose uptake, performed reduced amounts of glycolysis, secreted less lactate, and instead, relied on oxidative phosphorylation as their major source of energy [9]. When stimulated to divide in response to growth factors or cytokines, mouse hematopoietic cells and lymphocytes exhibited a surprisingly strong shift to increased glucose consumption and elevated rate of glycolysis [9]."

MAIS 

"[Adult Stem cells] metabolic pattern seems to contradict the pattern and principles developed for fibroblasts, endothelial cells and lymphocytes in which increased glycolysis is associated with more proliferation."
"For quiescent hematopoietic stem cells, a reliance on anaerobic glycolysis in a hypoxic environment is thought to provide the benefit of shielding them from reactive oxygen species."



-Yalamancili reports a 25\% percent decrease in ATP level in quiescent human fibroblast compared to proliferating ones.

**Mitochondrial respiration supports autophagy to provide stress resistance during quiescence 
- "and quiescent cells are better equipped to withstand oxidative insult"
-"OXPHOS-mediated ROS protection is specific to quiescent cells"
-"Oxidative stress resistance during quiescence is provided by OXPHOS-stimulated autophagy"

**Targeting Mitochondrial Function to Treat Quiescent Tumor Cells in Solid Tumors
-fig 2 suggest one other main factor behind quiescence may well be hypoxia

**A facile in vitro platform to study cancer cell dormancy under hypoxic microenvironments using CoCl2
-Immediate response of cells in culture to hypoxia
-recovery occurs in 48h 
- cells survive for more than three weeks

**A slow-cycling/quiescent cells subpopulation is involved in glioma invasiveness
-Using a genetic tool to visualize and ablate quiescent cells in mouse brain cancer and human cancer
organoids, we reveal their localization at both the core and the edge of the tumors, and we demonstrate that quiescent cells are involved in infiltration of brain cancer cells.

-Cytokine stimulation of aerobic glycolysis in hematopoietic cells exceeds proliferative demand
"As growth factor availability increases, glycolytic metabolism rises in excess of apparent cellular demand, resulting in the accumulation of unused energy equivalents both in the cytosol and as secreted metabolic products in the medium"

-Towards a Framework for Better Understanding of Quiescent Cancer Cells
-"Prior to the “restriction point” in early G1 phase, mitogens, such as insulin-like growth factor-1 (IGF-1) [19] and platelet-derived growth factor (PDGF) [20], are removed in serum-starved cells and cause cells to exit the cell cycle [3]."

"When the oxygen supply is insufficient, tumor cells survive by entering the G0 phase and may return to the G1 phase when oxygen is replenished [10,15]. Cancer cell lines can be experimentally exposed to hypoxia [8,15,27,28] through the use of a hypoxia chamber [8,15,26,27,28] or treatment with CoCl2 (concentration ranges from 100 μM to 500 μM) [15]. The degree of hypoxia in hypoxia chamber is severe, typically 1\% oxygen [26,27,28] or 0.1\% oxygen [15] for 5 [26] to 14 days [27]. ."

- Skp2 dictates cell cycle-dependent metabolic oscillation between glycolysis and TCA cycle
"This may be in part due to ATP needs, as cells rely mostly on the TCA cycle during G1 phase while switching to glycolysis, a less economic form of ATP production, to accumulate intermediate metabolites that are used as building blocks to synthesize biomacromolecules for subsequent DNA replication and cytokinesis.19"

-Effects of Glucose Deprivation on ATP and Proteoglycan Production of Intervertebral Disc Cells under Hypoxia
ATP content decreases by 2/3 in non cancer porcine cells when glucose falls from 5 mM to 1.25 mM (and 0.5 mM)  and also when oxygen in decreased from 20\% to 5\%

-Measuring and modeling energy and power consumption in living microbial cells with a synthetic ATP reporter
in bacteria, ATP consumption in maintenance has been found to be 1/10th of the consumption associated to growth

-Staying alive Valcourt2012

-Unraveling the Big Sleep: Molecular Aspects of Stem Cell Dormancy and Hibernation : Dias 2021 -> Cells in quiescence rely on autophagy and anaerobic glycolysis 

ATP threshold research
-Thresholds for cellular disruption and activation of the stress response in renal epithelia "Reducing cellular ATP below 50\% control consistently activated HSF"

-Graded ATP depletion can cause necrosis or apoptosis of cultured mouse proximal tubular cells
Wilfred Lieberthal
". We found that cells subjected to ATP depletion below ∼15\% of control died uniformly of necrosis."
"In contrast, cells subjected to ATP depletion between ∼25 and 70\% of control all died by apoptosis. The rapidity of cell death was proportional to the severity of reduction of cell ATP content and was independent of the mechanism of cell death."
"narrow range of ATP depletion (∼15 to 25\% of control) representing a threshold that determines whether cells die by necrosis or apoptosis."

“Wages of Fear”: transient threefold decrease in intracellular ATP level imposes apoptosis
-We have found that transient threefold decrease in [ATP] followed by recovery of the ATP level induces apoptosis accompanied by Bax translocation to mitochondria,...

Conclusion to this : The "behavior" of cells (what threhsold and dynamics) is not well studied 

**The tumour ecology of quiescence: Niches across scales of complexity 
-"and interestingly GBM quiescence can be induced via autophagy during glucose starvation triggering chemoresistance through orchestration of cell metabolism, cell cycle, and fitness [117]. Hence, metabolic plasticity in a low glucose environment can elicit immune escape and shift towards a quiescent cell state."

**High-pressure oxygen rewires glucose metabolism of patient-derived glioblastoma cells and fuels inflammasome response

**A slow-cycling/quiescent cells subpopulation is involved in glioma invasiveness
-"Here, we identify a population of malignant cells expressing Prominin-1 in a non-proliferating state in pediatric high-grade glioma patients. Using a genetic tool to visualize and ablate quiescent cells in mouse brain cancer and human cancer organoids, we reveal their localization at both the core and the edge of the tumors, and we demonstrate that quiescent cells are involved in infiltration of brain cancer cells. "

**Cancerous stem cells can arise from pediatric brain tumors
-There are neural stem cells in pediatric brain tumors

**Glioma Stem Cells in Pediatric High-Grade Gliomas: From Current Knowledge to Future Perspectives
-"Recently, genetic barcoding of freshly isolated adult GBM cells transplanted into PDOX mouse models further provided evidence for a conserved proliferative hierarchy, in which slow-cycling tumor stem-like cells give birth to a quick cycling, self-renewing, progenitor-like population"

**New in vivo avatars of diffuse intrinsic pontine gliomas (DIPG)
from stereotactic biopsies performed at diagnosis
-"Similarly,both cycling and non-cycling tumor cells (i.e. MIB-1
negative) were identified in the models as in the patient
tumor."


**Selective cell cycle arrest in glioblastoma cell lines by quantum molecular resonance alone or in combination with temozolomide
-"indicating cell cycle arrest in the G2/M phase" but only with temozolomide and QMR which we probably won't have.

wiki 
G1 phase : nutrient
This S phase block induces apoptosis in HeLa cells

**Putting the brakes on the cell cycle: mechanisms of cellular growth arrest
-"A point termed the “restriction point” was identified where cells will progress through to division even if growth factors are subsequently removed [8]."
-"While cells coming out of quiescence have a G1 restriction point, recent work using single-cell imaging has challenged whether this model is true in continuously cycling cells. Following mitosis, cells either enter a reversible, CDK2-activity low, mitogen-sensitive state or immediately begin to increase CDK2 activity, entering G1 in a mitogen-insensitive state [12]."
-"There are clear differences in the regulation of G1 in cells coming out of quiescence compared to continuously cycling cells."

**Cell cycle checkpoint signaling:: Cell cycle arrest versus apoptosis
-non pas plus clair
\end{document}