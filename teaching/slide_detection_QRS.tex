\documentclass{beamer}

\usepackage{epsfig}
\usepackage{multicol}
\usepackage{geometry}
%\usepackage[dvipsnames]{xcolor}
\usepackage{textcomp}
\usepackage{graphicx}
\usepackage{caption}
\usepackage{subcaption}
\usepackage{amsmath}
\usepackage{tcolorbox}
\usetheme{Boadilla}
\usepackage{pict2e}
\usepackage{tikz}
\usepackage{xcolor}

\title[Traitement du signal numérique]{Traitement du signal numérique - HEI4 IMS}
\author[Antony Bazir]{}

\setlength{\unitlength}{1cm}

\begin{document}

\section{Détection complexe QRS}
\begin{frame}
\frametitle{Filtrage numérique :  Extraction du complexe QRS}
\begin{center}
\includegraphics[scale=0.35]{tr_num_ecg.png}\\
\textit{\footnotesize Algorithme de détection du complexe QRS, Tompkins, 2000}\\
\end{center}
Filtrage temps réel du signal numérique \& Extraction du complexe QRS\\
\vspace{0.3cm}
Pourquoi extraire le complexe QRS ?\\
\vspace{0.2cm}
\begin{itemize}
\item suivi de la condition du patient en temps réel 
\item Etablissement des tachogrammes (RR)
\item \'A la base de beacoup d'algorithmes acutels...
\end{itemize}

\end{frame}

\begin{frame}
\frametitle{Filtrage numérique :  Extraction du complexe QRS}
\textbf{Caractéristiques du complexe QRS}
\begin{center}
\includegraphics[scale=0.35]{QRS.png}\\

\end{center}
\underline{Origine physique} : dépolarisation des ventricules\\
\vspace{0.3cm}
Caractéristiques: \\
\vspace{0.1cm}
\begin{itemize}
\item Durée :  80 -100 ms
\item Fréquence : 10-15 Hz 
\end{itemize}

\end{frame}

\subsection{Filtrage passe-bande}
\begin{frame}
\frametitle{Filtrage numérique :  Extraction du complexe QRS}
\begin{center}
\includegraphics[scale=0.3]{ECG_spectrum.png}\\
\textit{\scriptsize Spectre en amplitude de l'ECG, Tompkins,2000}\\
\vspace{0.1cm}
\end{center}
Importance du filtrage passe-bande...
\end{frame}

\begin{frame}
\frametitle{Filtrage numérique :  Extraction du complexe QRS}
\begin{center}
\includegraphics[scale=0.3]{ECG_spectrum.png}\\
\textit{\scriptsize Spectre en amplitude de l'ECG, Tompkins,2000}\\
\vspace{0.1cm}
\end{center}
Comment choisir $T_e$ ? \only<2-> {Shannon $\rightarrow f_e > 80 Hz$  \only<3->{ Dans les faits, $f_e$ 500 Hz} }


\end{frame}

\begin{frame}
\frametitle{Détection QRS: Filtrage passe-bande}
\begin{center}
\includegraphics[scale=0.35]{tr_num_ecg.png}\\
\textit{\footnotesize Algorithme de détection du complexe QRS, Tompkins, 2000}\\
\vspace{0.3cm}
\end{center}
Filtrage temps réel du signal numérique \& Extraction du QRS
\begin{enumerate}
\item \textbf{filtrage passe-bande numérique}
\item Dérivée 
\item Mise au carré 
\item Moyennage
\end{enumerate}
\end{frame}

\begin{frame}
\frametitle{Détection QRS: Filtrage passe-bande}
 \textbf{Filtrage passe-bande numérique}\\
 \vspace{0.3 cm}
 Plusieurs méthodes possibles (liste non-exhasutive) :\\
 \vspace{0.2cm}
 \begin{itemize}
 \item Filtre récursif à  2 pôles
 \vspace{0.2cm}
 \item Filtre à coefficients "entiers"
 \vspace{0.2cm}
 \end{itemize}
\end{frame}

\begin{frame}
\frametitle{Détection QRS: Filtrage passe-bande}
 \textbf{Filtrage passe-bande numérique: Filtre récursif à  2 pôles}\\
 \vspace{0.3 cm}
 Forme possible de la fonction de transfert :
 \[H(z) = \frac{c_1}{z-p_1} + \frac{c_2}{z-p_2} \]\\
  \vspace{0.3 cm}
  \'Equation aux différences :
 \[y[n] = b_1 y[n-1] - b_2 y[n-2] + x[n] - x[n-2] \]  
\end{frame}

\begin{frame}
\frametitle{Détection QRS: Filtrage passe-bande}
 \textbf{Filtrage passe-bande numérique: Filtre récursif à  2 pôles}\\

  Forme de l'équation aux différences :
 \[y[n] = b_1 y[n-1] - b_2 y[n-2] + x[n] - x[n-2] \]  \\
  \vspace{0.3 cm}
  Fonction de transfert :
   \[H(z) = \frac{1 - z^{-2}}{1 - b_1 z^{-1} + b_2 z^{-2}} \]  \\
  \vspace{0.3cm}
 \textbf{Comment se comporte ce filtre ?}
\end{frame}

\begin{frame}
\frametitle{Détection QRS: Filtrage passe-bande}
 \textbf{Filtrage passe-bande numérique: Filtre récursif à  2 pôles}\\

  \vspace{0.3 cm}
  Fonction de transfert :
   \[H(z) = \frac{1 - z^{-2}}{1 - b_1 z^{-1} + b_2 z^{-2}} \]  \\
  \vspace{0.3cm}
 On prend les valeurs données par Tompkins:
 \begin{itemize}
 \item $f_e$ = 500 Hz
 \item $b_1$ = 1.875
 \item $b_2$ = 0.9219
 \end{itemize}
\end{frame}

\begin{frame}
\frametitle{Détection QRS: Filtrage passe-bande}
 \textbf{Filtrage passe-bande numérique: Filtre récursif à  2 pôles}\\

  \vspace{0.3 cm}
  Fonction de transfert :
   \[H(z) = \frac{1 - z^{-2}}{1 - 1.875 z^{-1} + 0.9219 z^{-2}} \]  \\
  \vspace{0.3cm}
   \[H(z) = \frac{1 - e^{-4 j \pi \frac{f}{500}}}{1 - 1.875 e^{-2 j \pi \frac{f}{500}} + 0.9219  e^{-4 j \pi \frac{f}{500}}} \] 

\end{frame}

\begin{frame}
\frametitle{Détection QRS: Filtrage passe-bande}
 \textbf{Filtrage passe-bande numérique: Filtre récursif à  2 pôles}\\
\[H(z) = \frac{1 - e^{-4 j \pi \frac{f}{500}}}{1 - 1.875 e^{-2 j \pi \frac{f}{500}} + 0.9219  e^{-4 j \pi \frac{f}{500}}} \] 
\begin{center}
\begin{tikzpicture}
\begin{scope}[scale=2.5]
\draw[->] (0,0)-- (1.1,0)node[right] {\scriptsize $f$} ;
\draw[->] (0,0)-- (0,1.2)node[above] {\scriptsize  $|H(f)|$};
\draw[thick,blue] plot file {module_passebande1.txt};

\draw[dashed] (0.07,0) node[below]{\scriptsize 17 Hz}--(0.07,1);
\draw[dashed] (1,0) node[below]{\scriptsize 250 Hz}--(1,0);
\end{scope}

\end{tikzpicture}
\end{center}
\vspace{0.3cm}
On sélectionne bien dans une gamme de fréquence autour du QRS
\end{frame}

\begin{frame}
\frametitle{Détection QRS: Filtrage passe-bande}
 \textbf{Filtrage passe-bande numérique: Filtre récursif à  2 pôles}\\
\begin{columns}
\column{60mm}
\[H(z) = \frac{1 - e^{-4 j \pi \frac{f}{500}}}{1 - 1.875 e^{-2 j \pi \frac{f}{500}} + 0.9219  e^{-4 j \pi \frac{f}{500}}} \]
\column{60mm} 
\begin{center}
\begin{tikzpicture}
\begin{scope}[scale=2.5]
\draw[->] (0,0)-- (1.1,0)node[right] {\scriptsize $f$} ;
\draw[->] (0,0)-- (0,1.2)node[above] {\scriptsize  $|H(f)|$};
\draw[thick,blue] plot file {module_passebande1.txt};

\draw[dashed] (0.07,0) node[below]{\scriptsize 17 Hz}--(0.07,1);
\draw[dashed] (1,0) node[below]{\scriptsize 250 Hz}--(1,0);
\end{scope}

\end{tikzpicture}
\end{center}
\end{columns}
\vspace{0.3cm}
Remarques :
\begin{itemize}
\item Le numérateur sert à couper les basses fréquences
\item Coefficients optimisées pour augmenter la vitesse de calcul
\end{itemize}
\end{frame}

\begin{frame}
\frametitle{Détection QRS: Filtrage passe-bande}
 \textbf{Filtrage passe-bande numérique: Filtre récursif à  2 pôles}\\
\begin{columns}
\column{60mm}
\[H(z) = \frac{1 - e^{-4 j \pi \frac{f}{500}}}{1 - 1.875 e^{-2 j \pi \frac{f}{500}} + 0.9219  e^{-4 j \pi \frac{f}{500}}} \]
\column{60mm} 
\begin{center}
\begin{tikzpicture}
\begin{scope}[scale=2.5]
\draw[->] (0,0)-- (1.1,0)node[right] {\scriptsize $f$} ;
\draw[->] (0,-0.6)-- (0,0.6)node[above] {\scriptsize  $\text{arg}(H(f))$};
\draw[thick,blue] plot file {argument_passebande1.txt};

\draw[dashed] (0.07,0) node[below]{\scriptsize 17 Hz}--(0.07,0);
\draw[dashed] (1,0) node[below]{\scriptsize 250 Hz}--(1,0);
\end{scope}

\end{tikzpicture}
\end{center}
\end{columns}
Problèmes :
\begin{itemize}
\item Phase fortement non linéaire
\item Distorsion de phase importante au niveau de la bande d'intérêt
\end{itemize}
\only<2->{
On utilise un autre type de filtre...
}
\end{frame}




\begin{frame}
\frametitle{Détection QRS: Filtrage passe-bande}
 \textbf{Filtrage passe-bande numérique: Filtre à coeffs entiers}\\
 \vspace{0.2cm}
 \underline{Objectif} : Gagner du temps de calcul en prenant des coeffs entiers ET garder une phase linéaire (informatique) \\
 \vspace{0.3cm}
 Forme générale : 
 \[ H(z) = \frac{(1 -z^{-m})^p}{(1 - 2 \cos(\theta)z^{-1} + z^{-2})^t} \] \\
 \vspace{0.2cm}
 \begin{itemize}
 \item  $m$,$p$ et $t$ entiers positifs
 \item $\theta$ idéalement 0, $\pi/3$, $\pi/2$, $2\pi/3$ ou $\pi$
 \item Permet "d'annuler" des zéros
 \end{itemize}
\end{frame}

\begin{frame}
\frametitle{Détection QRS: Filtrage passe-bande}
 \textbf{Filtrage passe-bande numérique: Filtre à coeffs entiers}\\
 \vspace{0.2cm}
 Forme générale : 
 \[ H(z) = \frac{(1 -z^{-m})^p}{(1 - 2 \cos(\theta)z^{-1} + z^{-2})^t} \] \\
 \vspace{0.2cm}
 Mise en oeuvre pour la détection QRS :
 \begin{itemize}
 \item  Passe-haut et passe-bas en cascade
 \item \textbf{Phase linéaire} $\rightarrow$ pas de distorsion de phase
 \end{itemize}
 \vspace{0.2cm}
 \textbf{Important de ne pas trop déformer l'ECG pendant le traitement...}
\end{frame}

\begin{frame}
\frametitle{Détection QRS: Filtrage passe-bande}
 \textbf{Passe-bas à coeff entiers}\\
  \vspace{0.2cm}
   \[ H_{pb}(z) = \frac{(1 -z^{-6})^2}{(1 - z^{-1} )^2} \]
   \vspace{0.1cm}
   \begin{columns}
	\column{60mm}
	\begin{center}
	\begin{tikzpicture}
	\begin{scope}[scale=3]
\draw[->] (0,0)-- (1.1,0)node[right] {\scriptsize $f$} ;
\draw[->] (0,0)-- (0,1.2)node[above] {\scriptsize  $|H(f)|$};
\draw[thick,blue] plot file {module_passebas.txt};

\draw[dashed] (0.33,0) node[below]{\scriptsize 33 Hz}--(0.33,0);
\draw[dashed] (1,0) node[below]{\scriptsize 100 Hz}--(1,0);
	\end{scope}
	\end{tikzpicture}
	\end{center}
	\column{60mm}  
		\begin{center}
	\begin{tikzpicture}
	\begin{scope}[scale=3]
\draw[->] (0,0)-- (1.1,0)node[right] {\scriptsize $f$} ;
\draw[->] (0,-0.6)-- (0,0.6)node[above] {\scriptsize  $\text{arg}(H(f))$};
\draw[thick,blue] plot file {argument_passebas.txt};
	\end{scope}
	\end{tikzpicture}
	\end{center} 
   
   \end{columns}
  \end{frame}
  
  \begin{frame}
\frametitle{Détection QRS: Filtrage passe-bande}
 \textbf{Passe-haut à coeff entiers}\\
  \vspace{0.2cm}
   \[ H_{ph}(z) = z^{-18} - \frac{1 -z^{-36}}{36(1 - z^{-1})} \]
   \vspace{0.1cm}
   \begin{columns}
	\column{60mm}
	\begin{center}
	\begin{tikzpicture}
	\begin{scope}[scale=3]
\draw[->] (0,0)-- (1.1,0)node[right] {\scriptsize $f$} ;
\draw[->] (0,0)-- (0,1.2)node[above] {\scriptsize  $|H(f)|$};
\draw[thick,blue] plot file {module_passehaut.txt};

\draw[dashed] (0.09,0) node[below]{\scriptsize 9 Hz}--(0.09,1.2);
\draw[dashed] (1,0) node[below]{\scriptsize 100 Hz}--(1,0);
	\end{scope}
	\end{tikzpicture}
	\end{center}
	\column{60mm}  
		\begin{center}
	\begin{tikzpicture}
	\begin{scope}[scale=3]
\draw[->] (0,0)-- (1.1,0)node[right] {\scriptsize $f$} ;
\draw[->] (0,-0.6)-- (0,0.6)node[above] {\scriptsize  $\text{arg}(H(f))$};
\draw[thick,blue] plot file {argument_passehaut.txt};
	\end{scope}
	\end{tikzpicture}
	\end{center} 
   
   \end{columns}
  \end{frame}
  
  
    \begin{frame}
\frametitle{Détection QRS: Filtrage passe-bande}
 \textbf{Passe-bande à coeff entiers}\\
  \vspace{0.2cm}
   \[ H(z) = H_{pb}(z) \times H_{ph}(z) \]
   \vspace{0.1cm}
   \begin{columns}
	\column{60mm}
	\begin{center}
	\begin{tikzpicture}
	\begin{scope}[scale=3]
\draw[->] (0,0)-- (1.1,0)node[right] {\scriptsize $f$} ;
\draw[->] (0,0)-- (0,1.2)node[above] {\scriptsize  $|H(f)|$};
\draw[thick,blue] plot file {module_passebande.txt};

\draw[dashed] (0.09,0) node[below]{\scriptsize 9 Hz}--(0.09,1);
\draw[dashed] (1,0) node[below]{\scriptsize 100 Hz}--(1,0);
\draw[dashed] (0.33,0) node[below]{\scriptsize 33 Hz}--(0.33,0);
\draw[dashed] (1,0) node[below]{\scriptsize 100 Hz}--(1,0);
	\end{scope}
	\end{tikzpicture}
	\end{center}
	\column{60mm}  
		\begin{center}
	\begin{tikzpicture}
	\begin{scope}[scale=3]
\draw[->] (0,0)-- (1.1,0)node[right] {\scriptsize $f$} ;
\draw[->] (0,-0.6)-- (0,0.6)node[above] {\scriptsize  $\text{arg}(H(f))$};
\draw[thick,blue] plot file {argument_passebande.txt};
	\end{scope}
	\end{tikzpicture}
	\end{center} 
   
   \end{columns}
   \vspace{0.3cm}
   \only<2->{
   \textbf{Appliquons ce premier filtre à un ECG}
   }
  \end{frame}
  
   \begin{frame}
\frametitle{Détection QRS: Filtrage passe-bande}
 \textbf{Signal ECG} échantillonné à 200 Hz par Tompkins\\
 \vspace{0.3cm}
 	\begin{center}
 	\begin{tikzpicture}
	\begin{scope}[scale=3]
\draw[->] (0,0)-- (2.2,0)node[right] {\scriptsize $t$} ;
\draw[->] (0,0)-- (0,1.2)node[above] {\scriptsize  $s(t)$};
\draw[thick,blue] plot file {signal_ECG.txt};
\draw[dashed] (2,0) node[below]{\scriptsize 0.43 s}--(2,0);

\draw[dashed] (0.12,0.3) node[below]{\scriptsize P}--(0.12,0.3);

\draw[dashed] (0.66,-0.4) node[below]{\scriptsize QRS}--(0.66,-0.4);

\draw[dashed] (1.6,0.8) node[below]{\scriptsize T}--(1.6,0.8);

	\end{scope}
	\end{tikzpicture}
	\end{center}
 \only<2->{
 On retrace le spectre en fréquence du signal numérisé 
 }
 \end{frame} 
 
    \begin{frame}
\frametitle{Détection QRS: Filtrage passe-bande}
 \textbf{Spectre du Signal ECG numérique}\\
 \vspace{0.3cm}
 	\begin{center}
 	\begin{tikzpicture}
	\begin{scope}[scale=3]
\draw[->] (0,0)-- (2.2,0)node[right] {\scriptsize $f$} ;
\draw[->] (0,0)-- (0,1.2)node[above] {\scriptsize  $|\tilde{S}(f)|$};
\draw[thick,blue] plot file {module_ECG.txt};

\draw[dashed] (2,0) node[below]{\scriptsize 100 Hz}--(2,0);

\draw[dashed] (1,0) node[below]{\scriptsize 50 Hz}--(1,0);

\draw[dashed] (0.5,0) node[below]{\scriptsize 25 Hz}--(0.5,0);

\draw[dashed] (0.1,0) node[below]{\scriptsize 10 Hz}--(0.1,0);
	\end{scope}
	\end{tikzpicture}
	\end{center}

 \end{frame} 
 
 \begin{frame}
 \frametitle{Détection QRS: Filtrage passe-bande}
 On applique maintenant le filtre passe-bande au signal \\
 
\vspace{0.2cm}

Comment procéder ? \\
\vspace{0.2cm}
 \begin{itemize}
 \item Coder l'équation aux différences 
 \vspace{0.1cm}
 \item Réaliser la convolution dans le domaine temporel 
 \vspace{0.1cm}
 \item Faire le produit du spectre et du signal dans le domaine fréquentiel
 \end{itemize}
 \vspace{0.2cm}
 \only<2->{
\textbf{ Méthode la plus directe : Coder l'équation aux différences}
 }
 \end{frame}
 
  \begin{frame}
 \frametitle{Détection QRS: Filtrage passe-bande}
 \'Equations aux différences: \\
 \vspace{0.1cm}
  \[H_{pb}(z) = \frac{(1 -z^{-6})^2}{(1 - z^{-1} )^2}\] \[  \rightarrow  y[n] =  2y[n-1]- y[n-2] + x[n] - 2x[n-6] + x[n-12] \]\\
  \vspace{0.2cm}
  
  \[ H_{ph}(z) = z^{-36} - \frac{1 -z^{-36}}{36(1 - z^{-1})}\]
  \[ \rightarrow y[n] = x[n-18] -\frac{1}{36}[y[n-1] + x[n] - x[n-36] ]  \]\\
 \vspace{0.2cm}
 \only<2->{
 On les applique tour à tour au signal ECG
 }
  \end{frame}
  
  \begin{frame}
 \frametitle{Détection QRS: Filtrage passe-bande} 
 Résultat: 
 \begin{columns}
 \column{55mm}
  	\begin{center}
 	\begin{tikzpicture}
	\begin{scope}[scale=2]
\draw[->] (0,0)-- (2.2,0)node[right] {\scriptsize $t$} ;
\draw[->] (0,0)-- (0,1.2)node[above] {\scriptsize  $s(t)$};
\draw[thick,blue] plot file {signal_ECG.txt};
\draw[dashed] (2,0) node[below]{\scriptsize 0.43 s}--(2,0);

\draw[dashed] (0.12,0.4) node[below]{\scriptsize P}--(0.12,0.4);

\draw[dashed] (0.66,-0.4) node[below]{\scriptsize QRS}--(0.66,-0.4);

\draw[dashed] (1.6,0.8) node[below]{\scriptsize T}--(1.6,0.8);

%\draw[->] (2.4,0.6)--(3,0.6);

	\end{scope}
	\end{tikzpicture}
	\end{center}
	
	\column{10mm}
	  	\begin{center}
 	\begin{tikzpicture}
	\begin{scope}[scale=2]


\draw[->] (2.4,0.7)--(3,0.7);

	\end{scope}
	\end{tikzpicture}
	\end{center}
 
 \column{55mm}
   	\begin{center}
 	\begin{tikzpicture}
 	\begin{scope}[scale=2]
\draw[->] (0,0)-- (2.2,0)node[right] {\scriptsize $t$} ;
\draw[->] (0,0)-- (0,1.2)node[above] {\scriptsize  $s_1(t)$};
\draw[thick,blue] plot file {bp_sig_ECG.txt};
\draw[dashed] (2,-0.4) node[below]{\scriptsize 0.43 s}--(2,-0.4);

%\draw[dashed] (0.12,0.3) node[below]{\scriptsize P}--(0.12,0.3);

\draw[dashed] (1,-0.5) node[below]{\scriptsize QRS}--(1,-0.5);

%\draw[dashed] (1.6,0.8) node[below]{\scriptsize T}--(1.6,0.8);

	\end{scope}
	\end{tikzpicture}
	\end{center}
 \end{columns}
 \vspace{0.2cm}
 Le filtrage passe-bande "accentue" le complexe QRS dans le signal
 \end{frame}
 
\begin{frame}
\frametitle{Détection QRS: Dérivation}
\begin{center}
\includegraphics[scale=0.35]{tr_num_ecg.png}\\
\textit{\footnotesize Algorithme de détection du complexe QRS, Tompkins, 2000}\\
\vspace{0.3cm}
\end{center}
Filtrage temps réel du signal numérique \& Extraction du QRS
\begin{enumerate}
\item Filtrage passe-bande numérique
\item \textbf{Dérivée} 
\item Mise au carré 
\item Moyennage
\end{enumerate}
\end{frame}

\begin{frame}
\frametitle{Détection QRS: Dérivation}
\textbf{Pourquoi un dérivateur ?}
   	\begin{center}
 	\begin{tikzpicture}
 	\begin{scope}[scale=2]
\draw[->] (0,0)-- (2.2,0)node[right] {\scriptsize $t$} ;
\draw[->] (0,0)-- (0,1.2)node[above] {\scriptsize  $s_1(t)$};
\draw[thick,blue] plot file {bp_sig_ECG.txt};
\draw[dashed] (2,-0.4) node[below]{\scriptsize 0.43 s}--(2,-0.4);

%\draw[dashed] (0.12,0.3) node[below]{\scriptsize P}--(0.12,0.3);

\draw[dashed] (1,-0.5) node[below]{\scriptsize QRS}--(1,-0.5);

%\draw[dashed] (1.6,0.8) node[below]{\scriptsize T}--(1.6,0.8);

	\end{scope}
	\end{tikzpicture}
	\end{center}
	Le complexe QRS semble correspondre aux zones de pente maximale dans le signal...

\end{frame}
\vspace{0.3cm}

\subsection{Dérivation}
\begin{frame}
\frametitle{Détection QRS: Dérivation}
On implémente un dérivateur à 5 points 
\[ y[n] = \frac{2 x[n] + x[n-1] - x[n-3] -2 x[n-4]}{6} \] 

\[H(z) = \frac{1}{6}[2 + z^{-1} - z^{-3} -2z^{-4}] \] 

\begin{columns}
\column{60mm}
   	\begin{center}
 	\begin{tikzpicture}
 	\begin{scope}[scale=2]
\draw[->] (0,0)-- (2.2,0)node[right] {\scriptsize $f$} ;
\draw[->] (0,0)-- (0,1.2)node[above] {\scriptsize  $|H(f)|$};
\draw[thick,blue] plot file {module_deriv.txt};
\draw[dashed] (2,0) node[below]{\scriptsize 100 Hz}--(2,0);

	\end{scope}
	\end{tikzpicture}
	\end{center}
\column{60mm}
   	\begin{center}
 	\begin{tikzpicture}
 	\begin{scope}[scale=2]
\draw[->] (0,0)-- (2.2,0)node[right] {\scriptsize $f$} ;
\draw[->] (0,-0.6)-- (0,0.6)node[above] {\scriptsize  $\text{arg}(H(f))$};
\draw[thick,blue] plot file {argument_deriv.txt};
\draw[dashed] (2,0) node[below]{\scriptsize 100 Hz}--(2,0);

	\end{scope}
	\end{tikzpicture}
	\scriptsize Phase linéaire
	\end{center}
\vspace{0.1cm} 

\end{columns}
\end{frame}

\begin{frame}
\frametitle{Détection QRS: Dérivation}
On code l'équation aux différences
\[ y[n] = \frac{2 x[n] + x[n-1] - x[n-3] -2 x[n-4]}{6} \] 
\vspace{0.1cm}
Résultat:\\
\vspace{0.1cm}
   	\begin{center}
 	\begin{tikzpicture}
 	\begin{scope}[scale=2]
\draw[->] (0,0)-- (2.2,0)node[right] {\scriptsize $f$} ;
\draw[->] (0,-0.5)-- (0,0.5)node[above] {\scriptsize  $|H(f)|$};
\draw[thick,blue] plot file {deriv_sig_ECG.txt};
\draw[dashed] (2,0) node[below]{\scriptsize 0.43 s}--(2,0);
\draw[dotted,blue] plot file {bp_sig_ECG.txt};

	\end{scope}
	\end{tikzpicture}
	\end{center} 
Le filtre "réagit" aux régions à forte pente...
\end{frame}

\subsection{Mise au carré}


\begin{frame}
\frametitle{Détection QRS: Mise au carré}
\begin{center}
\includegraphics[scale=0.35]{tr_num_ecg.png}\\
\textit{\footnotesize Algorithme de détection du complexe QRS, Tompkins, 2000}\\
\vspace{0.3cm}
\end{center}
Filtrage temps réel du signal numérique \& Extraction du QRS
\begin{enumerate}
\item Filtrage passe-bande numérique
\item Dérivée 
\item \textbf{Mise au carré }
\item Moyennage
\end{enumerate}
\end{frame}

\begin{frame}
\frametitle{Détection QRS: Mise au carré}
On met le signal précédent au carré... 
\begin{columns}
\column{55mm}
   	\begin{center}
 	\begin{tikzpicture}
 	\begin{scope}[scale=2]
\draw[->] (0,0)-- (2.2,0)node[right] {\scriptsize $t$} ;
\draw[->] (0,-0.5)-- (0,0.5)node[above] {\scriptsize  $s_2(t)$};
\draw[thick,blue] plot file {deriv_sig_ECG.txt};
\draw[dashed] (2,0) node[below]{\scriptsize 0.43 s}--(2,0);
%\draw[dotted,blue] plot file {bp_sig_ECG.txt};

	\end{scope}
	\end{tikzpicture}
	\end{center} 
	\column{10mm}
	  	\begin{center}
 	\begin{tikzpicture}
	\begin{scope}[scale=2]


\draw[->] (2.4,0.7)--(3,0.7);

	\end{scope}
	\end{tikzpicture}
	\end{center}
	\column{55mm}
   	\begin{center}
 	\begin{tikzpicture}
 	\begin{scope}[scale=2]
\draw[->] (0,0)-- (2.2,0)node[right] {\scriptsize $t$} ;
\draw[->] (0,0)-- (0,1)node[above] {\scriptsize  $s_3(t)$};
\draw[thick,blue] plot file {square_sig_ECG.txt};
\draw[dashed] (2,0) node[below]{\scriptsize 0.43 s}--(2,0);
%\draw[dotted,blue] plot file {bp_sig_ECG.txt};

	\end{scope}
	\end{tikzpicture}
	\end{center}
\end{columns}
\vspace{1cm}
On a un signal maximisé au niveau de la QRS...\\
\vspace{0.5cm} \only<2->{Mais étape supplémentaire nécessaire pour éviter les faux positifs}
\end{frame}

\subsection{Moyennage}
\begin{frame}
\frametitle{Détection QRS: Moyennage}
\begin{center}
\includegraphics[scale=0.35]{tr_num_ecg.png}\\
\textit{\footnotesize Algorithme de détection du complexe QRS, Tompkins, 2000}\\
\vspace{0.3cm}
\end{center}
Filtrage temps réel du signal numérique \& Extraction du QRS
\begin{enumerate}
\item Filtrage passe-bande numérique
\item Dérivée 
\item Mise au carré
\item \textbf{Moyennage}
\end{enumerate}
\end{frame}

\begin{frame}
\frametitle{Détection QRS: Moyennage}
Moyennage du signal dérivé sur 32 échantillons:\\
\vspace{0.2cm}
\[ y[n] = \frac{1}{32} [x[n -31] + x[n-30] + x[n-29] + \cdots +  x[n] ]\]

\vspace{0.4cm}
Pourquoi 32 échantillons ?
\begin{itemize}
\item Durée typique d'un complexe QRS : 0.1 s
\vspace{0.2cm}
\item Fréquence d'échantillonnage : 200 Hz 
\vspace{0.2cm}
\item<2-> Durée minimale : 20 échantillons
\vspace{0.2cm}
\item<3-> On souhaite que la valeur max de la fenêtre dure plusieurs échantillons $\rightarrow $ \textbf{32 échantillons} 
\end{itemize}

\end{frame}

\begin{frame}
\frametitle{Détection QRS: Moyennage}
Moyennage du signal dérivé sur 32 échantillons:\\
\vspace{0.2cm}
\[ y[n] = \frac{1}{32} [x[n -31] + x[n-30] + x[n-29] + \cdots +  x[n] ]\]

\vspace{0.4cm}
\begin{columns}
\column{55mm}
   	\begin{center}
 	\begin{tikzpicture}
 	\begin{scope}[scale=2]
\draw[->] (0,0)-- (2.2,0)node[right] {\scriptsize $t$} ;
\draw[->] (0,0)-- (0,1)node[above] {\scriptsize  $s_3(t)$};
\draw[thick,blue] plot file {square_sig_ECG.txt};
\draw[dashed] (2,0) node[below]{\scriptsize 0.43 s}--(2,0);
%\draw[dotted,blue] plot file {bp_sig_ECG.txt};

	\end{scope}
	\end{tikzpicture}
	\end{center} 
	\column{10mm}
	  	\begin{center}
 	\begin{tikzpicture}
	\begin{scope}[scale=2]


\draw[->] (2.4,0.7)--(3,0.7);

	\end{scope}
	\end{tikzpicture}
	\end{center}
	\column{55mm}
   	\begin{center}
 	\begin{tikzpicture}
 	\begin{scope}[scale=2]
\draw[->] (0,0)-- (2.2,0)node[right] {\scriptsize $t$} ;
\draw[->] (0,0)-- (0,1)node[above] {\scriptsize  $s_4(t)$};
\draw[thick,blue] plot file {mvmn_sig_ECG.txt};
\draw[dashed] (2,0) node[below]{\scriptsize 0.43 s}--(2,0);
%\draw[dotted,blue] plot file {bp_sig_ECG.txt};

	\end{scope}
	\end{tikzpicture}
	\end{center}
\end{columns}
\vspace{0.2cm}
	Le signal passe et se maintient à  1 lors de la détection du complexe QRS


\end{frame}

\begin{frame}
\frametitle{Détection QRS: Résumé}
Filtre passant à 1 lors de la détection QRS:
  	\begin{center}
 	\begin{tikzpicture}
	\begin{scope}[scale=0.8]
\draw[->] (0,0)-- (2.2,0)node[right] {\scriptsize $t$} ;
\draw[->] (0,0)-- (0,1.2)node[above] {\scriptsize  $s(t)$};
\draw[thick,blue] plot file {signal_ECG.txt};
\draw (1,-0.8) node[text centered] {\scriptsize ECG};
\end{scope}

 	\begin{scope}[scale=0.5,xshift=6cm]
\draw[->] (0,0)-- (2.2,0)node[right] {\scriptsize $t$} ;
\draw[->] (0,0)-- (0,1.2)node[above] {\scriptsize  $s_1(t)$};
\draw[thick,blue] plot file {bp_sig_ECG.txt};
%\draw[dashed] (2,0) node[below]{\scriptsize 0.43 s}--(2,0);
\draw[dotted,blue] plot file {signal_ECG.txt};
\draw (1,-0.8) node[text centered] {\scriptsize passe-bande};
	\end{scope}
	
	\begin{scope}[scale=0.5,xshift=10.5cm]
\draw[->] (0,0)-- (2.2,0)node[right] {\scriptsize $t$} ;
\draw[->] (0,0)-- (0,1.2)node[above] {\scriptsize  $s_2(t)$};
\draw[thick,blue] plot file {deriv_sig_ECG.txt};
%\draw[dashed] (2,0) node[below]{\scriptsize 0.43 s}--(2,0);
\draw[dotted,blue] plot file {bp_sig_ECG.txt};
\draw (1,-0.8) node[text centered] {\scriptsize dérivée};
	\end{scope}
	
		\begin{scope}[scale=0.5,xshift=15cm]
\draw[->] (0,0)-- (2.2,0)node[right] {\scriptsize $t$} ;
\draw[->] (0,0)-- (0,1.2)node[above] {\scriptsize  $s_3(t)$};
\draw[thick,blue] plot file {square_sig_ECG.txt};
%\draw[dashed] (2,0) node[below]{\scriptsize 0.43 s}--(2,0);
\draw[dotted,blue] plot file {deriv_sig_ECG.txt};
\draw (1,-0.8) node[text centered] {\scriptsize carré};
	\end{scope}
	
			\begin{scope}[scale=0.8,xshift=12cm]
\draw[->] (0,0)-- (2.2,0)node[right] {\scriptsize $t$} ;
\draw[->] (0,0)-- (0,1.2)node[above] {\scriptsize  $s_4(t)$};
\draw[thick,blue] plot file {mvmn_sig_ECG.txt};
%\draw[dashed] (2,0) node[below]{\scriptsize 0.43 s}--(2,0);
\draw[dotted,blue] plot file {square_sig_ECG.txt};
\draw (1,-0.8) node[text centered] {\scriptsize détection};
	\end{scope}
\end{tikzpicture}
\end{center}
\vspace{0.5cm}
Remarques:
\begin{itemize}
\item Délai entre la détection et l'ECG dans le signal
\item Paramètres à rerégler si on change la fréquence d'échantillonnage
\end{itemize}

\end{frame}
\end{document}


