\documentclass[11pt,a4paper]{article}
%\usepackage[utf8]{inputenc}
%\usepackage[ascii]{inputenc}
\usepackage[margin=0.7in]{geometry}
%\usepackage{geometry}
\usepackage[dvipsnames]{xcolor}
\usepackage{textcomp}
\usepackage{graphicx}
\usepackage{caption}
\usepackage{subcaption}
\usepackage{amssymb}
\usepackage{amsmath}
\usepackage{tikz}


\begin{document}
\title{Résumé \& Méthodes chapitre II: Outils Mathématiques}
\maketitle
\section{Transformée de Laplace}
\subsection{Définition}
Transformée de Laplace monolatérale : 
	\[L\{f\}(s) = \displaystyle \int^{\infty}_{0} f(t) e^{-st} \; dt\]\\ 
	\vspace{0.3cm}
Transformée de Laplace bilatérale : \[L\{f\}(s) = \displaystyle \int^{\infty}_{-\infty} f(t) e^{-st} \; dt\]\\

Utile, entre autre, pour analyser des systèmes dynamiques de façon rapide. \\

Rappel :  $s$ est une variable complexe. La transformée de Laplace d'une fonction (réel ou complexe) est une fonction complexe.\\
\vspace{0.15cm}


\subsection{Propriétés}

\begin{itemize}
\item Linéarité:  $L\{af+g\}(s) = a L\{ f\}(s) + L \{ g \} (s)$
\vspace{0.1cm}
\item Dérivation : $ L\{f'\}(s) = \; s L\{f \}(s)$
\vspace{0.1cm}
\item Intégration : $ L\{F\}(s) = \frac{\displaystyle 1}{\displaystyle s} L\{f \}(s) $
\vspace{0.1cm}
\item Convolution : $ L\{f \star g \} = L\{f \}(s) \cdot L \{ g \} (s)  $
\end{itemize}

\vspace{0.15cm}


\subsection{Transformées Usuelles}
\begin{itemize}
\item $u(t)$  $\rightarrow \frac{\displaystyle 1}{\displaystyle p} $
\item $r(t) = t$  $\rightarrow \frac{\displaystyle 1}{ \displaystyle p^2} $
\item $f(t) = e^{-at}$ $\rightarrow \frac{\displaystyle 1}{\displaystyle p+a} $
\item $\delta (t)$ $\rightarrow 1$ 
\item $sin(\omega_0 t)$ = $\rightarrow \frac{\displaystyle \omega_0^2}{ \displaystyle p^2 + \omega_0^2} $
\end{itemize}
\vspace{1cm}
\section{Transformée de Fourier}
\subsection{Définition}
\[TF\{ f \}(\nu) = \displaystyle \int^{\infty}_{-\infty} f(t) e^{-j2\pi \nu t} dt\]

Rappel :  $\nu$ est une fréquence, et donc un réel. La transformée de Fourier d'une fonction est une fonction complexe
\\
\subsection{Lien avec Laplace}
\textbf{La transformée de Fourier} est un cas \textbf{particulier de la transformée de Laplace bilatérale.} En effet $s= \sigma + j \omega$ pour la variable de Laplace en général. La transformée de Fourier correspond au cas où $s = j \omega = j 2 \pi \nu$ avec $\nu$ la fréquence d'excitation.\\

\textbf{La transformée de Fourier a donc les mêmes propriétés que la transformée de Laplace.}
\\
\subsection{Notion de spectre \& Réponse en fréquence}
Si $f(t)$ est une fonction, alors $F(\nu) = TF\{f\}(\nu)$ est appelé \textbf{spectre de la fonction $f$}\\

\fbox{
\textbf{Réponse en fréquence = Module (ampl.) + argument (phase) du spectre } 
}
\\
\subsection{Extension temporelle vs. Extension fréquentielle}
Le point de départ ici est de constater que  :
\[ TF \{ \delta (t) \}  = 1 \]\\
Autrement dit, une fonction non nulle en un point unique de l'axe des temps a pour transformée de Fourier une fonction constante (et donc d'extension infinie) en fréquence.\\

La réciproque est également vraie:
\[ TF^{-1} \{ \delta (\nu) \}  = 1 \]\\

On peut généraliser en disant que \textbf{L'extension temporelle d'une fonction est inversement proportionnelle à son extension fréquentielle} donc,\\


\fbox{
\begin{minipage}{\textwidth}
\begin{center}
\textbf{Signal temporel court = Large spectre}
\begin{tikzpicture}
\draw[->] (-2,0)--(2,0) node[right]{$t$};
\draw[->] (0,-1)--(0,1) node[above]{$f(t)$};
\draw[thick,blue] (-2,0)--(0,0)--(0,1)--(0,0)--(2,0);

\begin{scope}[xshift=6cm]
\draw[->] (-2,0)--(2,0) node[right]{$\nu$};
\draw[->] (0,-1)--(0,1) node[above]{$F(\nu)$};
\draw[thick,blue] (-2,0.5)--(2,0.5);
\end{scope}
\end{tikzpicture}
\end{center}

\begin{center}
 \textbf{Signal temporel long  = Spectre court} 
\begin{tikzpicture}
\draw[->] (-2,0)--(2,0) node[right]{$t$};
\draw[->] (0,-1)--(0,1) node[above]{$f(t)$};
\draw[thick,blue] (-2,0.5)--(2,0.5);

\begin{scope}[xshift=6cm]
\draw[->] (-2,0)--(2,0) node[right]{$\nu$};
\draw[->] (0,-1)--(0,1) node[above]{$F(\nu)$};
\draw[thick,blue] (-2,0)--(0,0)--(0,1)--(0,0)--(2,0);
\end{scope}
\end{tikzpicture}
\end{center}
\end{minipage}
}

\vspace{1cm}
\subsection{Périodicité et échantillonnage}
\vspace{0.5cm}

Une fonction périodique dans le domaine temporel aura un spectre de raies espacées régulièrement et séparées de $\Delta f = 1/T$, où $T$ est la période du signal temporel. \\
\vspace{0.3cm}\\
\fbox{
\begin{minipage}{\textwidth}
\begin{center}
\textbf{Signal périodique = Spectre de raies}
\begin{tikzpicture}
\draw[->] (2.5,0) node[below] {0} -- (2.5,1.5)node[above] {$e(t)$};
\draw[->] (0,0)-- (5,0) node[right] {$t$};

\draw[thick,blue] (2,0)--(2.25,0)--(2.25,1)--(2.75,1)--(2.75,0)--(3,0);

\begin{scope}[xshift=1cm]
\draw[thick,blue] (2,0)--(2.25,0)--(2.25,1)--(2.75,1)--(2.75,0)--(3,0);
\draw[dashed,black] (2.5,0) node[below]{$T$}--(2.5,1);
\end{scope}

\begin{scope}[xshift=2cm]
\draw[thick,blue] (2,0)--(2.25,0)--(2.25,1)--(2.75,1)--(2.75,0)--(3,0);
\end{scope}


\begin{scope}[xshift=-1cm]
\draw[thick,blue] (2,0)--(2.25,0)--(2.25,1)--(2.75,1)--(2.75,0)--(3,0);
\end{scope}

\begin{scope}[xshift=-2cm]
\draw[thick,blue] (2,0)--(2.25,0)--(2.25,1)--(2.75,1)--(2.75,0)--(3,0);
\end{scope}



\begin{scope}[xshift=7cm]
\draw[->] (2.5,0) node[below] {0} -- (2.5,1.5)node[above] {$E(\nu)$};
\draw[->] (0,0)-- (5,0) node[right] {$\nu$};
\draw[ domain=0.1:4.9,color=blue,samples=24] plot[ycomb] (\x,{80*abs(sin(5*3.14*\x r -5*3.14*2.5r )/(5*3.14*\x r-5*3.14*2.5 r))});
\draw[<->] (3.25,-0.3)--(3.45,-0.3) node[below left] {$\frac{1}{T}$};
\end{scope}
\end{tikzpicture}

\begin{tikzpicture}
\draw[->] (2.5,0) node[below] {0} -- (2.5,1.5)node[above] {$e(t)$};
\draw[->] (0,0)-- (5,0) node[below right] {$t$};

\draw[thick,blue] (1,0)--(2.25,0)--(2.25,1)--(2.75,1)--(2.75,0)--(4,0);


\begin{scope}[xshift=2cm]
\draw[thick,blue] (1,0)--(2.25,0)--(2.25,1)--(2.75,1)--(2.75,0)--(3,0);
\end{scope}


\begin{scope}[xshift=-2cm]
\draw[thick,blue] (2,0)--(2.25,0)--(2.25,1)--(2.75,1)--(2.75,0)--(4,0);
\end{scope}



\begin{scope}[xshift=7cm]
\draw[->] (2.5,0) node[below] {0} -- (2.5,1.5)node[above] {$E(\nu)$};
\draw[->] (0,0)-- (5,0) node[right] {$\nu$};
\draw[ domain=0.1:4.9,color=blue,samples=96] plot[ycomb] (\x,{80*abs(sin(5*3.14*\x r -5*3.14*2.5r )/(5*3.14*\x r-5*3.14*2.5 r))});
%\draw[<->] (3.25,-0.3)--(3.45,-0.3) node[below left] {$\frac{1}{T}$};
\end{scope}
\end{tikzpicture}\\
Période infinie = Spectre continue\\
\end{center}
\end{minipage}
}\\


\vspace{0.3cm}
La réciproque est aussi vraie.\\
\vspace{0.3cm}\\
\fbox{
%\begin{center}
\begin{minipage}{\textwidth}
\begin{center}
\textbf{Spectre périodique = Signal de raies}
\begin{tikzpicture}
\draw[->] (2.5,0) node[below] {0} -- (2.5,1.5)node[above] {$e(t)$};
\draw[->] (0,0)-- (5,0) node[right] {$\nu$};

\draw[thick,blue] (2,0)--(2.25,0)--(2.25,1)--(2.75,1)--(2.75,0)--(3,0);

\begin{scope}[xshift=1cm]
\draw[thick,blue] (2,0)--(2.25,0)--(2.25,1)--(2.75,1)--(2.75,0)--(3,0);
\draw[dashed,black] (2.5,0) node[below]{$F$}--(2.5,1);
\end{scope}

\begin{scope}[xshift=2cm]
\draw[thick,blue] (2,0)--(2.25,0)--(2.25,1)--(2.75,1)--(2.75,0)--(3,0);
\end{scope}


\begin{scope}[xshift=-1cm]
\draw[thick,blue] (2,0)--(2.25,0)--(2.25,1)--(2.75,1)--(2.75,0)--(3,0);
\end{scope}

\begin{scope}[xshift=-2cm]
\draw[thick,blue] (2,0)--(2.25,0)--(2.25,1)--(2.75,1)--(2.75,0)--(3,0);
\end{scope}



\begin{scope}[xshift=7cm]
\draw[->] (2.5,0) node[below] {0} -- (2.5,1.5)node[above] {$E(t)$};
\draw[->] (0,0)-- (5,0) node[right] {$t$};
\draw[ domain=0.1:4.9,color=blue,samples=24] plot[ycomb] (\x,{80*abs(sin(5*3.14*\x r -5*3.14*2.5r )/(5*3.14*\x r-5*3.14*2.5 r))});
\draw[<->] (3.25,-0.3)--(3.45,-0.3) node[below left] {$\frac{1}{F}$};
\end{scope}
\end{tikzpicture}

\begin{tikzpicture}
\draw[->] (2.5,0) node[below] {0} -- (2.5,1.5)node[above] {$e(\nu)$};
\draw[->] (0,0)-- (5,0) node[below right] {$\nu$};

\draw[thick,blue] (1,0)--(2.25,0)--(2.25,1)--(2.75,1)--(2.75,0)--(4,0);


\begin{scope}[xshift=2cm]
\draw[thick,blue] (1,0)--(2.25,0)--(2.25,1)--(2.75,1)--(2.75,0)--(3,0);
\end{scope}


\begin{scope}[xshift=-2cm]
\draw[thick,blue] (2,0)--(2.25,0)--(2.25,1)--(2.75,1)--(2.75,0)--(4,0);
\end{scope}



\begin{scope}[xshift=7cm]
\draw[->] (2.5,0) node[below] {0} -- (2.5,1.5)node[above] {$E(t)$};
\draw[->] (0,0)-- (5,0) node[right] {$t$};
\draw[ domain=0.1:4.9,color=blue,samples=96] plot[ycomb] (\x,{80*abs(sin(5*3.14*\x r -5*3.14*2.5r )/(5*3.14*\x r-5*3.14*2.5 r))});
%\draw[<->] (3.25,-0.3)--(3.45,-0.3) node[below left] {$\frac{1}{T}$};
\end{scope}
\end{tikzpicture}
\end{center}
\end{minipage}
%
}

\end{document}