\documentclass[11pt,a4paper]{article}
%\usepackage[utf8]{inputenc}
%\usepackage[ascii]{inputenc}
\usepackage{geometry}
\usepackage[dvipsnames]{xcolor}
\usepackage{textcomp}
\usepackage{graphicx}
\usepackage{caption}
\usepackage{subcaption}
\usepackage{amssymb}
\usepackage{amsmath}
\usepackage{tikz}

\begin{document}

\section*{Résumé \& Méthodes chapitre II}
\subsection*{Transformée de Laplace}
\subsubsection*{Définition}
	Transformée de Laplace monolatérale : 
	\[L\{f\}(s) = \displaystyle \int^{\infty}_{0} f(t) e^{-st} \; dt\]\\ 
	\vspace{0.3cm}
	Transformée de Laplace bilatérale : \[L\{f\}(s) = \displaystyle \int^{\infty}_{-\infty} f(t) e^{-st} \; dt\]\\

Utile, entre autre, pour analyser des systèmes dynamiques de façon rapide. \\

Rappel :  $s$ est une variable complexe. La transformée de Laplace d'une fonction est une fonction complexe



\subsubsection*{Propriétés}

\begin{itemize}
\item Linéarité:  $L\{af+g\}(s) = a L\{ f\}(s) + L \{ g \} (s)$
\item Dérivation : $ L\{f'\}(s) = \; s L\{f \}(s)$
\item Intégration : $ L\{F\}(s) = \frac{\displaystyle 1}{\displaystyle s} L\{f \}(s) $
\item Convolution : $ L\{f \star g \} = L\{f \}(s) \cdot L \{ g \} (s)  $
\end{itemize}

\subsubsection*{Transformées Usuelles}
\begin{itemize}
\item $u(t)$  $\rightarrow \frac{\displaystyle 1}{\displaystyle p} $
\item $r(t) = t$  $\rightarrow \frac{\displaystyle 1}{ \displaystyle p^2} $
\item $f(t) = e^{-at}$ $\rightarrow \frac{\displaystyle 1}{\displaystyle p+a} $
\item $\delta (t)$ $\rightarrow 1$ 
\item $sin(\omega_0 t)$ = $\rightarrow \frac{\displaystyle \omega_0^2}{ \displaystyle p^2 + \omega_0^2} $
\end{itemize}

\subsection*{Transformée de Fourier}
\subsubsection*{Définition}
\[TF\{ f \}(\nu) = \displaystyle \int^{\infty}_{-\infty} f(t) e^{-j2\pi \nu t} dt\]

Rappel :  $\nu$ est une fréquence, et donc un réel. La transformée de Fourier d'une fonction est une fonction complexe

\subsubsection*{Lien avec Laplace}
\textbf{La transformée de Fourier} est un cas \textbf{particulier de la transformée de Laplace bilatérale.} En effet $s= \sigma + j \omega$ pour la variable de Laplace en général. La transformée de Fourier correspond au cas où $s = j \omega = j 2 \pi \nu$ avec $\nu$ la fréquence d'excitation.

\subsubsection*{Notion de spectre}
Si f(t) est une fonction, alors $F(\nu) = TF\{f\}(\nu)$ est appelé \textbf{spectre de la fonction $f$}
\end{document}