\documentclass[11pt,a4paper]{article}
%\usepackage[utf8]{inputenc}
%\usepackage[ascii]{inputenc}
%\usepackage{geometry}
\usepackage[margin=0.7in]{geometry}
\usepackage[dvipsnames]{xcolor}
\usepackage{textcomp}
\usepackage{graphicx}
\usepackage{caption}
\usepackage{subcaption}
\usepackage{amssymb}
\usepackage{amsmath}
\usepackage{tikz}

\begin{document}

\title{Résumé \& Méthodes: chapitre I}
\author{Antony Bazir}
\maketitle
Dans ce chapitre, on a donné quelques défintions générales... Puis 2 notions importantes ont été abordées:
\begin{itemize}
\item  Linéarité 
\item Invariance dans le temps
\end{itemize} 

\section{Définitions générales}
\textbf{Filtre} : "En traitement du signal, un filtre est un dispositif ou un processus permettant de retirer des composantes ou des parties indésirables d'un signal." (wikipedia.org)\\

\textbf{Comment définir un filtre ?} :
\begin{itemize}
\item Analogique/Numérique
\item Actif/Passif (analogique)
\item Causal/Non causal (numérique)
\item Linéaire/Non linéaire 
\item Invariant/Non invariant dans le temps
\end{itemize}

\section{Notion de linéarité}
\textbf{Définition Mathématique}:  $f : x \in \mathbb{R} \rightarrow f(x) \in \mathbb{R} $ et $(x_1,x_2,a,b) \in \mathbb{R}^4$ \\ alors $f$ linéaire si, \\

\[\boxed{f(a x_1 + b x_2) = a f(x_1) + b f(x_2)}\]
\vspace{0.2cm}\\
\textbf{Un système linéaire transforme donc une combinaison linéaire d'entrée en combinaison linéaire de sorties.}\\

\newpage
\section{Notion d'invariance dans le temps}
\textbf{Définition Mathématique}:
Soit $F$ un filtre de telle sorte que, pour $x(t)$ et $y(t)$ deux fonctions , $ \forall \; t \in \mathbb{R}, \; y(t) = F(x(t))$ ou $ \forall \; n \in \mathbb{N}, \; y[n] = F(x[n])$ \\

le filtre $F$  est dit \textbf{invariant dans le temps} si \\

\[\forall \tau \in \mathbb{R}, \;\; \boxed{F(x(t-\tau)) = y(t-\tau)}\]\\

\[\forall k \in \mathbb{N}, \;\; \boxed{F(x[n-k]) = y[n-k]}\]\\

En somme, \textbf{la sortie du système ne dépend pas du moment auquel l'entrée est appliquée}.

\section{Système Linéaire et Invariant dans le temps}
\textbf{Important}: Tous les systèmes décrits dans ce cours sont linéaires et invariants dans le temps. \textbf{C'est grâce à ces deux propriétés qu'on peut écrire des fonctions de transferts et prédire la sortie d'un système à partir de son entrée et de sa réponse impulsionnelle ou sa fonction de transfert.}\\

\begin{center}
\fbox{
\textbf{Linéarité + Invariance dans le temps $\rightarrow$ fonction de transfert}
}
\end{center}

\end{document}