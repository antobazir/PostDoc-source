\documentclass{beamer}

\usepackage{epsfig}
\usepackage{multicol}
\usepackage{geometry}
%\usepackage[dvipsnames]{xcolor}
\usepackage{textcomp}
\usepackage{graphicx}
\usepackage{caption}
\usepackage{subcaption}
\usepackage{amsmath}
\usepackage{tcolorbox}
\usetheme{Boadilla}
\usepackage{pict2e}
\usepackage{tikz}
\usepackage{xcolor}

\title[Traitement du signal numérique]{Traitement du signal numérique - HEI4 IMS}
\author[Antony Bazir]{}

\setlength{\unitlength}{1cm}

\begin{document}
\section{Doppler Ultrasound}
\subsection{Bases physiques}
\begin{frame}
\frametitle{\'Echographie Doppler:  Bases physiques}
\center
\includegraphics[scale=0.45]{doppler_principle.png}
\end{frame}

\begin{frame}
\frametitle{\'Echographie Doppler:  Bases physiques}
Principe : \\
\vspace{0.5cm}
\[ f_{rec} = f_{em} \cdot \frac{c - v_{rec}}{c - v_{em}} \]
\vspace{0.5cm}
\begin{itemize}
\item $f_{rec}$ : fréquence perçue par le récepteur
\vspace{0.2cm}
\item $f_{em}$ : fréquence émise par la source\\
\vspace{0.2cm}
\item $c$ : La vitesse du son dans le milieu
\vspace{0.2cm}
\item $v_{rec}$ : vitesse du récepteur dans le milieu
\vspace{0.2cm}
\item $v_{em}$ : vitesse de la source dans le milieu

\end{itemize} 
\end{frame}

\begin{frame}
\frametitle{\'Echographie Doppler:  Bases physiques}
En pratique:
\\
\vspace{0.5cm} 
\[ \Delta f = f_{em} \cdot 2 \frac{v}{c} \cos(\theta) \]
\vspace{0.5cm}
\begin{itemize}
\item $f_{em}$ : fréquence émise par la source\\
\vspace{0.2cm}
\item $c$ : La vitesse du son dans le milieu
\vspace{0.2cm}
\item $v$ : vitesse de la source dans le milieu
\vspace{0.2cm}
\item $\theta$ : angle de la trajectoire de la source par rapport au capteur
\vspace{0.2cm}
\item $\Delta_f$ : décalage en fréquence dû à l'effet Doppler
\end{itemize} 
\end{frame}

\begin{frame}
\frametitle{\'Echographie Doppler:  Bases physiques}
Application numérique
\\
\vspace{0.5cm} 
\[ \Delta f = f_{em} \cdot 2 \frac{v}{c} \cos(\theta) \]
\vspace{0.5cm}
\begin{itemize}
\item $f_{em}$ : 2 MHz
\item $c$ : 1540 m/s
\vspace{0.2cm}
\item $v$ : 0.3 m/s 
\vspace{0.2cm}
\item $\theta$ : 45°
\vspace{0.2cm}
\item $\Delta_f$ : 548 Hz
\end{itemize}
\end{frame}

\begin{frame}
\frametitle{\'Echographie Doppler: Situation réelle  }
\begin{columns}
\column{60mm}
\includegraphics[scale=0.3]{doppler_pratique.png}\\
\textit{Allan, 2012}
\column{60mm}
\begin{itemize}
\item Vitesse \& angle variées 
\vspace{0.4cm}
\item Interactions possibles entre les champs 
\end{itemize}
$\rightarrow$ bande de détection doppler
\end{columns}
\end{frame}

\begin{frame}
\frametitle{\'Echographie Doppler: Mode d'émission }
\begin{itemize}
\item Doppler en émission continu
\vspace{0.2cm}
\begin{itemize}
\item Permet d'analyser une grande gamme de vitesse de flux 
\vspace{0.2cm}
\item Localisation précise difficile
\vspace{0.2cm}
\end{itemize}
\item Doppler pulsé
\vspace{0.2cm}
\begin{itemize}
\item Vitesse max détectable limitée\\
\vspace{0.2cm}
\item Bonne propriété de localisation via temps de vol
\end{itemize}
\end{itemize}
\end{frame}

\begin{frame}
\frametitle{\'Echographie Doppler: Structure des sondes}
\begin{columns}
\column{60mm}
\begin{center}
\includegraphics[scale=0.4]{probes.png}\\
\textit{\footnotesize Configurations de sondes et volumes d'études, Allan, 2014}
\end{center}
\column{60mm}
\begin{itemize}
\item A. Sonde tandem pour le Doppler continu
\vspace{0.5cm}
\item B. Sonde unique pour le doppler pulsé 
\vspace{0.5cm}
\item C. Sonde unique en mode "multi-focus"
\end{itemize}
\vspace{0.4cm}
Possibilité de sondes matricielles 
\end{columns}
\end{frame}

\begin{frame}
\frametitle{\'Echographie Doppler: Fréquence des sondes}
\begin{columns}
\column{60mm}
\begin{center}
\includegraphics[scale=0.4]{probes.png}\\
\textit{\footnotesize Configurations de sondes et volumes d'études, Allan, 2014}
\end{center}
\column{60mm}
On cherche à distinguer des vaisseaux sanguins.\\
\vspace{0.5cm}
Diamètre vaisseaux:
\begin{itemize}
\item Capillaire : $\approx$ 10 \textmu m
\item veine : $\approx$ 1 mm
\item artères : $\approx$ 1 cm
\end{itemize}
\vspace{0.5cm}
$\lambda  =  \frac{\displaystyle c}{\displaystyle f} \approx 1$ mm $\rightarrow$ $f$ $\approx$ 1 MHz
\end{columns}
\end{frame}

\begin{frame}
\frametitle{\'Echographie Doppler: images Doppler}
\begin{columns}
\column{60mm}
\begin{center}
\includegraphics[scale=0.3]{imaging_doppler.png}\\
\textit{\footnotesize Sonde duplex, Allan, 2014}
\end{center}
\column{60mm}
\begin{center}
\includegraphics[scale=0.35]{image_doppler.png}\\
\textit{\footnotesize image duplex en mode normal et puissance, Allan, 2014}
\end{center}
\end{columns}
\end{frame}

\begin{frame}
\frametitle{\'Echographie Doppler: images Doppler}
\begin{columns}
\column{60mm}
\begin{center}
\includegraphics[scale=0.35]{image_doppler.png}\\
\textit{\footnotesize image duplex en mode normal et puissance, Allan, 2014}
\end{center}
\column{60mm}
\begin{itemize}
\item Doppler couleur
\vspace{0.2cm}
\begin{itemize}
\item Méthode pulsée
\vspace{0.2cm}
\item donne la vitesse et la direction du flux
\vspace{0.2cm} 
\end{itemize}
\item Doppler de puissance
\vspace{0.2cm}
\begin{itemize}
\item Mesure la puissance du signal doppler uniquement
\vspace{0.2cm}
\item Produit des images avec un meilleur rapport signal/bruit 
\end{itemize}
\end{itemize}
\end{columns}
\end{frame}

\begin{frame}
\frametitle{\'Echographie Doppler: Analyse spectrale}
\begin{center}
\includegraphics[scale=0.4]{spectra.png}\\
\textit{\footnotesize principe sonogramme instantané, Allan, 2014}
\end{center}
\end{frame}

\begin{frame}
\frametitle{\'Echographie Doppler: Analyse spectrale}
\begin{center}
\includegraphics[scale=0.4]{sonogram.png}\\
\textit{\footnotesize Principe sonogramme instantané, Allan, 2014}
\end{center}
\end{frame}

\begin{frame}
\frametitle{\'Echographie Doppler: images Doppler}
\begin{columns}
\column{60mm}
\begin{center}
\includegraphics[scale=0.5]{sonogram2.png}\\
\textit{\footnotesize Sonogramme, Allan, 2014}
\end{center}
\column{60mm}
Chaque spectre est mis à côte à côte et l'amplitude de chaque fréquence est en niveau de gris
\vspace{0.3cm}
\begin{itemize}
\item $T_s$ = 5 ms  $\rightarrow f_{min}$ = 200 Hz 
\vspace{0.3cm}
\item $T_e$ = 1 \textmu s $\rightarrow f_{max}$ = 500 kHz 
\vspace{0.3cm}
\item $f_{min} \rightarrow$ résolution spectrale
\end{itemize}
\end{columns}
\end{frame}

\begin{frame}
\frametitle{\'Echographie Doppler: images Doppler}
\begin{columns}
\column{80mm}
\begin{center}
\includegraphics[scale=0.5]{sonogram3.png}\\
\textit{\footnotesize Sonogramme avec vitesse moyenne et maximale mises en évidences, Allan, 2014}
\end{center}
\column{40mm}
Informations :
\vspace{0.3cm}
\begin{itemize}
\item Ligne blanche: vitesse max 
\vspace{0.3cm}
\item Ligne bleue: vitesse moyenne
\vspace{0.3cm}
\end{itemize}
On peut suivre le flux sanguin au cours du temps
\end{columns}
\end{frame}

\begin{frame}
\frametitle{\'Echographie Doppler: Artefacts possibles}
Artéfacts: phénomènes pouvant conduire à une mauvaise évaluation des variables étudiés
\begin{itemize}
\item \textbf{Atténuation }: perte de puissance acoustique avec la profondeur 
\vspace{0.2cm}
\item \textbf{Réfraction} : perte de puissance acoustique aux interfaces 
\vspace{0.2cm}
\item \textbf{Effet de faisceau}: Contribution non-désirée ou difficile à déchiffrer due au caractéristiques du faisceau
\vspace{0.2cm}
\item \textbf{Distorsion \& Génération d'harmoniques}
\vspace{0.2cm}
\item \textbf{Effet lié au patient} : Pression sur le vaisseau...
\vspace{0.2cm}
\item \textbf{Effet de repliment spectral}
\end{itemize}

\end{frame}

\begin{frame}
\frametitle{\'Echographie Doppler: Repliement spectral}
\begin{columns}
\column{60mm}
\begin{center}
\includegraphics[scale=0.35]{aliasing.png}\\
\textit{\footnotesize Repliment spectral, Allan, 2014}
\end{center}
\column{60mm}
Principe :
\vspace{0.3cm}
\begin{itemize}
\item Mode pulsée $\rightarrow$ "échantillonnage" signal 
\vspace{0.3cm}
\item échantillonnage $\rightarrow$ Shannon
\vspace{0.3cm}
\item Shannon non respecté $\rightarrow$  repliement
\vspace{0.3cm}
\end{itemize}
\end{columns}
\end{frame}

\begin{frame}
\frametitle{\'Echographie Doppler: Hémodynamique}
\begin{columns}
\column{70mm}
\begin{center}
\includegraphics[scale=0.3]{hemodynamique.png}\\
\textit{\footnotesize Système circulatoire humain}
\end{center}
\column{50mm}
Hémodynamique:
\vspace{0.3cm}
\begin{itemize}
\item Etude des flux sanguins dans le corps 
\vspace{0.3cm}
\item Objet d'étude du Doppler
\vspace{0.3cm}
\item Potentiel d'amélioration de la technologie Doppler
\vspace{0.3cm}
\end{itemize}
\end{columns}
\end{frame}

\begin{frame}
\frametitle{\'Echographie Doppler: Hémodynamique}
\underline{Vitesse du flux }: 
\begin{center}
\includegraphics[scale=0.45]{blood_flow.png}\\
\textit{\footnotesize vitesse flux sanguin, B. Cummings 2004}
\end{center}

Selon où on mesure le décalage Doppler peut varier d'un ordre de grandeur\\

\vspace{0.3cm}
Il faut donc adapter la résolution fréquentielle.

\end{frame}

\begin{frame}
\frametitle{\'Echographie Doppler: Hémodynamique}
\underline{Nature du flux }: 
\begin{center}
\includegraphics[scale=0.4]{flow.png}\\
\textit{\footnotesize Flux laminaire/turbulent, Allan, 2014}
\end{center}
\vspace{0.3cm}
nature flux $\rightarrow$ distribution des vitesses $\rightarrow$ spectre Doppler

\end{frame}

\begin{frame}
\frametitle{\'Echographie Doppler: Hémodynamique}
\underline{Sonogramme }: 
\begin{center}
\includegraphics[scale=0.35]{lam_turb.png}\\
\textit{\footnotesize Flux laminaire/turbulent, Allan, 2014}
\end{center}
\vspace{0.3cm}
flux turbulent $\rightarrow$ pics haute fréquence + contribution négative chaotique

\end{frame}

\begin{frame}
\frametitle{Doppler, coeur \& Ultrasons}
\begin{columns}
\column{60mm}
\begin{center}
\includegraphics[scale=0.3]{heart.png}\\
\textit{\footnotesize Schéma du coeur avec ses pacemakers
naturels, Najarian \& Splinter, 2010}
\end{center}
\column{60mm}
\begin{itemize}
\item On a étudié l'electroencéphalogramme
\vspace{0.5cm}
\item On a étudié le doppler 
\vspace{0.5cm}
\item \textbf{On peut mélanger les deux}
\end{itemize}
\end{columns}
\end{frame}

\begin{frame}
\frametitle{Doppler, coeur \& Ultrasons}
\begin{columns}
\column{60mm}
\begin{center}
\includegraphics[scale=0.3]{heart.png}\\
\textit{\footnotesize Schéma du coeur avec ses pacemakers
naturels, Najarian \& Splinter, 2010}
\end{center}
\column{60mm}
Mécanisme:
\begin{itemize}
\item Activation muscle cardiaque $\rightarrow$ ondes acoustiques
\vspace{0.5cm}
\item Propagation acoustique tissus $\rightarrow$ informations physiologiques
\vspace{0.5cm}

\end{itemize}

\end{columns}
\end{frame}

%
%\begin{frame}
%\frametitle{Doppler, coeur \& Ultrasons}
%\begin{columns}
%\column{60mm}
%\begin{center}
%\includegraphics[scale=0.4]{pwv_artery.png}\\
%\textit{\footnotesize Propagation ondes méca le long des parois artérielles}
%\end{center}
%\column{60mm}
%Soit $c$ la vitesse de l'onde mécanique dans l'artère  :
%
%\[ c = \sqrt{\frac{Eh}{2 \rho R }} \] 
%
%\begin{itemize}
%\item $E$ : module élastique de la paroi 
%\item $h$ : épaisseur de la paroi artérielle
%\item $\rho$ : la densité du sang
%\item $R$ : la rayon interne du vaisseaux
%\end{itemize}
%
%\end{columns}
%\end{frame}

\begin{frame}
\frametitle{Doppler, coeur \& Ultrasons}
\begin{columns}
\column{60mm}
\begin{center}
\includegraphics[scale=0.25]{TDI_myocard.png}\\
\textit{\footnotesize Images Doppler tissulaire, 1992}
\end{center}
\column{60mm}
\underline{Principe Doppler tissulaire} :
\vspace{0.3cm}
\begin{itemize}
\item Traducteur linéaire multiéléments
\vspace{0.3cm}
\item Filtrage "wall thump"/Clutter pour éliminer les basses fréquences non associées au mouvement recherché 
\vspace{0.3cm}
\item Réglage assez différent du Doppler "sanguin"
\end{itemize}

\end{columns}
\end{frame}


\begin{frame}
\frametitle{Doppler, coeur \& Ultrasons}
\begin{columns}
\column{60mm}
\begin{center}
\includegraphics[scale=0.3]{tdi_cfwi.png}\\
\textit{\footnotesize Images ultrasons du myocarde, Salles et al., 2019}
\end{center}
\column{60mm}
Depuis 1992... \\
\vspace{0.5cm}
\begin{itemize}
\item Amélioration des sensibilités des sondes ultrasonores
\vspace{0.5cm}
\item Augmentation des cadences d'imagerie
\vspace{0.5cm}
\item Augmentation des capacités de calcul/stockage pour le filtrage/traitement
\vspace{0.5cm}

\end{itemize}
Travaux de recherche de Sébastien Salles \textit{et al.}
\end{columns}
\end{frame}

\begin{frame}
\frametitle{Doppler, coeur \& Ultrasons}
\begin{columns}
\column{60mm}
\begin{center}
\includegraphics[scale=0.4]{cfwi2.png}\\
\textit{\footnotesize Principe filtrage image ultrasonore}
\end{center}
\column{60mm}
Principe "Clutter Filter Wave Imaging": \\
\vspace{0.5cm}
\begin{itemize}
\item Suite d'images ultrasonores $\rightarrow$ ensemble de fréquences Doppler
\vspace{0.5cm}
\item Spectre Doppler $\rightarrow$ Ensemble complexe de mouvement de tissu
\vspace{0.5cm}
\item Filtrage sélectif pour mettre en évidence l'onde tissulaire recherchée
\vspace{0.5cm}

\end{itemize}

\end{columns}
\end{frame}


\begin{frame}
\frametitle{Doppler, coeur \& Ultrasons}
\begin{columns}
\column{60mm}
\begin{center}
\includegraphics[scale=0.4]{cfwi2.png}\\
\textit{\footnotesize Principe filtrage image ultrasonore}
\end{center}
\column{60mm}
Etapes du traitement:\\
\vspace{0.5cm}
\begin{enumerate}
\item Filtre "clutter"/fouillis radar
\vspace{0.5cm}
\item Détection d'enveloppe et conversion en image "B-mode"
\vspace{0.5cm}
\item Filtrage passe-bas
\vspace{0.5cm}
\item Gradient temporel
\vspace{0.5cm}
\end{enumerate}
Travaux de recherche de Sébastien Salles \textit{et al.}
\end{columns}
\end{frame}


\begin{frame}
\frametitle{Doppler, coeur \& Ultrasons}
\underline{\'Etape 1 filtre Clutter}:
\begin{columns}
\column{70mm}
\begin{center}
\includegraphics[scale=0.4]{cfwi.png}\\
\textit{\footnotesize Illustration filtre Clutter, Salles \textit{et al.}, 2019}
\end{center}
\column{50mm}
Filtrage de suite d'images\\
\vspace{0.5cm}
\begin{enumerate}
\item Chaque pixel a une vitesse
\vspace{0.5cm}
\item \textbf{Coupe} les ondes venant des tissus stationnaires
\vspace{0.5cm}
\item \textbf{Atténue} sélectivement les vitesses recherchées
\end{enumerate}
\end{columns}
\end{frame}

\begin{frame}
\frametitle{Doppler, coeur \& Ultrasons}
\underline{\'Etape 2 Image B-mode}:

\begin{center}
\includegraphics[scale=0.4]{B-mode_envelope.png}\\
\textit{\footnotesize Principe formation d'image B-mode, Ouzir, 2018}
\end{center}
\vspace{0.2cm}
\begin{enumerate}
\item Ondes ultrasonores filtrées $\rightarrow$ Champ de vitesse
\vspace{0.2cm}
\item Enveloppes d'amplitude au cours du temps
\vspace{0.2cm}
\item Conversion en niveau de gris + Log
\end{enumerate}
\end{frame}

\begin{frame}
\frametitle{Doppler, coeur \& Ultrasons}
\underline{\'Etape 3 Lissage spatiotemporel}:
\begin{columns}
\column{60mm}
\begin{center}
\includegraphics[scale=0.4]{smoothing_cfwi.png}\\
\textit{\footnotesize Lissage spatiotemporel, Salles et al., 2019}
\end{center}
\column{60mm}
\vspace{0.2cm}
\begin{itemize}
\item Filtrage passe-bas spatiotemporel
\vspace{0.8cm}
\item \'Etape optionnelle d'optimisation

\end{itemize}
\end{columns}
\end{frame}

\begin{frame}
\frametitle{Doppler, coeur \& Ultrasons}
\underline{\'Etape 4 Dérivation}:
\begin{columns}
\column{60mm}
\begin{center}
\includegraphics[scale=0.4]{diff_cfwi.png}\\
\textit{\footnotesize Dérivation, Salles et al., 2019}
\end{center}
\column{60mm}
\vspace{0.2cm}
\begin{itemize}
\item Résultat étape 3 : cartes de vitesses
\vspace{0.5cm}
\item Onde méca recherchée définie via accélération
\vspace{0.5cm}
\item Dérivée temporelle vitesse = accélération
\end{itemize}
\end{columns}
\end{frame}


\begin{frame}
\frametitle{Doppler, coeur \& Ultrasons}
\underline{Résultats}:
\begin{columns}
\column{60mm}
\begin{center}
\includegraphics[scale=0.35]{cfwi_left_vent.png}\\
\textit{\footnotesize Résultat CFWI, Salles \textit{et al.}, 2019}
\end{center}
\column{60mm}
\underline{Filtrage de suite d'images}\\
\vspace{0.5cm}
\begin{enumerate}
\item Atténuation = zone sombre dans le doppler
\vspace{0.5cm}
\item Opérations successives pour remonter à l'onde dans le signal 
\vspace{0.5cm}
\end{enumerate}
On peut \textbf{suivre} l'onde mécanique 
\end{columns}
\end{frame}

\begin{frame}
\frametitle{Doppler, coeur \& Ultrasons}
Bilan
\begin{columns}
\column{60mm}
\begin{center}
\includegraphics[scale=0.35]{cfwi_left_vent.png}\\
\textit{\footnotesize Résultat CFWI, Salles \textit{et al.}, 2019}
\end{center}
\column{60mm}
Poursuites et applications\\
\vspace{0.5cm}
\begin{enumerate}
\item Technique désormais utilisée en 3D
\vspace{0.5cm}
\item Outils de diagnostic non invasif complémentaire  
\vspace{0.5cm}
\end{enumerate}
Toujours en développement... 
\end{columns}
\end{frame}

\begin{frame}
\frametitle{Bilan Doppler}
 
\begin{itemize}
\item Technique basée sur l'effet Doppler 
\vspace{0.5cm}
\item Utilisée régulièrement en clinique pour mesurer les flux sanguins
\vspace{0.5cm}
\item Utilisée sur les tissus depuis une trentaine d'année
\vspace{0.5cm}
\item Possibilité de suivre des ondes en 3D grâce aux progrès en imagerie
\vspace{0.5cm}
\end{itemize}
\end{frame}
\end{document}
